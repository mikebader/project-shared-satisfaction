\documentclass[]{article}
\usepackage{rotating}
\usepackage{setspace}
\usepackage{lmodern}
\usepackage{amssymb,amsmath}
\usepackage{ifxetex,ifluatex}
\usepackage{fixltx2e} % provides \textsubscript
\ifnum 0\ifxetex 1\fi\ifluatex 1\fi=0 % if pdftex
  \usepackage[T1]{fontenc}
  \usepackage[utf8]{inputenc}
\else % if luatex or xelatex
  \ifxetex
    \usepackage{mathspec}
  \else
    \usepackage{fontspec}
  \fi
  \defaultfontfeatures{Ligatures=TeX,Scale=MatchLowercase}
\fi
% use upquote if available, for straight quotes in verbatim environments
\IfFileExists{upquote.sty}{\usepackage{upquote}}{}
% use microtype if available
\IfFileExists{microtype.sty}{%
\usepackage{microtype}
\UseMicrotypeSet[protrusion]{basicmath} % disable protrusion for tt fonts
}{}
\usepackage[margin=1in]{geometry}
\usepackage{hyperref}
\hypersetup{unicode=true,
            pdftitle={Satisfaction and Stability in Multiethnic Neighborhoods},
            pdfauthor={Michael Bader},
            pdfborder={0 0 0},
            breaklinks=true}
\urlstyle{same}  % don't use monospace font for urls
\usepackage{color}
\usepackage{fancyvrb}
\newcommand{\VerbBar}{|}
\newcommand{\VERB}{\Verb[commandchars=\\\{\}]}
\DefineVerbatimEnvironment{Highlighting}{Verbatim}{commandchars=\\\{\}}
% Add ',fontsize=\small' for more characters per line
\usepackage{framed}
\definecolor{shadecolor}{RGB}{248,248,248}
\newenvironment{Shaded}{\begin{snugshade}}{\end{snugshade}}
\newcommand{\KeywordTok}[1]{\textcolor[rgb]{0.13,0.29,0.53}{\textbf{#1}}}
\newcommand{\DataTypeTok}[1]{\textcolor[rgb]{0.13,0.29,0.53}{#1}}
\newcommand{\DecValTok}[1]{\textcolor[rgb]{0.00,0.00,0.81}{#1}}
\newcommand{\BaseNTok}[1]{\textcolor[rgb]{0.00,0.00,0.81}{#1}}
\newcommand{\FloatTok}[1]{\textcolor[rgb]{0.00,0.00,0.81}{#1}}
\newcommand{\ConstantTok}[1]{\textcolor[rgb]{0.00,0.00,0.00}{#1}}
\newcommand{\CharTok}[1]{\textcolor[rgb]{0.31,0.60,0.02}{#1}}
\newcommand{\SpecialCharTok}[1]{\textcolor[rgb]{0.00,0.00,0.00}{#1}}
\newcommand{\StringTok}[1]{\textcolor[rgb]{0.31,0.60,0.02}{#1}}
\newcommand{\VerbatimStringTok}[1]{\textcolor[rgb]{0.31,0.60,0.02}{#1}}
\newcommand{\SpecialStringTok}[1]{\textcolor[rgb]{0.31,0.60,0.02}{#1}}
\newcommand{\ImportTok}[1]{#1}
\newcommand{\CommentTok}[1]{\textcolor[rgb]{0.56,0.35,0.01}{\textit{#1}}}
\newcommand{\DocumentationTok}[1]{\textcolor[rgb]{0.56,0.35,0.01}{\textbf{\textit{#1}}}}
\newcommand{\AnnotationTok}[1]{\textcolor[rgb]{0.56,0.35,0.01}{\textbf{\textit{#1}}}}
\newcommand{\CommentVarTok}[1]{\textcolor[rgb]{0.56,0.35,0.01}{\textbf{\textit{#1}}}}
\newcommand{\OtherTok}[1]{\textcolor[rgb]{0.56,0.35,0.01}{#1}}
\newcommand{\FunctionTok}[1]{\textcolor[rgb]{0.00,0.00,0.00}{#1}}
\newcommand{\VariableTok}[1]{\textcolor[rgb]{0.00,0.00,0.00}{#1}}
\newcommand{\ControlFlowTok}[1]{\textcolor[rgb]{0.13,0.29,0.53}{\textbf{#1}}}
\newcommand{\OperatorTok}[1]{\textcolor[rgb]{0.81,0.36,0.00}{\textbf{#1}}}
\newcommand{\BuiltInTok}[1]{#1}
\newcommand{\ExtensionTok}[1]{#1}
\newcommand{\PreprocessorTok}[1]{\textcolor[rgb]{0.56,0.35,0.01}{\textit{#1}}}
\newcommand{\AttributeTok}[1]{\textcolor[rgb]{0.77,0.63,0.00}{#1}}
\newcommand{\RegionMarkerTok}[1]{#1}
\newcommand{\InformationTok}[1]{\textcolor[rgb]{0.56,0.35,0.01}{\textbf{\textit{#1}}}}
\newcommand{\WarningTok}[1]{\textcolor[rgb]{0.56,0.35,0.01}{\textbf{\textit{#1}}}}
\newcommand{\AlertTok}[1]{\textcolor[rgb]{0.94,0.16,0.16}{#1}}
\newcommand{\ErrorTok}[1]{\textcolor[rgb]{0.64,0.00,0.00}{\textbf{#1}}}
\newcommand{\NormalTok}[1]{#1}
\usepackage{longtable,booktabs}
\usepackage{graphicx,grffile}
\makeatletter
\def\maxwidth{\ifdim\Gin@nat@width>\linewidth\linewidth\else\Gin@nat@width\fi}
\def\maxheight{\ifdim\Gin@nat@height>\textheight\textheight\else\Gin@nat@height\fi}
\makeatother
% Scale images if necessary, so that they will not overflow the page
% margins by default, and it is still possible to overwrite the defaults
% using explicit options in \includegraphics[width, height, ...]{}
\setkeys{Gin}{width=\maxwidth,height=\maxheight,keepaspectratio}
\IfFileExists{parskip.sty}{%
\usepackage{parskip}
}{% else
\setlength{\parindent}{0pt}
\setlength{\parskip}{6pt plus 2pt minus 1pt}
}
\setlength{\emergencystretch}{3em}  % prevent overfull lines
\providecommand{\tightlist}{%
  \setlength{\itemsep}{0pt}\setlength{\parskip}{0pt}}
\setcounter{secnumdepth}{5}
% Redefines (sub)paragraphs to behave more like sections
\ifx\paragraph\undefined\else
\let\oldparagraph\paragraph
\renewcommand{\paragraph}[1]{\oldparagraph{#1}\mbox{}}
\fi
\ifx\subparagraph\undefined\else
\let\oldsubparagraph\subparagraph
\renewcommand{\subparagraph}[1]{\oldsubparagraph{#1}\mbox{}}
\fi

%%% Use protect on footnotes to avoid problems with footnotes in titles
\let\rmarkdownfootnote\footnote%
\def\footnote{\protect\rmarkdownfootnote}

%%% Change title format to be more compact
\usepackage{titling}

% Create subtitle command for use in maketitle
\newcommand{\subtitle}[1]{
  \posttitle{
    \begin{center}\large#1\end{center}
    }
}

\setlength{\droptitle}{-2em}

  \title{Satisfaction and Stability in Multiethnic Neighborhoods}
    \pretitle{\vspace{\droptitle}\centering\huge}
  \posttitle{\par}
    \author{Michael Bader}
    \preauthor{\centering\large\emph}
  \postauthor{\par}
      \predate{\centering\large\emph}
  \postdate{\par}
    \date{2019-05-13}

\usepackage{array}
\usepackage{caption}
\usepackage{graphicx}
\usepackage{siunitx}
\usepackage[table]{xcolor}
\usepackage{multirow}
\usepackage{hhline}
\usepackage{calc}
\usepackage{tabularx}

\usepackage{amsthm}
\newtheorem{theorem}{Theorem}[section]
\newtheorem{lemma}{Lemma}[section]
\theoremstyle{definition}
\newtheorem{definition}{Definition}[section]
\newtheorem{corollary}{Corollary}[section]
\newtheorem{proposition}{Proposition}[section]
\theoremstyle{definition}
\newtheorem{example}{Example}[section]
\theoremstyle{definition}
\newtheorem{exercise}{Exercise}[section]
\theoremstyle{remark}
\newtheorem*{remark}{Remark}
\newtheorem*{solution}{Solution}
\begin{document}
\maketitle

%{
%\setcounter{tocdepth}{2}
%\tableofcontents
%}
\doublespacing
\section{Introduction}\label{introduction}

Multiethnic neighborhoods have emerged as a common feature in U.S.
metropolitan areas during the past half century. Federal
anti-discrimination laws, changing racial attitudes, and increased
immigration created new opportunities for people to become neighbors
with people across multiple racial and ethnic categories. Once
exceptional, racial integration among the four largest racial and ethnic
groups in the United States has become one of the most common forms--and
in some metropolitan areas, \emph{the} most common form--of neighborhood
racial change since the 1970s and 1980s \textbf{{[}cite: Logan \& Zhang;
Bader \& Warkentien{]}}. In a nation beset by its history of racial
discrimination and resulting segregation, multiethnic neighborhoods
offer hope for a more integrated future.

Academic research has not kept pace with the growing presence of
multiethnically integrated neighborhoods in American metropolitan areas.
A small body of qualitative works studying the unique dynamics of
integrated neighborhoods provides valuable insight\textbf{{[}w.ch{]}}
into the dynamics of racial integration. But research on neighborhood
gentrification and the concentration of poverty has dwarfed that which
focuses on this emergent form of neighborhood change. Without systematic
research on multiethnic neighborhoods, any predictions about the
sustainability of this newfound integration will be imprecise and
insufficient to inform policies to maintain multiethnic integration.

To help build the body of evidence on multiethnic neighborhoods, this
article examines whether residents of all races feel equally satisfied
living in multiethnic neighborhoods. It does so among a representative
sample of residents living in multiethnic neighborhoods in the
Washington, D.C. area, a major metropolitan area in the United States
with a large number of multiethnic neighborhoods. Evidence from the
sample shows that satisfaction is quite high, and that whites are
similarly satisfied living in multiethnic neighborhoods as their Asian,
black, and Latino neighbors. The findings also show, however, mixed
evidence that multiethnic segregation will remain sustainable in the
future as the neighborhoods continue to change. The findings reveal
directions of future research and possible policy intervention that
might be necessary to understand and promote racial integration in the
coming decades.

\section{Background}\label{background}

Multiethnic neighborhoods came about as Asian, black, and Latinx
newcomers moved into neighborhoods that had been all white, or nearly
so, starting in the 1980s \textbf{(cite: Logan; Bader)}. The trend
accelerated in the 1990s and 2000s. Unlike past decades where the entry
of non-white residents sparked violence and white flight that caused
racial transitions, many whites stayed put as their neighborhoods
integrated around them. The number of neighborhoods shared among white,
Asian, black, and Latinx neighbors rose rapidly in the 1990s and 2000s.
Logan and Zhang estimate that \textbf{TK\%} of neighborhoods in 2010
were what they termed ``global neighborhoods.'' Similarly, Bader and
Warkentien found stable multiethnic integration represented between
\textbf{TK\%} and \textbf{TK\%} of neighborhoods in the metropolitan
areas of the four largest cities.

Several currents made the growth of multiethnic neighborhoods possible.
Federal legislation, especially the Fair Housing Act, made housing
discrimination a federal crime. The impact of the FHA, which was passed
in 1968 in the wake of the Rev.~Dr.~Martin Luther King, Jr.'s
assassination, was neither immediate nor universal. Racial
discrimination has persisted in more subtle ways such as real-estate
agents ``steering'' clients to neighborhoods. The FHA provided, however,
provided a legal mechanism for people of color to enter white
neighborhoods and challenge overt forms of discrimination.

Second, \emph{multiethnic} integration would be impossble without a
multiethnic population. Increased immigration from Asian and Latin
American countries increased the number of Asian and Latinx people in
the United States. Immigration unsettled the stark black-white divide in
many American metropolitan areas. The trend\textbf{{[}w.ch{]}} was most
pronounced in metropolitan areas that became ``new immigrant''
destinations in the 1980s and 1990s. The growing economies that
attracted migrants also required new housing construction to accomodate
growing populations that made integrated neighborhoods possible from the
get-go\textbf{{[}w.ch.{]}}.

Finally, racial attitudes have been changing.\textbf{Add text.}

\subsection{Satisfaction and Long-Term
Stability}\label{satisfaction-and-long-term-stability}

\subsubsection{Satisfaction and
Mobility}\label{satisfaction-and-mobility}

Neighborhood satisfaction influences whether people stay put in their
neighborhoods. The process of white flight can be described as whites
having acted on the immediate and strong feelings of dissatisfaction
that the entry of a black family prompted. As whites vacated their
houses, additional black families--who were dissatisfied being forced to
live in overcrowded ghettos through law and custom--filled those
vacancies. The process resulted in the rapid ``succession'' of the
neighborhood's composition from all-white to all-black.

The more sanguine attitudes whites hold about people of color could mean
that whites are now satisfied living in integrated neighborhoods. If so,
then there is a stronger prospect of long-term integration. Evidence is
mixed. On the one hand, patterns of neighborhood change seem to bear
this out, with little evidence of rapid racial succession since the
1980s. On the other, studies consistently show lower levels of
satisfaction among whites who live in neighborhoods in which people of
color make up an increasing share of their neighbors.

Existing studies of satisfaction among residents measure the composition
of racial groups separately or measure the composition of all non-white
residents grouped into a single category. \textbf{{[}cite findings
here{]}} Multiethnic neighborhoods represent a distinctive form of
integration not captured by the method of measuring racial compositions
of groups. Instead, the \emph{presence} of all four racial groups
distinguishes these neighborhoods and satisfaction among residents of
different races in these distinctive neighborhoods have not been
measured.

Measuring whether residents think that the neighborhood has improved
provides an additional measure of satisfaction. While satisfied
residents might not feel compelled to leave, they could be more inclined
to leave if they perceive less improvement in the neighborhood. Previous
research shows that white perceptions of neighborhood trends diminish
with increasing non-white composition. Whites might not, therefore, flee
neighborhoods as they become integrated, but might be differentially
susceptible to being enticed to leave integrated neighborhoods. Again,
however, these studies do not sample \emph{multiethnic} neighborhoods.

\subsubsection{Long-Term Stability}\label{long-term-stability}

Even residents satisfied living in their neighborhood will find
themselves exiting for reasons unrelated to the racial composition of
the neighborhood. Residents will get married, retire, have children, get
divorced, and, in the most extreme case, exit upon their death. Life
course transitions such as these cause most moves; and, as residents
move out for those reasons, new residents will move in. Satisfaction
will be insufficient to predict future integration because integration
will survive only as long as the current cohort of residents does. To
maintain the multiethnic diversity of the neighborhood therefore
requires that the racial composition of households \emph{leaving}
multiethnic neighborhoods approximates the racial composition of those
\emph{entering} multiethnic neighborhoods.

Multiethnic neighborhoods must attract racially diverse newcomers to
maintain integration. Residents currently residing in multiethnic
neighborhoods provide an obvious clientele to fill vacancies in
multiethnic neighborhoods. Yet, we have neither a sense whether current
residents of multiethnic neighborhoods would consider moving to another
multiethnic neighborhood, nor whether the proportion who would consider
another multiethnic neighborhood differs across racial groups.

Rather than being individually derived or collectively negotiated within
households, housing preferences reflect a deeply social process.
Households cannot engage with the housing market across an entire
metropolitan area. Krysan and Crowder show that people approach housing
searches based on word of mouth that shape what people think about
different communities, and whether they even know anything about the
community at all. Settling on which neighborhoods to consider--and which
not to waste time considering--reflects social networks that are
segregated by race. Differences in the knowledge residents use as
starting points for their housing searches can help us predict the
future stability of multiethnic integration.

Race influences the familiarity residents have with communities and
where residents consider moving. Whites know more about and desire
moving to neighborhoods with larger shares of white residents. Whites
differ from other racial groups, who have stronger desires for
integration than whites. Havekes and colleagues find that racial
composition influences white searches more at each stage of the search
process. They find that whites move to communities with lower shares of
whites on average than communities that they search which, in turn, have
lower shares of whites on average than communities whites report
considering. Multiethnic neighborhoods might expose white residents to
more diverse social networks and increase the probability that whites
would consider another multiethnic neighborhood. But, again, we have no
evidence one way or the other among residents living in multiethnic
neighborhoods.

\subsubsection{Hypotheses}\label{hypotheses}

In light of this evidence, this article examines two hypotheses:

\begin{enumerate}
\def\labelenumi{\arabic{enumi}.}
\tightlist
\item
  Whites living in multiethnic neighborhoods will be a) less satisfied
  with their neighborhoods and b) less likely to perceive neighborhood
  improvement compared to their Asian, black, and Latinx neighbors; and
\item
  Whites living in multiethnic neighborhoods will be will be a) less
  familiar with, b) less likely to consider, and c) more likely to never
  consider moving to other multiethnic neighborhoods in their
  metropolitan area.
\end{enumerate}

\singlespacing
\section{Data}\label{data}

\begin{itemize}
\tightlist
\item
  Data come from the 2016 DC Area Survey

  \begin{itemize}
  \tightlist
  \item
    A mail survey using an address-based sample of households in the
    Washington, D.C. area that comprises the District of Columbia and
    the surrounding jurisdictions: Montgomery and Prince George's
    Counties in Maryland, Arlington and Fairfax Counties in Virginia
    (Fairfax County included the independent cities of Fairfax and Falls
    Church), and the city of Alexandria.
  \item
    Households were sampled from two types of neighborhoods:

    \begin{itemize}
    \tightlist
    \item
      Quadrivial: neighborhoods where Asians, blacks, Latinos, and
      whites each made up at least 10\% of the neighborhood and no group
      made up a majority

      \begin{itemize}
      \tightlist
      \item
        Prefer the term `quadrivial' based on Bader \& Warkentien rather
        than `global neighborhoods' because global can imply that
        residents of the neighborhoods are somehow un-American. While
        many residents of these neighborhoods were born abroad, most
        were born in the United States.
      \item
        114 of the 973 tracts in the DC area could be classified as
        quadrivial neighborhoods, with a population of 600,176. Almost
        half of the quadrivial neighborhoods (54) were in Montgomery
        County; none fell in the District of Columbia.
      \item
        \textbf{Insert Map of Quadrivial Neighborhoods in DC area}
      \end{itemize}
    \item
      Disproportionately Latino: neighborhoods where at least 25\% of
      the residents were Latinos that were not already classified as a
      quadrivial neighborhood
    \end{itemize}
  \item
    This study uses only the 641 respondents from quadrivial
    neighborhoods that also answered the race and ethnicity questions (0
    respondents provided insufficient data to determine their race).
  \item
    Sample details:

    \begin{itemize}
    \tightlist
    \item
      The survey oversampled households with Asian and Hispanic surnames
      as well as neighborhoods that were disproportionately black that
      satisfied the criteria for being a quadrivial neighborhood
    \item
      Surveys were mailed to households with a \$2 incentive enclosed
      and directions for the resident over 18 years of age with the most
      recent birthday to fill out the questionnaire; cover letter was
      printed in English and Spanish on reverse sides; booklet was in
      English with instructions on how to receive a booklet in Spanish
      in the Spanish-language cover letter
    \item
      A reminder was mailed out two weeks after the initial mailing
    \item
      The response rate (RR4, cite) was 12.8\%; this is in line with
      mail surveys (CITE)
    \end{itemize}
  \item
    Survey weights were calculated to allow for estimates represent the
    population of D.C.-area residents living in quadrivial (and
    disproportionately Latino) neighborhoods
  \end{itemize}
\end{itemize}

\subsection{Dependent Variables}\label{dependent-variables}

\subsubsection{Satisfaction Variable}\label{satisfaction-variable}

The current neighborhood satisfaction variable (\texttt{satisfied})
comes from Question 5: ``How satisfied are you with your neighborhood as
a place to live?'' Responses of ``Extremely satisfied'' and ``Very
satisfied'' are coded as ``satisfied'' (\texttt{1}); other responses
(``Somewhat satisfied'' and ``Not at all satisfied'' and coded as
``unsatisfied'' {[}\texttt{0}{]}).

\subsubsection{Neighborhood Change
Variable}\label{neighborhood-change-variable}

The satisfaction with neighborhood change variable (\texttt{better})
comes from Question 9: ``Looking back over the past five years or so,
would you say that your neighborhood has:'' Response options ``Become a
much better place to live'' and ``Become a somewhat better place to
live'' are coded as ``better'' (\texttt{1}) and the other three
responses are coded as ``not better'' (\texttt{0}).

\subsection{Independent Variable}\label{independent-variable}

The independent variable for these analyses is self-identified race that
has been recoded to represent non-Latinx white alone, non-Latinx black
alone or in combination with other races, Latinx of any race, and
non-Latinx Asian or Pacific Islander alone or in combination with any
race except black. Respondents with other races were dropped from the
analysis (N=641).

These data are ideal to answer the hypotheses because: * They are
representative of residents living in quadrivial neighborhoods * They
ask about satisfaction and neighborhood improvement using same questions
that have been used in previous studies * Have a sufficient sample of
all four racial groups to compare results across racial groups

\subsection{Control Variables}\label{control-variables}

In addition to racial variables, I also included variables to control
for other demographic characteristics of residents.

\textbf{Demographic characteristics}

\begin{itemize}
\item
  \emph{Age}, calculated from answer to Question 49: ``In what year were
  you born?''
\item
  \emph{Foreign born}, answer to Question 58: ``Where were you born?''
\item
  \emph{Gender}, male or female
\item
  \emph{Kids}, if answer to Question 2--``How many children ages 17 or
  under live in your household?''--is greater than zero
\item
  \emph{Marital status}, if answer to Question 60--``What is your
  marital status''-- is ``Now married or in a married-style
  arrangement''
\end{itemize}

\textbf{Economic Status}

\begin{itemize}
\tightlist
\item
  \emph{Educational attainment}, five categories based on Question 59,
  ``What is the highest level of school you have completed or the
  highest degree you have received?'' Used only educational attainment
  because it was highly correlated with income and would likely pick up
  stable features of status differences.
\end{itemize}

\textbf{Neighborhood Experience} In addition to demographic
characteristics, I also included controls for the respondent's
experience in the neighborhood.

\begin{itemize}
\item
  \emph{Years in neighborhood}, based on Question 4: ``How many years
  have you lived in your current neighborhood?'' (The length of time
  that respondents have lived in their neighborhoods could affect their
  perception of neighborhood quality and certainly neighborhood change)
\item
  \emph{Neighborhood size}, based on Question 7: ``How many blocks are
  in the area that you think of as your neighborhood?'' (The size of the
  area the respondent conjures as being part of their neighborhood could
  affect perceptions of neighborhood quality and change)
\end{itemize}

\textbf{Residence variables}

\begin{itemize}
\tightlist
\item
  \emph{State of residence}, factor variable indicating the state in
  which the respondent lives.
\end{itemize}

(set up data for survey weights \& multiple imputation)

\section{Analysis}\label{analysis}

\subsection{Analytical Approach}\label{analytical-approach}

For each analysis, ran three models.

\begin{enumerate}
\def\labelenumi{\arabic{enumi}.}
\tightlist
\item
  Race-only model
\item
  Race and demographic controls
\item
  Race, demographic, and neighborhood perception variables
\end{enumerate}

\begin{itemize}
\tightlist
\item
  Can't ``prove a null'' in this case, so will examine how much overlap
  there is in the confidence intervals (I need to work on explaining
  this better; any suggestions would be most welcome) {[}Create
  subsection at end of \textbf{Analysis} to address ``proving the
  null''?{]}
\end{itemize}

\subsection{Fixed Effects}\label{fixed-effects}

The correct test of the hypotheses compares the responses of white
residents to those of blacks, Latinos, and Asians, \emph{living in the
same neighborhood}. As a result, I include fixed effects for each census
tract in which respondents resided. Only one respondent lived in 9
tracts, and were therefore dropped from the analysis. Without these
tracts, the number of respondents per tract ranged from 2 to 19.

\subsection{Multiple Imputation \& Survey
Weights}\label{multiple-imputation-survey-weights}

Several variables have a substantial amount of missing data. I used the
Amelia package (cite) to create five imputation datasets that will be
combined for the final results. I then set up the survey dseign using
the list of imputed datasets to account for the complex survey design.

\subsection{Descriptive Statistics of Independent \& Control
Variables}\label{descriptive-statistics-of-independent-control-variables}

% latex table generated in R 3.5.0 by xtable 1.8-2 package
% 
\begin{sidewaystable}[ht]
\centering
\caption{Means and standard deviations of independent and control variables} 
\label{tab:descriptives}
\begin{tabular}{lR{.4in}R{.4in}R{.4in}R{.4in}R{.4in}R{.4in}R{.4in}R{.4in}R{.4in}R{.4in}}
  \toprule
& \multicolumn{2}{p{.8in}}{\centering \strong{Total Sample}} & \multicolumn{2}{p{.8in}}{\centering \strong{Asians}} & \multicolumn{2}{p{.8in}}{\centering \strong{Blacks}} & \multicolumn{2}{p{.8in}}{\centering \strong{Latinxs}} & \multicolumn{2}{p{.8in}}{\centering \strong{Whites}} \\
Variable & Mean & S.D. & Mean & S.D. & Mean & S.D. & Mean & S.D. & Mean & S.D. \\ 
  \midrule
\emph{Race}&&\\White &  0.32 &  &  &  &  &  &  &  &  &  \\ 
  Asian &  0.21 &  &  &  &  &  &  &  &  &  \\ 
  Black &  0.22 &  &  &  &  &  &  &  &  &  \\ 
  Latinx\vspace{1em} &  0.25 &  &  &  &  &  &  &  &  &  \\ 
  \emph{Demographics}&&\\Age & 47.11 & 0.82 & 45.04 & 1.57 & 46.78 & 1.78 & 47.25 & 1.89 & 48.63 & 1.39 \\ 
  Foreign Born &  0.46 &  & 0.79 &  & 0.37 &  & 0.65 &  & 0.14 &  \\ 
  Man &  0.49 &  & 0.51 &  & 0.43 &  & 0.52 &  & 0.48 &  \\ 
  Children present &  0.40 &  & 0.36 &  & 0.45 &  & 0.45 &  & 0.35 &  \\ 
  Married\vspace{1em} &  0.65 &  & 0.7 &  & 0.61 &  & 0.55 &  & 0.72 &  \\ 
  \emph{Education}&&\\Less than H.S. &  0.04 &  & 0.07 &  & 0 &  & 0.09 &  & 0.01 &  \\ 
  H.S. or G.E.D. &  0.09 &  & 0.08 &  & 0.14 &  & 0.08 &  & 0.07 &  \\ 
  Some college &  0.21 &  & 0.13 &  & 0.21 &  & 0.3 &  & 0.2 &  \\ 
  Bachelor's degree &  0.31 &  & 0.42 &  & 0.3 &  & 0.24 &  & 0.3 &  \\ 
  Professional degree\vspace{1em} &  0.34 &  & 0.3 &  & 0.34 &  & 0.28 &  & 0.42 &  \\ 
  \emph{Neighborhood Experience}\\Years in Neighborhood & 11.89 & 0.52 & 10.44 & 0.92 & 10.24 & 1.03 & 12.59 & 1.12 & 13.47 & 0.96 \\ 
  Perceived neighborhood size\\1-9 blocks &  0.60 &  & 0.66 &  & 0.58 &  & 0.52 &  & 0.65 &  \\ 
  10-50 blocks &  0.33 &  & 0.26 &  & 0.38 &  & 0.35 &  & 0.32 &  \\ 
  >50 blocks &  0.07 &  & 0.08 &  & 0.04 &  & 0.14 &  & 0.03 &  \\ 
   \bottomrule
\end{tabular}
\end{sidewaystable}


Residents of multiethnic neighborhoods are:

\begin{itemize}
\tightlist
\item
  Racially diverse
\item
  Likely to be foreign-born: 46\% of respondents are foreign born. This
  varied by racial group, however: 79\% of Asian residents were born
  abroad, 66\% of Latinos, and even 37\% of blacks while an overwhelming
  majority of whites (86\%) were native-born.
\item
  Residents are highly educated as almost a third hold a bachelor's
  degree and another third hold an advanced degree
\item
  Nearly two-thirds report being married
\end{itemize}

\doublespacing
\section{Results}\label{results}

\subsection{Satisfication Living in Multiethnic
Neighborhoods}\label{satisfication-living-in-multiethnic-neighborhoods}

Most residents of multiethnic neighborhoods, 71 percent, are satisfied
living in their neighborhood. Satisfaction was equally high among all
four racial groups: 72 percent of Asians, 76 percent of Latinxs, 67
percent of blacks, and 69 percent of whites in the sample were either
``extremely'' or ``very'' satisfied living in their multiethnic
neighborhoods.

The agreement across racial groups held when I statistically compared
groups within the same neighborhoods by modeling satisfaction with
neighborhood fixed effects. The first column of Table
\ref{tab:satisfaction} reports the results of a model including only
respondent race and neighborhood fixed effects. The estimates for racial
groups are small and not distinguishable from zero.

{[}Table \ref{tab:satisfaction} about here{]}

To further show the lack of difference between racial groups, I plotted
the predicted probability from the model in the left panel of Figure
\ref{fig:satisfaction}. The plotted figures represent the predicted
probability in the neighborhood with median satisfaction in the data,
and reflect the estimates predicted across the five imputed data sets
combined using Rubin's rules \textbf{{[}need cite{]}}. \textbf{{[}Look
into replacing this estimate with the AME{]}} Only 8 percent separated
the most satisfied group (Latinxs) from the least satisfied (Asians).

{[}Figure \ref{fig:satisfaction} about here{]}

The second column of Table \ref{tab:satisfaction} further confirmed that
satisfaction was an interracial sentiment among residents of multiethnic
neighborhoods. The estimates were again small and could not be
distinguished from a null effect. The right panel of Figure
@ref\{fig:satisfaction\} shows the predicted probabilities of
satisfaction from the model with controls. I estimated the probabilities
and standard errors at the reference level for all other variables in
the model, meaning the figure represents the satisfaction of a
50-year-old, native-born, unmarried woman with a high school degree and
no children who has lived in her neighborhood for 10 years.
\textbf{{[}Look into replacing this estimate with the AME{]}} In the
complete model, only 6 separated the least satisfied group (now whites)
from the most satisfied (still Latinxs).

Not only are individual coefficients not large or statistically
singificant, but including race does not improve the explanatory power
of the model. The final column of Table \ref{tab:satisfaction} reports
the results of model estimating satisfaction in which I did not include
race. The bottom row of the table reports the Akaike information
criterion (AIC), a measure that balances a model's parsimony with its
goodness of fit to the data \textbf{{[}need cite{]}}. A lower AIC
represents a better fit to the data, and we can observe from Table
\ref{tab:satisfaction} that the model without race fits the data better
than the model that includes race. I further affirmed the lack of
explanatory power attributable to race by conducting Wald's F-test
between the models with and without race. The mean value of the test
across the imputed data sets was 0.59, and the p-values ranged from 0.58
to 0.66.

These results establish that residents are satisfied living in
multiethnic neighborhoods regardless of their own race. Two other
factors, education and the perceived size of the neighborhood, are
associated with neighborhood satisfaction rather than race. Education
had a curvilinear relationship with satisfaction among multiethnic
neighborhood residents. Residents of multiethnic neighborhoods with a
high school degree were the most satisfied, while residents with less
than a high school education and those with a postgraduate degree felt
satisfaction the least often. The perceived size of the neighborhood
also mattered, as those residents who thought their neighborhood
comprised 10-50 blocks were 3.02 times more likely to be satisfied than
those who perceived their neighborhood to be only one to nine blocks.
\textbf{{[}Connect back to idea that this is a sufficient size to avoid
the worst of microsegregation, so people are actually satisfied with
their neighborhood with multiracial composition.{]}}

\subsection{Views on Neighborhood
Improvement}\label{views-on-neighborhood-improvement}

While seven in ten residents of multiethnic neighborhoods were
satisfied, half that proportion (34\%) thought that their neighborhood
had become a ``much better'' or ``somewhat better'' place to live over
the past five years. Also unlike satisfaction, the probability that
residents perceived improvements varied by race. Thirty-eight percent of
Asian and Latinx residents, and 35 percent of black residents, felt that
their neighborhoods had improved, while only 27 percent of whites felt
that way.

The first column of Table \ref{tab:improvement} shows that the racial
differences were unlikely due to the randomness of the sample. The
column reports the coefficients estimated for racial groups in a
neighborhood fixed-effect model that also controls for respondents'
satisfaction. The results show a strong association between perceived
improvement and race: whites were much less likely than Asian, black, or
Latinx residents to believe that their neighborhoods had improved. The
left panel of Figure \ref{fig:improvement} shows the lower probability
at which whites were predicted to report improvement in multiethnic
neighborhoods than the probabilities at which people of color were
predicted to report improvement. \textbf{{[}Look into replacing this
estimate with the AME{]}}

{[}Table \ref{tab:improvement} about here{]}

The lower propensity of whites to report improvement in multiethnic
neighborhoods persisted after including controls. The second column of
Table \ref{tab:improvement} shows positive associations for all three
non-white racial groups. Asians were 2.35 times more likely to report
improvement in their neighborhood, blacks were 2.64 times more likely,
and Latinos were 3.12 times more likely. The latter two differences were
large enough to be confident that the difference was not due to sampling
variation at the p=0.05 level. The right panel of Figure
\ref(fig:improvement) shows the predicted probabilities of reporting
neighborhood improvement across racial groups. \textbf{{[}Look into
replacing this estimate with the AME{]}}

{[}Figure \ref{fig:improvement} about here{]}

In addition to the differences between whites and people of color in the
outcome, the model that included race fit the data better than the model
that did not. The third column of Table \ref{tab:improvement} reports
coefficient estimates and standard errors of the model without race and
shows that the AIC value increases (fits less well) between the full
model and the model without race. Wald F-tests of including race versus
not provided evidence in the same direction: the mean value of the tests
across imputed data sets was 2.16 and the mean p-value was 0.09.

In summary, a lower proportion of whites perceived improvement in
multiethnic neighborhoods than their neighbors of color. Whites,
therefore, might be satisfied living in their multiethnic neighborhood,
but not inclined to see their neighborhood being a better place to live
in the future. Neighborhood racial diversity would have been growing in
the time period whites perceived less improvement than their neighbors.
The present study provides insufficient evidence to conclude that
increasing diversity \emph{caused} whites to perceive less improvement
than their neighbors, but it does indicate\textbf{{[}w.ch{]}} a temporal
correlation between growing diversity and a lower white endorsement of
neighborhood change.

Projecting forward, too, the less positive trajectory whites perceive
could portend a future in which whites might be more inclined, on the
margin, to move out of the neighborhood than their non-white neighbors.
Such marginal differences combined with the high rates of satisfaction
among whites would be unlikely to instigate white flight, but the higher
probability of losing white households could increase the chances of
long-term racial turnover. The chances of long-term turnover would
increase further if multiethnic neighborhoods attracted people of color
at higher rates than whites.

%\textbf{Where does this go?} The model also included whether respondents
%reported feeling satisfied living in their neighborhoods as a means to
%adjust for the levels off of which a respondent would evaluate change
%(satisfied residents might be less likely to perceive improvement, for
%example, since they are already satisfied with the
%neighborhood).\textbf{fix wording}

\singlespacing
\section{Familiarity with and Preferences for Other Multiethnic
Places}\label{familiarity-with-and-preferences-for-other-multiethnic-places}

The long-term stability of multiethnic neighborhoods requires that
residents of all races consider moving to existing multiethnic
neighborhoods. The subsequent analyses assess how likely residents are
to consider neighborhoods and how likely they are to actively not
consider multiethnic neighborhoods.

\subsection{Familiarity with Multiethnic
Places}\label{familiarity-with-multiethnic-places}

Two measures of familiarity with other multiethnic places in the
Washington, D.C. area:

\begin{enumerate}
\def\labelenumi{\arabic{enumi}.}
\tightlist
\item
  Whether the respondent reports knowing anything about the place
\item
  Whether respondent reports having friends or family in the place
\end{enumerate}

Uses the same analytical strategy, except that we add two covariates to
each model:

\begin{enumerate}
\def\labelenumi{\arabic{enumi}.}
\tightlist
\item
  Whether the respondent reports living in the place
\end{enumerate}

%\section{Notes}\label{notes}
%
%Correlation of education and income:
%
%\begin{Shaded}
%\begin{Highlighting}[]
%\KeywordTok{ggplot}\NormalTok{(dcas, }\KeywordTok{aes}\NormalTok{(}\DataTypeTok{x=}\NormalTok{dem.income.cat4,}\DataTypeTok{y=}\NormalTok{dem.educ.attain)) }\OperatorTok{+}\StringTok{ }\KeywordTok{geom_jitter}\NormalTok{()}
%\end{Highlighting}
%\end{Shaded}
%
%%\includegraphics{multiethnic-nhoods_files/figure-latex/education-income-correlation-1.pdf}
%
%\subsection{Notes from Writing Group April 4,
%2019}\label{notes-from-writing-group-april-4-2019}
%
%\subsubsection{Comments}\label{comments}
%
%\begin{description}
%\item[David]
%\begin{itemize}
%\tightlist
%\item
%  Like the setup on the front end (but didn't have a whole lot of time
%  to read it)
%\item
%  The graphs with point estimates and confidence intervals were helpful
%  ``to prove the null''
%\item
%  Examine whether a joint F-test for significance of the race variables
%  together provide significant improvement in fit; alternatively, use
%  the AIC \& BIC values to run the full model without race variable and
%  then with race variable
%\item
%  \textbf{Add fixed effects}
%
%  \begin{itemize}
%  \tightlist
%  \item
%    Have fixed effects in the models the entire time
%  \item
%    Used fixed effects to frame the method as being about comparing
%    residents of different races \emph{in the same neighborhood}
%  \end{itemize}
%\end{itemize}
%\item[Alex]
%\begin{itemize}
%\tightlist
%\item
%  Was left with the question, so what? Bring out that this fits with my
%  research showing that white acceptance of non-white neighbors is
%  insufficient, they need to seek diverse neighborhoods for integration
%  to persist
%\item
%  I am selecting on people who stayed, so these would be the whites
%  \emph{most} inclined to be satisfied with their neighborhoods; bring
%  out that in most of these neighborhoods, whites are still the majority
%  -- so if they \emph{are} selected, there is still a large proportion
%  of whites satisfied living in multiethnic neighborhoods
%\item
%  Subset by parental status (link to Ann Owens' ASR paper)
%\item
%  Point out that this study is unique to some degree to DC, which is a
%  type of multiethnic metropolitan area that is often overlooked (it's
%  not a black/white city, which are now the anomolies, places like DC
%  are the future)
%\end{itemize}
%\end{description}
%
%\subsubsection{Potential Journals}\label{potential-journals}
%
%\begin{itemize}
%\tightlist
%\item
%  \emph{Demography} especially with Mark Hayward as the editor
%\item
%  Possibly worth trying at a general sociology journal like \emph{Social
%  Forces}; subject-matter also fits at \emph{Social Problems} and
%  \emph{Social Science Research}
%\item
%  \emph{Urban Affairs Review}
%\end{itemize}
%
%\subsubsection{Future Directions}\label{future-directions}
%
%\begin{itemize}
%\tightlist
%\item
%  Do white conservatives look different than white liberals? Maybe find
%  a political scientist interested in neighborhoods to collaborate
%\item
%  Look at paper by Ryan Enos about how the demolition of public housing
%  shifted whites' political views:
%  \url{https://onlinelibrary.wiley.com/doi/epdf/10.1111/ajps.12156}
%\end{itemize}

\section{Supplement}\label{supplement}

\subsection{Model of Neighborhood Improvement without Including
Neighborhood Satisfaction as a
Variable}\label{model-of-neighborhood-improvement-without-including-neighborhood-satisfaction-as-a-variable}

The following reports an analysis of neighborhood improvement that does
not include neighborhood satisfaction as a covariate in the model.

\subsection{Full Analytic Model, Don't Know About
Communities}\label{full-analytic-model-dont-know-about-communities}

The following reports the full analytic model minus the coefficients and
standard errors for neighborhood fixed effects (second and third models)
predicting whether respondent doesn't know anything about the community.

\subsection{Full Analytic Model, Knowing Friends or
Family}\label{full-analytic-model-knowing-friends-or-family}

The following reports the full analytic model minus the coefficients and
standard errors for neighborhood fixed effects (second and third models)
predicting whether respondent knew friends or family who lived in the
neighborhood.

\clearpage
\section{Tables}\label{sec:tables}
\begin{table}[h]
\centering\captionsetup{justification=centering,singlelinecheck=off}
\caption{Estimated coefficients predicting  neighborhood satisfaction}
\label{tab:satisfaction}

    \providecommand{\huxb}[2][0,0,0]{\arrayrulecolor[RGB]{#1}\global\arrayrulewidth=#2pt}
    \providecommand{\huxvb}[2][0,0,0]{\color[RGB]{#1}\vrule width #2pt}
    \providecommand{\huxtpad}[1]{\rule{0pt}{\baselineskip+#1}}
    \providecommand{\huxbpad}[1]{\rule[-#1]{0pt}{#1}}
  \begin{tabularx}{0.5\textwidth}{p{0.1\textwidth} p{0.1\textwidth} p{0.1\textwidth} p{0.1\textwidth} p{0.1\textwidth}}


\hhline{>{\huxb{0.8}}->{\huxb{0.8}}->{\huxb{0.8}}->{\huxb{0.8}}->{\huxb{0.8}}-}
\arrayrulecolor{black}

\multicolumn{1}{!{\huxvb{0}}c!{\huxvb{0}}}{\huxtpad{4pt}\centering {\fontsize{9.5pt}{11.4pt}\selectfont }\huxbpad{4pt}} &
\multicolumn{1}{c!{\huxvb{0}}}{\huxtpad{4pt}\centering {\fontsize{9.5pt}{11.4pt}\selectfont (1)}\huxbpad{4pt}} &
\multicolumn{1}{c!{\huxvb{0}}}{\huxtpad{4pt}\centering {\fontsize{9.5pt}{11.4pt}\selectfont (2)}\huxbpad{4pt}} &
\multicolumn{1}{c!{\huxvb{0}}}{\huxtpad{4pt}\centering {\fontsize{9.5pt}{11.4pt}\selectfont (3)}\huxbpad{4pt}} &
\multicolumn{1}{c!{\huxvb{0}}}{\huxtpad{4pt}\centering {\fontsize{9.5pt}{11.4pt}\selectfont (4)}\huxbpad{4pt}} \tabularnewline[-0.5pt]


\hhline{>{\huxb{1}}->{\huxb{1}}->{\huxb{1}}->{\huxb{1}}->{\huxb{1}}-}
\arrayrulecolor{black}

\multicolumn{1}{!{\huxvb{0}}l!{\huxvb{0}}}{\huxtpad{4pt}\raggedright {\fontsize{9.5pt}{11.4pt}\selectfont (Intercept)}\huxbpad{4pt}} &
\multicolumn{1}{r!{\huxvb{0}}}{\huxtpad{4pt}\raggedleft {\fontsize{9.5pt}{11.4pt}\selectfont 0.944~}\huxbpad{4pt}} &
\multicolumn{1}{r!{\huxvb{0}}}{\huxtpad{4pt}\raggedleft {\fontsize{9.5pt}{11.4pt}\selectfont 1.714~~}\huxbpad{4pt}} &
\multicolumn{1}{r!{\huxvb{0}}}{\huxtpad{4pt}\raggedleft {\fontsize{9.5pt}{11.4pt}\selectfont 1.606~~~}\huxbpad{4pt}} &
\multicolumn{1}{r!{\huxvb{0}}}{\huxtpad{4pt}\raggedleft {\fontsize{9.5pt}{11.4pt}\selectfont 1.900~~~}\huxbpad{4pt}} \tabularnewline[-0.5pt]


\hhline{}
\arrayrulecolor{black}

\multicolumn{1}{!{\huxvb{0}}l!{\huxvb{0}}}{\huxtpad{4pt}\raggedright {\fontsize{9.5pt}{11.4pt}\selectfont }\huxbpad{4pt}} &
\multicolumn{1}{r!{\huxvb{0}}}{\huxtpad{4pt}\raggedleft {\fontsize{9.5pt}{11.4pt}\selectfont (0.732)}\huxbpad{4pt}} &
\multicolumn{1}{r!{\huxvb{0}}}{\huxtpad{4pt}\raggedleft {\fontsize{9.5pt}{11.4pt}\selectfont (1.089)~}\huxbpad{4pt}} &
\multicolumn{1}{r!{\huxvb{0}}}{\huxtpad{4pt}\raggedleft {\fontsize{9.5pt}{11.4pt}\selectfont (1.164)~~}\huxbpad{4pt}} &
\multicolumn{1}{r!{\huxvb{0}}}{\huxtpad{4pt}\raggedleft {\fontsize{9.5pt}{11.4pt}\selectfont (1.142)~~}\huxbpad{4pt}} \tabularnewline[-0.5pt]


\hhline{}
\arrayrulecolor{black}

\multicolumn{1}{!{\huxvb{0}}l!{\huxvb{0}}}{\huxtpad{4pt}\raggedright {\fontsize{9.5pt}{11.4pt}\selectfont Race}\huxbpad{4pt}} &
\multicolumn{1}{r!{\huxvb{0}}}{\huxtpad{4pt}\raggedleft {\fontsize{9.5pt}{11.4pt}\selectfont ~~~~~}\huxbpad{4pt}} &
\multicolumn{1}{r!{\huxvb{0}}}{\huxtpad{4pt}\raggedleft {\fontsize{9.5pt}{11.4pt}\selectfont ~~~~~~}\huxbpad{4pt}} &
\multicolumn{1}{r!{\huxvb{0}}}{\huxtpad{4pt}\raggedleft {\fontsize{9.5pt}{11.4pt}\selectfont ~~~~~~~}\huxbpad{4pt}} &
\multicolumn{1}{r!{\huxvb{0}}}{\huxtpad{4pt}\raggedleft {\fontsize{9.5pt}{11.4pt}\selectfont ~~~~~~~}\huxbpad{4pt}} \tabularnewline[-0.5pt]


\hhline{}
\arrayrulecolor{black}

\multicolumn{1}{!{\huxvb{0}}l!{\huxvb{0}}}{\huxtpad{4pt}\raggedright {\fontsize{9.5pt}{11.4pt}\selectfont Asian}\huxbpad{4pt}} &
\multicolumn{1}{r!{\huxvb{0}}}{\huxtpad{4pt}\raggedleft {\fontsize{9.5pt}{11.4pt}\selectfont -0.050~}\huxbpad{4pt}} &
\multicolumn{1}{r!{\huxvb{0}}}{\huxtpad{4pt}\raggedleft {\fontsize{9.5pt}{11.4pt}\selectfont -0.072~~}\huxbpad{4pt}} &
\multicolumn{1}{r!{\huxvb{0}}}{\huxtpad{4pt}\raggedleft {\fontsize{9.5pt}{11.4pt}\selectfont 0.061~~~}\huxbpad{4pt}} &
\multicolumn{1}{r!{\huxvb{0}}}{\huxtpad{4pt}\raggedleft {\fontsize{9.5pt}{11.4pt}\selectfont ~~~~~~~}\huxbpad{4pt}} \tabularnewline[-0.5pt]


\hhline{}
\arrayrulecolor{black}

\multicolumn{1}{!{\huxvb{0}}l!{\huxvb{0}}}{\huxtpad{4pt}\raggedright {\fontsize{9.5pt}{11.4pt}\selectfont }\huxbpad{4pt}} &
\multicolumn{1}{r!{\huxvb{0}}}{\huxtpad{4pt}\raggedleft {\fontsize{9.5pt}{11.4pt}\selectfont (0.384)}\huxbpad{4pt}} &
\multicolumn{1}{r!{\huxvb{0}}}{\huxtpad{4pt}\raggedleft {\fontsize{9.5pt}{11.4pt}\selectfont (0.484)~}\huxbpad{4pt}} &
\multicolumn{1}{r!{\huxvb{0}}}{\huxtpad{4pt}\raggedleft {\fontsize{9.5pt}{11.4pt}\selectfont (0.494)~~}\huxbpad{4pt}} &
\multicolumn{1}{r!{\huxvb{0}}}{\huxtpad{4pt}\raggedleft {\fontsize{9.5pt}{11.4pt}\selectfont ~~~~~~~}\huxbpad{4pt}} \tabularnewline[-0.5pt]


\hhline{}
\arrayrulecolor{black}

\multicolumn{1}{!{\huxvb{0}}l!{\huxvb{0}}}{\huxtpad{4pt}\raggedright {\fontsize{9.5pt}{11.4pt}\selectfont Black}\huxbpad{4pt}} &
\multicolumn{1}{r!{\huxvb{0}}}{\huxtpad{4pt}\raggedleft {\fontsize{9.5pt}{11.4pt}\selectfont 0.364~}\huxbpad{4pt}} &
\multicolumn{1}{r!{\huxvb{0}}}{\huxtpad{4pt}\raggedleft {\fontsize{9.5pt}{11.4pt}\selectfont 0.464~~}\huxbpad{4pt}} &
\multicolumn{1}{r!{\huxvb{0}}}{\huxtpad{4pt}\raggedleft {\fontsize{9.5pt}{11.4pt}\selectfont 0.363~~~}\huxbpad{4pt}} &
\multicolumn{1}{r!{\huxvb{0}}}{\huxtpad{4pt}\raggedleft {\fontsize{9.5pt}{11.4pt}\selectfont ~~~~~~~}\huxbpad{4pt}} \tabularnewline[-0.5pt]


\hhline{}
\arrayrulecolor{black}

\multicolumn{1}{!{\huxvb{0}}l!{\huxvb{0}}}{\huxtpad{4pt}\raggedright {\fontsize{9.5pt}{11.4pt}\selectfont }\huxbpad{4pt}} &
\multicolumn{1}{r!{\huxvb{0}}}{\huxtpad{4pt}\raggedleft {\fontsize{9.5pt}{11.4pt}\selectfont (0.400)}\huxbpad{4pt}} &
\multicolumn{1}{r!{\huxvb{0}}}{\huxtpad{4pt}\raggedleft {\fontsize{9.5pt}{11.4pt}\selectfont (0.422)~}\huxbpad{4pt}} &
\multicolumn{1}{r!{\huxvb{0}}}{\huxtpad{4pt}\raggedleft {\fontsize{9.5pt}{11.4pt}\selectfont (0.441)~~}\huxbpad{4pt}} &
\multicolumn{1}{r!{\huxvb{0}}}{\huxtpad{4pt}\raggedleft {\fontsize{9.5pt}{11.4pt}\selectfont ~~~~~~~}\huxbpad{4pt}} \tabularnewline[-0.5pt]


\hhline{}
\arrayrulecolor{black}

\multicolumn{1}{!{\huxvb{0}}l!{\huxvb{0}}}{\huxtpad{4pt}\raggedright {\fontsize{9.5pt}{11.4pt}\selectfont Latinx}\huxbpad{4pt}} &
\multicolumn{1}{r!{\huxvb{0}}}{\huxtpad{4pt}\raggedleft {\fontsize{9.5pt}{11.4pt}\selectfont 0.207~}\huxbpad{4pt}} &
\multicolumn{1}{r!{\huxvb{0}}}{\huxtpad{4pt}\raggedleft {\fontsize{9.5pt}{11.4pt}\selectfont 0.505~~}\huxbpad{4pt}} &
\multicolumn{1}{r!{\huxvb{0}}}{\huxtpad{4pt}\raggedleft {\fontsize{9.5pt}{11.4pt}\selectfont 0.564~~~}\huxbpad{4pt}} &
\multicolumn{1}{r!{\huxvb{0}}}{\huxtpad{4pt}\raggedleft {\fontsize{9.5pt}{11.4pt}\selectfont ~~~~~~~}\huxbpad{4pt}} \tabularnewline[-0.5pt]


\hhline{}
\arrayrulecolor{black}

\multicolumn{1}{!{\huxvb{0}}l!{\huxvb{0}}}{\huxtpad{4pt}\raggedright {\fontsize{9.5pt}{11.4pt}\selectfont }\huxbpad{4pt}} &
\multicolumn{1}{r!{\huxvb{0}}}{\huxtpad{4pt}\raggedleft {\fontsize{9.5pt}{11.4pt}\selectfont (0.424)}\huxbpad{4pt}} &
\multicolumn{1}{r!{\huxvb{0}}}{\huxtpad{4pt}\raggedleft {\fontsize{9.5pt}{11.4pt}\selectfont (0.495)~}\huxbpad{4pt}} &
\multicolumn{1}{r!{\huxvb{0}}}{\huxtpad{4pt}\raggedleft {\fontsize{9.5pt}{11.4pt}\selectfont (0.538)~~}\huxbpad{4pt}} &
\multicolumn{1}{r!{\huxvb{0}}}{\huxtpad{4pt}\raggedleft {\fontsize{9.5pt}{11.4pt}\selectfont ~~~~~~~}\huxbpad{4pt}} \tabularnewline[-0.5pt]


\hhline{}
\arrayrulecolor{black}

\multicolumn{1}{!{\huxvb{0}}l!{\huxvb{0}}}{\huxtpad{4pt}\raggedright {\fontsize{9.5pt}{11.4pt}\selectfont Demographics}\huxbpad{4pt}} &
\multicolumn{1}{r!{\huxvb{0}}}{\huxtpad{4pt}\raggedleft {\fontsize{9.5pt}{11.4pt}\selectfont ~~~~~}\huxbpad{4pt}} &
\multicolumn{1}{r!{\huxvb{0}}}{\huxtpad{4pt}\raggedleft {\fontsize{9.5pt}{11.4pt}\selectfont ~~~~~~}\huxbpad{4pt}} &
\multicolumn{1}{r!{\huxvb{0}}}{\huxtpad{4pt}\raggedleft {\fontsize{9.5pt}{11.4pt}\selectfont ~~~~~~~}\huxbpad{4pt}} &
\multicolumn{1}{r!{\huxvb{0}}}{\huxtpad{4pt}\raggedleft {\fontsize{9.5pt}{11.4pt}\selectfont ~~~~~~~}\huxbpad{4pt}} \tabularnewline[-0.5pt]


\hhline{}
\arrayrulecolor{black}

\multicolumn{1}{!{\huxvb{0}}l!{\huxvb{0}}}{\huxtpad{4pt}\raggedright {\fontsize{9.5pt}{11.4pt}\selectfont Age}\huxbpad{4pt}} &
\multicolumn{1}{r!{\huxvb{0}}}{\huxtpad{4pt}\raggedleft {\fontsize{9.5pt}{11.4pt}\selectfont ~~~~~}\huxbpad{4pt}} &
\multicolumn{1}{r!{\huxvb{0}}}{\huxtpad{4pt}\raggedleft {\fontsize{9.5pt}{11.4pt}\selectfont 0.002~~}\huxbpad{4pt}} &
\multicolumn{1}{r!{\huxvb{0}}}{\huxtpad{4pt}\raggedleft {\fontsize{9.5pt}{11.4pt}\selectfont 0.004~~~}\huxbpad{4pt}} &
\multicolumn{1}{r!{\huxvb{0}}}{\huxtpad{4pt}\raggedleft {\fontsize{9.5pt}{11.4pt}\selectfont 0.004~~~}\huxbpad{4pt}} \tabularnewline[-0.5pt]


\hhline{}
\arrayrulecolor{black}

\multicolumn{1}{!{\huxvb{0}}l!{\huxvb{0}}}{\huxtpad{4pt}\raggedright {\fontsize{9.5pt}{11.4pt}\selectfont }\huxbpad{4pt}} &
\multicolumn{1}{r!{\huxvb{0}}}{\huxtpad{4pt}\raggedleft {\fontsize{9.5pt}{11.4pt}\selectfont ~~~~~}\huxbpad{4pt}} &
\multicolumn{1}{r!{\huxvb{0}}}{\huxtpad{4pt}\raggedleft {\fontsize{9.5pt}{11.4pt}\selectfont (0.010)~}\huxbpad{4pt}} &
\multicolumn{1}{r!{\huxvb{0}}}{\huxtpad{4pt}\raggedleft {\fontsize{9.5pt}{11.4pt}\selectfont (0.012)~~}\huxbpad{4pt}} &
\multicolumn{1}{r!{\huxvb{0}}}{\huxtpad{4pt}\raggedleft {\fontsize{9.5pt}{11.4pt}\selectfont (0.012)~~}\huxbpad{4pt}} \tabularnewline[-0.5pt]


\hhline{}
\arrayrulecolor{black}

\multicolumn{1}{!{\huxvb{0}}l!{\huxvb{0}}}{\huxtpad{4pt}\raggedright {\fontsize{9.5pt}{11.4pt}\selectfont Foreign Born}\huxbpad{4pt}} &
\multicolumn{1}{r!{\huxvb{0}}}{\huxtpad{4pt}\raggedleft {\fontsize{9.5pt}{11.4pt}\selectfont ~~~~~}\huxbpad{4pt}} &
\multicolumn{1}{r!{\huxvb{0}}}{\huxtpad{4pt}\raggedleft {\fontsize{9.5pt}{11.4pt}\selectfont -0.075~~}\huxbpad{4pt}} &
\multicolumn{1}{r!{\huxvb{0}}}{\huxtpad{4pt}\raggedleft {\fontsize{9.5pt}{11.4pt}\selectfont -0.084~~~}\huxbpad{4pt}} &
\multicolumn{1}{r!{\huxvb{0}}}{\huxtpad{4pt}\raggedleft {\fontsize{9.5pt}{11.4pt}\selectfont -0.016~~~}\huxbpad{4pt}} \tabularnewline[-0.5pt]


\hhline{}
\arrayrulecolor{black}

\multicolumn{1}{!{\huxvb{0}}l!{\huxvb{0}}}{\huxtpad{4pt}\raggedright {\fontsize{9.5pt}{11.4pt}\selectfont }\huxbpad{4pt}} &
\multicolumn{1}{r!{\huxvb{0}}}{\huxtpad{4pt}\raggedleft {\fontsize{9.5pt}{11.4pt}\selectfont ~~~~~}\huxbpad{4pt}} &
\multicolumn{1}{r!{\huxvb{0}}}{\huxtpad{4pt}\raggedleft {\fontsize{9.5pt}{11.4pt}\selectfont (0.399)~}\huxbpad{4pt}} &
\multicolumn{1}{r!{\huxvb{0}}}{\huxtpad{4pt}\raggedleft {\fontsize{9.5pt}{11.4pt}\selectfont (0.428)~~}\huxbpad{4pt}} &
\multicolumn{1}{r!{\huxvb{0}}}{\huxtpad{4pt}\raggedleft {\fontsize{9.5pt}{11.4pt}\selectfont (0.330)~~}\huxbpad{4pt}} \tabularnewline[-0.5pt]


\hhline{}
\arrayrulecolor{black}

\multicolumn{1}{!{\huxvb{0}}l!{\huxvb{0}}}{\huxtpad{4pt}\raggedright {\fontsize{9.5pt}{11.4pt}\selectfont Male}\huxbpad{4pt}} &
\multicolumn{1}{r!{\huxvb{0}}}{\huxtpad{4pt}\raggedleft {\fontsize{9.5pt}{11.4pt}\selectfont ~~~~~}\huxbpad{4pt}} &
\multicolumn{1}{r!{\huxvb{0}}}{\huxtpad{4pt}\raggedleft {\fontsize{9.5pt}{11.4pt}\selectfont 0.228~~}\huxbpad{4pt}} &
\multicolumn{1}{r!{\huxvb{0}}}{\huxtpad{4pt}\raggedleft {\fontsize{9.5pt}{11.4pt}\selectfont 0.294~~~}\huxbpad{4pt}} &
\multicolumn{1}{r!{\huxvb{0}}}{\huxtpad{4pt}\raggedleft {\fontsize{9.5pt}{11.4pt}\selectfont 0.264~~~}\huxbpad{4pt}} \tabularnewline[-0.5pt]


\hhline{}
\arrayrulecolor{black}

\multicolumn{1}{!{\huxvb{0}}l!{\huxvb{0}}}{\huxtpad{4pt}\raggedright {\fontsize{9.5pt}{11.4pt}\selectfont }\huxbpad{4pt}} &
\multicolumn{1}{r!{\huxvb{0}}}{\huxtpad{4pt}\raggedleft {\fontsize{9.5pt}{11.4pt}\selectfont ~~~~~}\huxbpad{4pt}} &
\multicolumn{1}{r!{\huxvb{0}}}{\huxtpad{4pt}\raggedleft {\fontsize{9.5pt}{11.4pt}\selectfont (0.290)~}\huxbpad{4pt}} &
\multicolumn{1}{r!{\huxvb{0}}}{\huxtpad{4pt}\raggedleft {\fontsize{9.5pt}{11.4pt}\selectfont (0.294)~~}\huxbpad{4pt}} &
\multicolumn{1}{r!{\huxvb{0}}}{\huxtpad{4pt}\raggedleft {\fontsize{9.5pt}{11.4pt}\selectfont (0.295)~~}\huxbpad{4pt}} \tabularnewline[-0.5pt]


\hhline{}
\arrayrulecolor{black}

\multicolumn{1}{!{\huxvb{0}}l!{\huxvb{0}}}{\huxtpad{4pt}\raggedright {\fontsize{9.5pt}{11.4pt}\selectfont Children Present}\huxbpad{4pt}} &
\multicolumn{1}{r!{\huxvb{0}}}{\huxtpad{4pt}\raggedleft {\fontsize{9.5pt}{11.4pt}\selectfont ~~~~~}\huxbpad{4pt}} &
\multicolumn{1}{r!{\huxvb{0}}}{\huxtpad{4pt}\raggedleft {\fontsize{9.5pt}{11.4pt}\selectfont -0.678 *}\huxbpad{4pt}} &
\multicolumn{1}{r!{\huxvb{0}}}{\huxtpad{4pt}\raggedleft {\fontsize{9.5pt}{11.4pt}\selectfont -0.665~~~}\huxbpad{4pt}} &
\multicolumn{1}{r!{\huxvb{0}}}{\huxtpad{4pt}\raggedleft {\fontsize{9.5pt}{11.4pt}\selectfont -0.573~~~}\huxbpad{4pt}} \tabularnewline[-0.5pt]


\hhline{}
\arrayrulecolor{black}

\multicolumn{1}{!{\huxvb{0}}l!{\huxvb{0}}}{\huxtpad{4pt}\raggedright {\fontsize{9.5pt}{11.4pt}\selectfont }\huxbpad{4pt}} &
\multicolumn{1}{r!{\huxvb{0}}}{\huxtpad{4pt}\raggedleft {\fontsize{9.5pt}{11.4pt}\selectfont ~~~~~}\huxbpad{4pt}} &
\multicolumn{1}{r!{\huxvb{0}}}{\huxtpad{4pt}\raggedleft {\fontsize{9.5pt}{11.4pt}\selectfont (0.345)~}\huxbpad{4pt}} &
\multicolumn{1}{r!{\huxvb{0}}}{\huxtpad{4pt}\raggedleft {\fontsize{9.5pt}{11.4pt}\selectfont (0.367)~~}\huxbpad{4pt}} &
\multicolumn{1}{r!{\huxvb{0}}}{\huxtpad{4pt}\raggedleft {\fontsize{9.5pt}{11.4pt}\selectfont (0.360)~~}\huxbpad{4pt}} \tabularnewline[-0.5pt]


\hhline{}
\arrayrulecolor{black}

\multicolumn{1}{!{\huxvb{0}}l!{\huxvb{0}}}{\huxtpad{4pt}\raggedright {\fontsize{9.5pt}{11.4pt}\selectfont Married}\huxbpad{4pt}} &
\multicolumn{1}{r!{\huxvb{0}}}{\huxtpad{4pt}\raggedleft {\fontsize{9.5pt}{11.4pt}\selectfont ~~~~~}\huxbpad{4pt}} &
\multicolumn{1}{r!{\huxvb{0}}}{\huxtpad{4pt}\raggedleft {\fontsize{9.5pt}{11.4pt}\selectfont 0.617 *}\huxbpad{4pt}} &
\multicolumn{1}{r!{\huxvb{0}}}{\huxtpad{4pt}\raggedleft {\fontsize{9.5pt}{11.4pt}\selectfont 0.575~~~}\huxbpad{4pt}} &
\multicolumn{1}{r!{\huxvb{0}}}{\huxtpad{4pt}\raggedleft {\fontsize{9.5pt}{11.4pt}\selectfont 0.483~~~}\huxbpad{4pt}} \tabularnewline[-0.5pt]


\hhline{}
\arrayrulecolor{black}

\multicolumn{1}{!{\huxvb{0}}l!{\huxvb{0}}}{\huxtpad{4pt}\raggedright {\fontsize{9.5pt}{11.4pt}\selectfont }\huxbpad{4pt}} &
\multicolumn{1}{r!{\huxvb{0}}}{\huxtpad{4pt}\raggedleft {\fontsize{9.5pt}{11.4pt}\selectfont ~~~~~}\huxbpad{4pt}} &
\multicolumn{1}{r!{\huxvb{0}}}{\huxtpad{4pt}\raggedleft {\fontsize{9.5pt}{11.4pt}\selectfont (0.312)~}\huxbpad{4pt}} &
\multicolumn{1}{r!{\huxvb{0}}}{\huxtpad{4pt}\raggedleft {\fontsize{9.5pt}{11.4pt}\selectfont (0.319)~~}\huxbpad{4pt}} &
\multicolumn{1}{r!{\huxvb{0}}}{\huxtpad{4pt}\raggedleft {\fontsize{9.5pt}{11.4pt}\selectfont (0.307)~~}\huxbpad{4pt}} \tabularnewline[-0.5pt]


\hhline{}
\arrayrulecolor{black}

\multicolumn{1}{!{\huxvb{0}}l!{\huxvb{0}}}{\huxtpad{4pt}\raggedright {\fontsize{9.5pt}{11.4pt}\selectfont Socioeconomic}\huxbpad{4pt}} &
\multicolumn{1}{r!{\huxvb{0}}}{\huxtpad{4pt}\raggedleft {\fontsize{9.5pt}{11.4pt}\selectfont ~~~~~}\huxbpad{4pt}} &
\multicolumn{1}{r!{\huxvb{0}}}{\huxtpad{4pt}\raggedleft {\fontsize{9.5pt}{11.4pt}\selectfont ~~~~~~}\huxbpad{4pt}} &
\multicolumn{1}{r!{\huxvb{0}}}{\huxtpad{4pt}\raggedleft {\fontsize{9.5pt}{11.4pt}\selectfont ~~~~~~~}\huxbpad{4pt}} &
\multicolumn{1}{r!{\huxvb{0}}}{\huxtpad{4pt}\raggedleft {\fontsize{9.5pt}{11.4pt}\selectfont ~~~~~~~}\huxbpad{4pt}} \tabularnewline[-0.5pt]


\hhline{}
\arrayrulecolor{black}

\multicolumn{1}{!{\huxvb{0}}l!{\huxvb{0}}}{\huxtpad{4pt}\raggedright {\fontsize{9.5pt}{11.4pt}\selectfont $<$H.S.}\huxbpad{4pt}} &
\multicolumn{1}{r!{\huxvb{0}}}{\huxtpad{4pt}\raggedleft {\fontsize{9.5pt}{11.4pt}\selectfont ~~~~~}\huxbpad{4pt}} &
\multicolumn{1}{r!{\huxvb{0}}}{\huxtpad{4pt}\raggedleft {\fontsize{9.5pt}{11.4pt}\selectfont -1.951 *}\huxbpad{4pt}} &
\multicolumn{1}{r!{\huxvb{0}}}{\huxtpad{4pt}\raggedleft {\fontsize{9.5pt}{11.4pt}\selectfont -1.932 *~}\huxbpad{4pt}} &
\multicolumn{1}{r!{\huxvb{0}}}{\huxtpad{4pt}\raggedleft {\fontsize{9.5pt}{11.4pt}\selectfont -1.885 *~}\huxbpad{4pt}} \tabularnewline[-0.5pt]


\hhline{}
\arrayrulecolor{black}

\multicolumn{1}{!{\huxvb{0}}l!{\huxvb{0}}}{\huxtpad{4pt}\raggedright {\fontsize{9.5pt}{11.4pt}\selectfont }\huxbpad{4pt}} &
\multicolumn{1}{r!{\huxvb{0}}}{\huxtpad{4pt}\raggedleft {\fontsize{9.5pt}{11.4pt}\selectfont ~~~~~}\huxbpad{4pt}} &
\multicolumn{1}{r!{\huxvb{0}}}{\huxtpad{4pt}\raggedleft {\fontsize{9.5pt}{11.4pt}\selectfont (0.889)~}\huxbpad{4pt}} &
\multicolumn{1}{r!{\huxvb{0}}}{\huxtpad{4pt}\raggedleft {\fontsize{9.5pt}{11.4pt}\selectfont (0.867)~~}\huxbpad{4pt}} &
\multicolumn{1}{r!{\huxvb{0}}}{\huxtpad{4pt}\raggedleft {\fontsize{9.5pt}{11.4pt}\selectfont (0.896)~~}\huxbpad{4pt}} \tabularnewline[-0.5pt]


\hhline{}
\arrayrulecolor{black}

\multicolumn{1}{!{\huxvb{0}}l!{\huxvb{0}}}{\huxtpad{4pt}\raggedright {\fontsize{9.5pt}{11.4pt}\selectfont Some college, no B.A.}\huxbpad{4pt}} &
\multicolumn{1}{r!{\huxvb{0}}}{\huxtpad{4pt}\raggedleft {\fontsize{9.5pt}{11.4pt}\selectfont ~~~~~}\huxbpad{4pt}} &
\multicolumn{1}{r!{\huxvb{0}}}{\huxtpad{4pt}\raggedleft {\fontsize{9.5pt}{11.4pt}\selectfont -1.122~~}\huxbpad{4pt}} &
\multicolumn{1}{r!{\huxvb{0}}}{\huxtpad{4pt}\raggedleft {\fontsize{9.5pt}{11.4pt}\selectfont -1.119~~~}\huxbpad{4pt}} &
\multicolumn{1}{r!{\huxvb{0}}}{\huxtpad{4pt}\raggedleft {\fontsize{9.5pt}{11.4pt}\selectfont -1.108~~~}\huxbpad{4pt}} \tabularnewline[-0.5pt]


\hhline{}
\arrayrulecolor{black}

\multicolumn{1}{!{\huxvb{0}}l!{\huxvb{0}}}{\huxtpad{4pt}\raggedright {\fontsize{9.5pt}{11.4pt}\selectfont }\huxbpad{4pt}} &
\multicolumn{1}{r!{\huxvb{0}}}{\huxtpad{4pt}\raggedleft {\fontsize{9.5pt}{11.4pt}\selectfont ~~~~~}\huxbpad{4pt}} &
\multicolumn{1}{r!{\huxvb{0}}}{\huxtpad{4pt}\raggedleft {\fontsize{9.5pt}{11.4pt}\selectfont (0.607)~}\huxbpad{4pt}} &
\multicolumn{1}{r!{\huxvb{0}}}{\huxtpad{4pt}\raggedleft {\fontsize{9.5pt}{11.4pt}\selectfont (0.631)~~}\huxbpad{4pt}} &
\multicolumn{1}{r!{\huxvb{0}}}{\huxtpad{4pt}\raggedleft {\fontsize{9.5pt}{11.4pt}\selectfont (0.625)~~}\huxbpad{4pt}} \tabularnewline[-0.5pt]


\hhline{}
\arrayrulecolor{black}

\multicolumn{1}{!{\huxvb{0}}l!{\huxvb{0}}}{\huxtpad{4pt}\raggedright {\fontsize{9.5pt}{11.4pt}\selectfont B.A.}\huxbpad{4pt}} &
\multicolumn{1}{r!{\huxvb{0}}}{\huxtpad{4pt}\raggedleft {\fontsize{9.5pt}{11.4pt}\selectfont ~~~~~}\huxbpad{4pt}} &
\multicolumn{1}{r!{\huxvb{0}}}{\huxtpad{4pt}\raggedleft {\fontsize{9.5pt}{11.4pt}\selectfont -0.933~~}\huxbpad{4pt}} &
\multicolumn{1}{r!{\huxvb{0}}}{\huxtpad{4pt}\raggedleft {\fontsize{9.5pt}{11.4pt}\selectfont -0.922~~~}\huxbpad{4pt}} &
\multicolumn{1}{r!{\huxvb{0}}}{\huxtpad{4pt}\raggedleft {\fontsize{9.5pt}{11.4pt}\selectfont -0.991~~~}\huxbpad{4pt}} \tabularnewline[-0.5pt]


\hhline{}
\arrayrulecolor{black}

\multicolumn{1}{!{\huxvb{0}}l!{\huxvb{0}}}{\huxtpad{4pt}\raggedright {\fontsize{9.5pt}{11.4pt}\selectfont }\huxbpad{4pt}} &
\multicolumn{1}{r!{\huxvb{0}}}{\huxtpad{4pt}\raggedleft {\fontsize{9.5pt}{11.4pt}\selectfont ~~~~~}\huxbpad{4pt}} &
\multicolumn{1}{r!{\huxvb{0}}}{\huxtpad{4pt}\raggedleft {\fontsize{9.5pt}{11.4pt}\selectfont (0.578)~}\huxbpad{4pt}} &
\multicolumn{1}{r!{\huxvb{0}}}{\huxtpad{4pt}\raggedleft {\fontsize{9.5pt}{11.4pt}\selectfont (0.598)~~}\huxbpad{4pt}} &
\multicolumn{1}{r!{\huxvb{0}}}{\huxtpad{4pt}\raggedleft {\fontsize{9.5pt}{11.4pt}\selectfont (0.594)~~}\huxbpad{4pt}} \tabularnewline[-0.5pt]


\hhline{}
\arrayrulecolor{black}

\multicolumn{1}{!{\huxvb{0}}l!{\huxvb{0}}}{\huxtpad{4pt}\raggedright {\fontsize{9.5pt}{11.4pt}\selectfont Neighborhood perceptions}\huxbpad{4pt}} &
\multicolumn{1}{r!{\huxvb{0}}}{\huxtpad{4pt}\raggedleft {\fontsize{9.5pt}{11.4pt}\selectfont ~~~~~}\huxbpad{4pt}} &
\multicolumn{1}{r!{\huxvb{0}}}{\huxtpad{4pt}\raggedleft {\fontsize{9.5pt}{11.4pt}\selectfont ~~~~~~}\huxbpad{4pt}} &
\multicolumn{1}{r!{\huxvb{0}}}{\huxtpad{4pt}\raggedleft {\fontsize{9.5pt}{11.4pt}\selectfont ~~~~~~~}\huxbpad{4pt}} &
\multicolumn{1}{r!{\huxvb{0}}}{\huxtpad{4pt}\raggedleft {\fontsize{9.5pt}{11.4pt}\selectfont ~~~~~~~}\huxbpad{4pt}} \tabularnewline[-0.5pt]


\hhline{}
\arrayrulecolor{black}

\multicolumn{1}{!{\huxvb{0}}l!{\huxvb{0}}}{\huxtpad{4pt}\raggedright {\fontsize{9.5pt}{11.4pt}\selectfont M.A.+}\huxbpad{4pt}} &
\multicolumn{1}{r!{\huxvb{0}}}{\huxtpad{4pt}\raggedleft {\fontsize{9.5pt}{11.4pt}\selectfont ~~~~~}\huxbpad{4pt}} &
\multicolumn{1}{r!{\huxvb{0}}}{\huxtpad{4pt}\raggedleft {\fontsize{9.5pt}{11.4pt}\selectfont -1.367 *}\huxbpad{4pt}} &
\multicolumn{1}{r!{\huxvb{0}}}{\huxtpad{4pt}\raggedleft {\fontsize{9.5pt}{11.4pt}\selectfont -1.353 *~}\huxbpad{4pt}} &
\multicolumn{1}{r!{\huxvb{0}}}{\huxtpad{4pt}\raggedleft {\fontsize{9.5pt}{11.4pt}\selectfont -1.415 *~}\huxbpad{4pt}} \tabularnewline[-0.5pt]


\hhline{}
\arrayrulecolor{black}

\multicolumn{1}{!{\huxvb{0}}l!{\huxvb{0}}}{\huxtpad{4pt}\raggedright {\fontsize{9.5pt}{11.4pt}\selectfont }\huxbpad{4pt}} &
\multicolumn{1}{r!{\huxvb{0}}}{\huxtpad{4pt}\raggedleft {\fontsize{9.5pt}{11.4pt}\selectfont ~~~~~}\huxbpad{4pt}} &
\multicolumn{1}{r!{\huxvb{0}}}{\huxtpad{4pt}\raggedleft {\fontsize{9.5pt}{11.4pt}\selectfont (0.562)~}\huxbpad{4pt}} &
\multicolumn{1}{r!{\huxvb{0}}}{\huxtpad{4pt}\raggedleft {\fontsize{9.5pt}{11.4pt}\selectfont (0.568)~~}\huxbpad{4pt}} &
\multicolumn{1}{r!{\huxvb{0}}}{\huxtpad{4pt}\raggedleft {\fontsize{9.5pt}{11.4pt}\selectfont (0.564)~~}\huxbpad{4pt}} \tabularnewline[-0.5pt]


\hhline{}
\arrayrulecolor{black}

\multicolumn{1}{!{\huxvb{0}}l!{\huxvb{0}}}{\huxtpad{4pt}\raggedright {\fontsize{9.5pt}{11.4pt}\selectfont Years in neighborhood}\huxbpad{4pt}} &
\multicolumn{1}{r!{\huxvb{0}}}{\huxtpad{4pt}\raggedleft {\fontsize{9.5pt}{11.4pt}\selectfont ~~~~~}\huxbpad{4pt}} &
\multicolumn{1}{r!{\huxvb{0}}}{\huxtpad{4pt}\raggedleft {\fontsize{9.5pt}{11.4pt}\selectfont ~~~~~~}\huxbpad{4pt}} &
\multicolumn{1}{r!{\huxvb{0}}}{\huxtpad{4pt}\raggedleft {\fontsize{9.5pt}{11.4pt}\selectfont -0.009~~~}\huxbpad{4pt}} &
\multicolumn{1}{r!{\huxvb{0}}}{\huxtpad{4pt}\raggedleft {\fontsize{9.5pt}{11.4pt}\selectfont -0.010~~~}\huxbpad{4pt}} \tabularnewline[-0.5pt]


\hhline{}
\arrayrulecolor{black}

\multicolumn{1}{!{\huxvb{0}}l!{\huxvb{0}}}{\huxtpad{4pt}\raggedright {\fontsize{9.5pt}{11.4pt}\selectfont }\huxbpad{4pt}} &
\multicolumn{1}{r!{\huxvb{0}}}{\huxtpad{4pt}\raggedleft {\fontsize{9.5pt}{11.4pt}\selectfont ~~~~~}\huxbpad{4pt}} &
\multicolumn{1}{r!{\huxvb{0}}}{\huxtpad{4pt}\raggedleft {\fontsize{9.5pt}{11.4pt}\selectfont ~~~~~~}\huxbpad{4pt}} &
\multicolumn{1}{r!{\huxvb{0}}}{\huxtpad{4pt}\raggedleft {\fontsize{9.5pt}{11.4pt}\selectfont (0.016)~~}\huxbpad{4pt}} &
\multicolumn{1}{r!{\huxvb{0}}}{\huxtpad{4pt}\raggedleft {\fontsize{9.5pt}{11.4pt}\selectfont (0.016)~~}\huxbpad{4pt}} \tabularnewline[-0.5pt]


\hhline{}
\arrayrulecolor{black}

\multicolumn{1}{!{\huxvb{0}}l!{\huxvb{0}}}{\huxtpad{4pt}\raggedright {\fontsize{9.5pt}{11.4pt}\selectfont 10-50 blocks}\huxbpad{4pt}} &
\multicolumn{1}{r!{\huxvb{0}}}{\huxtpad{4pt}\raggedleft {\fontsize{9.5pt}{11.4pt}\selectfont ~~~~~}\huxbpad{4pt}} &
\multicolumn{1}{r!{\huxvb{0}}}{\huxtpad{4pt}\raggedleft {\fontsize{9.5pt}{11.4pt}\selectfont ~~~~~~}\huxbpad{4pt}} &
\multicolumn{1}{r!{\huxvb{0}}}{\huxtpad{4pt}\raggedleft {\fontsize{9.5pt}{11.4pt}\selectfont 1.100 **}\huxbpad{4pt}} &
\multicolumn{1}{r!{\huxvb{0}}}{\huxtpad{4pt}\raggedleft {\fontsize{9.5pt}{11.4pt}\selectfont 1.125 **}\huxbpad{4pt}} \tabularnewline[-0.5pt]


\hhline{}
\arrayrulecolor{black}

\multicolumn{1}{!{\huxvb{0}}l!{\huxvb{0}}}{\huxtpad{4pt}\raggedright {\fontsize{9.5pt}{11.4pt}\selectfont }\huxbpad{4pt}} &
\multicolumn{1}{r!{\huxvb{0}}}{\huxtpad{4pt}\raggedleft {\fontsize{9.5pt}{11.4pt}\selectfont ~~~~~}\huxbpad{4pt}} &
\multicolumn{1}{r!{\huxvb{0}}}{\huxtpad{4pt}\raggedleft {\fontsize{9.5pt}{11.4pt}\selectfont ~~~~~~}\huxbpad{4pt}} &
\multicolumn{1}{r!{\huxvb{0}}}{\huxtpad{4pt}\raggedleft {\fontsize{9.5pt}{11.4pt}\selectfont (0.384)~~}\huxbpad{4pt}} &
\multicolumn{1}{r!{\huxvb{0}}}{\huxtpad{4pt}\raggedleft {\fontsize{9.5pt}{11.4pt}\selectfont (0.368)~~}\huxbpad{4pt}} \tabularnewline[-0.5pt]


\hhline{}
\arrayrulecolor{black}

\multicolumn{1}{!{\huxvb{0}}l!{\huxvb{0}}}{\huxtpad{4pt}\raggedright {\fontsize{9.5pt}{11.4pt}\selectfont $>$50 blocks}\huxbpad{4pt}} &
\multicolumn{1}{r!{\huxvb{0}}}{\huxtpad{4pt}\raggedleft {\fontsize{9.5pt}{11.4pt}\selectfont ~~~~~}\huxbpad{4pt}} &
\multicolumn{1}{r!{\huxvb{0}}}{\huxtpad{4pt}\raggedleft {\fontsize{9.5pt}{11.4pt}\selectfont ~~~~~~}\huxbpad{4pt}} &
\multicolumn{1}{r!{\huxvb{0}}}{\huxtpad{4pt}\raggedleft {\fontsize{9.5pt}{11.4pt}\selectfont 0.267~~~}\huxbpad{4pt}} &
\multicolumn{1}{r!{\huxvb{0}}}{\huxtpad{4pt}\raggedleft {\fontsize{9.5pt}{11.4pt}\selectfont 0.421~~~}\huxbpad{4pt}} \tabularnewline[-0.5pt]


\hhline{}
\arrayrulecolor{black}

\multicolumn{1}{!{\huxvb{0}}l!{\huxvb{0}}}{\huxtpad{4pt}\raggedright {\fontsize{9.5pt}{11.4pt}\selectfont }\huxbpad{4pt}} &
\multicolumn{1}{r!{\huxvb{0}}}{\huxtpad{4pt}\raggedleft {\fontsize{9.5pt}{11.4pt}\selectfont ~~~~~}\huxbpad{4pt}} &
\multicolumn{1}{r!{\huxvb{0}}}{\huxtpad{4pt}\raggedleft {\fontsize{9.5pt}{11.4pt}\selectfont ~~~~~~}\huxbpad{4pt}} &
\multicolumn{1}{r!{\huxvb{0}}}{\huxtpad{4pt}\raggedleft {\fontsize{9.5pt}{11.4pt}\selectfont (0.601)~~}\huxbpad{4pt}} &
\multicolumn{1}{r!{\huxvb{0}}}{\huxtpad{4pt}\raggedleft {\fontsize{9.5pt}{11.4pt}\selectfont (0.578)~~}\huxbpad{4pt}} \tabularnewline[-0.5pt]


\hhline{>{\huxb{1}}->{\huxb{1}}->{\huxb{1}}->{\huxb{1}}->{\huxb{1}}-}
\arrayrulecolor{black}

\multicolumn{1}{!{\huxvb{0}}l!{\huxvb{0}}}{\huxtpad{4pt}\raggedright {\fontsize{9.5pt}{11.4pt}\selectfont AIC}\huxbpad{4pt}} &
\multicolumn{1}{r!{\huxvb{0}}}{\huxtpad{4pt}\raggedleft {\fontsize{9.5pt}{11.4pt}\selectfont 673.618~}\huxbpad{4pt}} &
\multicolumn{1}{r!{\huxvb{0}}}{\huxtpad{4pt}\raggedleft {\fontsize{9.5pt}{11.4pt}\selectfont 663.346~~}\huxbpad{4pt}} &
\multicolumn{1}{r!{\huxvb{0}}}{\huxtpad{4pt}\raggedleft {\fontsize{9.5pt}{11.4pt}\selectfont 659.927~~~}\huxbpad{4pt}} &
\multicolumn{1}{r!{\huxvb{0}}}{\huxtpad{4pt}\raggedleft {\fontsize{9.5pt}{11.4pt}\selectfont 655.029~~~}\huxbpad{4pt}} \tabularnewline[-0.5pt]


\hhline{>{\huxb{1}}->{\huxb{1}}->{\huxb{1}}->{\huxb{1}}->{\huxb{1}}-}
\arrayrulecolor{black}

\multicolumn{5}{!{\huxvb{0}}p{0.5\textwidth+8\tabcolsep}!{\huxvb{0}}}{\parbox[b]{0.5\textwidth+8\tabcolsep-4pt-4pt}{\huxtpad{4pt}\raggedright {\fontsize{9.5pt}{11.4pt}\selectfont  *** p $<$ 0.001;  ** p $<$ 0.01;  * p $<$ 0.05.}\huxbpad{4pt}}} \tabularnewline[-0.5pt]


\hhline{}
\arrayrulecolor{black}
\end{tabularx}
\end{table}


\begin{table}[h]
\centering\captionsetup{justification=centering,singlelinecheck=off}
\caption{Estimated coefficients predicting neighborhood improvement}
\label{tab:improvement}

    \providecommand{\huxb}[2][0,0,0]{\arrayrulecolor[RGB]{#1}\global\arrayrulewidth=#2pt}
    \providecommand{\huxvb}[2][0,0,0]{\color[RGB]{#1}\vrule width #2pt}
    \providecommand{\huxtpad}[1]{\rule{0pt}{\baselineskip+#1}}
    \providecommand{\huxbpad}[1]{\rule[-#1]{0pt}{#1}}
  \begin{tabularx}{0.5\textwidth}{p{0.125\textwidth} p{0.125\textwidth} p{0.125\textwidth} p{0.125\textwidth}}


\hhline{>{\huxb{0.8}}->{\huxb{0.8}}->{\huxb{0.8}}->{\huxb{0.8}}-}
\arrayrulecolor{black}

\multicolumn{1}{!{\huxvb{0}}c!{\huxvb{0}}}{\huxtpad{4pt}\centering {\fontsize{9.5pt}{11.4pt}\selectfont }\huxbpad{4pt}} &
\multicolumn{1}{c!{\huxvb{0}}}{\huxtpad{4pt}\centering {\fontsize{9.5pt}{11.4pt}\selectfont (1)}\huxbpad{4pt}} &
\multicolumn{1}{c!{\huxvb{0}}}{\huxtpad{4pt}\centering {\fontsize{9.5pt}{11.4pt}\selectfont (2)}\huxbpad{4pt}} &
\multicolumn{1}{c!{\huxvb{0}}}{\huxtpad{4pt}\centering {\fontsize{9.5pt}{11.4pt}\selectfont (3)}\huxbpad{4pt}} \tabularnewline[-0.5pt]


\hhline{>{\huxb{1}}->{\huxb{1}}->{\huxb{1}}->{\huxb{1}}-}
\arrayrulecolor{black}

\multicolumn{1}{!{\huxvb{0}}l!{\huxvb{0}}}{\huxtpad{4pt}\raggedright {\fontsize{9.5pt}{11.4pt}\selectfont (Intercept)}\huxbpad{4pt}} &
\multicolumn{1}{r!{\huxvb{0}}}{\huxtpad{4pt}\raggedleft {\fontsize{9.5pt}{11.4pt}\selectfont -20.595 ***}\huxbpad{4pt}} &
\multicolumn{1}{r!{\huxvb{0}}}{\huxtpad{4pt}\raggedleft {\fontsize{9.5pt}{11.4pt}\selectfont -20.210 ***}\huxbpad{4pt}} &
\multicolumn{1}{r!{\huxvb{0}}}{\huxtpad{4pt}\raggedleft {\fontsize{9.5pt}{11.4pt}\selectfont -21.999 ***}\huxbpad{4pt}} \tabularnewline[-0.5pt]


\hhline{}
\arrayrulecolor{black}

\multicolumn{1}{!{\huxvb{0}}l!{\huxvb{0}}}{\huxtpad{4pt}\raggedright {\fontsize{9.5pt}{11.4pt}\selectfont }\huxbpad{4pt}} &
\multicolumn{1}{r!{\huxvb{0}}}{\huxtpad{4pt}\raggedleft {\fontsize{9.5pt}{11.4pt}\selectfont (0.948)~~~}\huxbpad{4pt}} &
\multicolumn{1}{r!{\huxvb{0}}}{\huxtpad{4pt}\raggedleft {\fontsize{9.5pt}{11.4pt}\selectfont (1.173)~~~}\huxbpad{4pt}} &
\multicolumn{1}{r!{\huxvb{0}}}{\huxtpad{4pt}\raggedleft {\fontsize{9.5pt}{11.4pt}\selectfont (1.091)~~~}\huxbpad{4pt}} \tabularnewline[-0.5pt]


\hhline{}
\arrayrulecolor{black}

\multicolumn{1}{!{\huxvb{0}}l!{\huxvb{0}}}{\huxtpad{4pt}\raggedright {\fontsize{9.5pt}{11.4pt}\selectfont Race}\huxbpad{4pt}} &
\multicolumn{1}{r!{\huxvb{0}}}{\huxtpad{4pt}\raggedleft {\fontsize{9.5pt}{11.4pt}\selectfont ~~~~~~~~}\huxbpad{4pt}} &
\multicolumn{1}{r!{\huxvb{0}}}{\huxtpad{4pt}\raggedleft {\fontsize{9.5pt}{11.4pt}\selectfont ~~~~~~~~}\huxbpad{4pt}} &
\multicolumn{1}{r!{\huxvb{0}}}{\huxtpad{4pt}\raggedleft {\fontsize{9.5pt}{11.4pt}\selectfont ~~~~~~~~}\huxbpad{4pt}} \tabularnewline[-0.5pt]


\hhline{}
\arrayrulecolor{black}

\multicolumn{1}{!{\huxvb{0}}l!{\huxvb{0}}}{\huxtpad{4pt}\raggedright {\fontsize{9.5pt}{11.4pt}\selectfont Asian}\huxbpad{4pt}} &
\multicolumn{1}{r!{\huxvb{0}}}{\huxtpad{4pt}\raggedleft {\fontsize{9.5pt}{11.4pt}\selectfont 1.325 **~}\huxbpad{4pt}} &
\multicolumn{1}{r!{\huxvb{0}}}{\huxtpad{4pt}\raggedleft {\fontsize{9.5pt}{11.4pt}\selectfont 0.909~~~~}\huxbpad{4pt}} &
\multicolumn{1}{r!{\huxvb{0}}}{\huxtpad{4pt}\raggedleft {\fontsize{9.5pt}{11.4pt}\selectfont 1.055 *~~}\huxbpad{4pt}} \tabularnewline[-0.5pt]


\hhline{}
\arrayrulecolor{black}

\multicolumn{1}{!{\huxvb{0}}l!{\huxvb{0}}}{\huxtpad{4pt}\raggedright {\fontsize{9.5pt}{11.4pt}\selectfont }\huxbpad{4pt}} &
\multicolumn{1}{r!{\huxvb{0}}}{\huxtpad{4pt}\raggedleft {\fontsize{9.5pt}{11.4pt}\selectfont (0.423)~~~}\huxbpad{4pt}} &
\multicolumn{1}{r!{\huxvb{0}}}{\huxtpad{4pt}\raggedleft {\fontsize{9.5pt}{11.4pt}\selectfont (0.519)~~~}\huxbpad{4pt}} &
\multicolumn{1}{r!{\huxvb{0}}}{\huxtpad{4pt}\raggedleft {\fontsize{9.5pt}{11.4pt}\selectfont (0.509)~~~}\huxbpad{4pt}} \tabularnewline[-0.5pt]


\hhline{}
\arrayrulecolor{black}

\multicolumn{1}{!{\huxvb{0}}l!{\huxvb{0}}}{\huxtpad{4pt}\raggedright {\fontsize{9.5pt}{11.4pt}\selectfont Black}\huxbpad{4pt}} &
\multicolumn{1}{r!{\huxvb{0}}}{\huxtpad{4pt}\raggedleft {\fontsize{9.5pt}{11.4pt}\selectfont 1.435 **~}\huxbpad{4pt}} &
\multicolumn{1}{r!{\huxvb{0}}}{\huxtpad{4pt}\raggedleft {\fontsize{9.5pt}{11.4pt}\selectfont 1.401 **~}\huxbpad{4pt}} &
\multicolumn{1}{r!{\huxvb{0}}}{\huxtpad{4pt}\raggedleft {\fontsize{9.5pt}{11.4pt}\selectfont 0.937 *~~}\huxbpad{4pt}} \tabularnewline[-0.5pt]


\hhline{}
\arrayrulecolor{black}

\multicolumn{1}{!{\huxvb{0}}l!{\huxvb{0}}}{\huxtpad{4pt}\raggedright {\fontsize{9.5pt}{11.4pt}\selectfont }\huxbpad{4pt}} &
\multicolumn{1}{r!{\huxvb{0}}}{\huxtpad{4pt}\raggedleft {\fontsize{9.5pt}{11.4pt}\selectfont (0.457)~~~}\huxbpad{4pt}} &
\multicolumn{1}{r!{\huxvb{0}}}{\huxtpad{4pt}\raggedleft {\fontsize{9.5pt}{11.4pt}\selectfont (0.474)~~~}\huxbpad{4pt}} &
\multicolumn{1}{r!{\huxvb{0}}}{\huxtpad{4pt}\raggedleft {\fontsize{9.5pt}{11.4pt}\selectfont (0.465)~~~}\huxbpad{4pt}} \tabularnewline[-0.5pt]


\hhline{}
\arrayrulecolor{black}

\multicolumn{1}{!{\huxvb{0}}l!{\huxvb{0}}}{\huxtpad{4pt}\raggedright {\fontsize{9.5pt}{11.4pt}\selectfont Latinx}\huxbpad{4pt}} &
\multicolumn{1}{r!{\huxvb{0}}}{\huxtpad{4pt}\raggedleft {\fontsize{9.5pt}{11.4pt}\selectfont 1.641 ***}\huxbpad{4pt}} &
\multicolumn{1}{r!{\huxvb{0}}}{\huxtpad{4pt}\raggedleft {\fontsize{9.5pt}{11.4pt}\selectfont 1.280 *~~}\huxbpad{4pt}} &
\multicolumn{1}{r!{\huxvb{0}}}{\huxtpad{4pt}\raggedleft {\fontsize{9.5pt}{11.4pt}\selectfont 1.341 *~~}\huxbpad{4pt}} \tabularnewline[-0.5pt]


\hhline{}
\arrayrulecolor{black}

\multicolumn{1}{!{\huxvb{0}}l!{\huxvb{0}}}{\huxtpad{4pt}\raggedright {\fontsize{9.5pt}{11.4pt}\selectfont }\huxbpad{4pt}} &
\multicolumn{1}{r!{\huxvb{0}}}{\huxtpad{4pt}\raggedleft {\fontsize{9.5pt}{11.4pt}\selectfont (0.463)~~~}\huxbpad{4pt}} &
\multicolumn{1}{r!{\huxvb{0}}}{\huxtpad{4pt}\raggedleft {\fontsize{9.5pt}{11.4pt}\selectfont (0.515)~~~}\huxbpad{4pt}} &
\multicolumn{1}{r!{\huxvb{0}}}{\huxtpad{4pt}\raggedleft {\fontsize{9.5pt}{11.4pt}\selectfont (0.548)~~~}\huxbpad{4pt}} \tabularnewline[-0.5pt]


\hhline{}
\arrayrulecolor{black}

\multicolumn{1}{!{\huxvb{0}}l!{\huxvb{0}}}{\huxtpad{4pt}\raggedright {\fontsize{9.5pt}{11.4pt}\selectfont Demographics}\huxbpad{4pt}} &
\multicolumn{1}{r!{\huxvb{0}}}{\huxtpad{4pt}\raggedleft {\fontsize{9.5pt}{11.4pt}\selectfont ~~~~~~~~}\huxbpad{4pt}} &
\multicolumn{1}{r!{\huxvb{0}}}{\huxtpad{4pt}\raggedleft {\fontsize{9.5pt}{11.4pt}\selectfont ~~~~~~~~}\huxbpad{4pt}} &
\multicolumn{1}{r!{\huxvb{0}}}{\huxtpad{4pt}\raggedleft {\fontsize{9.5pt}{11.4pt}\selectfont ~~~~~~~~}\huxbpad{4pt}} \tabularnewline[-0.5pt]


\hhline{}
\arrayrulecolor{black}

\multicolumn{1}{!{\huxvb{0}}l!{\huxvb{0}}}{\huxtpad{4pt}\raggedright {\fontsize{9.5pt}{11.4pt}\selectfont Age}\huxbpad{4pt}} &
\multicolumn{1}{r!{\huxvb{0}}}{\huxtpad{4pt}\raggedleft {\fontsize{9.5pt}{11.4pt}\selectfont 20.548 ***}\huxbpad{4pt}} &
\multicolumn{1}{r!{\huxvb{0}}}{\huxtpad{4pt}\raggedleft {\fontsize{9.5pt}{11.4pt}\selectfont 20.168 ***}\huxbpad{4pt}} &
\multicolumn{1}{r!{\huxvb{0}}}{\huxtpad{4pt}\raggedleft {\fontsize{9.5pt}{11.4pt}\selectfont 19.561 ***}\huxbpad{4pt}} \tabularnewline[-0.5pt]


\hhline{}
\arrayrulecolor{black}

\multicolumn{1}{!{\huxvb{0}}l!{\huxvb{0}}}{\huxtpad{4pt}\raggedright {\fontsize{9.5pt}{11.4pt}\selectfont }\huxbpad{4pt}} &
\multicolumn{1}{r!{\huxvb{0}}}{\huxtpad{4pt}\raggedleft {\fontsize{9.5pt}{11.4pt}\selectfont (1.401)~~~}\huxbpad{4pt}} &
\multicolumn{1}{r!{\huxvb{0}}}{\huxtpad{4pt}\raggedleft {\fontsize{9.5pt}{11.4pt}\selectfont (1.574)~~~}\huxbpad{4pt}} &
\multicolumn{1}{r!{\huxvb{0}}}{\huxtpad{4pt}\raggedleft {\fontsize{9.5pt}{11.4pt}\selectfont (1.393)~~~}\huxbpad{4pt}} \tabularnewline[-0.5pt]


\hhline{}
\arrayrulecolor{black}

\multicolumn{1}{!{\huxvb{0}}l!{\huxvb{0}}}{\huxtpad{4pt}\raggedright {\fontsize{9.5pt}{11.4pt}\selectfont Foreign Born}\huxbpad{4pt}} &
\multicolumn{1}{r!{\huxvb{0}}}{\huxtpad{4pt}\raggedleft {\fontsize{9.5pt}{11.4pt}\selectfont 17.919 ***}\huxbpad{4pt}} &
\multicolumn{1}{r!{\huxvb{0}}}{\huxtpad{4pt}\raggedleft {\fontsize{9.5pt}{11.4pt}\selectfont 17.708 ***}\huxbpad{4pt}} &
\multicolumn{1}{r!{\huxvb{0}}}{\huxtpad{4pt}\raggedleft {\fontsize{9.5pt}{11.4pt}\selectfont 18.553 ***}\huxbpad{4pt}} \tabularnewline[-0.5pt]


\hhline{}
\arrayrulecolor{black}

\multicolumn{1}{!{\huxvb{0}}l!{\huxvb{0}}}{\huxtpad{4pt}\raggedright {\fontsize{9.5pt}{11.4pt}\selectfont }\huxbpad{4pt}} &
\multicolumn{1}{r!{\huxvb{0}}}{\huxtpad{4pt}\raggedleft {\fontsize{9.5pt}{11.4pt}\selectfont (1.468)~~~}\huxbpad{4pt}} &
\multicolumn{1}{r!{\huxvb{0}}}{\huxtpad{4pt}\raggedleft {\fontsize{9.5pt}{11.4pt}\selectfont (1.576)~~~}\huxbpad{4pt}} &
\multicolumn{1}{r!{\huxvb{0}}}{\huxtpad{4pt}\raggedleft {\fontsize{9.5pt}{11.4pt}\selectfont (1.364)~~~}\huxbpad{4pt}} \tabularnewline[-0.5pt]


\hhline{}
\arrayrulecolor{black}

\multicolumn{1}{!{\huxvb{0}}l!{\huxvb{0}}}{\huxtpad{4pt}\raggedright {\fontsize{9.5pt}{11.4pt}\selectfont Male}\huxbpad{4pt}} &
\multicolumn{1}{r!{\huxvb{0}}}{\huxtpad{4pt}\raggedleft {\fontsize{9.5pt}{11.4pt}\selectfont 1.139~~~~}\huxbpad{4pt}} &
\multicolumn{1}{r!{\huxvb{0}}}{\huxtpad{4pt}\raggedleft {\fontsize{9.5pt}{11.4pt}\selectfont 0.192~~~~}\huxbpad{4pt}} &
\multicolumn{1}{r!{\huxvb{0}}}{\huxtpad{4pt}\raggedleft {\fontsize{9.5pt}{11.4pt}\selectfont 0.445~~~~}\huxbpad{4pt}} \tabularnewline[-0.5pt]


\hhline{}
\arrayrulecolor{black}

\multicolumn{1}{!{\huxvb{0}}l!{\huxvb{0}}}{\huxtpad{4pt}\raggedright {\fontsize{9.5pt}{11.4pt}\selectfont }\huxbpad{4pt}} &
\multicolumn{1}{r!{\huxvb{0}}}{\huxtpad{4pt}\raggedleft {\fontsize{9.5pt}{11.4pt}\selectfont (1.112)~~~}\huxbpad{4pt}} &
\multicolumn{1}{r!{\huxvb{0}}}{\huxtpad{4pt}\raggedleft {\fontsize{9.5pt}{11.4pt}\selectfont (1.395)~~~}\huxbpad{4pt}} &
\multicolumn{1}{r!{\huxvb{0}}}{\huxtpad{4pt}\raggedleft {\fontsize{9.5pt}{11.4pt}\selectfont (1.509)~~~}\huxbpad{4pt}} \tabularnewline[-0.5pt]


\hhline{}
\arrayrulecolor{black}

\multicolumn{1}{!{\huxvb{0}}l!{\huxvb{0}}}{\huxtpad{4pt}\raggedright {\fontsize{9.5pt}{11.4pt}\selectfont Children Present}\huxbpad{4pt}} &
\multicolumn{1}{r!{\huxvb{0}}}{\huxtpad{4pt}\raggedleft {\fontsize{9.5pt}{11.4pt}\selectfont 0.780~~~~}\huxbpad{4pt}} &
\multicolumn{1}{r!{\huxvb{0}}}{\huxtpad{4pt}\raggedleft {\fontsize{9.5pt}{11.4pt}\selectfont 0.782~~~~}\huxbpad{4pt}} &
\multicolumn{1}{r!{\huxvb{0}}}{\huxtpad{4pt}\raggedleft {\fontsize{9.5pt}{11.4pt}\selectfont -0.031~~~~}\huxbpad{4pt}} \tabularnewline[-0.5pt]


\hhline{}
\arrayrulecolor{black}

\multicolumn{1}{!{\huxvb{0}}l!{\huxvb{0}}}{\huxtpad{4pt}\raggedright {\fontsize{9.5pt}{11.4pt}\selectfont }\huxbpad{4pt}} &
\multicolumn{1}{r!{\huxvb{0}}}{\huxtpad{4pt}\raggedleft {\fontsize{9.5pt}{11.4pt}\selectfont (1.293)~~~}\huxbpad{4pt}} &
\multicolumn{1}{r!{\huxvb{0}}}{\huxtpad{4pt}\raggedleft {\fontsize{9.5pt}{11.4pt}\selectfont (1.392)~~~}\huxbpad{4pt}} &
\multicolumn{1}{r!{\huxvb{0}}}{\huxtpad{4pt}\raggedleft {\fontsize{9.5pt}{11.4pt}\selectfont (1.469)~~~}\huxbpad{4pt}} \tabularnewline[-0.5pt]


\hhline{}
\arrayrulecolor{black}

\multicolumn{1}{!{\huxvb{0}}l!{\huxvb{0}}}{\huxtpad{4pt}\raggedright {\fontsize{9.5pt}{11.4pt}\selectfont Married}\huxbpad{4pt}} &
\multicolumn{1}{r!{\huxvb{0}}}{\huxtpad{4pt}\raggedleft {\fontsize{9.5pt}{11.4pt}\selectfont 16.796 ***}\huxbpad{4pt}} &
\multicolumn{1}{r!{\huxvb{0}}}{\huxtpad{4pt}\raggedleft {\fontsize{9.5pt}{11.4pt}\selectfont 16.408 ***}\huxbpad{4pt}} &
\multicolumn{1}{r!{\huxvb{0}}}{\huxtpad{4pt}\raggedleft {\fontsize{9.5pt}{11.4pt}\selectfont 18.279 ***}\huxbpad{4pt}} \tabularnewline[-0.5pt]


\hhline{}
\arrayrulecolor{black}

\multicolumn{1}{!{\huxvb{0}}l!{\huxvb{0}}}{\huxtpad{4pt}\raggedright {\fontsize{9.5pt}{11.4pt}\selectfont }\huxbpad{4pt}} &
\multicolumn{1}{r!{\huxvb{0}}}{\huxtpad{4pt}\raggedleft {\fontsize{9.5pt}{11.4pt}\selectfont (1.279)~~~}\huxbpad{4pt}} &
\multicolumn{1}{r!{\huxvb{0}}}{\huxtpad{4pt}\raggedleft {\fontsize{9.5pt}{11.4pt}\selectfont (1.427)~~~}\huxbpad{4pt}} &
\multicolumn{1}{r!{\huxvb{0}}}{\huxtpad{4pt}\raggedleft {\fontsize{9.5pt}{11.4pt}\selectfont (1.352)~~~}\huxbpad{4pt}} \tabularnewline[-0.5pt]


\hhline{}
\arrayrulecolor{black}

\multicolumn{1}{!{\huxvb{0}}l!{\huxvb{0}}}{\huxtpad{4pt}\raggedright {\fontsize{9.5pt}{11.4pt}\selectfont Socioeconomic}\huxbpad{4pt}} &
\multicolumn{1}{r!{\huxvb{0}}}{\huxtpad{4pt}\raggedleft {\fontsize{9.5pt}{11.4pt}\selectfont ~~~~~~~~}\huxbpad{4pt}} &
\multicolumn{1}{r!{\huxvb{0}}}{\huxtpad{4pt}\raggedleft {\fontsize{9.5pt}{11.4pt}\selectfont ~~~~~~~~}\huxbpad{4pt}} &
\multicolumn{1}{r!{\huxvb{0}}}{\huxtpad{4pt}\raggedleft {\fontsize{9.5pt}{11.4pt}\selectfont ~~~~~~~~}\huxbpad{4pt}} \tabularnewline[-0.5pt]


\hhline{}
\arrayrulecolor{black}

\multicolumn{1}{!{\huxvb{0}}l!{\huxvb{0}}}{\huxtpad{4pt}\raggedright {\fontsize{9.5pt}{11.4pt}\selectfont $<$H.S.}\huxbpad{4pt}} &
\multicolumn{1}{r!{\huxvb{0}}}{\huxtpad{4pt}\raggedleft {\fontsize{9.5pt}{11.4pt}\selectfont 17.245 ***}\huxbpad{4pt}} &
\multicolumn{1}{r!{\huxvb{0}}}{\huxtpad{4pt}\raggedleft {\fontsize{9.5pt}{11.4pt}\selectfont 16.970 ***}\huxbpad{4pt}} &
\multicolumn{1}{r!{\huxvb{0}}}{\huxtpad{4pt}\raggedleft {\fontsize{9.5pt}{11.4pt}\selectfont 17.851 ***}\huxbpad{4pt}} \tabularnewline[-0.5pt]


\hhline{}
\arrayrulecolor{black}

\multicolumn{1}{!{\huxvb{0}}l!{\huxvb{0}}}{\huxtpad{4pt}\raggedright {\fontsize{9.5pt}{11.4pt}\selectfont }\huxbpad{4pt}} &
\multicolumn{1}{r!{\huxvb{0}}}{\huxtpad{4pt}\raggedleft {\fontsize{9.5pt}{11.4pt}\selectfont (1.277)~~~}\huxbpad{4pt}} &
\multicolumn{1}{r!{\huxvb{0}}}{\huxtpad{4pt}\raggedleft {\fontsize{9.5pt}{11.4pt}\selectfont (1.429)~~~}\huxbpad{4pt}} &
\multicolumn{1}{r!{\huxvb{0}}}{\huxtpad{4pt}\raggedleft {\fontsize{9.5pt}{11.4pt}\selectfont (1.518)~~~}\huxbpad{4pt}} \tabularnewline[-0.5pt]


\hhline{}
\arrayrulecolor{black}

\multicolumn{1}{!{\huxvb{0}}l!{\huxvb{0}}}{\huxtpad{4pt}\raggedright {\fontsize{9.5pt}{11.4pt}\selectfont H.S.}\huxbpad{4pt}} &
\multicolumn{1}{r!{\huxvb{0}}}{\huxtpad{4pt}\raggedleft {\fontsize{9.5pt}{11.4pt}\selectfont 20.921 ***}\huxbpad{4pt}} &
\multicolumn{1}{r!{\huxvb{0}}}{\huxtpad{4pt}\raggedleft {\fontsize{9.5pt}{11.4pt}\selectfont 20.529 ***}\huxbpad{4pt}} &
\multicolumn{1}{r!{\huxvb{0}}}{\huxtpad{4pt}\raggedleft {\fontsize{9.5pt}{11.4pt}\selectfont 21.776 ***}\huxbpad{4pt}} \tabularnewline[-0.5pt]


\hhline{}
\arrayrulecolor{black}

\multicolumn{1}{!{\huxvb{0}}l!{\huxvb{0}}}{\huxtpad{4pt}\raggedright {\fontsize{9.5pt}{11.4pt}\selectfont }\huxbpad{4pt}} &
\multicolumn{1}{r!{\huxvb{0}}}{\huxtpad{4pt}\raggedleft {\fontsize{9.5pt}{11.4pt}\selectfont (1.109)~~~}\huxbpad{4pt}} &
\multicolumn{1}{r!{\huxvb{0}}}{\huxtpad{4pt}\raggedleft {\fontsize{9.5pt}{11.4pt}\selectfont (1.223)~~~}\huxbpad{4pt}} &
\multicolumn{1}{r!{\huxvb{0}}}{\huxtpad{4pt}\raggedleft {\fontsize{9.5pt}{11.4pt}\selectfont (1.055)~~~}\huxbpad{4pt}} \tabularnewline[-0.5pt]


\hhline{}
\arrayrulecolor{black}

\multicolumn{1}{!{\huxvb{0}}l!{\huxvb{0}}}{\huxtpad{4pt}\raggedright {\fontsize{9.5pt}{11.4pt}\selectfont Some college, no B.A.}\huxbpad{4pt}} &
\multicolumn{1}{r!{\huxvb{0}}}{\huxtpad{4pt}\raggedleft {\fontsize{9.5pt}{11.4pt}\selectfont 20.173 ***}\huxbpad{4pt}} &
\multicolumn{1}{r!{\huxvb{0}}}{\huxtpad{4pt}\raggedleft {\fontsize{9.5pt}{11.4pt}\selectfont 19.785 ***}\huxbpad{4pt}} &
\multicolumn{1}{r!{\huxvb{0}}}{\huxtpad{4pt}\raggedleft {\fontsize{9.5pt}{11.4pt}\selectfont 21.555 ***}\huxbpad{4pt}} \tabularnewline[-0.5pt]


\hhline{}
\arrayrulecolor{black}

\multicolumn{1}{!{\huxvb{0}}l!{\huxvb{0}}}{\huxtpad{4pt}\raggedright {\fontsize{9.5pt}{11.4pt}\selectfont }\huxbpad{4pt}} &
\multicolumn{1}{r!{\huxvb{0}}}{\huxtpad{4pt}\raggedleft {\fontsize{9.5pt}{11.4pt}\selectfont (1.155)~~~}\huxbpad{4pt}} &
\multicolumn{1}{r!{\huxvb{0}}}{\huxtpad{4pt}\raggedleft {\fontsize{9.5pt}{11.4pt}\selectfont (1.292)~~~}\huxbpad{4pt}} &
\multicolumn{1}{r!{\huxvb{0}}}{\huxtpad{4pt}\raggedleft {\fontsize{9.5pt}{11.4pt}\selectfont (1.238)~~~}\huxbpad{4pt}} \tabularnewline[-0.5pt]


\hhline{}
\arrayrulecolor{black}

\multicolumn{1}{!{\huxvb{0}}l!{\huxvb{0}}}{\huxtpad{4pt}\raggedright {\fontsize{9.5pt}{11.4pt}\selectfont B.A.}\huxbpad{4pt}} &
\multicolumn{1}{r!{\huxvb{0}}}{\huxtpad{4pt}\raggedleft {\fontsize{9.5pt}{11.4pt}\selectfont 37.543 ***}\huxbpad{4pt}} &
\multicolumn{1}{r!{\huxvb{0}}}{\huxtpad{4pt}\raggedleft {\fontsize{9.5pt}{11.4pt}\selectfont 37.053 ***}\huxbpad{4pt}} &
\multicolumn{1}{r!{\huxvb{0}}}{\huxtpad{4pt}\raggedleft {\fontsize{9.5pt}{11.4pt}\selectfont 38.731 ***}\huxbpad{4pt}} \tabularnewline[-0.5pt]


\hhline{}
\arrayrulecolor{black}

\multicolumn{1}{!{\huxvb{0}}l!{\huxvb{0}}}{\huxtpad{4pt}\raggedright {\fontsize{9.5pt}{11.4pt}\selectfont }\huxbpad{4pt}} &
\multicolumn{1}{r!{\huxvb{0}}}{\huxtpad{4pt}\raggedleft {\fontsize{9.5pt}{11.4pt}\selectfont (1.371)~~~}\huxbpad{4pt}} &
\multicolumn{1}{r!{\huxvb{0}}}{\huxtpad{4pt}\raggedleft {\fontsize{9.5pt}{11.4pt}\selectfont (1.473)~~~}\huxbpad{4pt}} &
\multicolumn{1}{r!{\huxvb{0}}}{\huxtpad{4pt}\raggedleft {\fontsize{9.5pt}{11.4pt}\selectfont (1.408)~~~}\huxbpad{4pt}} \tabularnewline[-0.5pt]


\hhline{}
\arrayrulecolor{black}

\multicolumn{1}{!{\huxvb{0}}l!{\huxvb{0}}}{\huxtpad{4pt}\raggedright {\fontsize{9.5pt}{11.4pt}\selectfont Neighborhood experience}\huxbpad{4pt}} &
\multicolumn{1}{r!{\huxvb{0}}}{\huxtpad{4pt}\raggedleft {\fontsize{9.5pt}{11.4pt}\selectfont ~~~~~~~~}\huxbpad{4pt}} &
\multicolumn{1}{r!{\huxvb{0}}}{\huxtpad{4pt}\raggedleft {\fontsize{9.5pt}{11.4pt}\selectfont ~~~~~~~~}\huxbpad{4pt}} &
\multicolumn{1}{r!{\huxvb{0}}}{\huxtpad{4pt}\raggedleft {\fontsize{9.5pt}{11.4pt}\selectfont ~~~~~~~~}\huxbpad{4pt}} \tabularnewline[-0.5pt]


\hhline{}
\arrayrulecolor{black}

\multicolumn{1}{!{\huxvb{0}}l!{\huxvb{0}}}{\huxtpad{4pt}\raggedright {\fontsize{9.5pt}{11.4pt}\selectfont Years in neighborhood}\huxbpad{4pt}} &
\multicolumn{1}{r!{\huxvb{0}}}{\huxtpad{4pt}\raggedleft {\fontsize{9.5pt}{11.4pt}\selectfont 21.063 ***}\huxbpad{4pt}} &
\multicolumn{1}{r!{\huxvb{0}}}{\huxtpad{4pt}\raggedleft {\fontsize{9.5pt}{11.4pt}\selectfont 20.587 ***}\huxbpad{4pt}} &
\multicolumn{1}{r!{\huxvb{0}}}{\huxtpad{4pt}\raggedleft {\fontsize{9.5pt}{11.4pt}\selectfont 21.965 ***}\huxbpad{4pt}} \tabularnewline[-0.5pt]


\hhline{}
\arrayrulecolor{black}

\multicolumn{1}{!{\huxvb{0}}l!{\huxvb{0}}}{\huxtpad{4pt}\raggedright {\fontsize{9.5pt}{11.4pt}\selectfont }\huxbpad{4pt}} &
\multicolumn{1}{r!{\huxvb{0}}}{\huxtpad{4pt}\raggedleft {\fontsize{9.5pt}{11.4pt}\selectfont (1.461)~~~}\huxbpad{4pt}} &
\multicolumn{1}{r!{\huxvb{0}}}{\huxtpad{4pt}\raggedleft {\fontsize{9.5pt}{11.4pt}\selectfont (1.598)~~~}\huxbpad{4pt}} &
\multicolumn{1}{r!{\huxvb{0}}}{\huxtpad{4pt}\raggedleft {\fontsize{9.5pt}{11.4pt}\selectfont (1.641)~~~}\huxbpad{4pt}} \tabularnewline[-0.5pt]


\hhline{}
\arrayrulecolor{black}

\multicolumn{1}{!{\huxvb{0}}l!{\huxvb{0}}}{\huxtpad{4pt}\raggedright {\fontsize{9.5pt}{11.4pt}\selectfont 1-9 blocks}\huxbpad{4pt}} &
\multicolumn{1}{r!{\huxvb{0}}}{\huxtpad{4pt}\raggedleft {\fontsize{9.5pt}{11.4pt}\selectfont 17.296 ***}\huxbpad{4pt}} &
\multicolumn{1}{r!{\huxvb{0}}}{\huxtpad{4pt}\raggedleft {\fontsize{9.5pt}{11.4pt}\selectfont 17.031 ***}\huxbpad{4pt}} &
\multicolumn{1}{r!{\huxvb{0}}}{\huxtpad{4pt}\raggedleft {\fontsize{9.5pt}{11.4pt}\selectfont 19.466 ***}\huxbpad{4pt}} \tabularnewline[-0.5pt]


\hhline{}
\arrayrulecolor{black}

\multicolumn{1}{!{\huxvb{0}}l!{\huxvb{0}}}{\huxtpad{4pt}\raggedright {\fontsize{9.5pt}{11.4pt}\selectfont }\huxbpad{4pt}} &
\multicolumn{1}{r!{\huxvb{0}}}{\huxtpad{4pt}\raggedleft {\fontsize{9.5pt}{11.4pt}\selectfont (1.596)~~~}\huxbpad{4pt}} &
\multicolumn{1}{r!{\huxvb{0}}}{\huxtpad{4pt}\raggedleft {\fontsize{9.5pt}{11.4pt}\selectfont (1.508)~~~}\huxbpad{4pt}} &
\multicolumn{1}{r!{\huxvb{0}}}{\huxtpad{4pt}\raggedleft {\fontsize{9.5pt}{11.4pt}\selectfont (1.323)~~~}\huxbpad{4pt}} \tabularnewline[-0.5pt]


\hhline{}
\arrayrulecolor{black}

\multicolumn{1}{!{\huxvb{0}}l!{\huxvb{0}}}{\huxtpad{4pt}\raggedright {\fontsize{9.5pt}{11.4pt}\selectfont 10-50 blocks}\huxbpad{4pt}} &
\multicolumn{1}{r!{\huxvb{0}}}{\huxtpad{4pt}\raggedleft {\fontsize{9.5pt}{11.4pt}\selectfont 18.377 ***}\huxbpad{4pt}} &
\multicolumn{1}{r!{\huxvb{0}}}{\huxtpad{4pt}\raggedleft {\fontsize{9.5pt}{11.4pt}\selectfont 17.968 ***}\huxbpad{4pt}} &
\multicolumn{1}{r!{\huxvb{0}}}{\huxtpad{4pt}\raggedleft {\fontsize{9.5pt}{11.4pt}\selectfont 19.748 ***}\huxbpad{4pt}} \tabularnewline[-0.5pt]


\hhline{}
\arrayrulecolor{black}

\multicolumn{1}{!{\huxvb{0}}l!{\huxvb{0}}}{\huxtpad{4pt}\raggedright {\fontsize{9.5pt}{11.4pt}\selectfont }\huxbpad{4pt}} &
\multicolumn{1}{r!{\huxvb{0}}}{\huxtpad{4pt}\raggedleft {\fontsize{9.5pt}{11.4pt}\selectfont (1.529)~~~}\huxbpad{4pt}} &
\multicolumn{1}{r!{\huxvb{0}}}{\huxtpad{4pt}\raggedleft {\fontsize{9.5pt}{11.4pt}\selectfont (1.548)~~~}\huxbpad{4pt}} &
\multicolumn{1}{r!{\huxvb{0}}}{\huxtpad{4pt}\raggedleft {\fontsize{9.5pt}{11.4pt}\selectfont (1.331)~~~}\huxbpad{4pt}} \tabularnewline[-0.5pt]


\hhline{}
\arrayrulecolor{black}

\multicolumn{1}{!{\huxvb{0}}l!{\huxvb{0}}}{\huxtpad{4pt}\raggedright {\fontsize{9.5pt}{11.4pt}\selectfont Extremely satisfied}\huxbpad{4pt}} &
\multicolumn{1}{r!{\huxvb{0}}}{\huxtpad{4pt}\raggedleft {\fontsize{9.5pt}{11.4pt}\selectfont 19.556 ***}\huxbpad{4pt}} &
\multicolumn{1}{r!{\huxvb{0}}}{\huxtpad{4pt}\raggedleft {\fontsize{9.5pt}{11.4pt}\selectfont 18.566 ***}\huxbpad{4pt}} &
\multicolumn{1}{r!{\huxvb{0}}}{\huxtpad{4pt}\raggedleft {\fontsize{9.5pt}{11.4pt}\selectfont 20.032 ***}\huxbpad{4pt}} \tabularnewline[-0.5pt]


\hhline{}
\arrayrulecolor{black}

\multicolumn{1}{!{\huxvb{0}}l!{\huxvb{0}}}{\huxtpad{4pt}\raggedright {\fontsize{9.5pt}{11.4pt}\selectfont }\huxbpad{4pt}} &
\multicolumn{1}{r!{\huxvb{0}}}{\huxtpad{4pt}\raggedleft {\fontsize{9.5pt}{11.4pt}\selectfont (1.807)~~~}\huxbpad{4pt}} &
\multicolumn{1}{r!{\huxvb{0}}}{\huxtpad{4pt}\raggedleft {\fontsize{9.5pt}{11.4pt}\selectfont (1.773)~~~}\huxbpad{4pt}} &
\multicolumn{1}{r!{\huxvb{0}}}{\huxtpad{4pt}\raggedleft {\fontsize{9.5pt}{11.4pt}\selectfont (1.637)~~~}\huxbpad{4pt}} \tabularnewline[-0.5pt]


\hhline{}
\arrayrulecolor{black}

\multicolumn{1}{!{\huxvb{0}}l!{\huxvb{0}}}{\huxtpad{4pt}\raggedright {\fontsize{9.5pt}{11.4pt}\selectfont sample\_tract24031700715}\huxbpad{4pt}} &
\multicolumn{1}{r!{\huxvb{0}}}{\huxtpad{4pt}\raggedleft {\fontsize{9.5pt}{11.4pt}\selectfont 18.550 ***}\huxbpad{4pt}} &
\multicolumn{1}{r!{\huxvb{0}}}{\huxtpad{4pt}\raggedleft {\fontsize{9.5pt}{11.4pt}\selectfont 17.856 ***}\huxbpad{4pt}} &
\multicolumn{1}{r!{\huxvb{0}}}{\huxtpad{4pt}\raggedleft {\fontsize{9.5pt}{11.4pt}\selectfont 18.422 ***}\huxbpad{4pt}} \tabularnewline[-0.5pt]


\hhline{}
\arrayrulecolor{black}

\multicolumn{1}{!{\huxvb{0}}l!{\huxvb{0}}}{\huxtpad{4pt}\raggedright {\fontsize{9.5pt}{11.4pt}\selectfont }\huxbpad{4pt}} &
\multicolumn{1}{r!{\huxvb{0}}}{\huxtpad{4pt}\raggedleft {\fontsize{9.5pt}{11.4pt}\selectfont (1.418)~~~}\huxbpad{4pt}} &
\multicolumn{1}{r!{\huxvb{0}}}{\huxtpad{4pt}\raggedleft {\fontsize{9.5pt}{11.4pt}\selectfont (1.532)~~~}\huxbpad{4pt}} &
\multicolumn{1}{r!{\huxvb{0}}}{\huxtpad{4pt}\raggedleft {\fontsize{9.5pt}{11.4pt}\selectfont (2.010)~~~}\huxbpad{4pt}} \tabularnewline[-0.5pt]


\hhline{}
\arrayrulecolor{black}

\multicolumn{1}{!{\huxvb{0}}l!{\huxvb{0}}}{\huxtpad{4pt}\raggedright {\fontsize{9.5pt}{11.4pt}\selectfont sample\_tract24031700716}\huxbpad{4pt}} &
\multicolumn{1}{r!{\huxvb{0}}}{\huxtpad{4pt}\raggedleft {\fontsize{9.5pt}{11.4pt}\selectfont 0.667~~~~}\huxbpad{4pt}} &
\multicolumn{1}{r!{\huxvb{0}}}{\huxtpad{4pt}\raggedleft {\fontsize{9.5pt}{11.4pt}\selectfont -0.099~~~~}\huxbpad{4pt}} &
\multicolumn{1}{r!{\huxvb{0}}}{\huxtpad{4pt}\raggedleft {\fontsize{9.5pt}{11.4pt}\selectfont 0.583~~~~}\huxbpad{4pt}} \tabularnewline[-0.5pt]


\hhline{}
\arrayrulecolor{black}

\multicolumn{1}{!{\huxvb{0}}l!{\huxvb{0}}}{\huxtpad{4pt}\raggedright {\fontsize{9.5pt}{11.4pt}\selectfont }\huxbpad{4pt}} &
\multicolumn{1}{r!{\huxvb{0}}}{\huxtpad{4pt}\raggedleft {\fontsize{9.5pt}{11.4pt}\selectfont (1.022)~~~}\huxbpad{4pt}} &
\multicolumn{1}{r!{\huxvb{0}}}{\huxtpad{4pt}\raggedleft {\fontsize{9.5pt}{11.4pt}\selectfont (1.156)~~~}\huxbpad{4pt}} &
\multicolumn{1}{r!{\huxvb{0}}}{\huxtpad{4pt}\raggedleft {\fontsize{9.5pt}{11.4pt}\selectfont (1.192)~~~}\huxbpad{4pt}} \tabularnewline[-0.5pt]


\hhline{}
\arrayrulecolor{black}

\multicolumn{1}{!{\huxvb{0}}l!{\huxvb{0}}}{\huxtpad{4pt}\raggedright {\fontsize{9.5pt}{11.4pt}\selectfont sample\_tract24031700720}\huxbpad{4pt}} &
\multicolumn{1}{r!{\huxvb{0}}}{\huxtpad{4pt}\raggedleft {\fontsize{9.5pt}{11.4pt}\selectfont 0.283~~~~}\huxbpad{4pt}} &
\multicolumn{1}{r!{\huxvb{0}}}{\huxtpad{4pt}\raggedleft {\fontsize{9.5pt}{11.4pt}\selectfont -0.072~~~~}\huxbpad{4pt}} &
\multicolumn{1}{r!{\huxvb{0}}}{\huxtpad{4pt}\raggedleft {\fontsize{9.5pt}{11.4pt}\selectfont -0.029~~~~}\huxbpad{4pt}} \tabularnewline[-0.5pt]


\hhline{}
\arrayrulecolor{black}

\multicolumn{1}{!{\huxvb{0}}l!{\huxvb{0}}}{\huxtpad{4pt}\raggedright {\fontsize{9.5pt}{11.4pt}\selectfont }\huxbpad{4pt}} &
\multicolumn{1}{r!{\huxvb{0}}}{\huxtpad{4pt}\raggedleft {\fontsize{9.5pt}{11.4pt}\selectfont (1.109)~~~}\huxbpad{4pt}} &
\multicolumn{1}{r!{\huxvb{0}}}{\huxtpad{4pt}\raggedleft {\fontsize{9.5pt}{11.4pt}\selectfont (1.187)~~~}\huxbpad{4pt}} &
\multicolumn{1}{r!{\huxvb{0}}}{\huxtpad{4pt}\raggedleft {\fontsize{9.5pt}{11.4pt}\selectfont (1.287)~~~}\huxbpad{4pt}} \tabularnewline[-0.5pt]


\hhline{}
\arrayrulecolor{black}

\multicolumn{1}{!{\huxvb{0}}l!{\huxvb{0}}}{\huxtpad{4pt}\raggedright {\fontsize{9.5pt}{11.4pt}\selectfont sample\_tract24031700721}\huxbpad{4pt}} &
\multicolumn{1}{r!{\huxvb{0}}}{\huxtpad{4pt}\raggedleft {\fontsize{9.5pt}{11.4pt}\selectfont 21.972 ***}\huxbpad{4pt}} &
\multicolumn{1}{r!{\huxvb{0}}}{\huxtpad{4pt}\raggedleft {\fontsize{9.5pt}{11.4pt}\selectfont 21.815 ***}\huxbpad{4pt}} &
\multicolumn{1}{r!{\huxvb{0}}}{\huxtpad{4pt}\raggedleft {\fontsize{9.5pt}{11.4pt}\selectfont 23.752 ***}\huxbpad{4pt}} \tabularnewline[-0.5pt]


\hhline{}
\arrayrulecolor{black}

\multicolumn{1}{!{\huxvb{0}}l!{\huxvb{0}}}{\huxtpad{4pt}\raggedright {\fontsize{9.5pt}{11.4pt}\selectfont }\huxbpad{4pt}} &
\multicolumn{1}{r!{\huxvb{0}}}{\huxtpad{4pt}\raggedleft {\fontsize{9.5pt}{11.4pt}\selectfont (1.704)~~~}\huxbpad{4pt}} &
\multicolumn{1}{r!{\huxvb{0}}}{\huxtpad{4pt}\raggedleft {\fontsize{9.5pt}{11.4pt}\selectfont (1.729)~~~}\huxbpad{4pt}} &
\multicolumn{1}{r!{\huxvb{0}}}{\huxtpad{4pt}\raggedleft {\fontsize{9.5pt}{11.4pt}\selectfont (1.584)~~~}\huxbpad{4pt}} \tabularnewline[-0.5pt]


\hhline{}
\arrayrulecolor{black}

\multicolumn{1}{!{\huxvb{0}}l!{\huxvb{0}}}{\huxtpad{4pt}\raggedright {\fontsize{9.5pt}{11.4pt}\selectfont sample\_tract24031700722}\huxbpad{4pt}} &
\multicolumn{1}{r!{\huxvb{0}}}{\huxtpad{4pt}\raggedleft {\fontsize{9.5pt}{11.4pt}\selectfont 19.637 ***}\huxbpad{4pt}} &
\multicolumn{1}{r!{\huxvb{0}}}{\huxtpad{4pt}\raggedleft {\fontsize{9.5pt}{11.4pt}\selectfont 19.742 ***}\huxbpad{4pt}} &
\multicolumn{1}{r!{\huxvb{0}}}{\huxtpad{4pt}\raggedleft {\fontsize{9.5pt}{11.4pt}\selectfont 21.787 ***}\huxbpad{4pt}} \tabularnewline[-0.5pt]


\hhline{}
\arrayrulecolor{black}

\multicolumn{1}{!{\huxvb{0}}l!{\huxvb{0}}}{\huxtpad{4pt}\raggedright {\fontsize{9.5pt}{11.4pt}\selectfont }\huxbpad{4pt}} &
\multicolumn{1}{r!{\huxvb{0}}}{\huxtpad{4pt}\raggedleft {\fontsize{9.5pt}{11.4pt}\selectfont (1.216)~~~}\huxbpad{4pt}} &
\multicolumn{1}{r!{\huxvb{0}}}{\huxtpad{4pt}\raggedleft {\fontsize{9.5pt}{11.4pt}\selectfont (1.405)~~~}\huxbpad{4pt}} &
\multicolumn{1}{r!{\huxvb{0}}}{\huxtpad{4pt}\raggedleft {\fontsize{9.5pt}{11.4pt}\selectfont (1.971)~~~}\huxbpad{4pt}} \tabularnewline[-0.5pt]


\hhline{}
\arrayrulecolor{black}

\multicolumn{1}{!{\huxvb{0}}l!{\huxvb{0}}}{\huxtpad{4pt}\raggedright {\fontsize{9.5pt}{11.4pt}\selectfont sample\_tract24031700724}\huxbpad{4pt}} &
\multicolumn{1}{r!{\huxvb{0}}}{\huxtpad{4pt}\raggedleft {\fontsize{9.5pt}{11.4pt}\selectfont 18.432 ***}\huxbpad{4pt}} &
\multicolumn{1}{r!{\huxvb{0}}}{\huxtpad{4pt}\raggedleft {\fontsize{9.5pt}{11.4pt}\selectfont 17.724 ***}\huxbpad{4pt}} &
\multicolumn{1}{r!{\huxvb{0}}}{\huxtpad{4pt}\raggedleft {\fontsize{9.5pt}{11.4pt}\selectfont 19.578 ***}\huxbpad{4pt}} \tabularnewline[-0.5pt]


\hhline{}
\arrayrulecolor{black}

\multicolumn{1}{!{\huxvb{0}}l!{\huxvb{0}}}{\huxtpad{4pt}\raggedright {\fontsize{9.5pt}{11.4pt}\selectfont }\huxbpad{4pt}} &
\multicolumn{1}{r!{\huxvb{0}}}{\huxtpad{4pt}\raggedleft {\fontsize{9.5pt}{11.4pt}\selectfont (1.424)~~~}\huxbpad{4pt}} &
\multicolumn{1}{r!{\huxvb{0}}}{\huxtpad{4pt}\raggedleft {\fontsize{9.5pt}{11.4pt}\selectfont (1.515)~~~}\huxbpad{4pt}} &
\multicolumn{1}{r!{\huxvb{0}}}{\huxtpad{4pt}\raggedleft {\fontsize{9.5pt}{11.4pt}\selectfont (1.308)~~~}\huxbpad{4pt}} \tabularnewline[-0.5pt]


\hhline{}
\arrayrulecolor{black}

\multicolumn{1}{!{\huxvb{0}}l!{\huxvb{0}}}{\huxtpad{4pt}\raggedright {\fontsize{9.5pt}{11.4pt}\selectfont sample\_tract24031700810}\huxbpad{4pt}} &
\multicolumn{1}{r!{\huxvb{0}}}{\huxtpad{4pt}\raggedleft {\fontsize{9.5pt}{11.4pt}\selectfont 0.823~~~~}\huxbpad{4pt}} &
\multicolumn{1}{r!{\huxvb{0}}}{\huxtpad{4pt}\raggedleft {\fontsize{9.5pt}{11.4pt}\selectfont 0.358~~~~}\huxbpad{4pt}} &
\multicolumn{1}{r!{\huxvb{0}}}{\huxtpad{4pt}\raggedleft {\fontsize{9.5pt}{11.4pt}\selectfont 0.183~~~~}\huxbpad{4pt}} \tabularnewline[-0.5pt]


\hhline{}
\arrayrulecolor{black}

\multicolumn{1}{!{\huxvb{0}}l!{\huxvb{0}}}{\huxtpad{4pt}\raggedright {\fontsize{9.5pt}{11.4pt}\selectfont }\huxbpad{4pt}} &
\multicolumn{1}{r!{\huxvb{0}}}{\huxtpad{4pt}\raggedleft {\fontsize{9.5pt}{11.4pt}\selectfont (1.122)~~~}\huxbpad{4pt}} &
\multicolumn{1}{r!{\huxvb{0}}}{\huxtpad{4pt}\raggedleft {\fontsize{9.5pt}{11.4pt}\selectfont (1.234)~~~}\huxbpad{4pt}} &
\multicolumn{1}{r!{\huxvb{0}}}{\huxtpad{4pt}\raggedleft {\fontsize{9.5pt}{11.4pt}\selectfont (1.326)~~~}\huxbpad{4pt}} \tabularnewline[-0.5pt]


\hhline{}
\arrayrulecolor{black}

\multicolumn{1}{!{\huxvb{0}}l!{\huxvb{0}}}{\huxtpad{4pt}\raggedright {\fontsize{9.5pt}{11.4pt}\selectfont sample\_tract24031700815}\huxbpad{4pt}} &
\multicolumn{1}{r!{\huxvb{0}}}{\huxtpad{4pt}\raggedleft {\fontsize{9.5pt}{11.4pt}\selectfont 18.418 ***}\huxbpad{4pt}} &
\multicolumn{1}{r!{\huxvb{0}}}{\huxtpad{4pt}\raggedleft {\fontsize{9.5pt}{11.4pt}\selectfont 17.938 ***}\huxbpad{4pt}} &
\multicolumn{1}{r!{\huxvb{0}}}{\huxtpad{4pt}\raggedleft {\fontsize{9.5pt}{11.4pt}\selectfont 19.561 ***}\huxbpad{4pt}} \tabularnewline[-0.5pt]


\hhline{}
\arrayrulecolor{black}

\multicolumn{1}{!{\huxvb{0}}l!{\huxvb{0}}}{\huxtpad{4pt}\raggedright {\fontsize{9.5pt}{11.4pt}\selectfont }\huxbpad{4pt}} &
\multicolumn{1}{r!{\huxvb{0}}}{\huxtpad{4pt}\raggedleft {\fontsize{9.5pt}{11.4pt}\selectfont (1.397)~~~}\huxbpad{4pt}} &
\multicolumn{1}{r!{\huxvb{0}}}{\huxtpad{4pt}\raggedleft {\fontsize{9.5pt}{11.4pt}\selectfont (1.538)~~~}\huxbpad{4pt}} &
\multicolumn{1}{r!{\huxvb{0}}}{\huxtpad{4pt}\raggedleft {\fontsize{9.5pt}{11.4pt}\selectfont (1.593)~~~}\huxbpad{4pt}} \tabularnewline[-0.5pt]


\hhline{}
\arrayrulecolor{black}

\multicolumn{1}{!{\huxvb{0}}l!{\huxvb{0}}}{\huxtpad{4pt}\raggedright {\fontsize{9.5pt}{11.4pt}\selectfont sample\_tract24031700816}\huxbpad{4pt}} &
\multicolumn{1}{r!{\huxvb{0}}}{\huxtpad{4pt}\raggedleft {\fontsize{9.5pt}{11.4pt}\selectfont 0.410~~~~}\huxbpad{4pt}} &
\multicolumn{1}{r!{\huxvb{0}}}{\huxtpad{4pt}\raggedleft {\fontsize{9.5pt}{11.4pt}\selectfont 0.546~~~~}\huxbpad{4pt}} &
\multicolumn{1}{r!{\huxvb{0}}}{\huxtpad{4pt}\raggedleft {\fontsize{9.5pt}{11.4pt}\selectfont 0.973~~~~}\huxbpad{4pt}} \tabularnewline[-0.5pt]


\hhline{}
\arrayrulecolor{black}

\multicolumn{1}{!{\huxvb{0}}l!{\huxvb{0}}}{\huxtpad{4pt}\raggedright {\fontsize{9.5pt}{11.4pt}\selectfont }\huxbpad{4pt}} &
\multicolumn{1}{r!{\huxvb{0}}}{\huxtpad{4pt}\raggedleft {\fontsize{9.5pt}{11.4pt}\selectfont (1.169)~~~}\huxbpad{4pt}} &
\multicolumn{1}{r!{\huxvb{0}}}{\huxtpad{4pt}\raggedleft {\fontsize{9.5pt}{11.4pt}\selectfont (1.246)~~~}\huxbpad{4pt}} &
\multicolumn{1}{r!{\huxvb{0}}}{\huxtpad{4pt}\raggedleft {\fontsize{9.5pt}{11.4pt}\selectfont (1.191)~~~}\huxbpad{4pt}} \tabularnewline[-0.5pt]


\hhline{}
\arrayrulecolor{black}

\multicolumn{1}{!{\huxvb{0}}l!{\huxvb{0}}}{\huxtpad{4pt}\raggedright {\fontsize{9.5pt}{11.4pt}\selectfont sample\_tract24031700817}\huxbpad{4pt}} &
\multicolumn{1}{r!{\huxvb{0}}}{\huxtpad{4pt}\raggedleft {\fontsize{9.5pt}{11.4pt}\selectfont 21.765 ***}\huxbpad{4pt}} &
\multicolumn{1}{r!{\huxvb{0}}}{\huxtpad{4pt}\raggedleft {\fontsize{9.5pt}{11.4pt}\selectfont 21.478 ***}\huxbpad{4pt}} &
\multicolumn{1}{r!{\huxvb{0}}}{\huxtpad{4pt}\raggedleft {\fontsize{9.5pt}{11.4pt}\selectfont 23.061 ***}\huxbpad{4pt}} \tabularnewline[-0.5pt]


\hhline{}
\arrayrulecolor{black}

\multicolumn{1}{!{\huxvb{0}}l!{\huxvb{0}}}{\huxtpad{4pt}\raggedright {\fontsize{9.5pt}{11.4pt}\selectfont }\huxbpad{4pt}} &
\multicolumn{1}{r!{\huxvb{0}}}{\huxtpad{4pt}\raggedleft {\fontsize{9.5pt}{11.4pt}\selectfont (1.367)~~~}\huxbpad{4pt}} &
\multicolumn{1}{r!{\huxvb{0}}}{\huxtpad{4pt}\raggedleft {\fontsize{9.5pt}{11.4pt}\selectfont (1.423)~~~}\huxbpad{4pt}} &
\multicolumn{1}{r!{\huxvb{0}}}{\huxtpad{4pt}\raggedleft {\fontsize{9.5pt}{11.4pt}\selectfont (1.409)~~~}\huxbpad{4pt}} \tabularnewline[-0.5pt]


\hhline{}
\arrayrulecolor{black}

\multicolumn{1}{!{\huxvb{0}}l!{\huxvb{0}}}{\huxtpad{4pt}\raggedright {\fontsize{9.5pt}{11.4pt}\selectfont sample\_tract24031700818}\huxbpad{4pt}} &
\multicolumn{1}{r!{\huxvb{0}}}{\huxtpad{4pt}\raggedleft {\fontsize{9.5pt}{11.4pt}\selectfont 20.159 ***}\huxbpad{4pt}} &
\multicolumn{1}{r!{\huxvb{0}}}{\huxtpad{4pt}\raggedleft {\fontsize{9.5pt}{11.4pt}\selectfont 19.639 ***}\huxbpad{4pt}} &
\multicolumn{1}{r!{\huxvb{0}}}{\huxtpad{4pt}\raggedleft {\fontsize{9.5pt}{11.4pt}\selectfont 20.634 ***}\huxbpad{4pt}} \tabularnewline[-0.5pt]


\hhline{}
\arrayrulecolor{black}

\multicolumn{1}{!{\huxvb{0}}l!{\huxvb{0}}}{\huxtpad{4pt}\raggedright {\fontsize{9.5pt}{11.4pt}\selectfont }\huxbpad{4pt}} &
\multicolumn{1}{r!{\huxvb{0}}}{\huxtpad{4pt}\raggedleft {\fontsize{9.5pt}{11.4pt}\selectfont (1.278)~~~}\huxbpad{4pt}} &
\multicolumn{1}{r!{\huxvb{0}}}{\huxtpad{4pt}\raggedleft {\fontsize{9.5pt}{11.4pt}\selectfont (1.407)~~~}\huxbpad{4pt}} &
\multicolumn{1}{r!{\huxvb{0}}}{\huxtpad{4pt}\raggedleft {\fontsize{9.5pt}{11.4pt}\selectfont (1.282)~~~}\huxbpad{4pt}} \tabularnewline[-0.5pt]


\hhline{}
\arrayrulecolor{black}

\multicolumn{1}{!{\huxvb{0}}l!{\huxvb{0}}}{\huxtpad{4pt}\raggedright {\fontsize{9.5pt}{11.4pt}\selectfont sample\_tract24031700819}\huxbpad{4pt}} &
\multicolumn{1}{r!{\huxvb{0}}}{\huxtpad{4pt}\raggedleft {\fontsize{9.5pt}{11.4pt}\selectfont 19.301 ***}\huxbpad{4pt}} &
\multicolumn{1}{r!{\huxvb{0}}}{\huxtpad{4pt}\raggedleft {\fontsize{9.5pt}{11.4pt}\selectfont 18.867 ***}\huxbpad{4pt}} &
\multicolumn{1}{r!{\huxvb{0}}}{\huxtpad{4pt}\raggedleft {\fontsize{9.5pt}{11.4pt}\selectfont 20.342 ***}\huxbpad{4pt}} \tabularnewline[-0.5pt]


\hhline{}
\arrayrulecolor{black}

\multicolumn{1}{!{\huxvb{0}}l!{\huxvb{0}}}{\huxtpad{4pt}\raggedright {\fontsize{9.5pt}{11.4pt}\selectfont }\huxbpad{4pt}} &
\multicolumn{1}{r!{\huxvb{0}}}{\huxtpad{4pt}\raggedleft {\fontsize{9.5pt}{11.4pt}\selectfont (1.241)~~~}\huxbpad{4pt}} &
\multicolumn{1}{r!{\huxvb{0}}}{\huxtpad{4pt}\raggedleft {\fontsize{9.5pt}{11.4pt}\selectfont (1.331)~~~}\huxbpad{4pt}} &
\multicolumn{1}{r!{\huxvb{0}}}{\huxtpad{4pt}\raggedleft {\fontsize{9.5pt}{11.4pt}\selectfont (1.222)~~~}\huxbpad{4pt}} \tabularnewline[-0.5pt]


\hhline{}
\arrayrulecolor{black}

\multicolumn{1}{!{\huxvb{0}}l!{\huxvb{0}}}{\huxtpad{4pt}\raggedright {\fontsize{9.5pt}{11.4pt}\selectfont sample\_tract24031700820}\huxbpad{4pt}} &
\multicolumn{1}{r!{\huxvb{0}}}{\huxtpad{4pt}\raggedleft {\fontsize{9.5pt}{11.4pt}\selectfont 17.540 ***}\huxbpad{4pt}} &
\multicolumn{1}{r!{\huxvb{0}}}{\huxtpad{4pt}\raggedleft {\fontsize{9.5pt}{11.4pt}\selectfont 16.994 ***}\huxbpad{4pt}} &
\multicolumn{1}{r!{\huxvb{0}}}{\huxtpad{4pt}\raggedleft {\fontsize{9.5pt}{11.4pt}\selectfont 17.871 ***}\huxbpad{4pt}} \tabularnewline[-0.5pt]


\hhline{}
\arrayrulecolor{black}

\multicolumn{1}{!{\huxvb{0}}l!{\huxvb{0}}}{\huxtpad{4pt}\raggedright {\fontsize{9.5pt}{11.4pt}\selectfont }\huxbpad{4pt}} &
\multicolumn{1}{r!{\huxvb{0}}}{\huxtpad{4pt}\raggedleft {\fontsize{9.5pt}{11.4pt}\selectfont (1.470)~~~}\huxbpad{4pt}} &
\multicolumn{1}{r!{\huxvb{0}}}{\huxtpad{4pt}\raggedleft {\fontsize{9.5pt}{11.4pt}\selectfont (1.645)~~~}\huxbpad{4pt}} &
\multicolumn{1}{r!{\huxvb{0}}}{\huxtpad{4pt}\raggedleft {\fontsize{9.5pt}{11.4pt}\selectfont (1.423)~~~}\huxbpad{4pt}} \tabularnewline[-0.5pt]


\hhline{}
\arrayrulecolor{black}

\multicolumn{1}{!{\huxvb{0}}l!{\huxvb{0}}}{\huxtpad{4pt}\raggedright {\fontsize{9.5pt}{11.4pt}\selectfont sample\_tract24031700822}\huxbpad{4pt}} &
\multicolumn{1}{r!{\huxvb{0}}}{\huxtpad{4pt}\raggedleft {\fontsize{9.5pt}{11.4pt}\selectfont 37.533 ***}\huxbpad{4pt}} &
\multicolumn{1}{r!{\huxvb{0}}}{\huxtpad{4pt}\raggedleft {\fontsize{9.5pt}{11.4pt}\selectfont 37.137 ***}\huxbpad{4pt}} &
\multicolumn{1}{r!{\huxvb{0}}}{\huxtpad{4pt}\raggedleft {\fontsize{9.5pt}{11.4pt}\selectfont 39.490 ***}\huxbpad{4pt}} \tabularnewline[-0.5pt]


\hhline{}
\arrayrulecolor{black}

\multicolumn{1}{!{\huxvb{0}}l!{\huxvb{0}}}{\huxtpad{4pt}\raggedright {\fontsize{9.5pt}{11.4pt}\selectfont }\huxbpad{4pt}} &
\multicolumn{1}{r!{\huxvb{0}}}{\huxtpad{4pt}\raggedleft {\fontsize{9.5pt}{11.4pt}\selectfont (1.371)~~~}\huxbpad{4pt}} &
\multicolumn{1}{r!{\huxvb{0}}}{\huxtpad{4pt}\raggedleft {\fontsize{9.5pt}{11.4pt}\selectfont (1.536)~~~}\huxbpad{4pt}} &
\multicolumn{1}{r!{\huxvb{0}}}{\huxtpad{4pt}\raggedleft {\fontsize{9.5pt}{11.4pt}\selectfont (1.733)~~~}\huxbpad{4pt}} \tabularnewline[-0.5pt]


\hhline{}
\arrayrulecolor{black}

\multicolumn{1}{!{\huxvb{0}}l!{\huxvb{0}}}{\huxtpad{4pt}\raggedright {\fontsize{9.5pt}{11.4pt}\selectfont sample\_tract24031700830}\huxbpad{4pt}} &
\multicolumn{1}{r!{\huxvb{0}}}{\huxtpad{4pt}\raggedleft {\fontsize{9.5pt}{11.4pt}\selectfont 1.230~~~~}\huxbpad{4pt}} &
\multicolumn{1}{r!{\huxvb{0}}}{\huxtpad{4pt}\raggedleft {\fontsize{9.5pt}{11.4pt}\selectfont 1.603~~~~}\huxbpad{4pt}} &
\multicolumn{1}{r!{\huxvb{0}}}{\huxtpad{4pt}\raggedleft {\fontsize{9.5pt}{11.4pt}\selectfont 1.344~~~~}\huxbpad{4pt}} \tabularnewline[-0.5pt]


\hhline{}
\arrayrulecolor{black}

\multicolumn{1}{!{\huxvb{0}}l!{\huxvb{0}}}{\huxtpad{4pt}\raggedright {\fontsize{9.5pt}{11.4pt}\selectfont }\huxbpad{4pt}} &
\multicolumn{1}{r!{\huxvb{0}}}{\huxtpad{4pt}\raggedleft {\fontsize{9.5pt}{11.4pt}\selectfont (1.112)~~~}\huxbpad{4pt}} &
\multicolumn{1}{r!{\huxvb{0}}}{\huxtpad{4pt}\raggedleft {\fontsize{9.5pt}{11.4pt}\selectfont (1.191)~~~}\huxbpad{4pt}} &
\multicolumn{1}{r!{\huxvb{0}}}{\huxtpad{4pt}\raggedleft {\fontsize{9.5pt}{11.4pt}\selectfont (1.261)~~~}\huxbpad{4pt}} \tabularnewline[-0.5pt]


\hhline{}
\arrayrulecolor{black}

\multicolumn{1}{!{\huxvb{0}}l!{\huxvb{0}}}{\huxtpad{4pt}\raggedright {\fontsize{9.5pt}{11.4pt}\selectfont sample\_tract24031700833}\huxbpad{4pt}} &
\multicolumn{1}{r!{\huxvb{0}}}{\huxtpad{4pt}\raggedleft {\fontsize{9.5pt}{11.4pt}\selectfont 0.814~~~~}\huxbpad{4pt}} &
\multicolumn{1}{r!{\huxvb{0}}}{\huxtpad{4pt}\raggedleft {\fontsize{9.5pt}{11.4pt}\selectfont 0.514~~~~}\huxbpad{4pt}} &
\multicolumn{1}{r!{\huxvb{0}}}{\huxtpad{4pt}\raggedleft {\fontsize{9.5pt}{11.4pt}\selectfont 0.503~~~~}\huxbpad{4pt}} \tabularnewline[-0.5pt]


\hhline{}
\arrayrulecolor{black}

\multicolumn{1}{!{\huxvb{0}}l!{\huxvb{0}}}{\huxtpad{4pt}\raggedright {\fontsize{9.5pt}{11.4pt}\selectfont }\huxbpad{4pt}} &
\multicolumn{1}{r!{\huxvb{0}}}{\huxtpad{4pt}\raggedleft {\fontsize{9.5pt}{11.4pt}\selectfont (1.080)~~~}\huxbpad{4pt}} &
\multicolumn{1}{r!{\huxvb{0}}}{\huxtpad{4pt}\raggedleft {\fontsize{9.5pt}{11.4pt}\selectfont (1.215)~~~}\huxbpad{4pt}} &
\multicolumn{1}{r!{\huxvb{0}}}{\huxtpad{4pt}\raggedleft {\fontsize{9.5pt}{11.4pt}\selectfont (1.109)~~~}\huxbpad{4pt}} \tabularnewline[-0.5pt]


\hhline{}
\arrayrulecolor{black}

\multicolumn{1}{!{\huxvb{0}}l!{\huxvb{0}}}{\huxtpad{4pt}\raggedright {\fontsize{9.5pt}{11.4pt}\selectfont sample\_tract24031700834}\huxbpad{4pt}} &
\multicolumn{1}{r!{\huxvb{0}}}{\huxtpad{4pt}\raggedleft {\fontsize{9.5pt}{11.4pt}\selectfont 15.567 ***}\huxbpad{4pt}} &
\multicolumn{1}{r!{\huxvb{0}}}{\huxtpad{4pt}\raggedleft {\fontsize{9.5pt}{11.4pt}\selectfont 14.542 ***}\huxbpad{4pt}} &
\multicolumn{1}{r!{\huxvb{0}}}{\huxtpad{4pt}\raggedleft {\fontsize{9.5pt}{11.4pt}\selectfont 16.325 ***}\huxbpad{4pt}} \tabularnewline[-0.5pt]


\hhline{}
\arrayrulecolor{black}

\multicolumn{1}{!{\huxvb{0}}l!{\huxvb{0}}}{\huxtpad{4pt}\raggedright {\fontsize{9.5pt}{11.4pt}\selectfont }\huxbpad{4pt}} &
\multicolumn{1}{r!{\huxvb{0}}}{\huxtpad{4pt}\raggedleft {\fontsize{9.5pt}{11.4pt}\selectfont (1.415)~~~}\huxbpad{4pt}} &
\multicolumn{1}{r!{\huxvb{0}}}{\huxtpad{4pt}\raggedleft {\fontsize{9.5pt}{11.4pt}\selectfont (1.572)~~~}\huxbpad{4pt}} &
\multicolumn{1}{r!{\huxvb{0}}}{\huxtpad{4pt}\raggedleft {\fontsize{9.5pt}{11.4pt}\selectfont (1.555)~~~}\huxbpad{4pt}} \tabularnewline[-0.5pt]


\hhline{}
\arrayrulecolor{black}

\multicolumn{1}{!{\huxvb{0}}l!{\huxvb{0}}}{\huxtpad{4pt}\raggedright {\fontsize{9.5pt}{11.4pt}\selectfont sample\_tract24031700835}\huxbpad{4pt}} &
\multicolumn{1}{r!{\huxvb{0}}}{\huxtpad{4pt}\raggedleft {\fontsize{9.5pt}{11.4pt}\selectfont 0.461~~~~}\huxbpad{4pt}} &
\multicolumn{1}{r!{\huxvb{0}}}{\huxtpad{4pt}\raggedleft {\fontsize{9.5pt}{11.4pt}\selectfont 0.147~~~~}\huxbpad{4pt}} &
\multicolumn{1}{r!{\huxvb{0}}}{\huxtpad{4pt}\raggedleft {\fontsize{9.5pt}{11.4pt}\selectfont 0.039~~~~}\huxbpad{4pt}} \tabularnewline[-0.5pt]


\hhline{}
\arrayrulecolor{black}

\multicolumn{1}{!{\huxvb{0}}l!{\huxvb{0}}}{\huxtpad{4pt}\raggedright {\fontsize{9.5pt}{11.4pt}\selectfont }\huxbpad{4pt}} &
\multicolumn{1}{r!{\huxvb{0}}}{\huxtpad{4pt}\raggedleft {\fontsize{9.5pt}{11.4pt}\selectfont (1.119)~~~}\huxbpad{4pt}} &
\multicolumn{1}{r!{\huxvb{0}}}{\huxtpad{4pt}\raggedleft {\fontsize{9.5pt}{11.4pt}\selectfont (1.219)~~~}\huxbpad{4pt}} &
\multicolumn{1}{r!{\huxvb{0}}}{\huxtpad{4pt}\raggedleft {\fontsize{9.5pt}{11.4pt}\selectfont (1.260)~~~}\huxbpad{4pt}} \tabularnewline[-0.5pt]


\hhline{}
\arrayrulecolor{black}

\multicolumn{1}{!{\huxvb{0}}l!{\huxvb{0}}}{\huxtpad{4pt}\raggedright {\fontsize{9.5pt}{11.4pt}\selectfont sample\_tract24031700902}\huxbpad{4pt}} &
\multicolumn{1}{r!{\huxvb{0}}}{\huxtpad{4pt}\raggedleft {\fontsize{9.5pt}{11.4pt}\selectfont 0.410~~~~}\huxbpad{4pt}} &
\multicolumn{1}{r!{\huxvb{0}}}{\huxtpad{4pt}\raggedleft {\fontsize{9.5pt}{11.4pt}\selectfont -1.263~~~~}\huxbpad{4pt}} &
\multicolumn{1}{r!{\huxvb{0}}}{\huxtpad{4pt}\raggedleft {\fontsize{9.5pt}{11.4pt}\selectfont 0.827~~~~}\huxbpad{4pt}} \tabularnewline[-0.5pt]


\hhline{}
\arrayrulecolor{black}

\multicolumn{1}{!{\huxvb{0}}l!{\huxvb{0}}}{\huxtpad{4pt}\raggedright {\fontsize{9.5pt}{11.4pt}\selectfont }\huxbpad{4pt}} &
\multicolumn{1}{r!{\huxvb{0}}}{\huxtpad{4pt}\raggedleft {\fontsize{9.5pt}{11.4pt}\selectfont (1.266)~~~}\huxbpad{4pt}} &
\multicolumn{1}{r!{\huxvb{0}}}{\huxtpad{4pt}\raggedleft {\fontsize{9.5pt}{11.4pt}\selectfont (1.687)~~~}\huxbpad{4pt}} &
\multicolumn{1}{r!{\huxvb{0}}}{\huxtpad{4pt}\raggedleft {\fontsize{9.5pt}{11.4pt}\selectfont (1.577)~~~}\huxbpad{4pt}} \tabularnewline[-0.5pt]


\hhline{}
\arrayrulecolor{black}

\multicolumn{1}{!{\huxvb{0}}l!{\huxvb{0}}}{\huxtpad{4pt}\raggedright {\fontsize{9.5pt}{11.4pt}\selectfont sample\_tract24031700903}\huxbpad{4pt}} &
\multicolumn{1}{r!{\huxvb{0}}}{\huxtpad{4pt}\raggedleft {\fontsize{9.5pt}{11.4pt}\selectfont 20.218 ***}\huxbpad{4pt}} &
\multicolumn{1}{r!{\huxvb{0}}}{\huxtpad{4pt}\raggedleft {\fontsize{9.5pt}{11.4pt}\selectfont 19.522 ***}\huxbpad{4pt}} &
\multicolumn{1}{r!{\huxvb{0}}}{\huxtpad{4pt}\raggedleft {\fontsize{9.5pt}{11.4pt}\selectfont 20.324 ***}\huxbpad{4pt}} \tabularnewline[-0.5pt]


\hhline{}
\arrayrulecolor{black}

\multicolumn{1}{!{\huxvb{0}}l!{\huxvb{0}}}{\huxtpad{4pt}\raggedright {\fontsize{9.5pt}{11.4pt}\selectfont }\huxbpad{4pt}} &
\multicolumn{1}{r!{\huxvb{0}}}{\huxtpad{4pt}\raggedleft {\fontsize{9.5pt}{11.4pt}\selectfont (1.388)~~~}\huxbpad{4pt}} &
\multicolumn{1}{r!{\huxvb{0}}}{\huxtpad{4pt}\raggedleft {\fontsize{9.5pt}{11.4pt}\selectfont (1.485)~~~}\huxbpad{4pt}} &
\multicolumn{1}{r!{\huxvb{0}}}{\huxtpad{4pt}\raggedleft {\fontsize{9.5pt}{11.4pt}\selectfont (1.910)~~~}\huxbpad{4pt}} \tabularnewline[-0.5pt]


\hhline{}
\arrayrulecolor{black}

\multicolumn{1}{!{\huxvb{0}}l!{\huxvb{0}}}{\huxtpad{4pt}\raggedright {\fontsize{9.5pt}{11.4pt}\selectfont sample\_tract24031701211}\huxbpad{4pt}} &
\multicolumn{1}{r!{\huxvb{0}}}{\huxtpad{4pt}\raggedleft {\fontsize{9.5pt}{11.4pt}\selectfont 18.442 ***}\huxbpad{4pt}} &
\multicolumn{1}{r!{\huxvb{0}}}{\huxtpad{4pt}\raggedleft {\fontsize{9.5pt}{11.4pt}\selectfont 18.004 ***}\huxbpad{4pt}} &
\multicolumn{1}{r!{\huxvb{0}}}{\huxtpad{4pt}\raggedleft {\fontsize{9.5pt}{11.4pt}\selectfont 18.458 ***}\huxbpad{4pt}} \tabularnewline[-0.5pt]


\hhline{}
\arrayrulecolor{black}

\multicolumn{1}{!{\huxvb{0}}l!{\huxvb{0}}}{\huxtpad{4pt}\raggedright {\fontsize{9.5pt}{11.4pt}\selectfont }\huxbpad{4pt}} &
\multicolumn{1}{r!{\huxvb{0}}}{\huxtpad{4pt}\raggedleft {\fontsize{9.5pt}{11.4pt}\selectfont (1.170)~~~}\huxbpad{4pt}} &
\multicolumn{1}{r!{\huxvb{0}}}{\huxtpad{4pt}\raggedleft {\fontsize{9.5pt}{11.4pt}\selectfont (1.279)~~~}\huxbpad{4pt}} &
\multicolumn{1}{r!{\huxvb{0}}}{\huxtpad{4pt}\raggedleft {\fontsize{9.5pt}{11.4pt}\selectfont (1.437)~~~}\huxbpad{4pt}} \tabularnewline[-0.5pt]


\hhline{}
\arrayrulecolor{black}

\multicolumn{1}{!{\huxvb{0}}l!{\huxvb{0}}}{\huxtpad{4pt}\raggedright {\fontsize{9.5pt}{11.4pt}\selectfont sample\_tract24031701219}\huxbpad{4pt}} &
\multicolumn{1}{r!{\huxvb{0}}}{\huxtpad{4pt}\raggedleft {\fontsize{9.5pt}{11.4pt}\selectfont 39.175 ***}\huxbpad{4pt}} &
\multicolumn{1}{r!{\huxvb{0}}}{\huxtpad{4pt}\raggedleft {\fontsize{9.5pt}{11.4pt}\selectfont 38.258 ***}\huxbpad{4pt}} &
\multicolumn{1}{r!{\huxvb{0}}}{\huxtpad{4pt}\raggedleft {\fontsize{9.5pt}{11.4pt}\selectfont 40.630 ***}\huxbpad{4pt}} \tabularnewline[-0.5pt]


\hhline{}
\arrayrulecolor{black}

\multicolumn{1}{!{\huxvb{0}}l!{\huxvb{0}}}{\huxtpad{4pt}\raggedright {\fontsize{9.5pt}{11.4pt}\selectfont }\huxbpad{4pt}} &
\multicolumn{1}{r!{\huxvb{0}}}{\huxtpad{4pt}\raggedleft {\fontsize{9.5pt}{11.4pt}\selectfont (1.372)~~~}\huxbpad{4pt}} &
\multicolumn{1}{r!{\huxvb{0}}}{\huxtpad{4pt}\raggedleft {\fontsize{9.5pt}{11.4pt}\selectfont (1.534)~~~}\huxbpad{4pt}} &
\multicolumn{1}{r!{\huxvb{0}}}{\huxtpad{4pt}\raggedleft {\fontsize{9.5pt}{11.4pt}\selectfont (1.482)~~~}\huxbpad{4pt}} \tabularnewline[-0.5pt]


\hhline{}
\arrayrulecolor{black}

\multicolumn{1}{!{\huxvb{0}}l!{\huxvb{0}}}{\huxtpad{4pt}\raggedright {\fontsize{9.5pt}{11.4pt}\selectfont sample\_tract24031701407}\huxbpad{4pt}} &
\multicolumn{1}{r!{\huxvb{0}}}{\huxtpad{4pt}\raggedleft {\fontsize{9.5pt}{11.4pt}\selectfont 17.740 ***}\huxbpad{4pt}} &
\multicolumn{1}{r!{\huxvb{0}}}{\huxtpad{4pt}\raggedleft {\fontsize{9.5pt}{11.4pt}\selectfont 17.588 ***}\huxbpad{4pt}} &
\multicolumn{1}{r!{\huxvb{0}}}{\huxtpad{4pt}\raggedleft {\fontsize{9.5pt}{11.4pt}\selectfont 19.833 ***}\huxbpad{4pt}} \tabularnewline[-0.5pt]


\hhline{}
\arrayrulecolor{black}

\multicolumn{1}{!{\huxvb{0}}l!{\huxvb{0}}}{\huxtpad{4pt}\raggedright {\fontsize{9.5pt}{11.4pt}\selectfont }\huxbpad{4pt}} &
\multicolumn{1}{r!{\huxvb{0}}}{\huxtpad{4pt}\raggedleft {\fontsize{9.5pt}{11.4pt}\selectfont (1.227)~~~}\huxbpad{4pt}} &
\multicolumn{1}{r!{\huxvb{0}}}{\huxtpad{4pt}\raggedleft {\fontsize{9.5pt}{11.4pt}\selectfont (1.296)~~~}\huxbpad{4pt}} &
\multicolumn{1}{r!{\huxvb{0}}}{\huxtpad{4pt}\raggedleft {\fontsize{9.5pt}{11.4pt}\selectfont (1.276)~~~}\huxbpad{4pt}} \tabularnewline[-0.5pt]


\hhline{}
\arrayrulecolor{black}

\multicolumn{1}{!{\huxvb{0}}l!{\huxvb{0}}}{\huxtpad{4pt}\raggedright {\fontsize{9.5pt}{11.4pt}\selectfont sample\_tract24031701408}\huxbpad{4pt}} &
\multicolumn{1}{r!{\huxvb{0}}}{\huxtpad{4pt}\raggedleft {\fontsize{9.5pt}{11.4pt}\selectfont 18.255 ***}\huxbpad{4pt}} &
\multicolumn{1}{r!{\huxvb{0}}}{\huxtpad{4pt}\raggedleft {\fontsize{9.5pt}{11.4pt}\selectfont 17.418 ***}\huxbpad{4pt}} &
\multicolumn{1}{r!{\huxvb{0}}}{\huxtpad{4pt}\raggedleft {\fontsize{9.5pt}{11.4pt}\selectfont 18.331 ***}\huxbpad{4pt}} \tabularnewline[-0.5pt]


\hhline{}
\arrayrulecolor{black}

\multicolumn{1}{!{\huxvb{0}}l!{\huxvb{0}}}{\huxtpad{4pt}\raggedright {\fontsize{9.5pt}{11.4pt}\selectfont }\huxbpad{4pt}} &
\multicolumn{1}{r!{\huxvb{0}}}{\huxtpad{4pt}\raggedleft {\fontsize{9.5pt}{11.4pt}\selectfont (1.701)~~~}\huxbpad{4pt}} &
\multicolumn{1}{r!{\huxvb{0}}}{\huxtpad{4pt}\raggedleft {\fontsize{9.5pt}{11.4pt}\selectfont (1.845)~~~}\huxbpad{4pt}} &
\multicolumn{1}{r!{\huxvb{0}}}{\huxtpad{4pt}\raggedleft {\fontsize{9.5pt}{11.4pt}\selectfont (1.469)~~~}\huxbpad{4pt}} \tabularnewline[-0.5pt]


\hhline{}
\arrayrulecolor{black}

\multicolumn{1}{!{\huxvb{0}}l!{\huxvb{0}}}{\huxtpad{4pt}\raggedright {\fontsize{9.5pt}{11.4pt}\selectfont sample\_tract24031701414}\huxbpad{4pt}} &
\multicolumn{1}{r!{\huxvb{0}}}{\huxtpad{4pt}\raggedleft {\fontsize{9.5pt}{11.4pt}\selectfont 17.459 ***}\huxbpad{4pt}} &
\multicolumn{1}{r!{\huxvb{0}}}{\huxtpad{4pt}\raggedleft {\fontsize{9.5pt}{11.4pt}\selectfont 17.522 ***}\huxbpad{4pt}} &
\multicolumn{1}{r!{\huxvb{0}}}{\huxtpad{4pt}\raggedleft {\fontsize{9.5pt}{11.4pt}\selectfont 18.667 ***}\huxbpad{4pt}} \tabularnewline[-0.5pt]


\hhline{}
\arrayrulecolor{black}

\multicolumn{1}{!{\huxvb{0}}l!{\huxvb{0}}}{\huxtpad{4pt}\raggedright {\fontsize{9.5pt}{11.4pt}\selectfont }\huxbpad{4pt}} &
\multicolumn{1}{r!{\huxvb{0}}}{\huxtpad{4pt}\raggedleft {\fontsize{9.5pt}{11.4pt}\selectfont (1.139)~~~}\huxbpad{4pt}} &
\multicolumn{1}{r!{\huxvb{0}}}{\huxtpad{4pt}\raggedleft {\fontsize{9.5pt}{11.4pt}\selectfont (1.179)~~~}\huxbpad{4pt}} &
\multicolumn{1}{r!{\huxvb{0}}}{\huxtpad{4pt}\raggedleft {\fontsize{9.5pt}{11.4pt}\selectfont (1.094)~~~}\huxbpad{4pt}} \tabularnewline[-0.5pt]


\hhline{}
\arrayrulecolor{black}

\multicolumn{1}{!{\huxvb{0}}l!{\huxvb{0}}}{\huxtpad{4pt}\raggedright {\fontsize{9.5pt}{11.4pt}\selectfont sample\_tract24031701415}\huxbpad{4pt}} &
\multicolumn{1}{r!{\huxvb{0}}}{\huxtpad{4pt}\raggedleft {\fontsize{9.5pt}{11.4pt}\selectfont 17.444 ***}\huxbpad{4pt}} &
\multicolumn{1}{r!{\huxvb{0}}}{\huxtpad{4pt}\raggedleft {\fontsize{9.5pt}{11.4pt}\selectfont 16.980 ***}\huxbpad{4pt}} &
\multicolumn{1}{r!{\huxvb{0}}}{\huxtpad{4pt}\raggedleft {\fontsize{9.5pt}{11.4pt}\selectfont 18.411 ***}\huxbpad{4pt}} \tabularnewline[-0.5pt]


\hhline{}
\arrayrulecolor{black}

\multicolumn{1}{!{\huxvb{0}}l!{\huxvb{0}}}{\huxtpad{4pt}\raggedright {\fontsize{9.5pt}{11.4pt}\selectfont }\huxbpad{4pt}} &
\multicolumn{1}{r!{\huxvb{0}}}{\huxtpad{4pt}\raggedleft {\fontsize{9.5pt}{11.4pt}\selectfont (1.152)~~~}\huxbpad{4pt}} &
\multicolumn{1}{r!{\huxvb{0}}}{\huxtpad{4pt}\raggedleft {\fontsize{9.5pt}{11.4pt}\selectfont (1.235)~~~}\huxbpad{4pt}} &
\multicolumn{1}{r!{\huxvb{0}}}{\huxtpad{4pt}\raggedleft {\fontsize{9.5pt}{11.4pt}\selectfont (1.193)~~~}\huxbpad{4pt}} \tabularnewline[-0.5pt]


\hhline{}
\arrayrulecolor{black}

\multicolumn{1}{!{\huxvb{0}}l!{\huxvb{0}}}{\huxtpad{4pt}\raggedright {\fontsize{9.5pt}{11.4pt}\selectfont sample\_tract24031701503}\huxbpad{4pt}} &
\multicolumn{1}{r!{\huxvb{0}}}{\huxtpad{4pt}\raggedleft {\fontsize{9.5pt}{11.4pt}\selectfont 19.358 ***}\huxbpad{4pt}} &
\multicolumn{1}{r!{\huxvb{0}}}{\huxtpad{4pt}\raggedleft {\fontsize{9.5pt}{11.4pt}\selectfont 18.703 ***}\huxbpad{4pt}} &
\multicolumn{1}{r!{\huxvb{0}}}{\huxtpad{4pt}\raggedleft {\fontsize{9.5pt}{11.4pt}\selectfont 19.953 ***}\huxbpad{4pt}} \tabularnewline[-0.5pt]


\hhline{}
\arrayrulecolor{black}

\multicolumn{1}{!{\huxvb{0}}l!{\huxvb{0}}}{\huxtpad{4pt}\raggedright {\fontsize{9.5pt}{11.4pt}\selectfont }\huxbpad{4pt}} &
\multicolumn{1}{r!{\huxvb{0}}}{\huxtpad{4pt}\raggedleft {\fontsize{9.5pt}{11.4pt}\selectfont (1.196)~~~}\huxbpad{4pt}} &
\multicolumn{1}{r!{\huxvb{0}}}{\huxtpad{4pt}\raggedleft {\fontsize{9.5pt}{11.4pt}\selectfont (1.265)~~~}\huxbpad{4pt}} &
\multicolumn{1}{r!{\huxvb{0}}}{\huxtpad{4pt}\raggedleft {\fontsize{9.5pt}{11.4pt}\selectfont (1.168)~~~}\huxbpad{4pt}} \tabularnewline[-0.5pt]


\hhline{}
\arrayrulecolor{black}

\multicolumn{1}{!{\huxvb{0}}l!{\huxvb{0}}}{\huxtpad{4pt}\raggedright {\fontsize{9.5pt}{11.4pt}\selectfont sample\_tract24031701507}\huxbpad{4pt}} &
\multicolumn{1}{r!{\huxvb{0}}}{\huxtpad{4pt}\raggedleft {\fontsize{9.5pt}{11.4pt}\selectfont 17.914 ***}\huxbpad{4pt}} &
\multicolumn{1}{r!{\huxvb{0}}}{\huxtpad{4pt}\raggedleft {\fontsize{9.5pt}{11.4pt}\selectfont 17.163 ***}\huxbpad{4pt}} &
\multicolumn{1}{r!{\huxvb{0}}}{\huxtpad{4pt}\raggedleft {\fontsize{9.5pt}{11.4pt}\selectfont 18.847 ***}\huxbpad{4pt}} \tabularnewline[-0.5pt]


\hhline{}
\arrayrulecolor{black}

\multicolumn{1}{!{\huxvb{0}}l!{\huxvb{0}}}{\huxtpad{4pt}\raggedright {\fontsize{9.5pt}{11.4pt}\selectfont }\huxbpad{4pt}} &
\multicolumn{1}{r!{\huxvb{0}}}{\huxtpad{4pt}\raggedleft {\fontsize{9.5pt}{11.4pt}\selectfont (1.323)~~~}\huxbpad{4pt}} &
\multicolumn{1}{r!{\huxvb{0}}}{\huxtpad{4pt}\raggedleft {\fontsize{9.5pt}{11.4pt}\selectfont (1.247)~~~}\huxbpad{4pt}} &
\multicolumn{1}{r!{\huxvb{0}}}{\huxtpad{4pt}\raggedleft {\fontsize{9.5pt}{11.4pt}\selectfont (1.238)~~~}\huxbpad{4pt}} \tabularnewline[-0.5pt]


\hhline{}
\arrayrulecolor{black}

\multicolumn{1}{!{\huxvb{0}}l!{\huxvb{0}}}{\huxtpad{4pt}\raggedright {\fontsize{9.5pt}{11.4pt}\selectfont sample\_tract24031703206}\huxbpad{4pt}} &
\multicolumn{1}{r!{\huxvb{0}}}{\huxtpad{4pt}\raggedleft {\fontsize{9.5pt}{11.4pt}\selectfont 18.264 ***}\huxbpad{4pt}} &
\multicolumn{1}{r!{\huxvb{0}}}{\huxtpad{4pt}\raggedleft {\fontsize{9.5pt}{11.4pt}\selectfont 17.717 ***}\huxbpad{4pt}} &
\multicolumn{1}{r!{\huxvb{0}}}{\huxtpad{4pt}\raggedleft {\fontsize{9.5pt}{11.4pt}\selectfont 18.547 ***}\huxbpad{4pt}} \tabularnewline[-0.5pt]


\hhline{}
\arrayrulecolor{black}

\multicolumn{1}{!{\huxvb{0}}l!{\huxvb{0}}}{\huxtpad{4pt}\raggedright {\fontsize{9.5pt}{11.4pt}\selectfont }\huxbpad{4pt}} &
\multicolumn{1}{r!{\huxvb{0}}}{\huxtpad{4pt}\raggedleft {\fontsize{9.5pt}{11.4pt}\selectfont (1.212)~~~}\huxbpad{4pt}} &
\multicolumn{1}{r!{\huxvb{0}}}{\huxtpad{4pt}\raggedleft {\fontsize{9.5pt}{11.4pt}\selectfont (1.569)~~~}\huxbpad{4pt}} &
\multicolumn{1}{r!{\huxvb{0}}}{\huxtpad{4pt}\raggedleft {\fontsize{9.5pt}{11.4pt}\selectfont (2.462)~~~}\huxbpad{4pt}} \tabularnewline[-0.5pt]


\hhline{}
\arrayrulecolor{black}

\multicolumn{1}{!{\huxvb{0}}l!{\huxvb{0}}}{\huxtpad{4pt}\raggedright {\fontsize{9.5pt}{11.4pt}\selectfont sample\_tract24031703209}\huxbpad{4pt}} &
\multicolumn{1}{r!{\huxvb{0}}}{\huxtpad{4pt}\raggedleft {\fontsize{9.5pt}{11.4pt}\selectfont 1.048~~~~}\huxbpad{4pt}} &
\multicolumn{1}{r!{\huxvb{0}}}{\huxtpad{4pt}\raggedleft {\fontsize{9.5pt}{11.4pt}\selectfont 0.820~~~~}\huxbpad{4pt}} &
\multicolumn{1}{r!{\huxvb{0}}}{\huxtpad{4pt}\raggedleft {\fontsize{9.5pt}{11.4pt}\selectfont 0.630~~~~}\huxbpad{4pt}} \tabularnewline[-0.5pt]


\hhline{}
\arrayrulecolor{black}

\multicolumn{1}{!{\huxvb{0}}l!{\huxvb{0}}}{\huxtpad{4pt}\raggedright {\fontsize{9.5pt}{11.4pt}\selectfont }\huxbpad{4pt}} &
\multicolumn{1}{r!{\huxvb{0}}}{\huxtpad{4pt}\raggedleft {\fontsize{9.5pt}{11.4pt}\selectfont (1.147)~~~}\huxbpad{4pt}} &
\multicolumn{1}{r!{\huxvb{0}}}{\huxtpad{4pt}\raggedleft {\fontsize{9.5pt}{11.4pt}\selectfont (1.312)~~~}\huxbpad{4pt}} &
\multicolumn{1}{r!{\huxvb{0}}}{\huxtpad{4pt}\raggedleft {\fontsize{9.5pt}{11.4pt}\selectfont (1.130)~~~}\huxbpad{4pt}} \tabularnewline[-0.5pt]


\hhline{}
\arrayrulecolor{black}

\multicolumn{1}{!{\huxvb{0}}l!{\huxvb{0}}}{\huxtpad{4pt}\raggedright {\fontsize{9.5pt}{11.4pt}\selectfont sample\_tract24031703210}\huxbpad{4pt}} &
\multicolumn{1}{r!{\huxvb{0}}}{\huxtpad{4pt}\raggedleft {\fontsize{9.5pt}{11.4pt}\selectfont 0.363~~~~}\huxbpad{4pt}} &
\multicolumn{1}{r!{\huxvb{0}}}{\huxtpad{4pt}\raggedleft {\fontsize{9.5pt}{11.4pt}\selectfont -0.512~~~~}\huxbpad{4pt}} &
\multicolumn{1}{r!{\huxvb{0}}}{\huxtpad{4pt}\raggedleft {\fontsize{9.5pt}{11.4pt}\selectfont -0.208~~~~}\huxbpad{4pt}} \tabularnewline[-0.5pt]


\hhline{}
\arrayrulecolor{black}

\multicolumn{1}{!{\huxvb{0}}l!{\huxvb{0}}}{\huxtpad{4pt}\raggedright {\fontsize{9.5pt}{11.4pt}\selectfont }\huxbpad{4pt}} &
\multicolumn{1}{r!{\huxvb{0}}}{\huxtpad{4pt}\raggedleft {\fontsize{9.5pt}{11.4pt}\selectfont (1.157)~~~}\huxbpad{4pt}} &
\multicolumn{1}{r!{\huxvb{0}}}{\huxtpad{4pt}\raggedleft {\fontsize{9.5pt}{11.4pt}\selectfont (1.399)~~~}\huxbpad{4pt}} &
\multicolumn{1}{r!{\huxvb{0}}}{\huxtpad{4pt}\raggedleft {\fontsize{9.5pt}{11.4pt}\selectfont (1.352)~~~}\huxbpad{4pt}} \tabularnewline[-0.5pt]


\hhline{}
\arrayrulecolor{black}

\multicolumn{1}{!{\huxvb{0}}l!{\huxvb{0}}}{\huxtpad{4pt}\raggedright {\fontsize{9.5pt}{11.4pt}\selectfont sample\_tract24031703212}\huxbpad{4pt}} &
\multicolumn{1}{r!{\huxvb{0}}}{\huxtpad{4pt}\raggedleft {\fontsize{9.5pt}{11.4pt}\selectfont 18.604 ***}\huxbpad{4pt}} &
\multicolumn{1}{r!{\huxvb{0}}}{\huxtpad{4pt}\raggedleft {\fontsize{9.5pt}{11.4pt}\selectfont 18.231 ***}\huxbpad{4pt}} &
\multicolumn{1}{r!{\huxvb{0}}}{\huxtpad{4pt}\raggedleft {\fontsize{9.5pt}{11.4pt}\selectfont 19.784 ***}\huxbpad{4pt}} \tabularnewline[-0.5pt]


\hhline{}
\arrayrulecolor{black}

\multicolumn{1}{!{\huxvb{0}}l!{\huxvb{0}}}{\huxtpad{4pt}\raggedright {\fontsize{9.5pt}{11.4pt}\selectfont }\huxbpad{4pt}} &
\multicolumn{1}{r!{\huxvb{0}}}{\huxtpad{4pt}\raggedleft {\fontsize{9.5pt}{11.4pt}\selectfont (1.113)~~~}\huxbpad{4pt}} &
\multicolumn{1}{r!{\huxvb{0}}}{\huxtpad{4pt}\raggedleft {\fontsize{9.5pt}{11.4pt}\selectfont (1.265)~~~}\huxbpad{4pt}} &
\multicolumn{1}{r!{\huxvb{0}}}{\huxtpad{4pt}\raggedleft {\fontsize{9.5pt}{11.4pt}\selectfont (1.252)~~~}\huxbpad{4pt}} \tabularnewline[-0.5pt]


\hhline{}
\arrayrulecolor{black}

\multicolumn{1}{!{\huxvb{0}}l!{\huxvb{0}}}{\huxtpad{4pt}\raggedright {\fontsize{9.5pt}{11.4pt}\selectfont sample\_tract24031703215}\huxbpad{4pt}} &
\multicolumn{1}{r!{\huxvb{0}}}{\huxtpad{4pt}\raggedleft {\fontsize{9.5pt}{11.4pt}\selectfont 17.952 ***}\huxbpad{4pt}} &
\multicolumn{1}{r!{\huxvb{0}}}{\huxtpad{4pt}\raggedleft {\fontsize{9.5pt}{11.4pt}\selectfont 17.786 ***}\huxbpad{4pt}} &
\multicolumn{1}{r!{\huxvb{0}}}{\huxtpad{4pt}\raggedleft {\fontsize{9.5pt}{11.4pt}\selectfont 18.609 ***}\huxbpad{4pt}} \tabularnewline[-0.5pt]


\hhline{}
\arrayrulecolor{black}

\multicolumn{1}{!{\huxvb{0}}l!{\huxvb{0}}}{\huxtpad{4pt}\raggedright {\fontsize{9.5pt}{11.4pt}\selectfont }\huxbpad{4pt}} &
\multicolumn{1}{r!{\huxvb{0}}}{\huxtpad{4pt}\raggedleft {\fontsize{9.5pt}{11.4pt}\selectfont (1.726)~~~}\huxbpad{4pt}} &
\multicolumn{1}{r!{\huxvb{0}}}{\huxtpad{4pt}\raggedleft {\fontsize{9.5pt}{11.4pt}\selectfont (1.667)~~~}\huxbpad{4pt}} &
\multicolumn{1}{r!{\huxvb{0}}}{\huxtpad{4pt}\raggedleft {\fontsize{9.5pt}{11.4pt}\selectfont (1.505)~~~}\huxbpad{4pt}} \tabularnewline[-0.5pt]


\hhline{}
\arrayrulecolor{black}

\multicolumn{1}{!{\huxvb{0}}l!{\huxvb{0}}}{\huxtpad{4pt}\raggedright {\fontsize{9.5pt}{11.4pt}\selectfont sample\_tract24031703220}\huxbpad{4pt}} &
\multicolumn{1}{r!{\huxvb{0}}}{\huxtpad{4pt}\raggedleft {\fontsize{9.5pt}{11.4pt}\selectfont 19.442 ***}\huxbpad{4pt}} &
\multicolumn{1}{r!{\huxvb{0}}}{\huxtpad{4pt}\raggedleft {\fontsize{9.5pt}{11.4pt}\selectfont 19.063 ***}\huxbpad{4pt}} &
\multicolumn{1}{r!{\huxvb{0}}}{\huxtpad{4pt}\raggedleft {\fontsize{9.5pt}{11.4pt}\selectfont 20.886 ***}\huxbpad{4pt}} \tabularnewline[-0.5pt]


\hhline{}
\arrayrulecolor{black}

\multicolumn{1}{!{\huxvb{0}}l!{\huxvb{0}}}{\huxtpad{4pt}\raggedright {\fontsize{9.5pt}{11.4pt}\selectfont }\huxbpad{4pt}} &
\multicolumn{1}{r!{\huxvb{0}}}{\huxtpad{4pt}\raggedleft {\fontsize{9.5pt}{11.4pt}\selectfont (1.475)~~~}\huxbpad{4pt}} &
\multicolumn{1}{r!{\huxvb{0}}}{\huxtpad{4pt}\raggedleft {\fontsize{9.5pt}{11.4pt}\selectfont (1.837)~~~}\huxbpad{4pt}} &
\multicolumn{1}{r!{\huxvb{0}}}{\huxtpad{4pt}\raggedleft {\fontsize{9.5pt}{11.4pt}\selectfont (1.902)~~~}\huxbpad{4pt}} \tabularnewline[-0.5pt]


\hhline{}
\arrayrulecolor{black}

\multicolumn{1}{!{\huxvb{0}}l!{\huxvb{0}}}{\huxtpad{4pt}\raggedright {\fontsize{9.5pt}{11.4pt}\selectfont sample\_tract24031703221}\huxbpad{4pt}} &
\multicolumn{1}{r!{\huxvb{0}}}{\huxtpad{4pt}\raggedleft {\fontsize{9.5pt}{11.4pt}\selectfont 17.586 ***}\huxbpad{4pt}} &
\multicolumn{1}{r!{\huxvb{0}}}{\huxtpad{4pt}\raggedleft {\fontsize{9.5pt}{11.4pt}\selectfont 17.238 ***}\huxbpad{4pt}} &
\multicolumn{1}{r!{\huxvb{0}}}{\huxtpad{4pt}\raggedleft {\fontsize{9.5pt}{11.4pt}\selectfont 18.474 ***}\huxbpad{4pt}} \tabularnewline[-0.5pt]


\hhline{}
\arrayrulecolor{black}

\multicolumn{1}{!{\huxvb{0}}l!{\huxvb{0}}}{\huxtpad{4pt}\raggedright {\fontsize{9.5pt}{11.4pt}\selectfont }\huxbpad{4pt}} &
\multicolumn{1}{r!{\huxvb{0}}}{\huxtpad{4pt}\raggedleft {\fontsize{9.5pt}{11.4pt}\selectfont (1.132)~~~}\huxbpad{4pt}} &
\multicolumn{1}{r!{\huxvb{0}}}{\huxtpad{4pt}\raggedleft {\fontsize{9.5pt}{11.4pt}\selectfont (1.258)~~~}\huxbpad{4pt}} &
\multicolumn{1}{r!{\huxvb{0}}}{\huxtpad{4pt}\raggedleft {\fontsize{9.5pt}{11.4pt}\selectfont (1.263)~~~}\huxbpad{4pt}} \tabularnewline[-0.5pt]


\hhline{}
\arrayrulecolor{black}

\multicolumn{1}{!{\huxvb{0}}l!{\huxvb{0}}}{\huxtpad{4pt}\raggedright {\fontsize{9.5pt}{11.4pt}\selectfont sample\_tract24031703501}\huxbpad{4pt}} &
\multicolumn{1}{r!{\huxvb{0}}}{\huxtpad{4pt}\raggedleft {\fontsize{9.5pt}{11.4pt}\selectfont 19.771 ***}\huxbpad{4pt}} &
\multicolumn{1}{r!{\huxvb{0}}}{\huxtpad{4pt}\raggedleft {\fontsize{9.5pt}{11.4pt}\selectfont 18.870 ***}\huxbpad{4pt}} &
\multicolumn{1}{r!{\huxvb{0}}}{\huxtpad{4pt}\raggedleft {\fontsize{9.5pt}{11.4pt}\selectfont 20.491 ***}\huxbpad{4pt}} \tabularnewline[-0.5pt]


\hhline{}
\arrayrulecolor{black}

\multicolumn{1}{!{\huxvb{0}}l!{\huxvb{0}}}{\huxtpad{4pt}\raggedright {\fontsize{9.5pt}{11.4pt}\selectfont }\huxbpad{4pt}} &
\multicolumn{1}{r!{\huxvb{0}}}{\huxtpad{4pt}\raggedleft {\fontsize{9.5pt}{11.4pt}\selectfont (1.097)~~~}\huxbpad{4pt}} &
\multicolumn{1}{r!{\huxvb{0}}}{\huxtpad{4pt}\raggedleft {\fontsize{9.5pt}{11.4pt}\selectfont (1.233)~~~}\huxbpad{4pt}} &
\multicolumn{1}{r!{\huxvb{0}}}{\huxtpad{4pt}\raggedleft {\fontsize{9.5pt}{11.4pt}\selectfont (1.303)~~~}\huxbpad{4pt}} \tabularnewline[-0.5pt]


\hhline{}
\arrayrulecolor{black}

\multicolumn{1}{!{\huxvb{0}}l!{\huxvb{0}}}{\huxtpad{4pt}\raggedright {\fontsize{9.5pt}{11.4pt}\selectfont sample\_tract24031703601}\huxbpad{4pt}} &
\multicolumn{1}{r!{\huxvb{0}}}{\huxtpad{4pt}\raggedleft {\fontsize{9.5pt}{11.4pt}\selectfont 19.814 ***}\huxbpad{4pt}} &
\multicolumn{1}{r!{\huxvb{0}}}{\huxtpad{4pt}\raggedleft {\fontsize{9.5pt}{11.4pt}\selectfont 19.290 ***}\huxbpad{4pt}} &
\multicolumn{1}{r!{\huxvb{0}}}{\huxtpad{4pt}\raggedleft {\fontsize{9.5pt}{11.4pt}\selectfont 21.120 ***}\huxbpad{4pt}} \tabularnewline[-0.5pt]


\hhline{}
\arrayrulecolor{black}

\multicolumn{1}{!{\huxvb{0}}l!{\huxvb{0}}}{\huxtpad{4pt}\raggedright {\fontsize{9.5pt}{11.4pt}\selectfont }\huxbpad{4pt}} &
\multicolumn{1}{r!{\huxvb{0}}}{\huxtpad{4pt}\raggedleft {\fontsize{9.5pt}{11.4pt}\selectfont (1.251)~~~}\huxbpad{4pt}} &
\multicolumn{1}{r!{\huxvb{0}}}{\huxtpad{4pt}\raggedleft {\fontsize{9.5pt}{11.4pt}\selectfont (1.313)~~~}\huxbpad{4pt}} &
\multicolumn{1}{r!{\huxvb{0}}}{\huxtpad{4pt}\raggedleft {\fontsize{9.5pt}{11.4pt}\selectfont (1.247)~~~}\huxbpad{4pt}} \tabularnewline[-0.5pt]


\hhline{}
\arrayrulecolor{black}

\multicolumn{1}{!{\huxvb{0}}l!{\huxvb{0}}}{\huxtpad{4pt}\raggedright {\fontsize{9.5pt}{11.4pt}\selectfont sample\_tract24031703902}\huxbpad{4pt}} &
\multicolumn{1}{r!{\huxvb{0}}}{\huxtpad{4pt}\raggedleft {\fontsize{9.5pt}{11.4pt}\selectfont 20.143 ***}\huxbpad{4pt}} &
\multicolumn{1}{r!{\huxvb{0}}}{\huxtpad{4pt}\raggedleft {\fontsize{9.5pt}{11.4pt}\selectfont 19.784 ***}\huxbpad{4pt}} &
\multicolumn{1}{r!{\huxvb{0}}}{\huxtpad{4pt}\raggedleft {\fontsize{9.5pt}{11.4pt}\selectfont 21.314 ***}\huxbpad{4pt}} \tabularnewline[-0.5pt]


\hhline{}
\arrayrulecolor{black}

\multicolumn{1}{!{\huxvb{0}}l!{\huxvb{0}}}{\huxtpad{4pt}\raggedright {\fontsize{9.5pt}{11.4pt}\selectfont }\huxbpad{4pt}} &
\multicolumn{1}{r!{\huxvb{0}}}{\huxtpad{4pt}\raggedleft {\fontsize{9.5pt}{11.4pt}\selectfont (2.261)~~~}\huxbpad{4pt}} &
\multicolumn{1}{r!{\huxvb{0}}}{\huxtpad{4pt}\raggedleft {\fontsize{9.5pt}{11.4pt}\selectfont (2.488)~~~}\huxbpad{4pt}} &
\multicolumn{1}{r!{\huxvb{0}}}{\huxtpad{4pt}\raggedleft {\fontsize{9.5pt}{11.4pt}\selectfont (1.951)~~~}\huxbpad{4pt}} \tabularnewline[-0.5pt]


\hhline{}
\arrayrulecolor{black}

\multicolumn{1}{!{\huxvb{0}}l!{\huxvb{0}}}{\huxtpad{4pt}\raggedright {\fontsize{9.5pt}{11.4pt}\selectfont sample\_tract24031704000}\huxbpad{4pt}} &
\multicolumn{1}{r!{\huxvb{0}}}{\huxtpad{4pt}\raggedleft {\fontsize{9.5pt}{11.4pt}\selectfont 18.395 ***}\huxbpad{4pt}} &
\multicolumn{1}{r!{\huxvb{0}}}{\huxtpad{4pt}\raggedleft {\fontsize{9.5pt}{11.4pt}\selectfont 18.197 ***}\huxbpad{4pt}} &
\multicolumn{1}{r!{\huxvb{0}}}{\huxtpad{4pt}\raggedleft {\fontsize{9.5pt}{11.4pt}\selectfont 19.159 ***}\huxbpad{4pt}} \tabularnewline[-0.5pt]


\hhline{}
\arrayrulecolor{black}

\multicolumn{1}{!{\huxvb{0}}l!{\huxvb{0}}}{\huxtpad{4pt}\raggedright {\fontsize{9.5pt}{11.4pt}\selectfont }\huxbpad{4pt}} &
\multicolumn{1}{r!{\huxvb{0}}}{\huxtpad{4pt}\raggedleft {\fontsize{9.5pt}{11.4pt}\selectfont (1.511)~~~}\huxbpad{4pt}} &
\multicolumn{1}{r!{\huxvb{0}}}{\huxtpad{4pt}\raggedleft {\fontsize{9.5pt}{11.4pt}\selectfont (1.632)~~~}\huxbpad{4pt}} &
\multicolumn{1}{r!{\huxvb{0}}}{\huxtpad{4pt}\raggedleft {\fontsize{9.5pt}{11.4pt}\selectfont (1.554)~~~}\huxbpad{4pt}} \tabularnewline[-0.5pt]


\hhline{}
\arrayrulecolor{black}

\multicolumn{1}{!{\huxvb{0}}l!{\huxvb{0}}}{\huxtpad{4pt}\raggedright {\fontsize{9.5pt}{11.4pt}\selectfont sample\_tract24031706012}\huxbpad{4pt}} &
\multicolumn{1}{r!{\huxvb{0}}}{\huxtpad{4pt}\raggedleft {\fontsize{9.5pt}{11.4pt}\selectfont 18.856 ***}\huxbpad{4pt}} &
\multicolumn{1}{r!{\huxvb{0}}}{\huxtpad{4pt}\raggedleft {\fontsize{9.5pt}{11.4pt}\selectfont 18.498 ***}\huxbpad{4pt}} &
\multicolumn{1}{r!{\huxvb{0}}}{\huxtpad{4pt}\raggedleft {\fontsize{9.5pt}{11.4pt}\selectfont 19.582 ***}\huxbpad{4pt}} \tabularnewline[-0.5pt]


\hhline{}
\arrayrulecolor{black}

\multicolumn{1}{!{\huxvb{0}}l!{\huxvb{0}}}{\huxtpad{4pt}\raggedright {\fontsize{9.5pt}{11.4pt}\selectfont }\huxbpad{4pt}} &
\multicolumn{1}{r!{\huxvb{0}}}{\huxtpad{4pt}\raggedleft {\fontsize{9.5pt}{11.4pt}\selectfont (1.364)~~~}\huxbpad{4pt}} &
\multicolumn{1}{r!{\huxvb{0}}}{\huxtpad{4pt}\raggedleft {\fontsize{9.5pt}{11.4pt}\selectfont (1.793)~~~}\huxbpad{4pt}} &
\multicolumn{1}{r!{\huxvb{0}}}{\huxtpad{4pt}\raggedleft {\fontsize{9.5pt}{11.4pt}\selectfont (1.339)~~~}\huxbpad{4pt}} \tabularnewline[-0.5pt]


\hhline{}
\arrayrulecolor{black}

\multicolumn{1}{!{\huxvb{0}}l!{\huxvb{0}}}{\huxtpad{4pt}\raggedright {\fontsize{9.5pt}{11.4pt}\selectfont sample\_tract24033800103}\huxbpad{4pt}} &
\multicolumn{1}{r!{\huxvb{0}}}{\huxtpad{4pt}\raggedleft {\fontsize{9.5pt}{11.4pt}\selectfont 20.047 ***}\huxbpad{4pt}} &
\multicolumn{1}{r!{\huxvb{0}}}{\huxtpad{4pt}\raggedleft {\fontsize{9.5pt}{11.4pt}\selectfont 19.815 ***}\huxbpad{4pt}} &
\multicolumn{1}{r!{\huxvb{0}}}{\huxtpad{4pt}\raggedleft {\fontsize{9.5pt}{11.4pt}\selectfont 21.379 ***}\huxbpad{4pt}} \tabularnewline[-0.5pt]


\hhline{}
\arrayrulecolor{black}

\multicolumn{1}{!{\huxvb{0}}l!{\huxvb{0}}}{\huxtpad{4pt}\raggedright {\fontsize{9.5pt}{11.4pt}\selectfont }\huxbpad{4pt}} &
\multicolumn{1}{r!{\huxvb{0}}}{\huxtpad{4pt}\raggedleft {\fontsize{9.5pt}{11.4pt}\selectfont (1.277)~~~}\huxbpad{4pt}} &
\multicolumn{1}{r!{\huxvb{0}}}{\huxtpad{4pt}\raggedleft {\fontsize{9.5pt}{11.4pt}\selectfont (1.423)~~~}\huxbpad{4pt}} &
\multicolumn{1}{r!{\huxvb{0}}}{\huxtpad{4pt}\raggedleft {\fontsize{9.5pt}{11.4pt}\selectfont (1.245)~~~}\huxbpad{4pt}} \tabularnewline[-0.5pt]


\hhline{}
\arrayrulecolor{black}

\multicolumn{1}{!{\huxvb{0}}l!{\huxvb{0}}}{\huxtpad{4pt}\raggedright {\fontsize{9.5pt}{11.4pt}\selectfont sample\_tract24033805909}\huxbpad{4pt}} &
\multicolumn{1}{r!{\huxvb{0}}}{\huxtpad{4pt}\raggedleft {\fontsize{9.5pt}{11.4pt}\selectfont 17.692 ***}\huxbpad{4pt}} &
\multicolumn{1}{r!{\huxvb{0}}}{\huxtpad{4pt}\raggedleft {\fontsize{9.5pt}{11.4pt}\selectfont 17.648 ***}\huxbpad{4pt}} &
\multicolumn{1}{r!{\huxvb{0}}}{\huxtpad{4pt}\raggedleft {\fontsize{9.5pt}{11.4pt}\selectfont 19.292 ***}\huxbpad{4pt}} \tabularnewline[-0.5pt]


\hhline{}
\arrayrulecolor{black}

\multicolumn{1}{!{\huxvb{0}}l!{\huxvb{0}}}{\huxtpad{4pt}\raggedright {\fontsize{9.5pt}{11.4pt}\selectfont }\huxbpad{4pt}} &
\multicolumn{1}{r!{\huxvb{0}}}{\huxtpad{4pt}\raggedleft {\fontsize{9.5pt}{11.4pt}\selectfont (1.617)~~~}\huxbpad{4pt}} &
\multicolumn{1}{r!{\huxvb{0}}}{\huxtpad{4pt}\raggedleft {\fontsize{9.5pt}{11.4pt}\selectfont (1.608)~~~}\huxbpad{4pt}} &
\multicolumn{1}{r!{\huxvb{0}}}{\huxtpad{4pt}\raggedleft {\fontsize{9.5pt}{11.4pt}\selectfont (1.332)~~~}\huxbpad{4pt}} \tabularnewline[-0.5pt]


\hhline{}
\arrayrulecolor{black}

\multicolumn{1}{!{\huxvb{0}}l!{\huxvb{0}}}{\huxtpad{4pt}\raggedright {\fontsize{9.5pt}{11.4pt}\selectfont sample\_tract24033806706}\huxbpad{4pt}} &
\multicolumn{1}{r!{\huxvb{0}}}{\huxtpad{4pt}\raggedleft {\fontsize{9.5pt}{11.4pt}\selectfont 17.782 ***}\huxbpad{4pt}} &
\multicolumn{1}{r!{\huxvb{0}}}{\huxtpad{4pt}\raggedleft {\fontsize{9.5pt}{11.4pt}\selectfont 17.238 ***}\huxbpad{4pt}} &
\multicolumn{1}{r!{\huxvb{0}}}{\huxtpad{4pt}\raggedleft {\fontsize{9.5pt}{11.4pt}\selectfont 18.929 ***}\huxbpad{4pt}} \tabularnewline[-0.5pt]


\hhline{}
\arrayrulecolor{black}

\multicolumn{1}{!{\huxvb{0}}l!{\huxvb{0}}}{\huxtpad{4pt}\raggedright {\fontsize{9.5pt}{11.4pt}\selectfont }\huxbpad{4pt}} &
\multicolumn{1}{r!{\huxvb{0}}}{\huxtpad{4pt}\raggedleft {\fontsize{9.5pt}{11.4pt}\selectfont (1.272)~~~}\huxbpad{4pt}} &
\multicolumn{1}{r!{\huxvb{0}}}{\huxtpad{4pt}\raggedleft {\fontsize{9.5pt}{11.4pt}\selectfont (1.278)~~~}\huxbpad{4pt}} &
\multicolumn{1}{r!{\huxvb{0}}}{\huxtpad{4pt}\raggedleft {\fontsize{9.5pt}{11.4pt}\selectfont (1.266)~~~}\huxbpad{4pt}} \tabularnewline[-0.5pt]


\hhline{}
\arrayrulecolor{black}

\multicolumn{1}{!{\huxvb{0}}l!{\huxvb{0}}}{\huxtpad{4pt}\raggedright {\fontsize{9.5pt}{11.4pt}\selectfont sample\_tract24033806900}\huxbpad{4pt}} &
\multicolumn{1}{r!{\huxvb{0}}}{\huxtpad{4pt}\raggedleft {\fontsize{9.5pt}{11.4pt}\selectfont 19.346 ***}\huxbpad{4pt}} &
\multicolumn{1}{r!{\huxvb{0}}}{\huxtpad{4pt}\raggedleft {\fontsize{9.5pt}{11.4pt}\selectfont 19.344 ***}\huxbpad{4pt}} &
\multicolumn{1}{r!{\huxvb{0}}}{\huxtpad{4pt}\raggedleft {\fontsize{9.5pt}{11.4pt}\selectfont 21.753 ***}\huxbpad{4pt}} \tabularnewline[-0.5pt]


\hhline{}
\arrayrulecolor{black}

\multicolumn{1}{!{\huxvb{0}}l!{\huxvb{0}}}{\huxtpad{4pt}\raggedright {\fontsize{9.5pt}{11.4pt}\selectfont }\huxbpad{4pt}} &
\multicolumn{1}{r!{\huxvb{0}}}{\huxtpad{4pt}\raggedleft {\fontsize{9.5pt}{11.4pt}\selectfont (1.400)~~~}\huxbpad{4pt}} &
\multicolumn{1}{r!{\huxvb{0}}}{\huxtpad{4pt}\raggedleft {\fontsize{9.5pt}{11.4pt}\selectfont (1.561)~~~}\huxbpad{4pt}} &
\multicolumn{1}{r!{\huxvb{0}}}{\huxtpad{4pt}\raggedleft {\fontsize{9.5pt}{11.4pt}\selectfont (1.417)~~~}\huxbpad{4pt}} \tabularnewline[-0.5pt]


\hhline{}
\arrayrulecolor{black}

\multicolumn{1}{!{\huxvb{0}}l!{\huxvb{0}}}{\huxtpad{4pt}\raggedright {\fontsize{9.5pt}{11.4pt}\selectfont sample\_tract24033807301}\huxbpad{4pt}} &
\multicolumn{1}{r!{\huxvb{0}}}{\huxtpad{4pt}\raggedleft {\fontsize{9.5pt}{11.4pt}\selectfont 17.983 ***}\huxbpad{4pt}} &
\multicolumn{1}{r!{\huxvb{0}}}{\huxtpad{4pt}\raggedleft {\fontsize{9.5pt}{11.4pt}\selectfont 17.960 ***}\huxbpad{4pt}} &
\multicolumn{1}{r!{\huxvb{0}}}{\huxtpad{4pt}\raggedleft {\fontsize{9.5pt}{11.4pt}\selectfont 19.460 ***}\huxbpad{4pt}} \tabularnewline[-0.5pt]


\hhline{}
\arrayrulecolor{black}

\multicolumn{1}{!{\huxvb{0}}l!{\huxvb{0}}}{\huxtpad{4pt}\raggedright {\fontsize{9.5pt}{11.4pt}\selectfont }\huxbpad{4pt}} &
\multicolumn{1}{r!{\huxvb{0}}}{\huxtpad{4pt}\raggedleft {\fontsize{9.5pt}{11.4pt}\selectfont (1.482)~~~}\huxbpad{4pt}} &
\multicolumn{1}{r!{\huxvb{0}}}{\huxtpad{4pt}\raggedleft {\fontsize{9.5pt}{11.4pt}\selectfont (1.596)~~~}\huxbpad{4pt}} &
\multicolumn{1}{r!{\huxvb{0}}}{\huxtpad{4pt}\raggedleft {\fontsize{9.5pt}{11.4pt}\selectfont (1.287)~~~}\huxbpad{4pt}} \tabularnewline[-0.5pt]


\hhline{}
\arrayrulecolor{black}

\multicolumn{1}{!{\huxvb{0}}l!{\huxvb{0}}}{\huxtpad{4pt}\raggedright {\fontsize{9.5pt}{11.4pt}\selectfont sample\_tract24033807305}\huxbpad{4pt}} &
\multicolumn{1}{r!{\huxvb{0}}}{\huxtpad{4pt}\raggedleft {\fontsize{9.5pt}{11.4pt}\selectfont 19.140 ***}\huxbpad{4pt}} &
\multicolumn{1}{r!{\huxvb{0}}}{\huxtpad{4pt}\raggedleft {\fontsize{9.5pt}{11.4pt}\selectfont 18.696 ***}\huxbpad{4pt}} &
\multicolumn{1}{r!{\huxvb{0}}}{\huxtpad{4pt}\raggedleft {\fontsize{9.5pt}{11.4pt}\selectfont 20.658 ***}\huxbpad{4pt}} \tabularnewline[-0.5pt]


\hhline{}
\arrayrulecolor{black}

\multicolumn{1}{!{\huxvb{0}}l!{\huxvb{0}}}{\huxtpad{4pt}\raggedright {\fontsize{9.5pt}{11.4pt}\selectfont }\huxbpad{4pt}} &
\multicolumn{1}{r!{\huxvb{0}}}{\huxtpad{4pt}\raggedleft {\fontsize{9.5pt}{11.4pt}\selectfont (1.454)~~~}\huxbpad{4pt}} &
\multicolumn{1}{r!{\huxvb{0}}}{\huxtpad{4pt}\raggedleft {\fontsize{9.5pt}{11.4pt}\selectfont (1.393)~~~}\huxbpad{4pt}} &
\multicolumn{1}{r!{\huxvb{0}}}{\huxtpad{4pt}\raggedleft {\fontsize{9.5pt}{11.4pt}\selectfont (1.283)~~~}\huxbpad{4pt}} \tabularnewline[-0.5pt]


\hhline{}
\arrayrulecolor{black}

\multicolumn{1}{!{\huxvb{0}}l!{\huxvb{0}}}{\huxtpad{4pt}\raggedright {\fontsize{9.5pt}{11.4pt}\selectfont sample\_tract24033807404}\huxbpad{4pt}} &
\multicolumn{1}{r!{\huxvb{0}}}{\huxtpad{4pt}\raggedleft {\fontsize{9.5pt}{11.4pt}\selectfont 21.266 ***}\huxbpad{4pt}} &
\multicolumn{1}{r!{\huxvb{0}}}{\huxtpad{4pt}\raggedleft {\fontsize{9.5pt}{11.4pt}\selectfont 20.872 ***}\huxbpad{4pt}} &
\multicolumn{1}{r!{\huxvb{0}}}{\huxtpad{4pt}\raggedleft {\fontsize{9.5pt}{11.4pt}\selectfont 23.149 ***}\huxbpad{4pt}} \tabularnewline[-0.5pt]


\hhline{}
\arrayrulecolor{black}

\multicolumn{1}{!{\huxvb{0}}l!{\huxvb{0}}}{\huxtpad{4pt}\raggedright {\fontsize{9.5pt}{11.4pt}\selectfont }\huxbpad{4pt}} &
\multicolumn{1}{r!{\huxvb{0}}}{\huxtpad{4pt}\raggedleft {\fontsize{9.5pt}{11.4pt}\selectfont (1.644)~~~}\huxbpad{4pt}} &
\multicolumn{1}{r!{\huxvb{0}}}{\huxtpad{4pt}\raggedleft {\fontsize{9.5pt}{11.4pt}\selectfont (1.739)~~~}\huxbpad{4pt}} &
\multicolumn{1}{r!{\huxvb{0}}}{\huxtpad{4pt}\raggedleft {\fontsize{9.5pt}{11.4pt}\selectfont (1.776)~~~}\huxbpad{4pt}} \tabularnewline[-0.5pt]


\hhline{}
\arrayrulecolor{black}

\multicolumn{1}{!{\huxvb{0}}l!{\huxvb{0}}}{\huxtpad{4pt}\raggedright {\fontsize{9.5pt}{11.4pt}\selectfont sample\_tract24033807405}\huxbpad{4pt}} &
\multicolumn{1}{r!{\huxvb{0}}}{\huxtpad{4pt}\raggedleft {\fontsize{9.5pt}{11.4pt}\selectfont 20.781 ***}\huxbpad{4pt}} &
\multicolumn{1}{r!{\huxvb{0}}}{\huxtpad{4pt}\raggedleft {\fontsize{9.5pt}{11.4pt}\selectfont 20.632 ***}\huxbpad{4pt}} &
\multicolumn{1}{r!{\huxvb{0}}}{\huxtpad{4pt}\raggedleft {\fontsize{9.5pt}{11.4pt}\selectfont 22.259 ***}\huxbpad{4pt}} \tabularnewline[-0.5pt]


\hhline{}
\arrayrulecolor{black}

\multicolumn{1}{!{\huxvb{0}}l!{\huxvb{0}}}{\huxtpad{4pt}\raggedright {\fontsize{9.5pt}{11.4pt}\selectfont }\huxbpad{4pt}} &
\multicolumn{1}{r!{\huxvb{0}}}{\huxtpad{4pt}\raggedleft {\fontsize{9.5pt}{11.4pt}\selectfont (1.516)~~~}\huxbpad{4pt}} &
\multicolumn{1}{r!{\huxvb{0}}}{\huxtpad{4pt}\raggedleft {\fontsize{9.5pt}{11.4pt}\selectfont (1.598)~~~}\huxbpad{4pt}} &
\multicolumn{1}{r!{\huxvb{0}}}{\huxtpad{4pt}\raggedleft {\fontsize{9.5pt}{11.4pt}\selectfont (1.224)~~~}\huxbpad{4pt}} \tabularnewline[-0.5pt]


\hhline{}
\arrayrulecolor{black}

\multicolumn{1}{!{\huxvb{0}}l!{\huxvb{0}}}{\huxtpad{4pt}\raggedright {\fontsize{9.5pt}{11.4pt}\selectfont sample\_tract24033807407}\huxbpad{4pt}} &
\multicolumn{1}{r!{\huxvb{0}}}{\huxtpad{4pt}\raggedleft {\fontsize{9.5pt}{11.4pt}\selectfont 19.665 ***}\huxbpad{4pt}} &
\multicolumn{1}{r!{\huxvb{0}}}{\huxtpad{4pt}\raggedleft {\fontsize{9.5pt}{11.4pt}\selectfont 19.176 ***}\huxbpad{4pt}} &
\multicolumn{1}{r!{\huxvb{0}}}{\huxtpad{4pt}\raggedleft {\fontsize{9.5pt}{11.4pt}\selectfont 20.701 ***}\huxbpad{4pt}} \tabularnewline[-0.5pt]


\hhline{}
\arrayrulecolor{black}

\multicolumn{1}{!{\huxvb{0}}l!{\huxvb{0}}}{\huxtpad{4pt}\raggedright {\fontsize{9.5pt}{11.4pt}\selectfont }\huxbpad{4pt}} &
\multicolumn{1}{r!{\huxvb{0}}}{\huxtpad{4pt}\raggedleft {\fontsize{9.5pt}{11.4pt}\selectfont (1.311)~~~}\huxbpad{4pt}} &
\multicolumn{1}{r!{\huxvb{0}}}{\huxtpad{4pt}\raggedleft {\fontsize{9.5pt}{11.4pt}\selectfont (1.497)~~~}\huxbpad{4pt}} &
\multicolumn{1}{r!{\huxvb{0}}}{\huxtpad{4pt}\raggedleft {\fontsize{9.5pt}{11.4pt}\selectfont (1.809)~~~}\huxbpad{4pt}} \tabularnewline[-0.5pt]


\hhline{}
\arrayrulecolor{black}

\multicolumn{1}{!{\huxvb{0}}l!{\huxvb{0}}}{\huxtpad{4pt}\raggedright {\fontsize{9.5pt}{11.4pt}\selectfont sample\_tract51013102500}\huxbpad{4pt}} &
\multicolumn{1}{r!{\huxvb{0}}}{\huxtpad{4pt}\raggedleft {\fontsize{9.5pt}{11.4pt}\selectfont 39.336 ***}\huxbpad{4pt}} &
\multicolumn{1}{r!{\huxvb{0}}}{\huxtpad{4pt}\raggedleft {\fontsize{9.5pt}{11.4pt}\selectfont 39.152 ***}\huxbpad{4pt}} &
\multicolumn{1}{r!{\huxvb{0}}}{\huxtpad{4pt}\raggedleft {\fontsize{9.5pt}{11.4pt}\selectfont 41.308 ***}\huxbpad{4pt}} \tabularnewline[-0.5pt]


\hhline{}
\arrayrulecolor{black}

\multicolumn{1}{!{\huxvb{0}}l!{\huxvb{0}}}{\huxtpad{4pt}\raggedright {\fontsize{9.5pt}{11.4pt}\selectfont }\huxbpad{4pt}} &
\multicolumn{1}{r!{\huxvb{0}}}{\huxtpad{4pt}\raggedleft {\fontsize{9.5pt}{11.4pt}\selectfont (1.052)~~~}\huxbpad{4pt}} &
\multicolumn{1}{r!{\huxvb{0}}}{\huxtpad{4pt}\raggedleft {\fontsize{9.5pt}{11.4pt}\selectfont (1.151)~~~}\huxbpad{4pt}} &
\multicolumn{1}{r!{\huxvb{0}}}{\huxtpad{4pt}\raggedleft {\fontsize{9.5pt}{11.4pt}\selectfont (1.094)~~~}\huxbpad{4pt}} \tabularnewline[-0.5pt]


\hhline{}
\arrayrulecolor{black}

\multicolumn{1}{!{\huxvb{0}}l!{\huxvb{0}}}{\huxtpad{4pt}\raggedright {\fontsize{9.5pt}{11.4pt}\selectfont sample\_tract51013102701}\huxbpad{4pt}} &
\multicolumn{1}{r!{\huxvb{0}}}{\huxtpad{4pt}\raggedleft {\fontsize{9.5pt}{11.4pt}\selectfont 18.695 ***}\huxbpad{4pt}} &
\multicolumn{1}{r!{\huxvb{0}}}{\huxtpad{4pt}\raggedleft {\fontsize{9.5pt}{11.4pt}\selectfont 18.145 ***}\huxbpad{4pt}} &
\multicolumn{1}{r!{\huxvb{0}}}{\huxtpad{4pt}\raggedleft {\fontsize{9.5pt}{11.4pt}\selectfont 19.419 ***}\huxbpad{4pt}} \tabularnewline[-0.5pt]


\hhline{}
\arrayrulecolor{black}

\multicolumn{1}{!{\huxvb{0}}l!{\huxvb{0}}}{\huxtpad{4pt}\raggedright {\fontsize{9.5pt}{11.4pt}\selectfont }\huxbpad{4pt}} &
\multicolumn{1}{r!{\huxvb{0}}}{\huxtpad{4pt}\raggedleft {\fontsize{9.5pt}{11.4pt}\selectfont (1.644)~~~}\huxbpad{4pt}} &
\multicolumn{1}{r!{\huxvb{0}}}{\huxtpad{4pt}\raggedleft {\fontsize{9.5pt}{11.4pt}\selectfont (1.767)~~~}\huxbpad{4pt}} &
\multicolumn{1}{r!{\huxvb{0}}}{\huxtpad{4pt}\raggedleft {\fontsize{9.5pt}{11.4pt}\selectfont (1.465)~~~}\huxbpad{4pt}} \tabularnewline[-0.5pt]


\hhline{}
\arrayrulecolor{black}

\multicolumn{1}{!{\huxvb{0}}l!{\huxvb{0}}}{\huxtpad{4pt}\raggedright {\fontsize{9.5pt}{11.4pt}\selectfont sample\_tract51013102801}\huxbpad{4pt}} &
\multicolumn{1}{r!{\huxvb{0}}}{\huxtpad{4pt}\raggedleft {\fontsize{9.5pt}{11.4pt}\selectfont 19.094 ***}\huxbpad{4pt}} &
\multicolumn{1}{r!{\huxvb{0}}}{\huxtpad{4pt}\raggedleft {\fontsize{9.5pt}{11.4pt}\selectfont 18.891 ***}\huxbpad{4pt}} &
\multicolumn{1}{r!{\huxvb{0}}}{\huxtpad{4pt}\raggedleft {\fontsize{9.5pt}{11.4pt}\selectfont 20.285 ***}\huxbpad{4pt}} \tabularnewline[-0.5pt]


\hhline{}
\arrayrulecolor{black}

\multicolumn{1}{!{\huxvb{0}}l!{\huxvb{0}}}{\huxtpad{4pt}\raggedright {\fontsize{9.5pt}{11.4pt}\selectfont }\huxbpad{4pt}} &
\multicolumn{1}{r!{\huxvb{0}}}{\huxtpad{4pt}\raggedleft {\fontsize{9.5pt}{11.4pt}\selectfont (1.312)~~~}\huxbpad{4pt}} &
\multicolumn{1}{r!{\huxvb{0}}}{\huxtpad{4pt}\raggedleft {\fontsize{9.5pt}{11.4pt}\selectfont (1.556)~~~}\huxbpad{4pt}} &
\multicolumn{1}{r!{\huxvb{0}}}{\huxtpad{4pt}\raggedleft {\fontsize{9.5pt}{11.4pt}\selectfont (1.273)~~~}\huxbpad{4pt}} \tabularnewline[-0.5pt]


\hhline{}
\arrayrulecolor{black}

\multicolumn{1}{!{\huxvb{0}}l!{\huxvb{0}}}{\huxtpad{4pt}\raggedright {\fontsize{9.5pt}{11.4pt}\selectfont sample\_tract51013103200}\huxbpad{4pt}} &
\multicolumn{1}{r!{\huxvb{0}}}{\huxtpad{4pt}\raggedleft {\fontsize{9.5pt}{11.4pt}\selectfont 20.909 ***}\huxbpad{4pt}} &
\multicolumn{1}{r!{\huxvb{0}}}{\huxtpad{4pt}\raggedleft {\fontsize{9.5pt}{11.4pt}\selectfont 20.907 ***}\huxbpad{4pt}} &
\multicolumn{1}{r!{\huxvb{0}}}{\huxtpad{4pt}\raggedleft {\fontsize{9.5pt}{11.4pt}\selectfont 22.241 ***}\huxbpad{4pt}} \tabularnewline[-0.5pt]


\hhline{}
\arrayrulecolor{black}

\multicolumn{1}{!{\huxvb{0}}l!{\huxvb{0}}}{\huxtpad{4pt}\raggedright {\fontsize{9.5pt}{11.4pt}\selectfont }\huxbpad{4pt}} &
\multicolumn{1}{r!{\huxvb{0}}}{\huxtpad{4pt}\raggedleft {\fontsize{9.5pt}{11.4pt}\selectfont (1.417)~~~}\huxbpad{4pt}} &
\multicolumn{1}{r!{\huxvb{0}}}{\huxtpad{4pt}\raggedleft {\fontsize{9.5pt}{11.4pt}\selectfont (1.419)~~~}\huxbpad{4pt}} &
\multicolumn{1}{r!{\huxvb{0}}}{\huxtpad{4pt}\raggedleft {\fontsize{9.5pt}{11.4pt}\selectfont (1.404)~~~}\huxbpad{4pt}} \tabularnewline[-0.5pt]


\hhline{}
\arrayrulecolor{black}

\multicolumn{1}{!{\huxvb{0}}l!{\huxvb{0}}}{\huxtpad{4pt}\raggedright {\fontsize{9.5pt}{11.4pt}\selectfont sample\_tract51059420100}\huxbpad{4pt}} &
\multicolumn{1}{r!{\huxvb{0}}}{\huxtpad{4pt}\raggedleft {\fontsize{9.5pt}{11.4pt}\selectfont 38.670 ***}\huxbpad{4pt}} &
\multicolumn{1}{r!{\huxvb{0}}}{\huxtpad{4pt}\raggedleft {\fontsize{9.5pt}{11.4pt}\selectfont 38.519 ***}\huxbpad{4pt}} &
\multicolumn{1}{r!{\huxvb{0}}}{\huxtpad{4pt}\raggedleft {\fontsize{9.5pt}{11.4pt}\selectfont 41.000 ***}\huxbpad{4pt}} \tabularnewline[-0.5pt]


\hhline{}
\arrayrulecolor{black}

\multicolumn{1}{!{\huxvb{0}}l!{\huxvb{0}}}{\huxtpad{4pt}\raggedright {\fontsize{9.5pt}{11.4pt}\selectfont }\huxbpad{4pt}} &
\multicolumn{1}{r!{\huxvb{0}}}{\huxtpad{4pt}\raggedleft {\fontsize{9.5pt}{11.4pt}\selectfont (1.289)~~~}\huxbpad{4pt}} &
\multicolumn{1}{r!{\huxvb{0}}}{\huxtpad{4pt}\raggedleft {\fontsize{9.5pt}{11.4pt}\selectfont (1.386)~~~}\huxbpad{4pt}} &
\multicolumn{1}{r!{\huxvb{0}}}{\huxtpad{4pt}\raggedleft {\fontsize{9.5pt}{11.4pt}\selectfont (1.319)~~~}\huxbpad{4pt}} \tabularnewline[-0.5pt]


\hhline{}
\arrayrulecolor{black}

\multicolumn{1}{!{\huxvb{0}}l!{\huxvb{0}}}{\huxtpad{4pt}\raggedright {\fontsize{9.5pt}{11.4pt}\selectfont sample\_tract51059420201}\huxbpad{4pt}} &
\multicolumn{1}{r!{\huxvb{0}}}{\huxtpad{4pt}\raggedleft {\fontsize{9.5pt}{11.4pt}\selectfont 20.332 ***}\huxbpad{4pt}} &
\multicolumn{1}{r!{\huxvb{0}}}{\huxtpad{4pt}\raggedleft {\fontsize{9.5pt}{11.4pt}\selectfont 20.087 ***}\huxbpad{4pt}} &
\multicolumn{1}{r!{\huxvb{0}}}{\huxtpad{4pt}\raggedleft {\fontsize{9.5pt}{11.4pt}\selectfont 21.898 ***}\huxbpad{4pt}} \tabularnewline[-0.5pt]


\hhline{}
\arrayrulecolor{black}

\multicolumn{1}{!{\huxvb{0}}l!{\huxvb{0}}}{\huxtpad{4pt}\raggedright {\fontsize{9.5pt}{11.4pt}\selectfont }\huxbpad{4pt}} &
\multicolumn{1}{r!{\huxvb{0}}}{\huxtpad{4pt}\raggedleft {\fontsize{9.5pt}{11.4pt}\selectfont (1.626)~~~}\huxbpad{4pt}} &
\multicolumn{1}{r!{\huxvb{0}}}{\huxtpad{4pt}\raggedleft {\fontsize{9.5pt}{11.4pt}\selectfont (1.778)~~~}\huxbpad{4pt}} &
\multicolumn{1}{r!{\huxvb{0}}}{\huxtpad{4pt}\raggedleft {\fontsize{9.5pt}{11.4pt}\selectfont (1.699)~~~}\huxbpad{4pt}} \tabularnewline[-0.5pt]


\hhline{}
\arrayrulecolor{black}

\multicolumn{1}{!{\huxvb{0}}l!{\huxvb{0}}}{\huxtpad{4pt}\raggedright {\fontsize{9.5pt}{11.4pt}\selectfont sample\_tract51059420400}\huxbpad{4pt}} &
\multicolumn{1}{r!{\huxvb{0}}}{\huxtpad{4pt}\raggedleft {\fontsize{9.5pt}{11.4pt}\selectfont 20.185 ***}\huxbpad{4pt}} &
\multicolumn{1}{r!{\huxvb{0}}}{\huxtpad{4pt}\raggedleft {\fontsize{9.5pt}{11.4pt}\selectfont 19.486 ***}\huxbpad{4pt}} &
\multicolumn{1}{r!{\huxvb{0}}}{\huxtpad{4pt}\raggedleft {\fontsize{9.5pt}{11.4pt}\selectfont 20.967 ***}\huxbpad{4pt}} \tabularnewline[-0.5pt]


\hhline{}
\arrayrulecolor{black}

\multicolumn{1}{!{\huxvb{0}}l!{\huxvb{0}}}{\huxtpad{4pt}\raggedright {\fontsize{9.5pt}{11.4pt}\selectfont }\huxbpad{4pt}} &
\multicolumn{1}{r!{\huxvb{0}}}{\huxtpad{4pt}\raggedleft {\fontsize{9.5pt}{11.4pt}\selectfont (1.432)~~~}\huxbpad{4pt}} &
\multicolumn{1}{r!{\huxvb{0}}}{\huxtpad{4pt}\raggedleft {\fontsize{9.5pt}{11.4pt}\selectfont (1.449)~~~}\huxbpad{4pt}} &
\multicolumn{1}{r!{\huxvb{0}}}{\huxtpad{4pt}\raggedleft {\fontsize{9.5pt}{11.4pt}\selectfont (1.371)~~~}\huxbpad{4pt}} \tabularnewline[-0.5pt]


\hhline{}
\arrayrulecolor{black}

\multicolumn{1}{!{\huxvb{0}}l!{\huxvb{0}}}{\huxtpad{4pt}\raggedright {\fontsize{9.5pt}{11.4pt}\selectfont sample\_tract51059420503}\huxbpad{4pt}} &
\multicolumn{1}{r!{\huxvb{0}}}{\huxtpad{4pt}\raggedleft {\fontsize{9.5pt}{11.4pt}\selectfont -0.213~~~~}\huxbpad{4pt}} &
\multicolumn{1}{r!{\huxvb{0}}}{\huxtpad{4pt}\raggedleft {\fontsize{9.5pt}{11.4pt}\selectfont -0.029~~~~}\huxbpad{4pt}} &
\multicolumn{1}{r!{\huxvb{0}}}{\huxtpad{4pt}\raggedleft {\fontsize{9.5pt}{11.4pt}\selectfont 0.751~~~~}\huxbpad{4pt}} \tabularnewline[-0.5pt]


\hhline{}
\arrayrulecolor{black}

\multicolumn{1}{!{\huxvb{0}}l!{\huxvb{0}}}{\huxtpad{4pt}\raggedright {\fontsize{9.5pt}{11.4pt}\selectfont }\huxbpad{4pt}} &
\multicolumn{1}{r!{\huxvb{0}}}{\huxtpad{4pt}\raggedleft {\fontsize{9.5pt}{11.4pt}\selectfont (1.322)~~~}\huxbpad{4pt}} &
\multicolumn{1}{r!{\huxvb{0}}}{\huxtpad{4pt}\raggedleft {\fontsize{9.5pt}{11.4pt}\selectfont (1.403)~~~}\huxbpad{4pt}} &
\multicolumn{1}{r!{\huxvb{0}}}{\huxtpad{4pt}\raggedleft {\fontsize{9.5pt}{11.4pt}\selectfont (1.364)~~~}\huxbpad{4pt}} \tabularnewline[-0.5pt]


\hhline{}
\arrayrulecolor{black}

\multicolumn{1}{!{\huxvb{0}}l!{\huxvb{0}}}{\huxtpad{4pt}\raggedright {\fontsize{9.5pt}{11.4pt}\selectfont sample\_tract51059421001}\huxbpad{4pt}} &
\multicolumn{1}{r!{\huxvb{0}}}{\huxtpad{4pt}\raggedleft {\fontsize{9.5pt}{11.4pt}\selectfont 20.004 ***}\huxbpad{4pt}} &
\multicolumn{1}{r!{\huxvb{0}}}{\huxtpad{4pt}\raggedleft {\fontsize{9.5pt}{11.4pt}\selectfont 19.817 ***}\huxbpad{4pt}} &
\multicolumn{1}{r!{\huxvb{0}}}{\huxtpad{4pt}\raggedleft {\fontsize{9.5pt}{11.4pt}\selectfont 20.897 ***}\huxbpad{4pt}} \tabularnewline[-0.5pt]


\hhline{}
\arrayrulecolor{black}

\multicolumn{1}{!{\huxvb{0}}l!{\huxvb{0}}}{\huxtpad{4pt}\raggedright {\fontsize{9.5pt}{11.4pt}\selectfont }\huxbpad{4pt}} &
\multicolumn{1}{r!{\huxvb{0}}}{\huxtpad{4pt}\raggedleft {\fontsize{9.5pt}{11.4pt}\selectfont (1.383)~~~}\huxbpad{4pt}} &
\multicolumn{1}{r!{\huxvb{0}}}{\huxtpad{4pt}\raggedleft {\fontsize{9.5pt}{11.4pt}\selectfont (1.512)~~~}\huxbpad{4pt}} &
\multicolumn{1}{r!{\huxvb{0}}}{\huxtpad{4pt}\raggedleft {\fontsize{9.5pt}{11.4pt}\selectfont (1.330)~~~}\huxbpad{4pt}} \tabularnewline[-0.5pt]


\hhline{}
\arrayrulecolor{black}

\multicolumn{1}{!{\huxvb{0}}l!{\huxvb{0}}}{\huxtpad{4pt}\raggedright {\fontsize{9.5pt}{11.4pt}\selectfont sample\_tract51059421002}\huxbpad{4pt}} &
\multicolumn{1}{r!{\huxvb{0}}}{\huxtpad{4pt}\raggedleft {\fontsize{9.5pt}{11.4pt}\selectfont 21.603 ***}\huxbpad{4pt}} &
\multicolumn{1}{r!{\huxvb{0}}}{\huxtpad{4pt}\raggedleft {\fontsize{9.5pt}{11.4pt}\selectfont 21.299 ***}\huxbpad{4pt}} &
\multicolumn{1}{r!{\huxvb{0}}}{\huxtpad{4pt}\raggedleft {\fontsize{9.5pt}{11.4pt}\selectfont 22.416 ***}\huxbpad{4pt}} \tabularnewline[-0.5pt]


\hhline{}
\arrayrulecolor{black}

\multicolumn{1}{!{\huxvb{0}}l!{\huxvb{0}}}{\huxtpad{4pt}\raggedright {\fontsize{9.5pt}{11.4pt}\selectfont }\huxbpad{4pt}} &
\multicolumn{1}{r!{\huxvb{0}}}{\huxtpad{4pt}\raggedleft {\fontsize{9.5pt}{11.4pt}\selectfont (1.512)~~~}\huxbpad{4pt}} &
\multicolumn{1}{r!{\huxvb{0}}}{\huxtpad{4pt}\raggedleft {\fontsize{9.5pt}{11.4pt}\selectfont (1.621)~~~}\huxbpad{4pt}} &
\multicolumn{1}{r!{\huxvb{0}}}{\huxtpad{4pt}\raggedleft {\fontsize{9.5pt}{11.4pt}\selectfont (1.627)~~~}\huxbpad{4pt}} \tabularnewline[-0.5pt]


\hhline{}
\arrayrulecolor{black}

\multicolumn{1}{!{\huxvb{0}}l!{\huxvb{0}}}{\huxtpad{4pt}\raggedright {\fontsize{9.5pt}{11.4pt}\selectfont sample\_tract51059421101}\huxbpad{4pt}} &
\multicolumn{1}{r!{\huxvb{0}}}{\huxtpad{4pt}\raggedleft {\fontsize{9.5pt}{11.4pt}\selectfont 19.237 ***}\huxbpad{4pt}} &
\multicolumn{1}{r!{\huxvb{0}}}{\huxtpad{4pt}\raggedleft {\fontsize{9.5pt}{11.4pt}\selectfont 18.875 ***}\huxbpad{4pt}} &
\multicolumn{1}{r!{\huxvb{0}}}{\huxtpad{4pt}\raggedleft {\fontsize{9.5pt}{11.4pt}\selectfont 19.914 ***}\huxbpad{4pt}} \tabularnewline[-0.5pt]


\hhline{}
\arrayrulecolor{black}

\multicolumn{1}{!{\huxvb{0}}l!{\huxvb{0}}}{\huxtpad{4pt}\raggedright {\fontsize{9.5pt}{11.4pt}\selectfont }\huxbpad{4pt}} &
\multicolumn{1}{r!{\huxvb{0}}}{\huxtpad{4pt}\raggedleft {\fontsize{9.5pt}{11.4pt}\selectfont (1.345)~~~}\huxbpad{4pt}} &
\multicolumn{1}{r!{\huxvb{0}}}{\huxtpad{4pt}\raggedleft {\fontsize{9.5pt}{11.4pt}\selectfont (1.731)~~~}\huxbpad{4pt}} &
\multicolumn{1}{r!{\huxvb{0}}}{\huxtpad{4pt}\raggedleft {\fontsize{9.5pt}{11.4pt}\selectfont (1.371)~~~}\huxbpad{4pt}} \tabularnewline[-0.5pt]


\hhline{}
\arrayrulecolor{black}

\multicolumn{1}{!{\huxvb{0}}l!{\huxvb{0}}}{\huxtpad{4pt}\raggedright {\fontsize{9.5pt}{11.4pt}\selectfont sample\_tract51059421102}\huxbpad{4pt}} &
\multicolumn{1}{r!{\huxvb{0}}}{\huxtpad{4pt}\raggedleft {\fontsize{9.5pt}{11.4pt}\selectfont 20.161 ***}\huxbpad{4pt}} &
\multicolumn{1}{r!{\huxvb{0}}}{\huxtpad{4pt}\raggedleft {\fontsize{9.5pt}{11.4pt}\selectfont 19.873 ***}\huxbpad{4pt}} &
\multicolumn{1}{r!{\huxvb{0}}}{\huxtpad{4pt}\raggedleft {\fontsize{9.5pt}{11.4pt}\selectfont 20.565 ***}\huxbpad{4pt}} \tabularnewline[-0.5pt]


\hhline{}
\arrayrulecolor{black}

\multicolumn{1}{!{\huxvb{0}}l!{\huxvb{0}}}{\huxtpad{4pt}\raggedright {\fontsize{9.5pt}{11.4pt}\selectfont }\huxbpad{4pt}} &
\multicolumn{1}{r!{\huxvb{0}}}{\huxtpad{4pt}\raggedleft {\fontsize{9.5pt}{11.4pt}\selectfont (1.350)~~~}\huxbpad{4pt}} &
\multicolumn{1}{r!{\huxvb{0}}}{\huxtpad{4pt}\raggedleft {\fontsize{9.5pt}{11.4pt}\selectfont (1.448)~~~}\huxbpad{4pt}} &
\multicolumn{1}{r!{\huxvb{0}}}{\huxtpad{4pt}\raggedleft {\fontsize{9.5pt}{11.4pt}\selectfont (1.718)~~~}\huxbpad{4pt}} \tabularnewline[-0.5pt]


\hhline{}
\arrayrulecolor{black}

\multicolumn{1}{!{\huxvb{0}}l!{\huxvb{0}}}{\huxtpad{4pt}\raggedright {\fontsize{9.5pt}{11.4pt}\selectfont sample\_tract51059421702}\huxbpad{4pt}} &
\multicolumn{1}{r!{\huxvb{0}}}{\huxtpad{4pt}\raggedleft {\fontsize{9.5pt}{11.4pt}\selectfont 2.033~~~~}\huxbpad{4pt}} &
\multicolumn{1}{r!{\huxvb{0}}}{\huxtpad{4pt}\raggedleft {\fontsize{9.5pt}{11.4pt}\selectfont 1.522~~~~}\huxbpad{4pt}} &
\multicolumn{1}{r!{\huxvb{0}}}{\huxtpad{4pt}\raggedleft {\fontsize{9.5pt}{11.4pt}\selectfont 1.854~~~~}\huxbpad{4pt}} \tabularnewline[-0.5pt]


\hhline{}
\arrayrulecolor{black}

\multicolumn{1}{!{\huxvb{0}}l!{\huxvb{0}}}{\huxtpad{4pt}\raggedright {\fontsize{9.5pt}{11.4pt}\selectfont }\huxbpad{4pt}} &
\multicolumn{1}{r!{\huxvb{0}}}{\huxtpad{4pt}\raggedleft {\fontsize{9.5pt}{11.4pt}\selectfont (1.232)~~~}\huxbpad{4pt}} &
\multicolumn{1}{r!{\huxvb{0}}}{\huxtpad{4pt}\raggedleft {\fontsize{9.5pt}{11.4pt}\selectfont (1.457)~~~}\huxbpad{4pt}} &
\multicolumn{1}{r!{\huxvb{0}}}{\huxtpad{4pt}\raggedleft {\fontsize{9.5pt}{11.4pt}\selectfont (1.442)~~~}\huxbpad{4pt}} \tabularnewline[-0.5pt]


\hhline{}
\arrayrulecolor{black}

\multicolumn{1}{!{\huxvb{0}}l!{\huxvb{0}}}{\huxtpad{4pt}\raggedright {\fontsize{9.5pt}{11.4pt}\selectfont sample\_tract51059421800}\huxbpad{4pt}} &
\multicolumn{1}{r!{\huxvb{0}}}{\huxtpad{4pt}\raggedleft {\fontsize{9.5pt}{11.4pt}\selectfont 0.873~~~~}\huxbpad{4pt}} &
\multicolumn{1}{r!{\huxvb{0}}}{\huxtpad{4pt}\raggedleft {\fontsize{9.5pt}{11.4pt}\selectfont -0.192~~~~}\huxbpad{4pt}} &
\multicolumn{1}{r!{\huxvb{0}}}{\huxtpad{4pt}\raggedleft {\fontsize{9.5pt}{11.4pt}\selectfont 0.339~~~~}\huxbpad{4pt}} \tabularnewline[-0.5pt]


\hhline{}
\arrayrulecolor{black}

\multicolumn{1}{!{\huxvb{0}}l!{\huxvb{0}}}{\huxtpad{4pt}\raggedright {\fontsize{9.5pt}{11.4pt}\selectfont }\huxbpad{4pt}} &
\multicolumn{1}{r!{\huxvb{0}}}{\huxtpad{4pt}\raggedleft {\fontsize{9.5pt}{11.4pt}\selectfont (0.998)~~~}\huxbpad{4pt}} &
\multicolumn{1}{r!{\huxvb{0}}}{\huxtpad{4pt}\raggedleft {\fontsize{9.5pt}{11.4pt}\selectfont (1.300)~~~}\huxbpad{4pt}} &
\multicolumn{1}{r!{\huxvb{0}}}{\huxtpad{4pt}\raggedleft {\fontsize{9.5pt}{11.4pt}\selectfont (1.478)~~~}\huxbpad{4pt}} \tabularnewline[-0.5pt]


\hhline{}
\arrayrulecolor{black}

\multicolumn{1}{!{\huxvb{0}}l!{\huxvb{0}}}{\huxtpad{4pt}\raggedright {\fontsize{9.5pt}{11.4pt}\selectfont sample\_tract51059422000}\huxbpad{4pt}} &
\multicolumn{1}{r!{\huxvb{0}}}{\huxtpad{4pt}\raggedleft {\fontsize{9.5pt}{11.4pt}\selectfont 0.612~~~~}\huxbpad{4pt}} &
\multicolumn{1}{r!{\huxvb{0}}}{\huxtpad{4pt}\raggedleft {\fontsize{9.5pt}{11.4pt}\selectfont -0.602~~~~}\huxbpad{4pt}} &
\multicolumn{1}{r!{\huxvb{0}}}{\huxtpad{4pt}\raggedleft {\fontsize{9.5pt}{11.4pt}\selectfont 0.539~~~~}\huxbpad{4pt}} \tabularnewline[-0.5pt]


\hhline{}
\arrayrulecolor{black}

\multicolumn{1}{!{\huxvb{0}}l!{\huxvb{0}}}{\huxtpad{4pt}\raggedright {\fontsize{9.5pt}{11.4pt}\selectfont }\huxbpad{4pt}} &
\multicolumn{1}{r!{\huxvb{0}}}{\huxtpad{4pt}\raggedleft {\fontsize{9.5pt}{11.4pt}\selectfont (1.217)~~~}\huxbpad{4pt}} &
\multicolumn{1}{r!{\huxvb{0}}}{\huxtpad{4pt}\raggedleft {\fontsize{9.5pt}{11.4pt}\selectfont (1.487)~~~}\huxbpad{4pt}} &
\multicolumn{1}{r!{\huxvb{0}}}{\huxtpad{4pt}\raggedleft {\fontsize{9.5pt}{11.4pt}\selectfont (1.217)~~~}\huxbpad{4pt}} \tabularnewline[-0.5pt]


\hhline{}
\arrayrulecolor{black}

\multicolumn{1}{!{\huxvb{0}}l!{\huxvb{0}}}{\huxtpad{4pt}\raggedright {\fontsize{9.5pt}{11.4pt}\selectfont sample\_tract51059422101}\huxbpad{4pt}} &
\multicolumn{1}{r!{\huxvb{0}}}{\huxtpad{4pt}\raggedleft {\fontsize{9.5pt}{11.4pt}\selectfont 19.381 ***}\huxbpad{4pt}} &
\multicolumn{1}{r!{\huxvb{0}}}{\huxtpad{4pt}\raggedleft {\fontsize{9.5pt}{11.4pt}\selectfont 18.767 ***}\huxbpad{4pt}} &
\multicolumn{1}{r!{\huxvb{0}}}{\huxtpad{4pt}\raggedleft {\fontsize{9.5pt}{11.4pt}\selectfont 20.453 ***}\huxbpad{4pt}} \tabularnewline[-0.5pt]


\hhline{}
\arrayrulecolor{black}

\multicolumn{1}{!{\huxvb{0}}l!{\huxvb{0}}}{\huxtpad{4pt}\raggedright {\fontsize{9.5pt}{11.4pt}\selectfont }\huxbpad{4pt}} &
\multicolumn{1}{r!{\huxvb{0}}}{\huxtpad{4pt}\raggedleft {\fontsize{9.5pt}{11.4pt}\selectfont (1.211)~~~}\huxbpad{4pt}} &
\multicolumn{1}{r!{\huxvb{0}}}{\huxtpad{4pt}\raggedleft {\fontsize{9.5pt}{11.4pt}\selectfont (1.301)~~~}\huxbpad{4pt}} &
\multicolumn{1}{r!{\huxvb{0}}}{\huxtpad{4pt}\raggedleft {\fontsize{9.5pt}{11.4pt}\selectfont (1.154)~~~}\huxbpad{4pt}} \tabularnewline[-0.5pt]


\hhline{}
\arrayrulecolor{black}

\multicolumn{1}{!{\huxvb{0}}l!{\huxvb{0}}}{\huxtpad{4pt}\raggedright {\fontsize{9.5pt}{11.4pt}\selectfont sample\_tract51059422102}\huxbpad{4pt}} &
\multicolumn{1}{r!{\huxvb{0}}}{\huxtpad{4pt}\raggedleft {\fontsize{9.5pt}{11.4pt}\selectfont 17.912 ***}\huxbpad{4pt}} &
\multicolumn{1}{r!{\huxvb{0}}}{\huxtpad{4pt}\raggedleft {\fontsize{9.5pt}{11.4pt}\selectfont 17.595 ***}\huxbpad{4pt}} &
\multicolumn{1}{r!{\huxvb{0}}}{\huxtpad{4pt}\raggedleft {\fontsize{9.5pt}{11.4pt}\selectfont 19.650 ***}\huxbpad{4pt}} \tabularnewline[-0.5pt]


\hhline{}
\arrayrulecolor{black}

\multicolumn{1}{!{\huxvb{0}}l!{\huxvb{0}}}{\huxtpad{4pt}\raggedright {\fontsize{9.5pt}{11.4pt}\selectfont }\huxbpad{4pt}} &
\multicolumn{1}{r!{\huxvb{0}}}{\huxtpad{4pt}\raggedleft {\fontsize{9.5pt}{11.4pt}\selectfont (1.538)~~~}\huxbpad{4pt}} &
\multicolumn{1}{r!{\huxvb{0}}}{\huxtpad{4pt}\raggedleft {\fontsize{9.5pt}{11.4pt}\selectfont (1.563)~~~}\huxbpad{4pt}} &
\multicolumn{1}{r!{\huxvb{0}}}{\huxtpad{4pt}\raggedleft {\fontsize{9.5pt}{11.4pt}\selectfont (1.187)~~~}\huxbpad{4pt}} \tabularnewline[-0.5pt]


\hhline{}
\arrayrulecolor{black}

\multicolumn{1}{!{\huxvb{0}}l!{\huxvb{0}}}{\huxtpad{4pt}\raggedright {\fontsize{9.5pt}{11.4pt}\selectfont sample\_tract51059422401}\huxbpad{4pt}} &
\multicolumn{1}{r!{\huxvb{0}}}{\huxtpad{4pt}\raggedleft {\fontsize{9.5pt}{11.4pt}\selectfont 38.728 ***}\huxbpad{4pt}} &
\multicolumn{1}{r!{\huxvb{0}}}{\huxtpad{4pt}\raggedleft {\fontsize{9.5pt}{11.4pt}\selectfont 38.159 ***}\huxbpad{4pt}} &
\multicolumn{1}{r!{\huxvb{0}}}{\huxtpad{4pt}\raggedleft {\fontsize{9.5pt}{11.4pt}\selectfont 40.086 ***}\huxbpad{4pt}} \tabularnewline[-0.5pt]


\hhline{}
\arrayrulecolor{black}

\multicolumn{1}{!{\huxvb{0}}l!{\huxvb{0}}}{\huxtpad{4pt}\raggedright {\fontsize{9.5pt}{11.4pt}\selectfont }\huxbpad{4pt}} &
\multicolumn{1}{r!{\huxvb{0}}}{\huxtpad{4pt}\raggedleft {\fontsize{9.5pt}{11.4pt}\selectfont (1.248)~~~}\huxbpad{4pt}} &
\multicolumn{1}{r!{\huxvb{0}}}{\huxtpad{4pt}\raggedleft {\fontsize{9.5pt}{11.4pt}\selectfont (1.339)~~~}\huxbpad{4pt}} &
\multicolumn{1}{r!{\huxvb{0}}}{\huxtpad{4pt}\raggedleft {\fontsize{9.5pt}{11.4pt}\selectfont (1.402)~~~}\huxbpad{4pt}} \tabularnewline[-0.5pt]


\hhline{}
\arrayrulecolor{black}

\multicolumn{1}{!{\huxvb{0}}l!{\huxvb{0}}}{\huxtpad{4pt}\raggedright {\fontsize{9.5pt}{11.4pt}\selectfont sample\_tract51059430901}\huxbpad{4pt}} &
\multicolumn{1}{r!{\huxvb{0}}}{\huxtpad{4pt}\raggedleft {\fontsize{9.5pt}{11.4pt}\selectfont 19.925 ***}\huxbpad{4pt}} &
\multicolumn{1}{r!{\huxvb{0}}}{\huxtpad{4pt}\raggedleft {\fontsize{9.5pt}{11.4pt}\selectfont 19.422 ***}\huxbpad{4pt}} &
\multicolumn{1}{r!{\huxvb{0}}}{\huxtpad{4pt}\raggedleft {\fontsize{9.5pt}{11.4pt}\selectfont 20.940 ***}\huxbpad{4pt}} \tabularnewline[-0.5pt]


\hhline{}
\arrayrulecolor{black}

\multicolumn{1}{!{\huxvb{0}}l!{\huxvb{0}}}{\huxtpad{4pt}\raggedright {\fontsize{9.5pt}{11.4pt}\selectfont }\huxbpad{4pt}} &
\multicolumn{1}{r!{\huxvb{0}}}{\huxtpad{4pt}\raggedleft {\fontsize{9.5pt}{11.4pt}\selectfont (1.216)~~~}\huxbpad{4pt}} &
\multicolumn{1}{r!{\huxvb{0}}}{\huxtpad{4pt}\raggedleft {\fontsize{9.5pt}{11.4pt}\selectfont (1.365)~~~}\huxbpad{4pt}} &
\multicolumn{1}{r!{\huxvb{0}}}{\huxtpad{4pt}\raggedleft {\fontsize{9.5pt}{11.4pt}\selectfont (1.222)~~~}\huxbpad{4pt}} \tabularnewline[-0.5pt]


\hhline{}
\arrayrulecolor{black}

\multicolumn{1}{!{\huxvb{0}}l!{\huxvb{0}}}{\huxtpad{4pt}\raggedright {\fontsize{9.5pt}{11.4pt}\selectfont sample\_tract51059431001}\huxbpad{4pt}} &
\multicolumn{1}{r!{\huxvb{0}}}{\huxtpad{4pt}\raggedleft {\fontsize{9.5pt}{11.4pt}\selectfont 18.209 ***}\huxbpad{4pt}} &
\multicolumn{1}{r!{\huxvb{0}}}{\huxtpad{4pt}\raggedleft {\fontsize{9.5pt}{11.4pt}\selectfont 18.027 ***}\huxbpad{4pt}} &
\multicolumn{1}{r!{\huxvb{0}}}{\huxtpad{4pt}\raggedleft {\fontsize{9.5pt}{11.4pt}\selectfont 19.140 ***}\huxbpad{4pt}} \tabularnewline[-0.5pt]


\hhline{}
\arrayrulecolor{black}

\multicolumn{1}{!{\huxvb{0}}l!{\huxvb{0}}}{\huxtpad{4pt}\raggedright {\fontsize{9.5pt}{11.4pt}\selectfont }\huxbpad{4pt}} &
\multicolumn{1}{r!{\huxvb{0}}}{\huxtpad{4pt}\raggedleft {\fontsize{9.5pt}{11.4pt}\selectfont (1.518)~~~}\huxbpad{4pt}} &
\multicolumn{1}{r!{\huxvb{0}}}{\huxtpad{4pt}\raggedleft {\fontsize{9.5pt}{11.4pt}\selectfont (1.505)~~~}\huxbpad{4pt}} &
\multicolumn{1}{r!{\huxvb{0}}}{\huxtpad{4pt}\raggedleft {\fontsize{9.5pt}{11.4pt}\selectfont (1.179)~~~}\huxbpad{4pt}} \tabularnewline[-0.5pt]


\hhline{}
\arrayrulecolor{black}

\multicolumn{1}{!{\huxvb{0}}l!{\huxvb{0}}}{\huxtpad{4pt}\raggedright {\fontsize{9.5pt}{11.4pt}\selectfont sample\_tract51059431002}\huxbpad{4pt}} &
\multicolumn{1}{r!{\huxvb{0}}}{\huxtpad{4pt}\raggedleft {\fontsize{9.5pt}{11.4pt}\selectfont 37.825 ***}\huxbpad{4pt}} &
\multicolumn{1}{r!{\huxvb{0}}}{\huxtpad{4pt}\raggedleft {\fontsize{9.5pt}{11.4pt}\selectfont 37.191 ***}\huxbpad{4pt}} &
\multicolumn{1}{r!{\huxvb{0}}}{\huxtpad{4pt}\raggedleft {\fontsize{9.5pt}{11.4pt}\selectfont 39.218 ***}\huxbpad{4pt}} \tabularnewline[-0.5pt]


\hhline{}
\arrayrulecolor{black}

\multicolumn{1}{!{\huxvb{0}}l!{\huxvb{0}}}{\huxtpad{4pt}\raggedright {\fontsize{9.5pt}{11.4pt}\selectfont }\huxbpad{4pt}} &
\multicolumn{1}{r!{\huxvb{0}}}{\huxtpad{4pt}\raggedleft {\fontsize{9.5pt}{11.4pt}\selectfont (1.385)~~~}\huxbpad{4pt}} &
\multicolumn{1}{r!{\huxvb{0}}}{\huxtpad{4pt}\raggedleft {\fontsize{9.5pt}{11.4pt}\selectfont (1.513)~~~}\huxbpad{4pt}} &
\multicolumn{1}{r!{\huxvb{0}}}{\huxtpad{4pt}\raggedleft {\fontsize{9.5pt}{11.4pt}\selectfont (1.464)~~~}\huxbpad{4pt}} \tabularnewline[-0.5pt]


\hhline{}
\arrayrulecolor{black}

\multicolumn{1}{!{\huxvb{0}}l!{\huxvb{0}}}{\huxtpad{4pt}\raggedright {\fontsize{9.5pt}{11.4pt}\selectfont sample\_tract51059432702}\huxbpad{4pt}} &
\multicolumn{1}{r!{\huxvb{0}}}{\huxtpad{4pt}\raggedleft {\fontsize{9.5pt}{11.4pt}\selectfont 21.654 ***}\huxbpad{4pt}} &
\multicolumn{1}{r!{\huxvb{0}}}{\huxtpad{4pt}\raggedleft {\fontsize{9.5pt}{11.4pt}\selectfont 21.307 ***}\huxbpad{4pt}} &
\multicolumn{1}{r!{\huxvb{0}}}{\huxtpad{4pt}\raggedleft {\fontsize{9.5pt}{11.4pt}\selectfont 21.821 ***}\huxbpad{4pt}} \tabularnewline[-0.5pt]


\hhline{}
\arrayrulecolor{black}

\multicolumn{1}{!{\huxvb{0}}l!{\huxvb{0}}}{\huxtpad{4pt}\raggedright {\fontsize{9.5pt}{11.4pt}\selectfont }\huxbpad{4pt}} &
\multicolumn{1}{r!{\huxvb{0}}}{\huxtpad{4pt}\raggedleft {\fontsize{9.5pt}{11.4pt}\selectfont (1.577)~~~}\huxbpad{4pt}} &
\multicolumn{1}{r!{\huxvb{0}}}{\huxtpad{4pt}\raggedleft {\fontsize{9.5pt}{11.4pt}\selectfont (1.676)~~~}\huxbpad{4pt}} &
\multicolumn{1}{r!{\huxvb{0}}}{\huxtpad{4pt}\raggedleft {\fontsize{9.5pt}{11.4pt}\selectfont (1.594)~~~}\huxbpad{4pt}} \tabularnewline[-0.5pt]


\hhline{}
\arrayrulecolor{black}

\multicolumn{1}{!{\huxvb{0}}l!{\huxvb{0}}}{\huxtpad{4pt}\raggedright {\fontsize{9.5pt}{11.4pt}\selectfont sample\_tract51059432800}\huxbpad{4pt}} &
\multicolumn{1}{r!{\huxvb{0}}}{\huxtpad{4pt}\raggedleft {\fontsize{9.5pt}{11.4pt}\selectfont 1.605~~~~}\huxbpad{4pt}} &
\multicolumn{1}{r!{\huxvb{0}}}{\huxtpad{4pt}\raggedleft {\fontsize{9.5pt}{11.4pt}\selectfont 0.543~~~~}\huxbpad{4pt}} &
\multicolumn{1}{r!{\huxvb{0}}}{\huxtpad{4pt}\raggedleft {\fontsize{9.5pt}{11.4pt}\selectfont 1.682~~~~}\huxbpad{4pt}} \tabularnewline[-0.5pt]


\hhline{}
\arrayrulecolor{black}

\multicolumn{1}{!{\huxvb{0}}l!{\huxvb{0}}}{\huxtpad{4pt}\raggedright {\fontsize{9.5pt}{11.4pt}\selectfont }\huxbpad{4pt}} &
\multicolumn{1}{r!{\huxvb{0}}}{\huxtpad{4pt}\raggedleft {\fontsize{9.5pt}{11.4pt}\selectfont (1.383)~~~}\huxbpad{4pt}} &
\multicolumn{1}{r!{\huxvb{0}}}{\huxtpad{4pt}\raggedleft {\fontsize{9.5pt}{11.4pt}\selectfont (1.584)~~~}\huxbpad{4pt}} &
\multicolumn{1}{r!{\huxvb{0}}}{\huxtpad{4pt}\raggedleft {\fontsize{9.5pt}{11.4pt}\selectfont (1.640)~~~}\huxbpad{4pt}} \tabularnewline[-0.5pt]


\hhline{}
\arrayrulecolor{black}

\multicolumn{1}{!{\huxvb{0}}l!{\huxvb{0}}}{\huxtpad{4pt}\raggedright {\fontsize{9.5pt}{11.4pt}\selectfont sample\_tract51059451900}\huxbpad{4pt}} &
\multicolumn{1}{r!{\huxvb{0}}}{\huxtpad{4pt}\raggedleft {\fontsize{9.5pt}{11.4pt}\selectfont 1.320~~~~}\huxbpad{4pt}} &
\multicolumn{1}{r!{\huxvb{0}}}{\huxtpad{4pt}\raggedleft {\fontsize{9.5pt}{11.4pt}\selectfont 0.968~~~~}\huxbpad{4pt}} &
\multicolumn{1}{r!{\huxvb{0}}}{\huxtpad{4pt}\raggedleft {\fontsize{9.5pt}{11.4pt}\selectfont 1.013~~~~}\huxbpad{4pt}} \tabularnewline[-0.5pt]


\hhline{}
\arrayrulecolor{black}

\multicolumn{1}{!{\huxvb{0}}l!{\huxvb{0}}}{\huxtpad{4pt}\raggedright {\fontsize{9.5pt}{11.4pt}\selectfont }\huxbpad{4pt}} &
\multicolumn{1}{r!{\huxvb{0}}}{\huxtpad{4pt}\raggedleft {\fontsize{9.5pt}{11.4pt}\selectfont (1.231)~~~}\huxbpad{4pt}} &
\multicolumn{1}{r!{\huxvb{0}}}{\huxtpad{4pt}\raggedleft {\fontsize{9.5pt}{11.4pt}\selectfont (1.378)~~~}\huxbpad{4pt}} &
\multicolumn{1}{r!{\huxvb{0}}}{\huxtpad{4pt}\raggedleft {\fontsize{9.5pt}{11.4pt}\selectfont (1.341)~~~}\huxbpad{4pt}} \tabularnewline[-0.5pt]


\hhline{}
\arrayrulecolor{black}

\multicolumn{1}{!{\huxvb{0}}l!{\huxvb{0}}}{\huxtpad{4pt}\raggedright {\fontsize{9.5pt}{11.4pt}\selectfont sample\_tract51059452101}\huxbpad{4pt}} &
\multicolumn{1}{r!{\huxvb{0}}}{\huxtpad{4pt}\raggedleft {\fontsize{9.5pt}{11.4pt}\selectfont 16.150 ***}\huxbpad{4pt}} &
\multicolumn{1}{r!{\huxvb{0}}}{\huxtpad{4pt}\raggedleft {\fontsize{9.5pt}{11.4pt}\selectfont 15.854 ***}\huxbpad{4pt}} &
\multicolumn{1}{r!{\huxvb{0}}}{\huxtpad{4pt}\raggedleft {\fontsize{9.5pt}{11.4pt}\selectfont 18.529 ***}\huxbpad{4pt}} \tabularnewline[-0.5pt]


\hhline{}
\arrayrulecolor{black}

\multicolumn{1}{!{\huxvb{0}}l!{\huxvb{0}}}{\huxtpad{4pt}\raggedright {\fontsize{9.5pt}{11.4pt}\selectfont }\huxbpad{4pt}} &
\multicolumn{1}{r!{\huxvb{0}}}{\huxtpad{4pt}\raggedleft {\fontsize{9.5pt}{11.4pt}\selectfont (1.698)~~~}\huxbpad{4pt}} &
\multicolumn{1}{r!{\huxvb{0}}}{\huxtpad{4pt}\raggedleft {\fontsize{9.5pt}{11.4pt}\selectfont (1.780)~~~}\huxbpad{4pt}} &
\multicolumn{1}{r!{\huxvb{0}}}{\huxtpad{4pt}\raggedleft {\fontsize{9.5pt}{11.4pt}\selectfont (1.793)~~~}\huxbpad{4pt}} \tabularnewline[-0.5pt]


\hhline{}
\arrayrulecolor{black}

\multicolumn{1}{!{\huxvb{0}}l!{\huxvb{0}}}{\huxtpad{4pt}\raggedright {\fontsize{9.5pt}{11.4pt}\selectfont sample\_tract51059452200}\huxbpad{4pt}} &
\multicolumn{1}{r!{\huxvb{0}}}{\huxtpad{4pt}\raggedleft {\fontsize{9.5pt}{11.4pt}\selectfont 18.920 ***}\huxbpad{4pt}} &
\multicolumn{1}{r!{\huxvb{0}}}{\huxtpad{4pt}\raggedleft {\fontsize{9.5pt}{11.4pt}\selectfont 18.570 ***}\huxbpad{4pt}} &
\multicolumn{1}{r!{\huxvb{0}}}{\huxtpad{4pt}\raggedleft {\fontsize{9.5pt}{11.4pt}\selectfont 20.156 ***}\huxbpad{4pt}} \tabularnewline[-0.5pt]


\hhline{}
\arrayrulecolor{black}

\multicolumn{1}{!{\huxvb{0}}l!{\huxvb{0}}}{\huxtpad{4pt}\raggedright {\fontsize{9.5pt}{11.4pt}\selectfont }\huxbpad{4pt}} &
\multicolumn{1}{r!{\huxvb{0}}}{\huxtpad{4pt}\raggedleft {\fontsize{9.5pt}{11.4pt}\selectfont (1.238)~~~}\huxbpad{4pt}} &
\multicolumn{1}{r!{\huxvb{0}}}{\huxtpad{4pt}\raggedleft {\fontsize{9.5pt}{11.4pt}\selectfont (1.288)~~~}\huxbpad{4pt}} &
\multicolumn{1}{r!{\huxvb{0}}}{\huxtpad{4pt}\raggedleft {\fontsize{9.5pt}{11.4pt}\selectfont (1.288)~~~}\huxbpad{4pt}} \tabularnewline[-0.5pt]


\hhline{}
\arrayrulecolor{black}

\multicolumn{1}{!{\huxvb{0}}l!{\huxvb{0}}}{\huxtpad{4pt}\raggedright {\fontsize{9.5pt}{11.4pt}\selectfont sample\_tract51059452301}\huxbpad{4pt}} &
\multicolumn{1}{r!{\huxvb{0}}}{\huxtpad{4pt}\raggedleft {\fontsize{9.5pt}{11.4pt}\selectfont -0.203~~~~}\huxbpad{4pt}} &
\multicolumn{1}{r!{\huxvb{0}}}{\huxtpad{4pt}\raggedleft {\fontsize{9.5pt}{11.4pt}\selectfont -0.558~~~~}\huxbpad{4pt}} &
\multicolumn{1}{r!{\huxvb{0}}}{\huxtpad{4pt}\raggedleft {\fontsize{9.5pt}{11.4pt}\selectfont -0.482~~~~}\huxbpad{4pt}} \tabularnewline[-0.5pt]


\hhline{}
\arrayrulecolor{black}

\multicolumn{1}{!{\huxvb{0}}l!{\huxvb{0}}}{\huxtpad{4pt}\raggedright {\fontsize{9.5pt}{11.4pt}\selectfont }\huxbpad{4pt}} &
\multicolumn{1}{r!{\huxvb{0}}}{\huxtpad{4pt}\raggedleft {\fontsize{9.5pt}{11.4pt}\selectfont (1.390)~~~}\huxbpad{4pt}} &
\multicolumn{1}{r!{\huxvb{0}}}{\huxtpad{4pt}\raggedleft {\fontsize{9.5pt}{11.4pt}\selectfont (1.589)~~~}\huxbpad{4pt}} &
\multicolumn{1}{r!{\huxvb{0}}}{\huxtpad{4pt}\raggedleft {\fontsize{9.5pt}{11.4pt}\selectfont (1.487)~~~}\huxbpad{4pt}} \tabularnewline[-0.5pt]


\hhline{}
\arrayrulecolor{black}

\multicolumn{1}{!{\huxvb{0}}l!{\huxvb{0}}}{\huxtpad{4pt}\raggedright {\fontsize{9.5pt}{11.4pt}\selectfont sample\_tract51059452502}\huxbpad{4pt}} &
\multicolumn{1}{r!{\huxvb{0}}}{\huxtpad{4pt}\raggedleft {\fontsize{9.5pt}{11.4pt}\selectfont 17.468 ***}\huxbpad{4pt}} &
\multicolumn{1}{r!{\huxvb{0}}}{\huxtpad{4pt}\raggedleft {\fontsize{9.5pt}{11.4pt}\selectfont 16.663 ***}\huxbpad{4pt}} &
\multicolumn{1}{r!{\huxvb{0}}}{\huxtpad{4pt}\raggedleft {\fontsize{9.5pt}{11.4pt}\selectfont 18.338 ***}\huxbpad{4pt}} \tabularnewline[-0.5pt]


\hhline{}
\arrayrulecolor{black}

\multicolumn{1}{!{\huxvb{0}}l!{\huxvb{0}}}{\huxtpad{4pt}\raggedright {\fontsize{9.5pt}{11.4pt}\selectfont }\huxbpad{4pt}} &
\multicolumn{1}{r!{\huxvb{0}}}{\huxtpad{4pt}\raggedleft {\fontsize{9.5pt}{11.4pt}\selectfont (1.316)~~~}\huxbpad{4pt}} &
\multicolumn{1}{r!{\huxvb{0}}}{\huxtpad{4pt}\raggedleft {\fontsize{9.5pt}{11.4pt}\selectfont (1.414)~~~}\huxbpad{4pt}} &
\multicolumn{1}{r!{\huxvb{0}}}{\huxtpad{4pt}\raggedleft {\fontsize{9.5pt}{11.4pt}\selectfont (1.409)~~~}\huxbpad{4pt}} \tabularnewline[-0.5pt]


\hhline{}
\arrayrulecolor{black}

\multicolumn{1}{!{\huxvb{0}}l!{\huxvb{0}}}{\huxtpad{4pt}\raggedright {\fontsize{9.5pt}{11.4pt}\selectfont sample\_tract51059452600}\huxbpad{4pt}} &
\multicolumn{1}{r!{\huxvb{0}}}{\huxtpad{4pt}\raggedleft {\fontsize{9.5pt}{11.4pt}\selectfont 19.221 ***}\huxbpad{4pt}} &
\multicolumn{1}{r!{\huxvb{0}}}{\huxtpad{4pt}\raggedleft {\fontsize{9.5pt}{11.4pt}\selectfont 18.976 ***}\huxbpad{4pt}} &
\multicolumn{1}{r!{\huxvb{0}}}{\huxtpad{4pt}\raggedleft {\fontsize{9.5pt}{11.4pt}\selectfont 20.510 ***}\huxbpad{4pt}} \tabularnewline[-0.5pt]


\hhline{}
\arrayrulecolor{black}

\multicolumn{1}{!{\huxvb{0}}l!{\huxvb{0}}}{\huxtpad{4pt}\raggedright {\fontsize{9.5pt}{11.4pt}\selectfont }\huxbpad{4pt}} &
\multicolumn{1}{r!{\huxvb{0}}}{\huxtpad{4pt}\raggedleft {\fontsize{9.5pt}{11.4pt}\selectfont (1.117)~~~}\huxbpad{4pt}} &
\multicolumn{1}{r!{\huxvb{0}}}{\huxtpad{4pt}\raggedleft {\fontsize{9.5pt}{11.4pt}\selectfont (1.252)~~~}\huxbpad{4pt}} &
\multicolumn{1}{r!{\huxvb{0}}}{\huxtpad{4pt}\raggedleft {\fontsize{9.5pt}{11.4pt}\selectfont (1.210)~~~}\huxbpad{4pt}} \tabularnewline[-0.5pt]


\hhline{}
\arrayrulecolor{black}

\multicolumn{1}{!{\huxvb{0}}l!{\huxvb{0}}}{\huxtpad{4pt}\raggedright {\fontsize{9.5pt}{11.4pt}\selectfont sample\_tract51059452700}\huxbpad{4pt}} &
\multicolumn{1}{r!{\huxvb{0}}}{\huxtpad{4pt}\raggedleft {\fontsize{9.5pt}{11.4pt}\selectfont 1.084~~~~}\huxbpad{4pt}} &
\multicolumn{1}{r!{\huxvb{0}}}{\huxtpad{4pt}\raggedleft {\fontsize{9.5pt}{11.4pt}\selectfont 0.538~~~~}\huxbpad{4pt}} &
\multicolumn{1}{r!{\huxvb{0}}}{\huxtpad{4pt}\raggedleft {\fontsize{9.5pt}{11.4pt}\selectfont 1.565~~~~}\huxbpad{4pt}} \tabularnewline[-0.5pt]


\hhline{}
\arrayrulecolor{black}

\multicolumn{1}{!{\huxvb{0}}l!{\huxvb{0}}}{\huxtpad{4pt}\raggedright {\fontsize{9.5pt}{11.4pt}\selectfont }\huxbpad{4pt}} &
\multicolumn{1}{r!{\huxvb{0}}}{\huxtpad{4pt}\raggedleft {\fontsize{9.5pt}{11.4pt}\selectfont (1.263)~~~}\huxbpad{4pt}} &
\multicolumn{1}{r!{\huxvb{0}}}{\huxtpad{4pt}\raggedleft {\fontsize{9.5pt}{11.4pt}\selectfont (1.459)~~~}\huxbpad{4pt}} &
\multicolumn{1}{r!{\huxvb{0}}}{\huxtpad{4pt}\raggedleft {\fontsize{9.5pt}{11.4pt}\selectfont (1.442)~~~}\huxbpad{4pt}} \tabularnewline[-0.5pt]


\hhline{}
\arrayrulecolor{black}

\multicolumn{1}{!{\huxvb{0}}l!{\huxvb{0}}}{\huxtpad{4pt}\raggedright {\fontsize{9.5pt}{11.4pt}\selectfont sample\_tract51059452801}\huxbpad{4pt}} &
\multicolumn{1}{r!{\huxvb{0}}}{\huxtpad{4pt}\raggedleft {\fontsize{9.5pt}{11.4pt}\selectfont 0.925~~~~}\huxbpad{4pt}} &
\multicolumn{1}{r!{\huxvb{0}}}{\huxtpad{4pt}\raggedleft {\fontsize{9.5pt}{11.4pt}\selectfont 0.355~~~~}\huxbpad{4pt}} &
\multicolumn{1}{r!{\huxvb{0}}}{\huxtpad{4pt}\raggedleft {\fontsize{9.5pt}{11.4pt}\selectfont 0.331~~~~}\huxbpad{4pt}} \tabularnewline[-0.5pt]


\hhline{}
\arrayrulecolor{black}

\multicolumn{1}{!{\huxvb{0}}l!{\huxvb{0}}}{\huxtpad{4pt}\raggedright {\fontsize{9.5pt}{11.4pt}\selectfont }\huxbpad{4pt}} &
\multicolumn{1}{r!{\huxvb{0}}}{\huxtpad{4pt}\raggedleft {\fontsize{9.5pt}{11.4pt}\selectfont (1.138)~~~}\huxbpad{4pt}} &
\multicolumn{1}{r!{\huxvb{0}}}{\huxtpad{4pt}\raggedleft {\fontsize{9.5pt}{11.4pt}\selectfont (1.337)~~~}\huxbpad{4pt}} &
\multicolumn{1}{r!{\huxvb{0}}}{\huxtpad{4pt}\raggedleft {\fontsize{9.5pt}{11.4pt}\selectfont (1.446)~~~}\huxbpad{4pt}} \tabularnewline[-0.5pt]


\hhline{}
\arrayrulecolor{black}

\multicolumn{1}{!{\huxvb{0}}l!{\huxvb{0}}}{\huxtpad{4pt}\raggedright {\fontsize{9.5pt}{11.4pt}\selectfont sample\_tract51059461901}\huxbpad{4pt}} &
\multicolumn{1}{r!{\huxvb{0}}}{\huxtpad{4pt}\raggedleft {\fontsize{9.5pt}{11.4pt}\selectfont 37.659 ***}\huxbpad{4pt}} &
\multicolumn{1}{r!{\huxvb{0}}}{\huxtpad{4pt}\raggedleft {\fontsize{9.5pt}{11.4pt}\selectfont 37.242 ***}\huxbpad{4pt}} &
\multicolumn{1}{r!{\huxvb{0}}}{\huxtpad{4pt}\raggedleft {\fontsize{9.5pt}{11.4pt}\selectfont 40.713 ***}\huxbpad{4pt}} \tabularnewline[-0.5pt]


\hhline{}
\arrayrulecolor{black}

\multicolumn{1}{!{\huxvb{0}}l!{\huxvb{0}}}{\huxtpad{4pt}\raggedright {\fontsize{9.5pt}{11.4pt}\selectfont }\huxbpad{4pt}} &
\multicolumn{1}{r!{\huxvb{0}}}{\huxtpad{4pt}\raggedleft {\fontsize{9.5pt}{11.4pt}\selectfont (1.051)~~~}\huxbpad{4pt}} &
\multicolumn{1}{r!{\huxvb{0}}}{\huxtpad{4pt}\raggedleft {\fontsize{9.5pt}{11.4pt}\selectfont (1.212)~~~}\huxbpad{4pt}} &
\multicolumn{1}{r!{\huxvb{0}}}{\huxtpad{4pt}\raggedleft {\fontsize{9.5pt}{11.4pt}\selectfont (1.285)~~~}\huxbpad{4pt}} \tabularnewline[-0.5pt]


\hhline{}
\arrayrulecolor{black}

\multicolumn{1}{!{\huxvb{0}}l!{\huxvb{0}}}{\huxtpad{4pt}\raggedright {\fontsize{9.5pt}{11.4pt}\selectfont sample\_tract51059471402}\huxbpad{4pt}} &
\multicolumn{1}{r!{\huxvb{0}}}{\huxtpad{4pt}\raggedleft {\fontsize{9.5pt}{11.4pt}\selectfont 18.270 ***}\huxbpad{4pt}} &
\multicolumn{1}{r!{\huxvb{0}}}{\huxtpad{4pt}\raggedleft {\fontsize{9.5pt}{11.4pt}\selectfont 18.382 ***}\huxbpad{4pt}} &
\multicolumn{1}{r!{\huxvb{0}}}{\huxtpad{4pt}\raggedleft {\fontsize{9.5pt}{11.4pt}\selectfont 20.066 ***}\huxbpad{4pt}} \tabularnewline[-0.5pt]


\hhline{}
\arrayrulecolor{black}

\multicolumn{1}{!{\huxvb{0}}l!{\huxvb{0}}}{\huxtpad{4pt}\raggedright {\fontsize{9.5pt}{11.4pt}\selectfont }\huxbpad{4pt}} &
\multicolumn{1}{r!{\huxvb{0}}}{\huxtpad{4pt}\raggedleft {\fontsize{9.5pt}{11.4pt}\selectfont (1.700)~~~}\huxbpad{4pt}} &
\multicolumn{1}{r!{\huxvb{0}}}{\huxtpad{4pt}\raggedleft {\fontsize{9.5pt}{11.4pt}\selectfont (1.834)~~~}\huxbpad{4pt}} &
\multicolumn{1}{r!{\huxvb{0}}}{\huxtpad{4pt}\raggedleft {\fontsize{9.5pt}{11.4pt}\selectfont (1.792)~~~}\huxbpad{4pt}} \tabularnewline[-0.5pt]


\hhline{}
\arrayrulecolor{black}

\multicolumn{1}{!{\huxvb{0}}l!{\huxvb{0}}}{\huxtpad{4pt}\raggedright {\fontsize{9.5pt}{11.4pt}\selectfont sample\_tract51059480901}\huxbpad{4pt}} &
\multicolumn{1}{r!{\huxvb{0}}}{\huxtpad{4pt}\raggedleft {\fontsize{9.5pt}{11.4pt}\selectfont 19.617 ***}\huxbpad{4pt}} &
\multicolumn{1}{r!{\huxvb{0}}}{\huxtpad{4pt}\raggedleft {\fontsize{9.5pt}{11.4pt}\selectfont 18.717 ***}\huxbpad{4pt}} &
\multicolumn{1}{r!{\huxvb{0}}}{\huxtpad{4pt}\raggedleft {\fontsize{9.5pt}{11.4pt}\selectfont 19.108 ***}\huxbpad{4pt}} \tabularnewline[-0.5pt]


\hhline{}
\arrayrulecolor{black}

\multicolumn{1}{!{\huxvb{0}}l!{\huxvb{0}}}{\huxtpad{4pt}\raggedright {\fontsize{9.5pt}{11.4pt}\selectfont }\huxbpad{4pt}} &
\multicolumn{1}{r!{\huxvb{0}}}{\huxtpad{4pt}\raggedleft {\fontsize{9.5pt}{11.4pt}\selectfont (1.359)~~~}\huxbpad{4pt}} &
\multicolumn{1}{r!{\huxvb{0}}}{\huxtpad{4pt}\raggedleft {\fontsize{9.5pt}{11.4pt}\selectfont (1.383)~~~}\huxbpad{4pt}} &
\multicolumn{1}{r!{\huxvb{0}}}{\huxtpad{4pt}\raggedleft {\fontsize{9.5pt}{11.4pt}\selectfont (1.198)~~~}\huxbpad{4pt}} \tabularnewline[-0.5pt]


\hhline{}
\arrayrulecolor{black}

\multicolumn{1}{!{\huxvb{0}}l!{\huxvb{0}}}{\huxtpad{4pt}\raggedright {\fontsize{9.5pt}{11.4pt}\selectfont sample\_tract51059480902}\huxbpad{4pt}} &
\multicolumn{1}{r!{\huxvb{0}}}{\huxtpad{4pt}\raggedleft {\fontsize{9.5pt}{11.4pt}\selectfont 20.155 ***}\huxbpad{4pt}} &
\multicolumn{1}{r!{\huxvb{0}}}{\huxtpad{4pt}\raggedleft {\fontsize{9.5pt}{11.4pt}\selectfont 19.766 ***}\huxbpad{4pt}} &
\multicolumn{1}{r!{\huxvb{0}}}{\huxtpad{4pt}\raggedleft {\fontsize{9.5pt}{11.4pt}\selectfont 19.720 ***}\huxbpad{4pt}} \tabularnewline[-0.5pt]


\hhline{}
\arrayrulecolor{black}

\multicolumn{1}{!{\huxvb{0}}l!{\huxvb{0}}}{\huxtpad{4pt}\raggedright {\fontsize{9.5pt}{11.4pt}\selectfont }\huxbpad{4pt}} &
\multicolumn{1}{r!{\huxvb{0}}}{\huxtpad{4pt}\raggedleft {\fontsize{9.5pt}{11.4pt}\selectfont (1.385)~~~}\huxbpad{4pt}} &
\multicolumn{1}{r!{\huxvb{0}}}{\huxtpad{4pt}\raggedleft {\fontsize{9.5pt}{11.4pt}\selectfont (1.677)~~~}\huxbpad{4pt}} &
\multicolumn{1}{r!{\huxvb{0}}}{\huxtpad{4pt}\raggedleft {\fontsize{9.5pt}{11.4pt}\selectfont (1.829)~~~}\huxbpad{4pt}} \tabularnewline[-0.5pt]


\hhline{}
\arrayrulecolor{black}

\multicolumn{1}{!{\huxvb{0}}l!{\huxvb{0}}}{\huxtpad{4pt}\raggedright {\fontsize{9.5pt}{11.4pt}\selectfont sample\_tract51059481103}\huxbpad{4pt}} &
\multicolumn{1}{r!{\huxvb{0}}}{\huxtpad{4pt}\raggedleft {\fontsize{9.5pt}{11.4pt}\selectfont 1.025~~~~}\huxbpad{4pt}} &
\multicolumn{1}{r!{\huxvb{0}}}{\huxtpad{4pt}\raggedleft {\fontsize{9.5pt}{11.4pt}\selectfont 0.795~~~~}\huxbpad{4pt}} &
\multicolumn{1}{r!{\huxvb{0}}}{\huxtpad{4pt}\raggedleft {\fontsize{9.5pt}{11.4pt}\selectfont 2.093~~~~}\huxbpad{4pt}} \tabularnewline[-0.5pt]


\hhline{}
\arrayrulecolor{black}

\multicolumn{1}{!{\huxvb{0}}l!{\huxvb{0}}}{\huxtpad{4pt}\raggedright {\fontsize{9.5pt}{11.4pt}\selectfont }\huxbpad{4pt}} &
\multicolumn{1}{r!{\huxvb{0}}}{\huxtpad{4pt}\raggedleft {\fontsize{9.5pt}{11.4pt}\selectfont (1.371)~~~}\huxbpad{4pt}} &
\multicolumn{1}{r!{\huxvb{0}}}{\huxtpad{4pt}\raggedleft {\fontsize{9.5pt}{11.4pt}\selectfont (1.532)~~~}\huxbpad{4pt}} &
\multicolumn{1}{r!{\huxvb{0}}}{\huxtpad{4pt}\raggedleft {\fontsize{9.5pt}{11.4pt}\selectfont (1.502)~~~}\huxbpad{4pt}} \tabularnewline[-0.5pt]


\hhline{}
\arrayrulecolor{black}

\multicolumn{1}{!{\huxvb{0}}l!{\huxvb{0}}}{\huxtpad{4pt}\raggedright {\fontsize{9.5pt}{11.4pt}\selectfont sample\_tract51059481202}\huxbpad{4pt}} &
\multicolumn{1}{r!{\huxvb{0}}}{\huxtpad{4pt}\raggedleft {\fontsize{9.5pt}{11.4pt}\selectfont 38.297 ***}\huxbpad{4pt}} &
\multicolumn{1}{r!{\huxvb{0}}}{\huxtpad{4pt}\raggedleft {\fontsize{9.5pt}{11.4pt}\selectfont 37.385 ***}\huxbpad{4pt}} &
\multicolumn{1}{r!{\huxvb{0}}}{\huxtpad{4pt}\raggedleft {\fontsize{9.5pt}{11.4pt}\selectfont 39.066 ***}\huxbpad{4pt}} \tabularnewline[-0.5pt]


\hhline{}
\arrayrulecolor{black}

\multicolumn{1}{!{\huxvb{0}}l!{\huxvb{0}}}{\huxtpad{4pt}\raggedright {\fontsize{9.5pt}{11.4pt}\selectfont }\huxbpad{4pt}} &
\multicolumn{1}{r!{\huxvb{0}}}{\huxtpad{4pt}\raggedleft {\fontsize{9.5pt}{11.4pt}\selectfont (1.268)~~~}\huxbpad{4pt}} &
\multicolumn{1}{r!{\huxvb{0}}}{\huxtpad{4pt}\raggedleft {\fontsize{9.5pt}{11.4pt}\selectfont (1.375)~~~}\huxbpad{4pt}} &
\multicolumn{1}{r!{\huxvb{0}}}{\huxtpad{4pt}\raggedleft {\fontsize{9.5pt}{11.4pt}\selectfont (1.456)~~~}\huxbpad{4pt}} \tabularnewline[-0.5pt]


\hhline{}
\arrayrulecolor{black}

\multicolumn{1}{!{\huxvb{0}}l!{\huxvb{0}}}{\huxtpad{4pt}\raggedright {\fontsize{9.5pt}{11.4pt}\selectfont sample\_tract51059482302}\huxbpad{4pt}} &
\multicolumn{1}{r!{\huxvb{0}}}{\huxtpad{4pt}\raggedleft {\fontsize{9.5pt}{11.4pt}\selectfont 39.005 ***}\huxbpad{4pt}} &
\multicolumn{1}{r!{\huxvb{0}}}{\huxtpad{4pt}\raggedleft {\fontsize{9.5pt}{11.4pt}\selectfont 38.691 ***}\huxbpad{4pt}} &
\multicolumn{1}{r!{\huxvb{0}}}{\huxtpad{4pt}\raggedleft {\fontsize{9.5pt}{11.4pt}\selectfont 42.789 ***}\huxbpad{4pt}} \tabularnewline[-0.5pt]


\hhline{}
\arrayrulecolor{black}

\multicolumn{1}{!{\huxvb{0}}l!{\huxvb{0}}}{\huxtpad{4pt}\raggedright {\fontsize{9.5pt}{11.4pt}\selectfont }\huxbpad{4pt}} &
\multicolumn{1}{r!{\huxvb{0}}}{\huxtpad{4pt}\raggedleft {\fontsize{9.5pt}{11.4pt}\selectfont (1.372)~~~}\huxbpad{4pt}} &
\multicolumn{1}{r!{\huxvb{0}}}{\huxtpad{4pt}\raggedleft {\fontsize{9.5pt}{11.4pt}\selectfont (1.493)~~~}\huxbpad{4pt}} &
\multicolumn{1}{r!{\huxvb{0}}}{\huxtpad{4pt}\raggedleft {\fontsize{9.5pt}{11.4pt}\selectfont (1.498)~~~}\huxbpad{4pt}} \tabularnewline[-0.5pt]


\hhline{}
\arrayrulecolor{black}

\multicolumn{1}{!{\huxvb{0}}l!{\huxvb{0}}}{\huxtpad{4pt}\raggedright {\fontsize{9.5pt}{11.4pt}\selectfont sample\_tract51059491301}\huxbpad{4pt}} &
\multicolumn{1}{r!{\huxvb{0}}}{\huxtpad{4pt}\raggedleft {\fontsize{9.5pt}{11.4pt}\selectfont 19.804 ***}\huxbpad{4pt}} &
\multicolumn{1}{r!{\huxvb{0}}}{\huxtpad{4pt}\raggedleft {\fontsize{9.5pt}{11.4pt}\selectfont 19.236 ***}\huxbpad{4pt}} &
\multicolumn{1}{r!{\huxvb{0}}}{\huxtpad{4pt}\raggedleft {\fontsize{9.5pt}{11.4pt}\selectfont 20.874 ***}\huxbpad{4pt}} \tabularnewline[-0.5pt]


\hhline{}
\arrayrulecolor{black}

\multicolumn{1}{!{\huxvb{0}}l!{\huxvb{0}}}{\huxtpad{4pt}\raggedright {\fontsize{9.5pt}{11.4pt}\selectfont }\huxbpad{4pt}} &
\multicolumn{1}{r!{\huxvb{0}}}{\huxtpad{4pt}\raggedleft {\fontsize{9.5pt}{11.4pt}\selectfont (1.279)~~~}\huxbpad{4pt}} &
\multicolumn{1}{r!{\huxvb{0}}}{\huxtpad{4pt}\raggedleft {\fontsize{9.5pt}{11.4pt}\selectfont (1.361)~~~}\huxbpad{4pt}} &
\multicolumn{1}{r!{\huxvb{0}}}{\huxtpad{4pt}\raggedleft {\fontsize{9.5pt}{11.4pt}\selectfont (1.324)~~~}\huxbpad{4pt}} \tabularnewline[-0.5pt]


\hhline{}
\arrayrulecolor{black}

\multicolumn{1}{!{\huxvb{0}}l!{\huxvb{0}}}{\huxtpad{4pt}\raggedright {\fontsize{9.5pt}{11.4pt}\selectfont sample\_tract51059491601}\huxbpad{4pt}} &
\multicolumn{1}{r!{\huxvb{0}}}{\huxtpad{4pt}\raggedleft {\fontsize{9.5pt}{11.4pt}\selectfont 21.609 ***}\huxbpad{4pt}} &
\multicolumn{1}{r!{\huxvb{0}}}{\huxtpad{4pt}\raggedleft {\fontsize{9.5pt}{11.4pt}\selectfont 21.219 ***}\huxbpad{4pt}} &
\multicolumn{1}{r!{\huxvb{0}}}{\huxtpad{4pt}\raggedleft {\fontsize{9.5pt}{11.4pt}\selectfont 22.479 ***}\huxbpad{4pt}} \tabularnewline[-0.5pt]


\hhline{}
\arrayrulecolor{black}

\multicolumn{1}{!{\huxvb{0}}l!{\huxvb{0}}}{\huxtpad{4pt}\raggedright {\fontsize{9.5pt}{11.4pt}\selectfont }\huxbpad{4pt}} &
\multicolumn{1}{r!{\huxvb{0}}}{\huxtpad{4pt}\raggedleft {\fontsize{9.5pt}{11.4pt}\selectfont (1.233)~~~}\huxbpad{4pt}} &
\multicolumn{1}{r!{\huxvb{0}}}{\huxtpad{4pt}\raggedleft {\fontsize{9.5pt}{11.4pt}\selectfont (1.363)~~~}\huxbpad{4pt}} &
\multicolumn{1}{r!{\huxvb{0}}}{\huxtpad{4pt}\raggedleft {\fontsize{9.5pt}{11.4pt}\selectfont (1.209)~~~}\huxbpad{4pt}} \tabularnewline[-0.5pt]


\hhline{}
\arrayrulecolor{black}

\multicolumn{1}{!{\huxvb{0}}l!{\huxvb{0}}}{\huxtpad{4pt}\raggedright {\fontsize{9.5pt}{11.4pt}\selectfont sample\_tract51059491702}\huxbpad{4pt}} &
\multicolumn{1}{r!{\huxvb{0}}}{\huxtpad{4pt}\raggedleft {\fontsize{9.5pt}{11.4pt}\selectfont 23.199 ***}\huxbpad{4pt}} &
\multicolumn{1}{r!{\huxvb{0}}}{\huxtpad{4pt}\raggedleft {\fontsize{9.5pt}{11.4pt}\selectfont 23.046 ***}\huxbpad{4pt}} &
\multicolumn{1}{r!{\huxvb{0}}}{\huxtpad{4pt}\raggedleft {\fontsize{9.5pt}{11.4pt}\selectfont 23.529 ***}\huxbpad{4pt}} \tabularnewline[-0.5pt]


\hhline{}
\arrayrulecolor{black}

\multicolumn{1}{!{\huxvb{0}}l!{\huxvb{0}}}{\huxtpad{4pt}\raggedright {\fontsize{9.5pt}{11.4pt}\selectfont }\huxbpad{4pt}} &
\multicolumn{1}{r!{\huxvb{0}}}{\huxtpad{4pt}\raggedleft {\fontsize{9.5pt}{11.4pt}\selectfont (1.712)~~~}\huxbpad{4pt}} &
\multicolumn{1}{r!{\huxvb{0}}}{\huxtpad{4pt}\raggedleft {\fontsize{9.5pt}{11.4pt}\selectfont (1.842)~~~}\huxbpad{4pt}} &
\multicolumn{1}{r!{\huxvb{0}}}{\huxtpad{4pt}\raggedleft {\fontsize{9.5pt}{11.4pt}\selectfont (1.822)~~~}\huxbpad{4pt}} \tabularnewline[-0.5pt]


\hhline{}
\arrayrulecolor{black}

\multicolumn{1}{!{\huxvb{0}}l!{\huxvb{0}}}{\huxtpad{4pt}\raggedright {\fontsize{9.5pt}{11.4pt}\selectfont sample\_tract51059491801}\huxbpad{4pt}} &
\multicolumn{1}{r!{\huxvb{0}}}{\huxtpad{4pt}\raggedleft {\fontsize{9.5pt}{11.4pt}\selectfont -0.172~~~~}\huxbpad{4pt}} &
\multicolumn{1}{r!{\huxvb{0}}}{\huxtpad{4pt}\raggedleft {\fontsize{9.5pt}{11.4pt}\selectfont -0.309~~~~}\huxbpad{4pt}} &
\multicolumn{1}{r!{\huxvb{0}}}{\huxtpad{4pt}\raggedleft {\fontsize{9.5pt}{11.4pt}\selectfont 0.626~~~~}\huxbpad{4pt}} \tabularnewline[-0.5pt]


\hhline{}
\arrayrulecolor{black}

\multicolumn{1}{!{\huxvb{0}}l!{\huxvb{0}}}{\huxtpad{4pt}\raggedright {\fontsize{9.5pt}{11.4pt}\selectfont }\huxbpad{4pt}} &
\multicolumn{1}{r!{\huxvb{0}}}{\huxtpad{4pt}\raggedleft {\fontsize{9.5pt}{11.4pt}\selectfont (1.189)~~~}\huxbpad{4pt}} &
\multicolumn{1}{r!{\huxvb{0}}}{\huxtpad{4pt}\raggedleft {\fontsize{9.5pt}{11.4pt}\selectfont (1.320)~~~}\huxbpad{4pt}} &
\multicolumn{1}{r!{\huxvb{0}}}{\huxtpad{4pt}\raggedleft {\fontsize{9.5pt}{11.4pt}\selectfont (1.249)~~~}\huxbpad{4pt}} \tabularnewline[-0.5pt]


\hhline{}
\arrayrulecolor{black}

\multicolumn{1}{!{\huxvb{0}}l!{\huxvb{0}}}{\huxtpad{4pt}\raggedright {\fontsize{9.5pt}{11.4pt}\selectfont sample\_tract51059492400}\huxbpad{4pt}} &
\multicolumn{1}{r!{\huxvb{0}}}{\huxtpad{4pt}\raggedleft {\fontsize{9.5pt}{11.4pt}\selectfont 0.725~~~~}\huxbpad{4pt}} &
\multicolumn{1}{r!{\huxvb{0}}}{\huxtpad{4pt}\raggedleft {\fontsize{9.5pt}{11.4pt}\selectfont -0.422~~~~}\huxbpad{4pt}} &
\multicolumn{1}{r!{\huxvb{0}}}{\huxtpad{4pt}\raggedleft {\fontsize{9.5pt}{11.4pt}\selectfont 0.176~~~~}\huxbpad{4pt}} \tabularnewline[-0.5pt]


\hhline{}
\arrayrulecolor{black}

\multicolumn{1}{!{\huxvb{0}}l!{\huxvb{0}}}{\huxtpad{4pt}\raggedright {\fontsize{9.5pt}{11.4pt}\selectfont }\huxbpad{4pt}} &
\multicolumn{1}{r!{\huxvb{0}}}{\huxtpad{4pt}\raggedleft {\fontsize{9.5pt}{11.4pt}\selectfont (1.100)~~~}\huxbpad{4pt}} &
\multicolumn{1}{r!{\huxvb{0}}}{\huxtpad{4pt}\raggedleft {\fontsize{9.5pt}{11.4pt}\selectfont (1.528)~~~}\huxbpad{4pt}} &
\multicolumn{1}{r!{\huxvb{0}}}{\huxtpad{4pt}\raggedleft {\fontsize{9.5pt}{11.4pt}\selectfont (1.357)~~~}\huxbpad{4pt}} \tabularnewline[-0.5pt]


\hhline{}
\arrayrulecolor{black}

\multicolumn{1}{!{\huxvb{0}}l!{\huxvb{0}}}{\huxtpad{4pt}\raggedright {\fontsize{9.5pt}{11.4pt}\selectfont sample\_tract51510200102}\huxbpad{4pt}} &
\multicolumn{1}{r!{\huxvb{0}}}{\huxtpad{4pt}\raggedleft {\fontsize{9.5pt}{11.4pt}\selectfont 17.312 ***}\huxbpad{4pt}} &
\multicolumn{1}{r!{\huxvb{0}}}{\huxtpad{4pt}\raggedleft {\fontsize{9.5pt}{11.4pt}\selectfont 16.794 ***}\huxbpad{4pt}} &
\multicolumn{1}{r!{\huxvb{0}}}{\huxtpad{4pt}\raggedleft {\fontsize{9.5pt}{11.4pt}\selectfont 17.436 ***}\huxbpad{4pt}} \tabularnewline[-0.5pt]


\hhline{}
\arrayrulecolor{black}

\multicolumn{1}{!{\huxvb{0}}l!{\huxvb{0}}}{\huxtpad{4pt}\raggedright {\fontsize{9.5pt}{11.4pt}\selectfont }\huxbpad{4pt}} &
\multicolumn{1}{r!{\huxvb{0}}}{\huxtpad{4pt}\raggedleft {\fontsize{9.5pt}{11.4pt}\selectfont (1.895)~~~}\huxbpad{4pt}} &
\multicolumn{1}{r!{\huxvb{0}}}{\huxtpad{4pt}\raggedleft {\fontsize{9.5pt}{11.4pt}\selectfont (2.125)~~~}\huxbpad{4pt}} &
\multicolumn{1}{r!{\huxvb{0}}}{\huxtpad{4pt}\raggedleft {\fontsize{9.5pt}{11.4pt}\selectfont (1.732)~~~}\huxbpad{4pt}} \tabularnewline[-0.5pt]


\hhline{}
\arrayrulecolor{black}

\multicolumn{1}{!{\huxvb{0}}l!{\huxvb{0}}}{\huxtpad{4pt}\raggedright {\fontsize{9.5pt}{11.4pt}\selectfont sample\_tract51510200103}\huxbpad{4pt}} &
\multicolumn{1}{r!{\huxvb{0}}}{\huxtpad{4pt}\raggedleft {\fontsize{9.5pt}{11.4pt}\selectfont 1.223~~~~}\huxbpad{4pt}} &
\multicolumn{1}{r!{\huxvb{0}}}{\huxtpad{4pt}\raggedleft {\fontsize{9.5pt}{11.4pt}\selectfont 0.569~~~~}\huxbpad{4pt}} &
\multicolumn{1}{r!{\huxvb{0}}}{\huxtpad{4pt}\raggedleft {\fontsize{9.5pt}{11.4pt}\selectfont 0.407~~~~}\huxbpad{4pt}} \tabularnewline[-0.5pt]


\hhline{}
\arrayrulecolor{black}

\multicolumn{1}{!{\huxvb{0}}l!{\huxvb{0}}}{\huxtpad{4pt}\raggedright {\fontsize{9.5pt}{11.4pt}\selectfont }\huxbpad{4pt}} &
\multicolumn{1}{r!{\huxvb{0}}}{\huxtpad{4pt}\raggedleft {\fontsize{9.5pt}{11.4pt}\selectfont (0.990)~~~}\huxbpad{4pt}} &
\multicolumn{1}{r!{\huxvb{0}}}{\huxtpad{4pt}\raggedleft {\fontsize{9.5pt}{11.4pt}\selectfont (1.141)~~~}\huxbpad{4pt}} &
\multicolumn{1}{r!{\huxvb{0}}}{\huxtpad{4pt}\raggedleft {\fontsize{9.5pt}{11.4pt}\selectfont (1.239)~~~}\huxbpad{4pt}} \tabularnewline[-0.5pt]


\hhline{}
\arrayrulecolor{black}

\multicolumn{1}{!{\huxvb{0}}l!{\huxvb{0}}}{\huxtpad{4pt}\raggedright {\fontsize{9.5pt}{11.4pt}\selectfont sample\_tract51510200107}\huxbpad{4pt}} &
\multicolumn{1}{r!{\huxvb{0}}}{\huxtpad{4pt}\raggedleft {\fontsize{9.5pt}{11.4pt}\selectfont 1.169~~~~}\huxbpad{4pt}} &
\multicolumn{1}{r!{\huxvb{0}}}{\huxtpad{4pt}\raggedleft {\fontsize{9.5pt}{11.4pt}\selectfont 0.986~~~~}\huxbpad{4pt}} &
\multicolumn{1}{r!{\huxvb{0}}}{\huxtpad{4pt}\raggedleft {\fontsize{9.5pt}{11.4pt}\selectfont 1.617~~~~}\huxbpad{4pt}} \tabularnewline[-0.5pt]


\hhline{}
\arrayrulecolor{black}

\multicolumn{1}{!{\huxvb{0}}l!{\huxvb{0}}}{\huxtpad{4pt}\raggedright {\fontsize{9.5pt}{11.4pt}\selectfont }\huxbpad{4pt}} &
\multicolumn{1}{r!{\huxvb{0}}}{\huxtpad{4pt}\raggedleft {\fontsize{9.5pt}{11.4pt}\selectfont (1.123)~~~}\huxbpad{4pt}} &
\multicolumn{1}{r!{\huxvb{0}}}{\huxtpad{4pt}\raggedleft {\fontsize{9.5pt}{11.4pt}\selectfont (1.224)~~~}\huxbpad{4pt}} &
\multicolumn{1}{r!{\huxvb{0}}}{\huxtpad{4pt}\raggedleft {\fontsize{9.5pt}{11.4pt}\selectfont (1.165)~~~}\huxbpad{4pt}} \tabularnewline[-0.5pt]


\hhline{}
\arrayrulecolor{black}

\multicolumn{1}{!{\huxvb{0}}l!{\huxvb{0}}}{\huxtpad{4pt}\raggedright {\fontsize{9.5pt}{11.4pt}\selectfont sample\_tract51510200301}\huxbpad{4pt}} &
\multicolumn{1}{r!{\huxvb{0}}}{\huxtpad{4pt}\raggedleft {\fontsize{9.5pt}{11.4pt}\selectfont 20.240 ***}\huxbpad{4pt}} &
\multicolumn{1}{r!{\huxvb{0}}}{\huxtpad{4pt}\raggedleft {\fontsize{9.5pt}{11.4pt}\selectfont 20.075 ***}\huxbpad{4pt}} &
\multicolumn{1}{r!{\huxvb{0}}}{\huxtpad{4pt}\raggedleft {\fontsize{9.5pt}{11.4pt}\selectfont 20.817 ***}\huxbpad{4pt}} \tabularnewline[-0.5pt]


\hhline{}
\arrayrulecolor{black}

\multicolumn{1}{!{\huxvb{0}}l!{\huxvb{0}}}{\huxtpad{4pt}\raggedright {\fontsize{9.5pt}{11.4pt}\selectfont }\huxbpad{4pt}} &
\multicolumn{1}{r!{\huxvb{0}}}{\huxtpad{4pt}\raggedleft {\fontsize{9.5pt}{11.4pt}\selectfont (1.637)~~~}\huxbpad{4pt}} &
\multicolumn{1}{r!{\huxvb{0}}}{\huxtpad{4pt}\raggedleft {\fontsize{9.5pt}{11.4pt}\selectfont (1.653)~~~}\huxbpad{4pt}} &
\multicolumn{1}{r!{\huxvb{0}}}{\huxtpad{4pt}\raggedleft {\fontsize{9.5pt}{11.4pt}\selectfont (1.495)~~~}\huxbpad{4pt}} \tabularnewline[-0.5pt]


\hhline{}
\arrayrulecolor{black}

\multicolumn{1}{!{\huxvb{0}}l!{\huxvb{0}}}{\huxtpad{4pt}\raggedright {\fontsize{9.5pt}{11.4pt}\selectfont sample\_tract51510200303}\huxbpad{4pt}} &
\multicolumn{1}{r!{\huxvb{0}}}{\huxtpad{4pt}\raggedleft {\fontsize{9.5pt}{11.4pt}\selectfont 20.531 ***}\huxbpad{4pt}} &
\multicolumn{1}{r!{\huxvb{0}}}{\huxtpad{4pt}\raggedleft {\fontsize{9.5pt}{11.4pt}\selectfont 20.395 ***}\huxbpad{4pt}} &
\multicolumn{1}{r!{\huxvb{0}}}{\huxtpad{4pt}\raggedleft {\fontsize{9.5pt}{11.4pt}\selectfont 21.085 ***}\huxbpad{4pt}} \tabularnewline[-0.5pt]


\hhline{}
\arrayrulecolor{black}

\multicolumn{1}{!{\huxvb{0}}l!{\huxvb{0}}}{\huxtpad{4pt}\raggedright {\fontsize{9.5pt}{11.4pt}\selectfont }\huxbpad{4pt}} &
\multicolumn{1}{r!{\huxvb{0}}}{\huxtpad{4pt}\raggedleft {\fontsize{9.5pt}{11.4pt}\selectfont (1.208)~~~}\huxbpad{4pt}} &
\multicolumn{1}{r!{\huxvb{0}}}{\huxtpad{4pt}\raggedleft {\fontsize{9.5pt}{11.4pt}\selectfont (1.340)~~~}\huxbpad{4pt}} &
\multicolumn{1}{r!{\huxvb{0}}}{\huxtpad{4pt}\raggedleft {\fontsize{9.5pt}{11.4pt}\selectfont (1.190)~~~}\huxbpad{4pt}} \tabularnewline[-0.5pt]


\hhline{}
\arrayrulecolor{black}

\multicolumn{1}{!{\huxvb{0}}l!{\huxvb{0}}}{\huxtpad{4pt}\raggedright {\fontsize{9.5pt}{11.4pt}\selectfont sample\_tract51510200600}\huxbpad{4pt}} &
\multicolumn{1}{r!{\huxvb{0}}}{\huxtpad{4pt}\raggedleft {\fontsize{9.5pt}{11.4pt}\selectfont 19.196 ***}\huxbpad{4pt}} &
\multicolumn{1}{r!{\huxvb{0}}}{\huxtpad{4pt}\raggedleft {\fontsize{9.5pt}{11.4pt}\selectfont 18.695 ***}\huxbpad{4pt}} &
\multicolumn{1}{r!{\huxvb{0}}}{\huxtpad{4pt}\raggedleft {\fontsize{9.5pt}{11.4pt}\selectfont 20.469 ***}\huxbpad{4pt}} \tabularnewline[-0.5pt]


\hhline{}
\arrayrulecolor{black}

\multicolumn{1}{!{\huxvb{0}}l!{\huxvb{0}}}{\huxtpad{4pt}\raggedright {\fontsize{9.5pt}{11.4pt}\selectfont }\huxbpad{4pt}} &
\multicolumn{1}{r!{\huxvb{0}}}{\huxtpad{4pt}\raggedleft {\fontsize{9.5pt}{11.4pt}\selectfont (1.366)~~~}\huxbpad{4pt}} &
\multicolumn{1}{r!{\huxvb{0}}}{\huxtpad{4pt}\raggedleft {\fontsize{9.5pt}{11.4pt}\selectfont (1.473)~~~}\huxbpad{4pt}} &
\multicolumn{1}{r!{\huxvb{0}}}{\huxtpad{4pt}\raggedleft {\fontsize{9.5pt}{11.4pt}\selectfont (1.429)~~~}\huxbpad{4pt}} \tabularnewline[-0.5pt]


\hhline{}
\arrayrulecolor{black}

\multicolumn{1}{!{\huxvb{0}}l!{\huxvb{0}}}{\huxtpad{4pt}\raggedright {\fontsize{9.5pt}{11.4pt}\selectfont age}\huxbpad{4pt}} &
\multicolumn{1}{r!{\huxvb{0}}}{\huxtpad{4pt}\raggedleft {\fontsize{9.5pt}{11.4pt}\selectfont ~~~~~~~~}\huxbpad{4pt}} &
\multicolumn{1}{r!{\huxvb{0}}}{\huxtpad{4pt}\raggedleft {\fontsize{9.5pt}{11.4pt}\selectfont -0.002~~~~}\huxbpad{4pt}} &
\multicolumn{1}{r!{\huxvb{0}}}{\huxtpad{4pt}\raggedleft {\fontsize{9.5pt}{11.4pt}\selectfont 0.030 *~~}\huxbpad{4pt}} \tabularnewline[-0.5pt]


\hhline{}
\arrayrulecolor{black}

\multicolumn{1}{!{\huxvb{0}}l!{\huxvb{0}}}{\huxtpad{4pt}\raggedright {\fontsize{9.5pt}{11.4pt}\selectfont }\huxbpad{4pt}} &
\multicolumn{1}{r!{\huxvb{0}}}{\huxtpad{4pt}\raggedleft {\fontsize{9.5pt}{11.4pt}\selectfont ~~~~~~~~}\huxbpad{4pt}} &
\multicolumn{1}{r!{\huxvb{0}}}{\huxtpad{4pt}\raggedleft {\fontsize{9.5pt}{11.4pt}\selectfont (0.010)~~~}\huxbpad{4pt}} &
\multicolumn{1}{r!{\huxvb{0}}}{\huxtpad{4pt}\raggedleft {\fontsize{9.5pt}{11.4pt}\selectfont (0.013)~~~}\huxbpad{4pt}} \tabularnewline[-0.5pt]


\hhline{}
\arrayrulecolor{black}

\multicolumn{1}{!{\huxvb{0}}l!{\huxvb{0}}}{\huxtpad{4pt}\raggedright {\fontsize{9.5pt}{11.4pt}\selectfont forbornTRUE}\huxbpad{4pt}} &
\multicolumn{1}{r!{\huxvb{0}}}{\huxtpad{4pt}\raggedleft {\fontsize{9.5pt}{11.4pt}\selectfont ~~~~~~~~}\huxbpad{4pt}} &
\multicolumn{1}{r!{\huxvb{0}}}{\huxtpad{4pt}\raggedleft {\fontsize{9.5pt}{11.4pt}\selectfont 0.586~~~~}\huxbpad{4pt}} &
\multicolumn{1}{r!{\huxvb{0}}}{\huxtpad{4pt}\raggedleft {\fontsize{9.5pt}{11.4pt}\selectfont 0.618~~~~}\huxbpad{4pt}} \tabularnewline[-0.5pt]


\hhline{}
\arrayrulecolor{black}

\multicolumn{1}{!{\huxvb{0}}l!{\huxvb{0}}}{\huxtpad{4pt}\raggedright {\fontsize{9.5pt}{11.4pt}\selectfont }\huxbpad{4pt}} &
\multicolumn{1}{r!{\huxvb{0}}}{\huxtpad{4pt}\raggedleft {\fontsize{9.5pt}{11.4pt}\selectfont ~~~~~~~~}\huxbpad{4pt}} &
\multicolumn{1}{r!{\huxvb{0}}}{\huxtpad{4pt}\raggedleft {\fontsize{9.5pt}{11.4pt}\selectfont (0.414)~~~}\huxbpad{4pt}} &
\multicolumn{1}{r!{\huxvb{0}}}{\huxtpad{4pt}\raggedleft {\fontsize{9.5pt}{11.4pt}\selectfont (0.409)~~~}\huxbpad{4pt}} \tabularnewline[-0.5pt]


\hhline{}
\arrayrulecolor{black}

\multicolumn{1}{!{\huxvb{0}}l!{\huxvb{0}}}{\huxtpad{4pt}\raggedright {\fontsize{9.5pt}{11.4pt}\selectfont manTRUE}\huxbpad{4pt}} &
\multicolumn{1}{r!{\huxvb{0}}}{\huxtpad{4pt}\raggedleft {\fontsize{9.5pt}{11.4pt}\selectfont ~~~~~~~~}\huxbpad{4pt}} &
\multicolumn{1}{r!{\huxvb{0}}}{\huxtpad{4pt}\raggedleft {\fontsize{9.5pt}{11.4pt}\selectfont 0.195~~~~}\huxbpad{4pt}} &
\multicolumn{1}{r!{\huxvb{0}}}{\huxtpad{4pt}\raggedleft {\fontsize{9.5pt}{11.4pt}\selectfont 0.223~~~~}\huxbpad{4pt}} \tabularnewline[-0.5pt]


\hhline{}
\arrayrulecolor{black}

\multicolumn{1}{!{\huxvb{0}}l!{\huxvb{0}}}{\huxtpad{4pt}\raggedright {\fontsize{9.5pt}{11.4pt}\selectfont }\huxbpad{4pt}} &
\multicolumn{1}{r!{\huxvb{0}}}{\huxtpad{4pt}\raggedleft {\fontsize{9.5pt}{11.4pt}\selectfont ~~~~~~~~}\huxbpad{4pt}} &
\multicolumn{1}{r!{\huxvb{0}}}{\huxtpad{4pt}\raggedleft {\fontsize{9.5pt}{11.4pt}\selectfont (0.318)~~~}\huxbpad{4pt}} &
\multicolumn{1}{r!{\huxvb{0}}}{\huxtpad{4pt}\raggedleft {\fontsize{9.5pt}{11.4pt}\selectfont (0.316)~~~}\huxbpad{4pt}} \tabularnewline[-0.5pt]


\hhline{}
\arrayrulecolor{black}

\multicolumn{1}{!{\huxvb{0}}l!{\huxvb{0}}}{\huxtpad{4pt}\raggedright {\fontsize{9.5pt}{11.4pt}\selectfont kidsTRUE}\huxbpad{4pt}} &
\multicolumn{1}{r!{\huxvb{0}}}{\huxtpad{4pt}\raggedleft {\fontsize{9.5pt}{11.4pt}\selectfont ~~~~~~~~}\huxbpad{4pt}} &
\multicolumn{1}{r!{\huxvb{0}}}{\huxtpad{4pt}\raggedleft {\fontsize{9.5pt}{11.4pt}\selectfont 0.067~~~~}\huxbpad{4pt}} &
\multicolumn{1}{r!{\huxvb{0}}}{\huxtpad{4pt}\raggedleft {\fontsize{9.5pt}{11.4pt}\selectfont -0.088~~~~}\huxbpad{4pt}} \tabularnewline[-0.5pt]


\hhline{}
\arrayrulecolor{black}

\multicolumn{1}{!{\huxvb{0}}l!{\huxvb{0}}}{\huxtpad{4pt}\raggedright {\fontsize{9.5pt}{11.4pt}\selectfont }\huxbpad{4pt}} &
\multicolumn{1}{r!{\huxvb{0}}}{\huxtpad{4pt}\raggedleft {\fontsize{9.5pt}{11.4pt}\selectfont ~~~~~~~~}\huxbpad{4pt}} &
\multicolumn{1}{r!{\huxvb{0}}}{\huxtpad{4pt}\raggedleft {\fontsize{9.5pt}{11.4pt}\selectfont (0.326)~~~}\huxbpad{4pt}} &
\multicolumn{1}{r!{\huxvb{0}}}{\huxtpad{4pt}\raggedleft {\fontsize{9.5pt}{11.4pt}\selectfont (0.346)~~~}\huxbpad{4pt}} \tabularnewline[-0.5pt]


\hhline{}
\arrayrulecolor{black}

\multicolumn{1}{!{\huxvb{0}}l!{\huxvb{0}}}{\huxtpad{4pt}\raggedright {\fontsize{9.5pt}{11.4pt}\selectfont marriedTRUE}\huxbpad{4pt}} &
\multicolumn{1}{r!{\huxvb{0}}}{\huxtpad{4pt}\raggedleft {\fontsize{9.5pt}{11.4pt}\selectfont ~~~~~~~~}\huxbpad{4pt}} &
\multicolumn{1}{r!{\huxvb{0}}}{\huxtpad{4pt}\raggedleft {\fontsize{9.5pt}{11.4pt}\selectfont 0.336~~~~}\huxbpad{4pt}} &
\multicolumn{1}{r!{\huxvb{0}}}{\huxtpad{4pt}\raggedleft {\fontsize{9.5pt}{11.4pt}\selectfont -0.155~~~~}\huxbpad{4pt}} \tabularnewline[-0.5pt]


\hhline{}
\arrayrulecolor{black}

\multicolumn{1}{!{\huxvb{0}}l!{\huxvb{0}}}{\huxtpad{4pt}\raggedright {\fontsize{9.5pt}{11.4pt}\selectfont }\huxbpad{4pt}} &
\multicolumn{1}{r!{\huxvb{0}}}{\huxtpad{4pt}\raggedleft {\fontsize{9.5pt}{11.4pt}\selectfont ~~~~~~~~}\huxbpad{4pt}} &
\multicolumn{1}{r!{\huxvb{0}}}{\huxtpad{4pt}\raggedleft {\fontsize{9.5pt}{11.4pt}\selectfont (0.331)~~~}\huxbpad{4pt}} &
\multicolumn{1}{r!{\huxvb{0}}}{\huxtpad{4pt}\raggedleft {\fontsize{9.5pt}{11.4pt}\selectfont (0.330)~~~}\huxbpad{4pt}} \tabularnewline[-0.5pt]


\hhline{}
\arrayrulecolor{black}

\multicolumn{1}{!{\huxvb{0}}l!{\huxvb{0}}}{\huxtpad{4pt}\raggedright {\fontsize{9.5pt}{11.4pt}\selectfont educ.L}\huxbpad{4pt}} &
\multicolumn{1}{r!{\huxvb{0}}}{\huxtpad{4pt}\raggedleft {\fontsize{9.5pt}{11.4pt}\selectfont ~~~~~~~~}\huxbpad{4pt}} &
\multicolumn{1}{r!{\huxvb{0}}}{\huxtpad{4pt}\raggedleft {\fontsize{9.5pt}{11.4pt}\selectfont -1.188~~~~}\huxbpad{4pt}} &
\multicolumn{1}{r!{\huxvb{0}}}{\huxtpad{4pt}\raggedleft {\fontsize{9.5pt}{11.4pt}\selectfont -1.222 *~~}\huxbpad{4pt}} \tabularnewline[-0.5pt]


\hhline{}
\arrayrulecolor{black}

\multicolumn{1}{!{\huxvb{0}}l!{\huxvb{0}}}{\huxtpad{4pt}\raggedright {\fontsize{9.5pt}{11.4pt}\selectfont }\huxbpad{4pt}} &
\multicolumn{1}{r!{\huxvb{0}}}{\huxtpad{4pt}\raggedleft {\fontsize{9.5pt}{11.4pt}\selectfont ~~~~~~~~}\huxbpad{4pt}} &
\multicolumn{1}{r!{\huxvb{0}}}{\huxtpad{4pt}\raggedleft {\fontsize{9.5pt}{11.4pt}\selectfont (0.607)~~~}\huxbpad{4pt}} &
\multicolumn{1}{r!{\huxvb{0}}}{\huxtpad{4pt}\raggedleft {\fontsize{9.5pt}{11.4pt}\selectfont (0.559)~~~}\huxbpad{4pt}} \tabularnewline[-0.5pt]


\hhline{}
\arrayrulecolor{black}

\multicolumn{1}{!{\huxvb{0}}l!{\huxvb{0}}}{\huxtpad{4pt}\raggedright {\fontsize{9.5pt}{11.4pt}\selectfont educ.Q}\huxbpad{4pt}} &
\multicolumn{1}{r!{\huxvb{0}}}{\huxtpad{4pt}\raggedleft {\fontsize{9.5pt}{11.4pt}\selectfont ~~~~~~~~}\huxbpad{4pt}} &
\multicolumn{1}{r!{\huxvb{0}}}{\huxtpad{4pt}\raggedleft {\fontsize{9.5pt}{11.4pt}\selectfont 0.455~~~~}\huxbpad{4pt}} &
\multicolumn{1}{r!{\huxvb{0}}}{\huxtpad{4pt}\raggedleft {\fontsize{9.5pt}{11.4pt}\selectfont 0.420~~~~}\huxbpad{4pt}} \tabularnewline[-0.5pt]


\hhline{}
\arrayrulecolor{black}

\multicolumn{1}{!{\huxvb{0}}l!{\huxvb{0}}}{\huxtpad{4pt}\raggedright {\fontsize{9.5pt}{11.4pt}\selectfont }\huxbpad{4pt}} &
\multicolumn{1}{r!{\huxvb{0}}}{\huxtpad{4pt}\raggedleft {\fontsize{9.5pt}{11.4pt}\selectfont ~~~~~~~~}\huxbpad{4pt}} &
\multicolumn{1}{r!{\huxvb{0}}}{\huxtpad{4pt}\raggedleft {\fontsize{9.5pt}{11.4pt}\selectfont (0.545)~~~}\huxbpad{4pt}} &
\multicolumn{1}{r!{\huxvb{0}}}{\huxtpad{4pt}\raggedleft {\fontsize{9.5pt}{11.4pt}\selectfont (0.532)~~~}\huxbpad{4pt}} \tabularnewline[-0.5pt]


\hhline{}
\arrayrulecolor{black}

\multicolumn{1}{!{\huxvb{0}}l!{\huxvb{0}}}{\huxtpad{4pt}\raggedright {\fontsize{9.5pt}{11.4pt}\selectfont educ.C}\huxbpad{4pt}} &
\multicolumn{1}{r!{\huxvb{0}}}{\huxtpad{4pt}\raggedleft {\fontsize{9.5pt}{11.4pt}\selectfont ~~~~~~~~}\huxbpad{4pt}} &
\multicolumn{1}{r!{\huxvb{0}}}{\huxtpad{4pt}\raggedleft {\fontsize{9.5pt}{11.4pt}\selectfont 0.083~~~~}\huxbpad{4pt}} &
\multicolumn{1}{r!{\huxvb{0}}}{\huxtpad{4pt}\raggedleft {\fontsize{9.5pt}{11.4pt}\selectfont -0.020~~~~}\huxbpad{4pt}} \tabularnewline[-0.5pt]


\hhline{}
\arrayrulecolor{black}

\multicolumn{1}{!{\huxvb{0}}l!{\huxvb{0}}}{\huxtpad{4pt}\raggedright {\fontsize{9.5pt}{11.4pt}\selectfont }\huxbpad{4pt}} &
\multicolumn{1}{r!{\huxvb{0}}}{\huxtpad{4pt}\raggedleft {\fontsize{9.5pt}{11.4pt}\selectfont ~~~~~~~~}\huxbpad{4pt}} &
\multicolumn{1}{r!{\huxvb{0}}}{\huxtpad{4pt}\raggedleft {\fontsize{9.5pt}{11.4pt}\selectfont (0.460)~~~}\huxbpad{4pt}} &
\multicolumn{1}{r!{\huxvb{0}}}{\huxtpad{4pt}\raggedleft {\fontsize{9.5pt}{11.4pt}\selectfont (0.447)~~~}\huxbpad{4pt}} \tabularnewline[-0.5pt]


\hhline{}
\arrayrulecolor{black}

\multicolumn{1}{!{\huxvb{0}}l!{\huxvb{0}}}{\huxtpad{4pt}\raggedright {\fontsize{9.5pt}{11.4pt}\selectfont educ\verb|^|4}\huxbpad{4pt}} &
\multicolumn{1}{r!{\huxvb{0}}}{\huxtpad{4pt}\raggedleft {\fontsize{9.5pt}{11.4pt}\selectfont ~~~~~~~~}\huxbpad{4pt}} &
\multicolumn{1}{r!{\huxvb{0}}}{\huxtpad{4pt}\raggedleft {\fontsize{9.5pt}{11.4pt}\selectfont 0.025~~~~}\huxbpad{4pt}} &
\multicolumn{1}{r!{\huxvb{0}}}{\huxtpad{4pt}\raggedleft {\fontsize{9.5pt}{11.4pt}\selectfont 0.073~~~~}\huxbpad{4pt}} \tabularnewline[-0.5pt]


\hhline{}
\arrayrulecolor{black}

\multicolumn{1}{!{\huxvb{0}}l!{\huxvb{0}}}{\huxtpad{4pt}\raggedright {\fontsize{9.5pt}{11.4pt}\selectfont }\huxbpad{4pt}} &
\multicolumn{1}{r!{\huxvb{0}}}{\huxtpad{4pt}\raggedleft {\fontsize{9.5pt}{11.4pt}\selectfont ~~~~~~~~}\huxbpad{4pt}} &
\multicolumn{1}{r!{\huxvb{0}}}{\huxtpad{4pt}\raggedleft {\fontsize{9.5pt}{11.4pt}\selectfont (0.333)~~~}\huxbpad{4pt}} &
\multicolumn{1}{r!{\huxvb{0}}}{\huxtpad{4pt}\raggedleft {\fontsize{9.5pt}{11.4pt}\selectfont (0.354)~~~}\huxbpad{4pt}} \tabularnewline[-0.5pt]


\hhline{}
\arrayrulecolor{black}

\multicolumn{1}{!{\huxvb{0}}l!{\huxvb{0}}}{\huxtpad{4pt}\raggedright {\fontsize{9.5pt}{11.4pt}\selectfont nhdyrs}\huxbpad{4pt}} &
\multicolumn{1}{r!{\huxvb{0}}}{\huxtpad{4pt}\raggedleft {\fontsize{9.5pt}{11.4pt}\selectfont ~~~~~~~~}\huxbpad{4pt}} &
\multicolumn{1}{r!{\huxvb{0}}}{\huxtpad{4pt}\raggedleft {\fontsize{9.5pt}{11.4pt}\selectfont ~~~~~~~~}\huxbpad{4pt}} &
\multicolumn{1}{r!{\huxvb{0}}}{\huxtpad{4pt}\raggedleft {\fontsize{9.5pt}{11.4pt}\selectfont -0.090 ***}\huxbpad{4pt}} \tabularnewline[-0.5pt]


\hhline{}
\arrayrulecolor{black}

\multicolumn{1}{!{\huxvb{0}}l!{\huxvb{0}}}{\huxtpad{4pt}\raggedright {\fontsize{9.5pt}{11.4pt}\selectfont }\huxbpad{4pt}} &
\multicolumn{1}{r!{\huxvb{0}}}{\huxtpad{4pt}\raggedleft {\fontsize{9.5pt}{11.4pt}\selectfont ~~~~~~~~}\huxbpad{4pt}} &
\multicolumn{1}{r!{\huxvb{0}}}{\huxtpad{4pt}\raggedleft {\fontsize{9.5pt}{11.4pt}\selectfont ~~~~~~~~}\huxbpad{4pt}} &
\multicolumn{1}{r!{\huxvb{0}}}{\huxtpad{4pt}\raggedleft {\fontsize{9.5pt}{11.4pt}\selectfont (0.021)~~~}\huxbpad{4pt}} \tabularnewline[-0.5pt]


\hhline{}
\arrayrulecolor{black}

\multicolumn{1}{!{\huxvb{0}}l!{\huxvb{0}}}{\huxtpad{4pt}\raggedright {\fontsize{9.5pt}{11.4pt}\selectfont nhdsize.L}\huxbpad{4pt}} &
\multicolumn{1}{r!{\huxvb{0}}}{\huxtpad{4pt}\raggedleft {\fontsize{9.5pt}{11.4pt}\selectfont ~~~~~~~~}\huxbpad{4pt}} &
\multicolumn{1}{r!{\huxvb{0}}}{\huxtpad{4pt}\raggedleft {\fontsize{9.5pt}{11.4pt}\selectfont ~~~~~~~~}\huxbpad{4pt}} &
\multicolumn{1}{r!{\huxvb{0}}}{\huxtpad{4pt}\raggedleft {\fontsize{9.5pt}{11.4pt}\selectfont 0.129~~~~}\huxbpad{4pt}} \tabularnewline[-0.5pt]


\hhline{}
\arrayrulecolor{black}

\multicolumn{1}{!{\huxvb{0}}l!{\huxvb{0}}}{\huxtpad{4pt}\raggedright {\fontsize{9.5pt}{11.4pt}\selectfont }\huxbpad{4pt}} &
\multicolumn{1}{r!{\huxvb{0}}}{\huxtpad{4pt}\raggedleft {\fontsize{9.5pt}{11.4pt}\selectfont ~~~~~~~~}\huxbpad{4pt}} &
\multicolumn{1}{r!{\huxvb{0}}}{\huxtpad{4pt}\raggedleft {\fontsize{9.5pt}{11.4pt}\selectfont ~~~~~~~~}\huxbpad{4pt}} &
\multicolumn{1}{r!{\huxvb{0}}}{\huxtpad{4pt}\raggedleft {\fontsize{9.5pt}{11.4pt}\selectfont (0.493)~~~}\huxbpad{4pt}} \tabularnewline[-0.5pt]


\hhline{}
\arrayrulecolor{black}

\multicolumn{1}{!{\huxvb{0}}l!{\huxvb{0}}}{\huxtpad{4pt}\raggedright {\fontsize{9.5pt}{11.4pt}\selectfont nhdsize.Q}\huxbpad{4pt}} &
\multicolumn{1}{r!{\huxvb{0}}}{\huxtpad{4pt}\raggedleft {\fontsize{9.5pt}{11.4pt}\selectfont ~~~~~~~~}\huxbpad{4pt}} &
\multicolumn{1}{r!{\huxvb{0}}}{\huxtpad{4pt}\raggedleft {\fontsize{9.5pt}{11.4pt}\selectfont ~~~~~~~~}\huxbpad{4pt}} &
\multicolumn{1}{r!{\huxvb{0}}}{\huxtpad{4pt}\raggedleft {\fontsize{9.5pt}{11.4pt}\selectfont -0.411~~~~}\huxbpad{4pt}} \tabularnewline[-0.5pt]


\hhline{}
\arrayrulecolor{black}

\multicolumn{1}{!{\huxvb{0}}l!{\huxvb{0}}}{\huxtpad{4pt}\raggedright {\fontsize{9.5pt}{11.4pt}\selectfont }\huxbpad{4pt}} &
\multicolumn{1}{r!{\huxvb{0}}}{\huxtpad{4pt}\raggedleft {\fontsize{9.5pt}{11.4pt}\selectfont ~~~~~~~~}\huxbpad{4pt}} &
\multicolumn{1}{r!{\huxvb{0}}}{\huxtpad{4pt}\raggedleft {\fontsize{9.5pt}{11.4pt}\selectfont ~~~~~~~~}\huxbpad{4pt}} &
\multicolumn{1}{r!{\huxvb{0}}}{\huxtpad{4pt}\raggedleft {\fontsize{9.5pt}{11.4pt}\selectfont (0.330)~~~}\huxbpad{4pt}} \tabularnewline[-0.5pt]


\hhline{}
\arrayrulecolor{black}

\multicolumn{1}{!{\huxvb{0}}l!{\huxvb{0}}}{\huxtpad{4pt}\raggedright {\fontsize{9.5pt}{11.4pt}\selectfont extremely\_satisfied}\huxbpad{4pt}} &
\multicolumn{1}{r!{\huxvb{0}}}{\huxtpad{4pt}\raggedleft {\fontsize{9.5pt}{11.4pt}\selectfont ~~~~~~~~}\huxbpad{4pt}} &
\multicolumn{1}{r!{\huxvb{0}}}{\huxtpad{4pt}\raggedleft {\fontsize{9.5pt}{11.4pt}\selectfont ~~~~~~~~}\huxbpad{4pt}} &
\multicolumn{1}{r!{\huxvb{0}}}{\huxtpad{4pt}\raggedleft {\fontsize{9.5pt}{11.4pt}\selectfont 1.377 ***}\huxbpad{4pt}} \tabularnewline[-0.5pt]


\hhline{}
\arrayrulecolor{black}

\multicolumn{1}{!{\huxvb{0}}l!{\huxvb{0}}}{\huxtpad{4pt}\raggedright {\fontsize{9.5pt}{11.4pt}\selectfont }\huxbpad{4pt}} &
\multicolumn{1}{r!{\huxvb{0}}}{\huxtpad{4pt}\raggedleft {\fontsize{9.5pt}{11.4pt}\selectfont ~~~~~~~~}\huxbpad{4pt}} &
\multicolumn{1}{r!{\huxvb{0}}}{\huxtpad{4pt}\raggedleft {\fontsize{9.5pt}{11.4pt}\selectfont ~~~~~~~~}\huxbpad{4pt}} &
\multicolumn{1}{r!{\huxvb{0}}}{\huxtpad{4pt}\raggedleft {\fontsize{9.5pt}{11.4pt}\selectfont (0.393)~~~}\huxbpad{4pt}} \tabularnewline[-0.5pt]


\hhline{>{\huxb{1}}->{\huxb{1}}->{\huxb{1}}->{\huxb{1}}-}
\arrayrulecolor{black}

\multicolumn{1}{!{\huxvb{0}}l!{\huxvb{0}}}{\huxtpad{4pt}\raggedright {\fontsize{9.5pt}{11.4pt}\selectfont nobs}\huxbpad{4pt}} &
\multicolumn{1}{r!{\huxvb{0}}}{\huxtpad{4pt}\raggedleft {\fontsize{9.5pt}{11.4pt}\selectfont ~~~~~~~~}\huxbpad{4pt}} &
\multicolumn{1}{r!{\huxvb{0}}}{\huxtpad{4pt}\raggedleft {\fontsize{9.5pt}{11.4pt}\selectfont ~~~~~~~~}\huxbpad{4pt}} &
\multicolumn{1}{r!{\huxvb{0}}}{\huxtpad{4pt}\raggedleft {\fontsize{9.5pt}{11.4pt}\selectfont ~~~~~~~~}\huxbpad{4pt}} \tabularnewline[-0.5pt]


\hhline{>{\huxb{0.8}}->{\huxb{0.8}}->{\huxb{0.8}}->{\huxb{0.8}}-}
\arrayrulecolor{black}

\multicolumn{4}{!{\huxvb{0}}p{0.5\textwidth+6\tabcolsep}!{\huxvb{0}}}{\parbox[b]{0.5\textwidth+6\tabcolsep-4pt-4pt}{\huxtpad{4pt}\raggedright {\fontsize{9.5pt}{11.4pt}\selectfont  *** p $<$ 0.001;  ** p $<$ 0.01;  * p $<$ 0.05.}\huxbpad{4pt}}} \tabularnewline[-0.5pt]


\hhline{}
\arrayrulecolor{black}
\end{tabularx}
\end{table}


\begin{table}[h]
\centering\captionsetup{justification=centering,singlelinecheck=off}
\caption{Estimated coefficients predicting unfamiliarity with selected multiethnic communities}
\label{tab:knowledge}

    \providecommand{\huxb}[2][0,0,0]{\arrayrulecolor[RGB]{#1}\global\arrayrulewidth=#2pt}
    \providecommand{\huxvb}[2][0,0,0]{\color[RGB]{#1}\vrule width #2pt}
    \providecommand{\huxtpad}[1]{\rule{0pt}{\baselineskip+#1}}
    \providecommand{\huxbpad}[1]{\rule[-#1]{0pt}{#1}}
  \begin{tabularx}{0.5\textwidth}{p{0.125\textwidth} p{0.125\textwidth} p{0.125\textwidth} p{0.125\textwidth}}


\hhline{>{\huxb{0.8}}->{\huxb{0.8}}->{\huxb{0.8}}->{\huxb{0.8}}-}
\arrayrulecolor{black}

\multicolumn{1}{!{\huxvb{0}}c!{\huxvb{0}}}{\huxtpad{4pt}\centering {\fontsize{9.5pt}{11.4pt}\selectfont }\huxbpad{4pt}} &
\multicolumn{1}{c!{\huxvb{0}}}{\huxtpad{4pt}\centering {\fontsize{9.5pt}{11.4pt}\selectfont Herndon}\huxbpad{4pt}} &
\multicolumn{1}{c!{\huxvb{0}}}{\huxtpad{4pt}\centering {\fontsize{9.5pt}{11.4pt}\selectfont ~~~~Germantown}\huxbpad{4pt}} &
\multicolumn{1}{c!{\huxvb{0}}}{\huxtpad{4pt}\centering {\fontsize{9.5pt}{11.4pt}\selectfont ~~~~~~Wheaton}\huxbpad{4pt}} \tabularnewline[-0.5pt]


\hhline{>{\huxb{1}}->{\huxb{1}}->{\huxb{1}}->{\huxb{1}}-}
\arrayrulecolor{black}

\multicolumn{1}{!{\huxvb{0}}l!{\huxvb{0}}}{\huxtpad{4pt}\raggedright {\fontsize{9.5pt}{11.4pt}\selectfont (Intercept)}\huxbpad{4pt}} &
\multicolumn{1}{r!{\huxvb{0}}}{\huxtpad{4pt}\raggedleft {\fontsize{9.5pt}{11.4pt}\selectfont -2.296 *}\huxbpad{4pt}} &
\multicolumn{1}{r!{\huxvb{0}}}{\huxtpad{4pt}\raggedleft {\fontsize{9.5pt}{11.4pt}\selectfont -0.315~}\huxbpad{4pt}} &
\multicolumn{1}{r!{\huxvb{0}}}{\huxtpad{4pt}\raggedleft {\fontsize{9.5pt}{11.4pt}\selectfont 0.559~}\huxbpad{4pt}} \tabularnewline[-0.5pt]


\hhline{}
\arrayrulecolor{black}

\multicolumn{1}{!{\huxvb{0}}l!{\huxvb{0}}}{\huxtpad{4pt}\raggedright {\fontsize{9.5pt}{11.4pt}\selectfont }\huxbpad{4pt}} &
\multicolumn{1}{r!{\huxvb{0}}}{\huxtpad{4pt}\raggedleft {\fontsize{9.5pt}{11.4pt}\selectfont (1.011)~}\huxbpad{4pt}} &
\multicolumn{1}{r!{\huxvb{0}}}{\huxtpad{4pt}\raggedleft {\fontsize{9.5pt}{11.4pt}\selectfont (0.925)}\huxbpad{4pt}} &
\multicolumn{1}{r!{\huxvb{0}}}{\huxtpad{4pt}\raggedleft {\fontsize{9.5pt}{11.4pt}\selectfont (0.982)}\huxbpad{4pt}} \tabularnewline[-0.5pt]


\hhline{}
\arrayrulecolor{black}

\multicolumn{1}{!{\huxvb{0}}l!{\huxvb{0}}}{\huxtpad{4pt}\raggedright {\fontsize{9.5pt}{11.4pt}\selectfont Race}\huxbpad{4pt}} &
\multicolumn{1}{r!{\huxvb{0}}}{\huxtpad{4pt}\raggedleft {\fontsize{9.5pt}{11.4pt}\selectfont ~~~~~~}\huxbpad{4pt}} &
\multicolumn{1}{r!{\huxvb{0}}}{\huxtpad{4pt}\raggedleft {\fontsize{9.5pt}{11.4pt}\selectfont ~~~~~}\huxbpad{4pt}} &
\multicolumn{1}{r!{\huxvb{0}}}{\huxtpad{4pt}\raggedleft {\fontsize{9.5pt}{11.4pt}\selectfont ~~~~~}\huxbpad{4pt}} \tabularnewline[-0.5pt]


\hhline{}
\arrayrulecolor{black}

\multicolumn{1}{!{\huxvb{0}}l!{\huxvb{0}}}{\huxtpad{4pt}\raggedright {\fontsize{9.5pt}{11.4pt}\selectfont Asian}\huxbpad{4pt}} &
\multicolumn{1}{r!{\huxvb{0}}}{\huxtpad{4pt}\raggedleft {\fontsize{9.5pt}{11.4pt}\selectfont 0.151~~}\huxbpad{4pt}} &
\multicolumn{1}{r!{\huxvb{0}}}{\huxtpad{4pt}\raggedleft {\fontsize{9.5pt}{11.4pt}\selectfont 0.008~}\huxbpad{4pt}} &
\multicolumn{1}{r!{\huxvb{0}}}{\huxtpad{4pt}\raggedleft {\fontsize{9.5pt}{11.4pt}\selectfont 0.655~}\huxbpad{4pt}} \tabularnewline[-0.5pt]


\hhline{}
\arrayrulecolor{black}

\multicolumn{1}{!{\huxvb{0}}l!{\huxvb{0}}}{\huxtpad{4pt}\raggedright {\fontsize{9.5pt}{11.4pt}\selectfont }\huxbpad{4pt}} &
\multicolumn{1}{r!{\huxvb{0}}}{\huxtpad{4pt}\raggedleft {\fontsize{9.5pt}{11.4pt}\selectfont (0.408)~}\huxbpad{4pt}} &
\multicolumn{1}{r!{\huxvb{0}}}{\huxtpad{4pt}\raggedleft {\fontsize{9.5pt}{11.4pt}\selectfont (0.541)}\huxbpad{4pt}} &
\multicolumn{1}{r!{\huxvb{0}}}{\huxtpad{4pt}\raggedleft {\fontsize{9.5pt}{11.4pt}\selectfont (0.533)}\huxbpad{4pt}} \tabularnewline[-0.5pt]


\hhline{}
\arrayrulecolor{black}

\multicolumn{1}{!{\huxvb{0}}l!{\huxvb{0}}}{\huxtpad{4pt}\raggedright {\fontsize{9.5pt}{11.4pt}\selectfont Black}\huxbpad{4pt}} &
\multicolumn{1}{r!{\huxvb{0}}}{\huxtpad{4pt}\raggedleft {\fontsize{9.5pt}{11.4pt}\selectfont 0.561~~}\huxbpad{4pt}} &
\multicolumn{1}{r!{\huxvb{0}}}{\huxtpad{4pt}\raggedleft {\fontsize{9.5pt}{11.4pt}\selectfont 0.170~}\huxbpad{4pt}} &
\multicolumn{1}{r!{\huxvb{0}}}{\huxtpad{4pt}\raggedleft {\fontsize{9.5pt}{11.4pt}\selectfont -0.691~}\huxbpad{4pt}} \tabularnewline[-0.5pt]


\hhline{}
\arrayrulecolor{black}

\multicolumn{1}{!{\huxvb{0}}l!{\huxvb{0}}}{\huxtpad{4pt}\raggedright {\fontsize{9.5pt}{11.4pt}\selectfont }\huxbpad{4pt}} &
\multicolumn{1}{r!{\huxvb{0}}}{\huxtpad{4pt}\raggedleft {\fontsize{9.5pt}{11.4pt}\selectfont (0.378)~}\huxbpad{4pt}} &
\multicolumn{1}{r!{\huxvb{0}}}{\huxtpad{4pt}\raggedleft {\fontsize{9.5pt}{11.4pt}\selectfont (0.479)}\huxbpad{4pt}} &
\multicolumn{1}{r!{\huxvb{0}}}{\huxtpad{4pt}\raggedleft {\fontsize{9.5pt}{11.4pt}\selectfont (0.498)}\huxbpad{4pt}} \tabularnewline[-0.5pt]


\hhline{}
\arrayrulecolor{black}

\multicolumn{1}{!{\huxvb{0}}l!{\huxvb{0}}}{\huxtpad{4pt}\raggedright {\fontsize{9.5pt}{11.4pt}\selectfont Latinx}\huxbpad{4pt}} &
\multicolumn{1}{r!{\huxvb{0}}}{\huxtpad{4pt}\raggedleft {\fontsize{9.5pt}{11.4pt}\selectfont 0.219~~}\huxbpad{4pt}} &
\multicolumn{1}{r!{\huxvb{0}}}{\huxtpad{4pt}\raggedleft {\fontsize{9.5pt}{11.4pt}\selectfont -0.429~}\huxbpad{4pt}} &
\multicolumn{1}{r!{\huxvb{0}}}{\huxtpad{4pt}\raggedleft {\fontsize{9.5pt}{11.4pt}\selectfont -0.242~}\huxbpad{4pt}} \tabularnewline[-0.5pt]


\hhline{}
\arrayrulecolor{black}

\multicolumn{1}{!{\huxvb{0}}l!{\huxvb{0}}}{\huxtpad{4pt}\raggedright {\fontsize{9.5pt}{11.4pt}\selectfont }\huxbpad{4pt}} &
\multicolumn{1}{r!{\huxvb{0}}}{\huxtpad{4pt}\raggedleft {\fontsize{9.5pt}{11.4pt}\selectfont (0.419)~}\huxbpad{4pt}} &
\multicolumn{1}{r!{\huxvb{0}}}{\huxtpad{4pt}\raggedleft {\fontsize{9.5pt}{11.4pt}\selectfont (0.533)}\huxbpad{4pt}} &
\multicolumn{1}{r!{\huxvb{0}}}{\huxtpad{4pt}\raggedleft {\fontsize{9.5pt}{11.4pt}\selectfont (0.482)}\huxbpad{4pt}} \tabularnewline[-0.5pt]


\hhline{}
\arrayrulecolor{black}

\multicolumn{1}{!{\huxvb{0}}l!{\huxvb{0}}}{\huxtpad{4pt}\raggedright {\fontsize{9.5pt}{11.4pt}\selectfont Demographics}\huxbpad{4pt}} &
\multicolumn{1}{r!{\huxvb{0}}}{\huxtpad{4pt}\raggedleft {\fontsize{9.5pt}{11.4pt}\selectfont ~~~~~~}\huxbpad{4pt}} &
\multicolumn{1}{r!{\huxvb{0}}}{\huxtpad{4pt}\raggedleft {\fontsize{9.5pt}{11.4pt}\selectfont ~~~~~}\huxbpad{4pt}} &
\multicolumn{1}{r!{\huxvb{0}}}{\huxtpad{4pt}\raggedleft {\fontsize{9.5pt}{11.4pt}\selectfont ~~~~~}\huxbpad{4pt}} \tabularnewline[-0.5pt]


\hhline{}
\arrayrulecolor{black}

\multicolumn{1}{!{\huxvb{0}}l!{\huxvb{0}}}{\huxtpad{4pt}\raggedright {\fontsize{9.5pt}{11.4pt}\selectfont Age}\huxbpad{4pt}} &
\multicolumn{1}{r!{\huxvb{0}}}{\huxtpad{4pt}\raggedleft {\fontsize{9.5pt}{11.4pt}\selectfont -0.016~~}\huxbpad{4pt}} &
\multicolumn{1}{r!{\huxvb{0}}}{\huxtpad{4pt}\raggedleft {\fontsize{9.5pt}{11.4pt}\selectfont 0.020~}\huxbpad{4pt}} &
\multicolumn{1}{r!{\huxvb{0}}}{\huxtpad{4pt}\raggedleft {\fontsize{9.5pt}{11.4pt}\selectfont -0.014~}\huxbpad{4pt}} \tabularnewline[-0.5pt]


\hhline{}
\arrayrulecolor{black}

\multicolumn{1}{!{\huxvb{0}}l!{\huxvb{0}}}{\huxtpad{4pt}\raggedright {\fontsize{9.5pt}{11.4pt}\selectfont }\huxbpad{4pt}} &
\multicolumn{1}{r!{\huxvb{0}}}{\huxtpad{4pt}\raggedleft {\fontsize{9.5pt}{11.4pt}\selectfont (0.010)~}\huxbpad{4pt}} &
\multicolumn{1}{r!{\huxvb{0}}}{\huxtpad{4pt}\raggedleft {\fontsize{9.5pt}{11.4pt}\selectfont (0.012)}\huxbpad{4pt}} &
\multicolumn{1}{r!{\huxvb{0}}}{\huxtpad{4pt}\raggedleft {\fontsize{9.5pt}{11.4pt}\selectfont (0.011)}\huxbpad{4pt}} \tabularnewline[-0.5pt]


\hhline{}
\arrayrulecolor{black}

\multicolumn{1}{!{\huxvb{0}}l!{\huxvb{0}}}{\huxtpad{4pt}\raggedright {\fontsize{9.5pt}{11.4pt}\selectfont Foreign Born}\huxbpad{4pt}} &
\multicolumn{1}{r!{\huxvb{0}}}{\huxtpad{4pt}\raggedleft {\fontsize{9.5pt}{11.4pt}\selectfont 0.361~~}\huxbpad{4pt}} &
\multicolumn{1}{r!{\huxvb{0}}}{\huxtpad{4pt}\raggedleft {\fontsize{9.5pt}{11.4pt}\selectfont 0.209~}\huxbpad{4pt}} &
\multicolumn{1}{r!{\huxvb{0}}}{\huxtpad{4pt}\raggedleft {\fontsize{9.5pt}{11.4pt}\selectfont -0.147~}\huxbpad{4pt}} \tabularnewline[-0.5pt]


\hhline{}
\arrayrulecolor{black}

\multicolumn{1}{!{\huxvb{0}}l!{\huxvb{0}}}{\huxtpad{4pt}\raggedright {\fontsize{9.5pt}{11.4pt}\selectfont }\huxbpad{4pt}} &
\multicolumn{1}{r!{\huxvb{0}}}{\huxtpad{4pt}\raggedleft {\fontsize{9.5pt}{11.4pt}\selectfont (0.357)~}\huxbpad{4pt}} &
\multicolumn{1}{r!{\huxvb{0}}}{\huxtpad{4pt}\raggedleft {\fontsize{9.5pt}{11.4pt}\selectfont (0.424)}\huxbpad{4pt}} &
\multicolumn{1}{r!{\huxvb{0}}}{\huxtpad{4pt}\raggedleft {\fontsize{9.5pt}{11.4pt}\selectfont (0.455)}\huxbpad{4pt}} \tabularnewline[-0.5pt]


\hhline{}
\arrayrulecolor{black}

\multicolumn{1}{!{\huxvb{0}}l!{\huxvb{0}}}{\huxtpad{4pt}\raggedright {\fontsize{9.5pt}{11.4pt}\selectfont Male}\huxbpad{4pt}} &
\multicolumn{1}{r!{\huxvb{0}}}{\huxtpad{4pt}\raggedleft {\fontsize{9.5pt}{11.4pt}\selectfont 0.087~~}\huxbpad{4pt}} &
\multicolumn{1}{r!{\huxvb{0}}}{\huxtpad{4pt}\raggedleft {\fontsize{9.5pt}{11.4pt}\selectfont 0.313~}\huxbpad{4pt}} &
\multicolumn{1}{r!{\huxvb{0}}}{\huxtpad{4pt}\raggedleft {\fontsize{9.5pt}{11.4pt}\selectfont -0.053~}\huxbpad{4pt}} \tabularnewline[-0.5pt]


\hhline{}
\arrayrulecolor{black}

\multicolumn{1}{!{\huxvb{0}}l!{\huxvb{0}}}{\huxtpad{4pt}\raggedright {\fontsize{9.5pt}{11.4pt}\selectfont }\huxbpad{4pt}} &
\multicolumn{1}{r!{\huxvb{0}}}{\huxtpad{4pt}\raggedleft {\fontsize{9.5pt}{11.4pt}\selectfont (0.280)~}\huxbpad{4pt}} &
\multicolumn{1}{r!{\huxvb{0}}}{\huxtpad{4pt}\raggedleft {\fontsize{9.5pt}{11.4pt}\selectfont (0.332)}\huxbpad{4pt}} &
\multicolumn{1}{r!{\huxvb{0}}}{\huxtpad{4pt}\raggedleft {\fontsize{9.5pt}{11.4pt}\selectfont (0.337)}\huxbpad{4pt}} \tabularnewline[-0.5pt]


\hhline{}
\arrayrulecolor{black}

\multicolumn{1}{!{\huxvb{0}}l!{\huxvb{0}}}{\huxtpad{4pt}\raggedright {\fontsize{9.5pt}{11.4pt}\selectfont Children Present}\huxbpad{4pt}} &
\multicolumn{1}{r!{\huxvb{0}}}{\huxtpad{4pt}\raggedleft {\fontsize{9.5pt}{11.4pt}\selectfont -0.124~~}\huxbpad{4pt}} &
\multicolumn{1}{r!{\huxvb{0}}}{\huxtpad{4pt}\raggedleft {\fontsize{9.5pt}{11.4pt}\selectfont -0.096~}\huxbpad{4pt}} &
\multicolumn{1}{r!{\huxvb{0}}}{\huxtpad{4pt}\raggedleft {\fontsize{9.5pt}{11.4pt}\selectfont -0.027~}\huxbpad{4pt}} \tabularnewline[-0.5pt]


\hhline{}
\arrayrulecolor{black}

\multicolumn{1}{!{\huxvb{0}}l!{\huxvb{0}}}{\huxtpad{4pt}\raggedright {\fontsize{9.5pt}{11.4pt}\selectfont }\huxbpad{4pt}} &
\multicolumn{1}{r!{\huxvb{0}}}{\huxtpad{4pt}\raggedleft {\fontsize{9.5pt}{11.4pt}\selectfont (0.301)~}\huxbpad{4pt}} &
\multicolumn{1}{r!{\huxvb{0}}}{\huxtpad{4pt}\raggedleft {\fontsize{9.5pt}{11.4pt}\selectfont (0.390)}\huxbpad{4pt}} &
\multicolumn{1}{r!{\huxvb{0}}}{\huxtpad{4pt}\raggedleft {\fontsize{9.5pt}{11.4pt}\selectfont (0.399)}\huxbpad{4pt}} \tabularnewline[-0.5pt]


\hhline{}
\arrayrulecolor{black}

\multicolumn{1}{!{\huxvb{0}}l!{\huxvb{0}}}{\huxtpad{4pt}\raggedright {\fontsize{9.5pt}{11.4pt}\selectfont Married}\huxbpad{4pt}} &
\multicolumn{1}{r!{\huxvb{0}}}{\huxtpad{4pt}\raggedleft {\fontsize{9.5pt}{11.4pt}\selectfont -0.394~~}\huxbpad{4pt}} &
\multicolumn{1}{r!{\huxvb{0}}}{\huxtpad{4pt}\raggedleft {\fontsize{9.5pt}{11.4pt}\selectfont 0.319~}\huxbpad{4pt}} &
\multicolumn{1}{r!{\huxvb{0}}}{\huxtpad{4pt}\raggedleft {\fontsize{9.5pt}{11.4pt}\selectfont 0.606~}\huxbpad{4pt}} \tabularnewline[-0.5pt]


\hhline{}
\arrayrulecolor{black}

\multicolumn{1}{!{\huxvb{0}}l!{\huxvb{0}}}{\huxtpad{4pt}\raggedright {\fontsize{9.5pt}{11.4pt}\selectfont }\huxbpad{4pt}} &
\multicolumn{1}{r!{\huxvb{0}}}{\huxtpad{4pt}\raggedleft {\fontsize{9.5pt}{11.4pt}\selectfont (0.301)~}\huxbpad{4pt}} &
\multicolumn{1}{r!{\huxvb{0}}}{\huxtpad{4pt}\raggedleft {\fontsize{9.5pt}{11.4pt}\selectfont (0.344)}\huxbpad{4pt}} &
\multicolumn{1}{r!{\huxvb{0}}}{\huxtpad{4pt}\raggedleft {\fontsize{9.5pt}{11.4pt}\selectfont (0.346)}\huxbpad{4pt}} \tabularnewline[-0.5pt]


\hhline{}
\arrayrulecolor{black}

\multicolumn{1}{!{\huxvb{0}}l!{\huxvb{0}}}{\huxtpad{4pt}\raggedright {\fontsize{9.5pt}{11.4pt}\selectfont Socioeconomic}\huxbpad{4pt}} &
\multicolumn{1}{r!{\huxvb{0}}}{\huxtpad{4pt}\raggedleft {\fontsize{9.5pt}{11.4pt}\selectfont ~~~~~~}\huxbpad{4pt}} &
\multicolumn{1}{r!{\huxvb{0}}}{\huxtpad{4pt}\raggedleft {\fontsize{9.5pt}{11.4pt}\selectfont ~~~~~}\huxbpad{4pt}} &
\multicolumn{1}{r!{\huxvb{0}}}{\huxtpad{4pt}\raggedleft {\fontsize{9.5pt}{11.4pt}\selectfont ~~~~~}\huxbpad{4pt}} \tabularnewline[-0.5pt]


\hhline{}
\arrayrulecolor{black}

\multicolumn{1}{!{\huxvb{0}}l!{\huxvb{0}}}{\huxtpad{4pt}\raggedright {\fontsize{9.5pt}{11.4pt}\selectfont $<$H.S.}\huxbpad{4pt}} &
\multicolumn{1}{r!{\huxvb{0}}}{\huxtpad{4pt}\raggedleft {\fontsize{9.5pt}{11.4pt}\selectfont 2.714 *}\huxbpad{4pt}} &
\multicolumn{1}{r!{\huxvb{0}}}{\huxtpad{4pt}\raggedleft {\fontsize{9.5pt}{11.4pt}\selectfont -0.336~}\huxbpad{4pt}} &
\multicolumn{1}{r!{\huxvb{0}}}{\huxtpad{4pt}\raggedleft {\fontsize{9.5pt}{11.4pt}\selectfont -1.387~}\huxbpad{4pt}} \tabularnewline[-0.5pt]


\hhline{}
\arrayrulecolor{black}

\multicolumn{1}{!{\huxvb{0}}l!{\huxvb{0}}}{\huxtpad{4pt}\raggedright {\fontsize{9.5pt}{11.4pt}\selectfont }\huxbpad{4pt}} &
\multicolumn{1}{r!{\huxvb{0}}}{\huxtpad{4pt}\raggedleft {\fontsize{9.5pt}{11.4pt}\selectfont (1.058)~}\huxbpad{4pt}} &
\multicolumn{1}{r!{\huxvb{0}}}{\huxtpad{4pt}\raggedleft {\fontsize{9.5pt}{11.4pt}\selectfont (1.041)}\huxbpad{4pt}} &
\multicolumn{1}{r!{\huxvb{0}}}{\huxtpad{4pt}\raggedleft {\fontsize{9.5pt}{11.4pt}\selectfont (1.092)}\huxbpad{4pt}} \tabularnewline[-0.5pt]


\hhline{}
\arrayrulecolor{black}

\multicolumn{1}{!{\huxvb{0}}l!{\huxvb{0}}}{\huxtpad{4pt}\raggedright {\fontsize{9.5pt}{11.4pt}\selectfont Some college, no B.A.}\huxbpad{4pt}} &
\multicolumn{1}{r!{\huxvb{0}}}{\huxtpad{4pt}\raggedleft {\fontsize{9.5pt}{11.4pt}\selectfont 1.344 *}\huxbpad{4pt}} &
\multicolumn{1}{r!{\huxvb{0}}}{\huxtpad{4pt}\raggedleft {\fontsize{9.5pt}{11.4pt}\selectfont 0.303~}\huxbpad{4pt}} &
\multicolumn{1}{r!{\huxvb{0}}}{\huxtpad{4pt}\raggedleft {\fontsize{9.5pt}{11.4pt}\selectfont -0.228~}\huxbpad{4pt}} \tabularnewline[-0.5pt]


\hhline{}
\arrayrulecolor{black}

\multicolumn{1}{!{\huxvb{0}}l!{\huxvb{0}}}{\huxtpad{4pt}\raggedright {\fontsize{9.5pt}{11.4pt}\selectfont }\huxbpad{4pt}} &
\multicolumn{1}{r!{\huxvb{0}}}{\huxtpad{4pt}\raggedleft {\fontsize{9.5pt}{11.4pt}\selectfont (0.567)~}\huxbpad{4pt}} &
\multicolumn{1}{r!{\huxvb{0}}}{\huxtpad{4pt}\raggedleft {\fontsize{9.5pt}{11.4pt}\selectfont (0.654)}\huxbpad{4pt}} &
\multicolumn{1}{r!{\huxvb{0}}}{\huxtpad{4pt}\raggedleft {\fontsize{9.5pt}{11.4pt}\selectfont (0.671)}\huxbpad{4pt}} \tabularnewline[-0.5pt]


\hhline{}
\arrayrulecolor{black}

\multicolumn{1}{!{\huxvb{0}}l!{\huxvb{0}}}{\huxtpad{4pt}\raggedright {\fontsize{9.5pt}{11.4pt}\selectfont B.A.}\huxbpad{4pt}} &
\multicolumn{1}{r!{\huxvb{0}}}{\huxtpad{4pt}\raggedleft {\fontsize{9.5pt}{11.4pt}\selectfont 0.935~~}\huxbpad{4pt}} &
\multicolumn{1}{r!{\huxvb{0}}}{\huxtpad{4pt}\raggedleft {\fontsize{9.5pt}{11.4pt}\selectfont 0.157~}\huxbpad{4pt}} &
\multicolumn{1}{r!{\huxvb{0}}}{\huxtpad{4pt}\raggedleft {\fontsize{9.5pt}{11.4pt}\selectfont -0.301~}\huxbpad{4pt}} \tabularnewline[-0.5pt]


\hhline{}
\arrayrulecolor{black}

\multicolumn{1}{!{\huxvb{0}}l!{\huxvb{0}}}{\huxtpad{4pt}\raggedright {\fontsize{9.5pt}{11.4pt}\selectfont }\huxbpad{4pt}} &
\multicolumn{1}{r!{\huxvb{0}}}{\huxtpad{4pt}\raggedleft {\fontsize{9.5pt}{11.4pt}\selectfont (0.550)~}\huxbpad{4pt}} &
\multicolumn{1}{r!{\huxvb{0}}}{\huxtpad{4pt}\raggedleft {\fontsize{9.5pt}{11.4pt}\selectfont (0.628)}\huxbpad{4pt}} &
\multicolumn{1}{r!{\huxvb{0}}}{\huxtpad{4pt}\raggedleft {\fontsize{9.5pt}{11.4pt}\selectfont (0.640)}\huxbpad{4pt}} \tabularnewline[-0.5pt]


\hhline{}
\arrayrulecolor{black}

\multicolumn{1}{!{\huxvb{0}}l!{\huxvb{0}}}{\huxtpad{4pt}\raggedright {\fontsize{9.5pt}{11.4pt}\selectfont M.A.+}\huxbpad{4pt}} &
\multicolumn{1}{r!{\huxvb{0}}}{\huxtpad{4pt}\raggedleft {\fontsize{9.5pt}{11.4pt}\selectfont 1.283 *}\huxbpad{4pt}} &
\multicolumn{1}{r!{\huxvb{0}}}{\huxtpad{4pt}\raggedleft {\fontsize{9.5pt}{11.4pt}\selectfont 0.692~}\huxbpad{4pt}} &
\multicolumn{1}{r!{\huxvb{0}}}{\huxtpad{4pt}\raggedleft {\fontsize{9.5pt}{11.4pt}\selectfont -0.442~}\huxbpad{4pt}} \tabularnewline[-0.5pt]


\hhline{}
\arrayrulecolor{black}

\multicolumn{1}{!{\huxvb{0}}l!{\huxvb{0}}}{\huxtpad{4pt}\raggedright {\fontsize{9.5pt}{11.4pt}\selectfont }\huxbpad{4pt}} &
\multicolumn{1}{r!{\huxvb{0}}}{\huxtpad{4pt}\raggedleft {\fontsize{9.5pt}{11.4pt}\selectfont (0.549)~}\huxbpad{4pt}} &
\multicolumn{1}{r!{\huxvb{0}}}{\huxtpad{4pt}\raggedleft {\fontsize{9.5pt}{11.4pt}\selectfont (0.640)}\huxbpad{4pt}} &
\multicolumn{1}{r!{\huxvb{0}}}{\huxtpad{4pt}\raggedleft {\fontsize{9.5pt}{11.4pt}\selectfont (0.653)}\huxbpad{4pt}} \tabularnewline[-0.5pt]


\hhline{>{\huxb{1}}->{\huxb{1}}->{\huxb{1}}->{\huxb{1}}-}
\arrayrulecolor{black}
\end{tabularx}
\end{table}


% latex table generated in R 3.5.0 by xtable 1.8-2 package
% Tue Nov 20 14:10:28 2018
\begin{sidewaystable}[ht]
\centering
\caption{Estimated race coefficients for willingness to consider communities} 
\label{tab:consider}
\begin{tabular}{lrlrlrlrlrlrlrlrl}
  \toprule
  & \multicolumn{8}{c}{Race only} & \multicolumn{8}{c}{With controls}\\
 & Intercept && Asian && Black && Latino&& Intercept && Asian && Black && Latino&\\
 \midrule
Columbia Heights & -2.212 & *** & -1.607 & ** & -0.976 & * & -0.911 &  & -3.275 & *** & -1.083 &  & -0.390 &  & -0.183 &  \\ 
  Brightwood & -4.762 & *** & -14.829 & *** & 1.886 & * & 0.879 &  & -7.127 & *** & -14.582 & *** & 2.271 & * & 1.778 &  \\ 
  Langley Park & -4.658 & *** & 1.282 &  & 1.066 &  & 1.093 &  & -6.652 & *** & 1.606 &  & 1.151 &  & 1.377 &  \\ 
  Hyattsville & -3.526 & *** & -0.001 &  & 0.128 &  & 0.182 &  & -4.272 & *** & 0.292 &  & 0.224 &  & 0.446 &  \\ 
  Greenbelt & -3.520 & *** & -1.045 &  & 1.307 & *** & 0.739 &  & -4.082 & *** & -0.412 &  & 1.496 & ** & 1.234 & * \\ 
  Wheaton & -3.706 & *** & 0.387 &  & 1.346 & ** & 1.691 & *** & -4.135 & *** & 0.614 &  & 1.329 & ** & 1.630 & ** \\ 
  Germantown & -2.634 & *** & -0.045 &  & -0.174 &  & 0.670 &  & -3.113 & *** & 0.167 &  & -0.130 &  & 0.920 &  \\ 
  Arlington & -1.307 & *** & -0.510 &  & -0.301 &  & -0.083 &  & -1.989 & *** & -0.194 &  & 0.044 &  & 0.369 &  \\ 
  Annandale & -2.278 & *** & -0.731 &  & -0.230 &  & -0.791 &  & -2.464 & *** & -0.706 &  & -0.340 &  & -0.807 &  \\ 
  Huntington & -3.592 & *** & -0.454 &  & 0.848 &  & 0.141 &  & -4.956 & *** & 0.385 &  & 1.401 & * & 0.859 &  \\ 
  Herndon & -2.125 & *** & -0.572 &  & -0.317 &  & -0.178 &  & -2.367 & *** & -0.502 &  & -0.161 &  & -0.079 &  \\ 
   \bottomrule
\end{tabular}
\end{sidewaystable}


% latex table generated in R 3.5.0 by xtable 1.8-2 package
% 
\begin{sidewaystable}[ht]
\centering
\caption{Estimated race coefficients for not considering communities} 
\label{tab:notconsider}
\begin{tabular}{lrlrlrlrlrlrlrlrl}
  \toprule
  & \multicolumn{8}{c}{Race only} & \multicolumn{8}{c}{With controls}\\
 & Intercept && Asian && Black && Latino&& Intercept && Asian && Black && Latino&\\
 \midrule
Columbia Heights & -0.846 & *** & -0.853 & * & -0.755 & * & -0.388 &  & -1.215 &  & -0.383 &  & -0.492 &  & 0.140 &  \\ 
  Brightwood & -1.245 & *** & -0.452 &  & -0.839 & * & -0.253 &  & -2.356 & *** & -0.063 &  & -0.473 &  & 0.183 &  \\ 
  Langley Park & -0.481 & ** & -0.929 & ** & -0.132 &  & 0.327 &  & -1.714 & ** & -0.654 &  & 0.159 &  & 0.680 &  \\ 
  Hyattsville & -0.537 & ** & -0.898 & ** & -0.038 &  & 0.013 &  & -1.439 & * & -0.562 &  & 0.344 &  & 0.646 &  \\ 
  Greenbelt & -0.559 & ** & -0.923 & ** & -0.475 &  & 0.204 &  & -0.548 &  & -0.718 &  & -0.203 &  & 0.595 &  \\ 
  Wheaton & -0.803 & *** & -0.822 & * & 0.015 &  & -0.127 &  & -1.569 & ** & -0.556 &  & 0.344 &  & 0.296 &  \\ 
  Germantown & -1.239 & *** & -0.811 & * & -0.453 &  & -0.361 &  & -2.001 & * & -0.056 &  & -0.087 &  & 0.317 &  \\ 
  Arlington & -1.704 & *** & -0.613 &  & -0.099 &  & -0.461 &  & -3.831 & *** & -0.384 &  & -0.048 &  & -0.318 &  \\ 
  Annandale & -1.463 & *** & -0.239 &  & -0.455 &  & -0.229 &  & -2.443 & *** & -0.227 &  & -0.408 &  & -0.035 &  \\ 
  Huntington & -0.855 & *** & -1.067 & ** & -0.733 &  & -0.422 &  & -1.904 & ** & -0.422 &  & -0.499 &  & 0.012 &  \\ 
  Herndon & -1.069 & *** & -0.647 &  & -0.163 &  & -0.314 &  & -1.260 &  & -0.042 &  & 0.098 &  & 0.187 &  \\ 
   \bottomrule
\end{tabular}
\end{sidewaystable}


\clearpage
\section{Figures}
\label{sec:figures}

\begin{figure}[h!]
\caption{Predicted probability that residents of multiethnic neighborhoods are satisfied in their neighborhood, by race}
\label{fig:satisfaction}
\includegraphics{../images/satisfied.png}
\end{figure}

\begin{figure}[h!]
\caption{Predicted probability that residents of multiethnic neighborhoods perceive improvement in their neighborhood in the past five years, by race}
\label{fig:improvement}
\includegraphics{../images/improvement.png}
\end{figure}
\end{document}
