\documentclass{baderart}

\title{Satisfaction and Stability of Multiethnic Neighborhoods}
\author{Michael D.\ M.\ Bader}

\providecommand{\tightlist}{%
  \setlength{\itemsep}{0pt}\setlength{\parskip}{0pt}}
%% Values recorded from analysis source files
\newcommand{\N}{641}
\newcommand{\miinc}{43}
\newcommand{\miedu}{2}
\newcommand{\Rversion}{3.5.0}
\newcommand{\oneresp}{9}
\newcommand{\medNpertract}{5}
\newcommand{\maxNpertract}{19}
\newcommand{\meansatisfied}{71}
\newcommand{\apisatisfied}{72}
\newcommand{\nhbsatisfied}{67}
\newcommand{\hspsatisfied}{76}
\newcommand{\nhwsatisfied}{69}
\newcommand{\satWaldF}{0.54}
\newcommand{\satWaldpMin}{0.61}
\newcommand{\satWaldpMax}{0.69}
\newcommand{\satnhdsize}{3.16}
\newcommand{\maxdiffone}{8}
\newcommand{\maxdiffthree}{6}
\newcommand{\meanimproved}{34}
\newcommand{\betWaldF}{2.33}
\newcommand{\betWaldp}{0.07}
\newcommand{\apibetter}{2.36}
\newcommand{\nhbbetter}{2.91}
\newcommand{\hspbetter}{3.35}
\newcommand{\dkherndon}{46}
\newcommand{\dkgermantown}{31}
\newcommand{\dkwheaton}{37}
\newcommand{\ffherndon}{25}
\newcommand{\ffgermantown}{40}
\newcommand{\ffwheaton}{28}
\newcommand{\consherndon}{10}
\newcommand{\consgermantown}{10}
\newcommand{\conswheaton}{6}
\newcommand{\ncherndon}{21}
\newcommand{\ncgermantown}{17}
\newcommand{\ncwheaton}{27}



\begin{document}
\maketitle

\doublespace

\section{Introduction}\label{introduction}
Federal anti-discrimination laws, changing racial attitudes, and increased immigration created new opportunities for people to become neighbors with people across multiple racial and ethnic categories. Once exceptional, multiethnic neighborhoods have emerged as a common feature in U.S. metropolitan areas during the past half century. In a nation beset by its history of racial segregation, multiethnic neighborhoods offer hope for a more integrated future.

Academic research has not kept pace with the growing presence of multiethnically integrated neighborhoods in American metropolitan areas. The body of research on gentrification dwarfs that focusing on emergent integration. The imbalance does not correspond to the respective size of each phenomenon since gentrification occurs in a smaller number of neighborhoods than multiethnic integration. 

A small body of qualitative research has studied the unique dynamics of integrated neighborhoods. Contrary to the model of neighborhoods forming around racial and ethnic solidarity, this work shows that residents of all races tend to enjoy living among diverse neighbors. Long-term integration seems possible. Without research that studies a broad sample of residents from multiethnic neighborhoods,however, we cannot know how sustainable multiethnic integration will be. 

In this article, I begin to build this evidence by examining how satisfied residents are living in multiethnic neighborhoods. I also examine factors that related to the long-term stability of integration. I used a representative sample of households in multiethnic neighborhoods in the Washington, D.C.\ area. Evidence shows that a large majority of all racial groups are satisfied living in their current neighborhoods. I also studied how much improvement residents perceived in their neighborhoods and their knowledge and willingness to consider moving to another multiethnic community. These measures show mixed prospects for sustaining multiethnic integration. Together, these results show an unmet need to understand the dynamics of multiethnic communities in America's metropolitan areas. 

\section{Background}\label{background}
Multiethnic neighborhoods emerged as one of many types of neighborhood change that emerged since the 1970s. Other trends that emerged included gentrification and ``ethnoburbs.'' Changing metropolitan economies, family structures, and \ldots\ fostered the gentrification of many urban--and even suburban--neighborhoods \textbf{cite: Lung-Amam, others}. A generation of whites who grew up after the Civil Rights Movement's successes in the 1960s started moving to urban neighborhoods that older generations avoided. People of color living in those neighborhoods have been ``displaced'' by economic, social, and legal means \textbf{cites}. 



They have come about as Asian, black, and Latinx newcomers moved into neighborhoods that had been all white, or nearly so \textbf{(cite: Logan; Bader)}. The trend started in the 1980s and accelerated in the 1990s and 2000s. In the past, the entry of non-white residents sparked violence and white flight. During and after the 1980s, whites stayed put as their neighborhoods integrated around them. 




Whites who as a group,  had once avoided dense neighborhoods filled with people of color started moving 

The racial impact of neighborhood change 


that expanded the types of racial change American neighborhoods experienced. One was the gentrification of urban (and even suburban) neighborhoods. 

The number of neighborhoods shared among white, Asian, black, and Latinx neighbors rose rapidly in the 1990s and 2000s. Logan and Zhang estimate that \textbf{TK\%} of neighborhoods in 2010 were what they termed ``global neighborhoods.'' Similarly, Bader and Warkentien found stable multiethnic integration represented between \textbf{TK\%} and \textbf{TK\%} of neighborhoods in the metropolitan areas of the four largest cities.

Several currents made the growth of multiethnic neighborhoods possible. Federal legislation, especially the Fair Housing Act, made housing discrimination a federal crime. The impact of the FHA, which was passed in 1968 in the wake of the Rev.~Dr.~Martin Luther King, Jr.'s assassination, was neither immediate nor universal. Racial discrimination has persisted in more subtle ways such as real-estate agents ``steering'' clients to neighborhoods. The FHA provided, however, provided a legal mechanism for people of color to enter white neighborhoods and challenge overt forms of discrimination.

Second, \emph{multiethnic} integration would be impossble without a multiethnic population. Increased immigration from Asian and Latin American countries increased the number of Asian and Latinx people in the United States. Immigration unsettled the stark black-white divide in many American metropolitan areas. The trend\textbf{{[}w.ch{]}} was most pronounced in metropolitan areas that became ``new immigrant'' destinations in the 1980s and 1990s. The growing economies that attracted migrants also required new housing construction to accomodate growing populations that made integrated neighborhoods possible from the get-go\textbf{{[}w.ch.{]}}.

Finally, racial attitudes have been changing.

\subsection{Satisfaction and Long-Term
Stability}\label{satisfaction-and-long-term-stability}

\subsubsection{Satisfaction and
Mobility}\label{satisfaction-and-mobility}

Neighborhood satisfaction influences whether people stay put in their
neighborhoods. The process of white flight can be described as whites
having acted on the immediate and strong feelings of dissatisfaction
that the entry of a black family prompted. As whites vacated their
houses, additional black families--who were dissatisfied being forced to
live in overcrowded ghettos through law and custom--filled those
vacancies. The process resulted in the rapid ``succession'' of the
neighborhood's composition from all-white to all-black.

The more sanguine attitudes whites hold about people of color could mean
that whites are now satisfied living in integrated neighborhoods. If so,
then there is a stronger prospect of long-term integration. Evidence is
mixed. On the one hand, patterns of neighborhood change seem to bear
this out, with little evidence of rapid racial succession since the
1980s. On the other, studies consistently show lower levels of
satisfaction among whites who live in neighborhoods in which people of
color make up an increasing share of their neighbors.

Existing studies of satisfaction among residents measure the composition
of racial groups separately or measure the composition of all non-white
residents grouped into a single category. \textbf{{[}cite findings
here{]}} Multiethnic neighborhoods represent a distinctive form of
integration not captured by the method of measuring racial compositions
of groups. Instead, the \emph{presence} of all four racial groups
distinguishes these neighborhoods and satisfaction among residents of
different races in these distinctive neighborhoods have not been
measured.

\begin{itemize}
\tightlist
\item
  Are groups equally satisifed living in integrated neighborhoods, and
  therefore likely to stay (maintaining integration)?
\end{itemize}

Measuring whether residents think that the neighborhood has improved
provides an additional measure of satisfaction. While satisfied
residents might not feel compelled to leave, they could be more inclined
to leave if they perceive less improvement in the neighborhood. Previous
research shows that white perceptions of neighborhood trends diminish
with increasing non-white composition. Even if whites feel satisfied
living in neighborhoods, they might feel less confident that they will
be satisfied in the future. If true, then whites might be enticed to
leave the neighborhood more readily than neighbors of color. Again,
however, existing studies do not sample \emph{multiethnic}
neighborhoods.

\subsection{Satisfaction and Long-Term Stability}\label{satisfaction-and-long-term-stability}

\subsubsection{Satisfaction and Mobility}\label{satisfaction-and-mobility}

Neighborhood satisfaction influences whether people stay put in their neighborhoods. The process of white flight can be described as whites having acted on the immediate and strong feelings of dissatisfaction that the entry of a black family prompted. As whites vacated their houses, additional black families--who were dissatisfied being forced to live in overcrowded ghettos through law and custom--filled those vacancies. The process resulted in the rapid ``succession'' of the neighborhood's composition from all-white to all-black.

The more sanguine attitudes whites hold about people of color could mean that whites are now satisfied living in integrated neighborhoods. If so, then there is a stronger prospect of long-term integration. Evidence is mixed. On the one hand, patterns of neighborhood change seem to bear this out, with little evidence of rapid racial succession since the 1980s. On the other, studies consistently show lower levels of satisfaction among whites who live in neighborhoods in which people of color make up an increasing share of their neighbors.

Existing studies of satisfaction among residents measure the composition of racial groups separately or measure the composition of all non-white residents grouped into a single category. \textbf{{[}cite findings here{]}} Multiethnic neighborhoods represent a distinctive form of integration not captured by the method of measuring racial compositions of groups. Instead, the \emph{presence} of all four racial groups distinguishes these neighborhoods and satisfaction among residents of different races in these distinctive neighborhoods have not been measured.

\begin{itemize} \tightlist \item   Are groups equally satisifed living in integrated neighborhoods, and   therefore likely to stay (maintaining integration)? \end{itemize}

Measuring whether residents think that the neighborhood has improved provides an additional measure of satisfaction. While satisfied residents might not feel compelled to leave, they could be more inclined to leave if they perceive less improvement in the neighborhood. Previous research shows that white perceptions of neighborhood trends diminish with increasing non-white composition. Even if whites feel satisfied living in neighborhoods, they might feel less confident that they will be satisfied in the future. If true, then whites might be enticed to leave the neighborhood more readily than neighbors of color. Again, however, existing studies do not sample \emph{multiethnic} neighborhoods.


\subsubsection{Long-Term Stability}\label{long-term-stability}

Even residents satisfied living in their neighborhood will find themselves exiting for reasons unrelated to the racial composition of the neighborhood. Residents will get married, retire, have children, get divorced, and, in the most extreme case, exit upon their death. Life course transitions such as these cause most moves; and, as residents move out for those reasons, new residents will move in. Satisfaction will be insufficient to predict future integration because integration will survive only as long as the current cohort of residents does. To maintain the multiethnic diversity of the neighborhood therefore requires that the racial composition of households \emph{leaving} multiethnic neighborhoods approximates the racial composition of those \emph{entering} multiethnic neighborhoods.

Multiethnic neighborhoods must attract racially diverse newcomers to maintain integration. Residents currently residing in multiethnic neighborhoods provide an obvious clientele to fill vacancies in multiethnic neighborhoods. Yet, we have neither a sense whether current residents of multiethnic neighborhoods would consider moving to another multiethnic neighborhood, nor whether the proportion who would consider another multiethnic neighborhood differs across racial groups.

Rather than being individually derived or collectively negotiated within households, housing preferences reflect a deeply social process. Households cannot engage with the housing market across an entire metropolitan area. Krysan and Crowder show that people approach housing searches based on word of mouth that shape what people think about different communities, and whether they even know anything about the community at all. Settling on which neighborhoods to consider--and which not to waste time considering--reflects social networks that are segregated by race. Differences in the knowledge residents use as starting points for their housing searches can help us predict the future stability of multiethnic integration.

Race influences the familiarity residents have with communities and where residents consider moving. Whites know more about and desire moving to neighborhoods with larger shares of white residents. Whites differ from other racial groups, who have stronger desires for integration than whites. Havekes and colleagues find that racial composition influences white searches more at each stage of the search process. They find that whites move to communities with lower shares of whites on average than communities that they search which, in turn, have lower shares of whites on average than communities whites report considering. Multiethnic neighborhoods might expose white residents to more diverse social networks and increase the probability that whites would consider another multiethnic neighborhood. But, again, we have no evidence one way or the other among residents living in multiethnic neighborhoods.

\subsubsection{Hypotheses}\label{hypotheses}

In light of this evidence, this article examines two hypotheses:


\begin{enumerate} \def\labelenumi{\arabic{enumi}.} \tightlist \item   Whites living in multiethnic neighborhoods will be a) less satisfied   with their neighborhoods and b) less likely to perceive neighborhood   improvement compared to their Asian, black, and Latinx neighbors; and \item   Whites living in multiethnic neighborhoods will be will be a) less   familiar with, b) less likely to consider, and c) more likely to never   consider moving to other multiethnic neighborhoods in their   metropolitan area compared to their Asian, black, and Latinx   neighbors. \end{enumerate}

\subsection{Multiethnic Neighborhoods in the Washington, D.C. Area}\label{multiethnic-neighborhoods-in-the-washington-d.c.-area}

Data to test the hypotheses come from the 2016 DC Area Survey (DCAS). Washington, D.C. has historically been a segregated metropolitan area. Almost all blacks lived in the eastern part of the city and nearly all whites in the west. This east-west pattern spilled out to the Maryland suburbs. Few blacks lived across the Potomac River in Northern Virginia, though they were highly clustered in a small number of neighborhoods. In 1970, only \textbf{TK}\% and \textbf{TK}\% of Asians and Latinxs lived in the DC area.

The region's economy grew and became less based on the federal government in the 1980s and 1990. The expansion of finance, insurance, and real estate services, collectively known by the acronym FIRE, mirrored the sector's prominence in redeveloping metropolitan areas in the U.S. and abroad. During the same period, the region emerged as an immigrant destination. As the nation's capital and importance in international relations, the region's foreign-born residents are more socioeconomically diverse than average. Latinos and Asians now comprise \textbf{TK}\% and \textbf{TK}\%, respectively, of the DC area while blacks make up \textbf{TK}\% and whites \textbf{TK}\%. \textbf{TK} residents were born outside of the United States, with the largest shares coming from El Salvador, \textbf{TK}, and \textbf{TK}.

Real estate developers capitalized on the expanding population by developing large swaths of land in middle-ring suburbs in the counties surrounding the District. Having been built after the Fair Housing Act passed, the residences developed in these suburbs were not subject to the history of redlining restrictive covenants of previous developments. Residents of all races were attracted to these new suburban homes, especially in Montgomery and Fairfax Counties that boasted nationally renowned school systems. As Figure \textbf{TK} shows, multiethnic neighborhoods were especially likely to emerge along the hub-and-spoke system of the region's commuter rail lines. Additionally, Montgomery County has, since 1974, mandated that all housing developments larger than 20 units include a set percentage of units that qualify as affordable housing.\footnote{Fairfax County passed an inclusionary   zoning ordinance in 1971, but it was struck down by the Virginia   Supreme Court. Fairfax County implemented a different mandatory   inclusionary zoning policy in 1990.}

The 2016 DCAS sought to represent residents of two types of neighborhoods--multiethnic and disproportionately Latinx--in the Washington, D.C. area, which comprised Washington, D.C. and the surrounding jurisdictions of Montgomery and Prince George's Counties in Maryland, Arlington and Fairfax (including the cities of Falls Church and Fairfax) Counties Virginia, and the independent city of Alexandria, Virginia. Only responses from residents sampled in the multiethnic neighborhoods were included in this analysis.\footnote{Disproportionately   Latinx neighborhoods were those in which Latinx residents made up at   least a quarter of the residents and were not already classified as a   multiethnic neighborhood.}

Multiethnic neighborhoods included in the sampling frame met two criteria. First, Asians, blacks, Latinxs, and whites each comprised at least 10\textasciitilde{}percent of the population of the neighborhood. This cutoff was used to ensure that each of the four racial groups represented a distinguishable subset of a neighborhood's residents. Second, none of those four groups could represent a majority of residents. This criterion was included to eliminate neighborhoods where a single group formed the dominant identity of the neighborhood. These criteria resulted in a sample frame of 114 neighborhoods that represented \textbf{TK} people. The majority of neighborhoods were in Montgomery County, Maryland, followed by Fairfax County, Virginia, then \textbf{TK}\ldots{} The neighborhoods included in the sampling frame are shown in Figure @ref(fig:map). An address-based sample of households was drawn from these eligible neighborhoods that included oversamples of households with Asian and Hispanic surnames and households located in from disproportionately black tracts (that still satisfied the criteria as multiethnic neighborhoods).

{[}Figure @ref(fig:map) about here{]}

Table @ref(tab:nhddescriptives) contains a comparison of multiethnic neighborhoods to all neighborhoods in the DC area. Whites make up about a third of multiethnic neighborhoods, on average, while Latinxs make up about a quarter. Blacks and Asians make up 22 and 18 percent of residents, respectively.

{[}Table @ref(tab:nhddescriptives) about here{]}

Immigration represented the largest deviation of multiethnic neighborhoods from DC area neighborhoods overall. Immigrants make up nearly two of every five residents in multiethnic neighborhoods, compared to just under one of every four residents in neighborhoods overall. Residents of multiethnic neighborhoods had slightly lower educational attainment than neighborhoods overall. One in four residents of multiethnic neighborhoods, however, had a bachelor's degree and one in five had a post-graduate degree. Married households and those with children comprised a larger share of households in multiethnic neighborhoods compared to DC area neighborhoods overall.

Eighteen neighborhoods met the first inclusion criterion but not the second. Of those eighteen, fifteen were neighborhoods that had a white majority; six were located in Fairfax County, Virginia (including one in Fairfax city), five in Montgomery County, Maryland, and two each in D.C. and Arlington County, Virginia. Two neighborhoods had a black majority, one each in Montgomery County and Prince George's County, and only one neighborhood, in Montgomery County, had a Latinx majority.

A 12-page survey booklet was mailed to 9,600 households in those two types of neighborhoods along with a one-page cover letter written in English and Spanish on opposite sides, a two-dollar bill as an incentive, and a postage-paid return envelope. A reminder was sent to all households that had not responded approximately two weeks after the original survey was sent. By the end of the 54-day field period, 1,222 households responded with complete surveys, 674 of which were from households in multiethnic neighborhoods. The responses from the entire sample represeted a 12.8 percent response rate (AAPOR RR4 standard).

\subsection{Dependent Variables}\label{dependent-variables}

The analyses use dichotomous measures of satisfaction and neighborhood improvement as dependent variables. To measure current neighborhood \emph{satisfaction}, respondents who indicated that they were ``extremely'' or ``very'' satisfied living in their neighborhood were coded as being satisfied. Those who indicated that they were ``somewhat'' or ``not at all'' satisfied were coded as being unsatisfied.

The measure of neighborhood improvement was taken from the respondents' answers to the question, ``Looking back over the past five years or so, would you say that your neighborhood has\(\ldots\)'' I coded respondents as having indicated that their neighborhood improved if they answered that their neighborhood has become a ``much better'' or ``somewhat better'' place to live. Respondents coded as not perceiving neighborhood improvement were those who answered that their neighborhoods ``about the same,'' ``somewhat worse,'' or ``much worse.''

\subsection{Independent Variable}\label{independent-variable}

A measure created from respondents' self-identified race and ethnicity is the independent variable in all analyses. Respondents were allowed to pick multiple racial categories and indicate if they identified as Latinx. I collapsed these two variables to create four categories: \emph{Latinx} for those of any race who identified as Latinx; \emph{white} for those who identified only as white; \emph{black} for those who non-Latinx respondents who identified as black, either alone or in combination with any other race; and \emph{Asian} for those non-Latinx respondents who identified as either Asian or Pacific Islander, either alone or in combination with any other race other than black. Respondents missing on either the race or ethnicity question and respondents who identified with other racial groups were not included in the analysis. This left a final analytic dataset of \{N\} respondents.

\subsection{Control Variables}\label{control-variables}

In addition to racial variables, I also included variables to control for other demographic characteristics of residents. I calculated \emph{age} based on respondents' birth year and \emph{gender} based on respondents' gender identity, for which they were given the choices ``male,'' ``female,'' and ``other.'' Male was set as the reference. Only two respondents chose ``other'' and were coded as missing. I included an indicator for \emph{being partnered} based on those respondents who indicated that they were ``now married or in a marriage-style arrangment,'' with those not currently partnered as the reference.

I included whether the respondent had \emph{children in the household} because the presence of children would likely affect the outlook residents have of their neighborhoods. The question did not limit the response to respondents' own children. I also included an indicator to measure whether respondents were \emph{foreign born} since immigrant status would likely correlate with residential perceptions and race simultaneously.

I represented economic status with a measure of \emph{educational attainment}. I created a measure with five categories based on the highest level of education respondents reported: less than high school, high school (including GED), some college (including associates degrees), bachelors degree, and graduate degree. I chose to use educational attainment to measure socioeconomic status because it is a more stable measure than income, and space did not permit the DCAS to ascertain income longitudinally. Income also had more missing data compared to educational attainment (N=\{miinc\}) instead of (N=\{mieduc\}). Among those respondents who answered both questions, however, education and income were highly correlated.

In addition to demographic characteristics, I also included controls for the respondent's experience in the neighborhood. First, I included the \emph{years respondents lived in their neighborhoods} since the length of time lived in neighborhoods could affect perceptions of quality and would certainly affect perceptions of neighborhood change. Second, I included a measure of \emph{neighborhood size}. This measure was self-reported by respondents and including it in models accounts for how subjective perceptions of neighborhood boundaries affect satisfaction and perceived change.

\subsection{Summary}\label{summary}

The 2016 DCAS data contain the elements required to test the hypotheses stated above. The data represent the whole of residents of multiethnic neighborhoods in a metropolitan sample. The data also provide sufficient power to compare the responses of all four racial groups to one another due to the oversamples. The questions used to measure satisfaction and improvement come from previous surveys and allow for the direct comparison of responses to the existing literature. Together, these attributes overcome problems with previous data sources to investigate interracial differences in multiethnic neighborhoods.

\section{Analytical Approach}\label{analytical-approach}

I used logistic regression analysis to test the hypotheses above. The general model estimated can be expressed as:

\[\mathbf{y} = \alpha + \mathbf{\beta^T X} + \mathbf{\gamma^T Z} + \mathbf{\delta_j} + \mathbf{\epsilon}\]

\noindent where \(\mathbf{y}\) is the vector of outcomes for respondents, \(\alpha\) is the intercept, \(\mathbf{\beta}\) is a vector of point estimates for racial groups (with whites omitted), \(\mathbf{X}\), and \(\mathbf{\epsilon}\) is a vector of individual-specific errors. Non-Latinx whites were the reference group since the hypotheses ask whether whites differ from other groups. The vector \(\mathbf{\gamma}\) contains point estimates of demographic and neighborhood experience controls, \(\mathbf{Z}\). I first estimated each outcome with only the intercept and race terms included. I then estimated each outcome with controls present. I conducted all analyses in R version \Rversion.

All models included a fixed effect, \(\mathbf{\delta_j}\), for the neighborhood of residence. Neighborhoods were defined as residents' census tracts. Including neighborhood fixed effects makes the estimates, \(\mathbf{\beta}\), reflect the differences between white outcomes and those of Asians, blacks, and Latinxs \emph{living in the same neighborhood}. Respondents who lived in neighborhoods without other respondents (N=\oneresp) were removed from the analysis. The remaining respondents lived in tracts with a median of \medNpertract and a maximum of \maxNpertract.

All analyses accounted for missing data and the complex survey design. I used the \texttt{Amelia} package (cite) to impute missing values in five datasets. I conducted all analyses using these five datasets weighting outcomes to account for the complex sample design using the \texttt{survey} library. I combined all results using Rubin's rules {[}TK:cite{]}. All code for the models presented here are available at {[}redacted{]}.

\section{Results}\label{results}

\subsection{Satisfication Living in Multiethnic Neighborhoods}\label{satisfication-living-in-multiethnic-neighborhoods}

Most residents of multiethnic neighborhoods, \meansatisfied percent, are satisfied living in their neighborhood. Satisfaction was equally high among all four racial groups: \apisatisfied percent of Asians, \hspsatisfied percent of Latinxs, \nhbsatisfied percent of blacks, and \nhwsatisfied percent of whites in the sample were either ``extremely'' or ``very'' satisfied living in their multiethnic neighborhoods.

The agreement across racial groups held when I statistically compared groups within the same neighborhoods by modeling satisfaction with neighborhood fixed effects. The first column of Table @ref(tab:satisfaction) reports the results of a model including only respondent race and neighborhood fixed effects. The estimates for racial groups are small and not distinguishable from zero.

{[}Table @ref(tab:satisfaction) about here{]}


To further show the lack of difference between racial groups, I plotted the predicted probability from the model in the left panel of Figure @ref(fig:satisfaction). The plotted figures represent the predicted probability in the neighborhood with median satisfaction in the data, and reflect the estimates predicted across the five imputed data sets combined using Rubin's rules \textbf{{[}need cite{]}}. \textbf{{[}Look into replacing this estimate with the AME{]}} Only \maxdiffone percent separated the most satisfied group (Latinxs) from the least satisfied (Asians).

{[}Figure @ref(fig:satisfaction) about here{]}

The second column of Table @ref(tab:satisfaction) further confirmed that satisfaction was an interracial sentiment among residents of multiethnic neighborhoods. The estimates were again small and could not be distinguished from a null effect. The right panel of Figure @ref\{fig:satisfaction\} shows the predicted probabilities of satisfaction from the model with controls. I estimated the probabilities and standard errors at the reference level for all other variables in the model, meaning the figure represents the satisfaction of a 50-year-old, native-born, unmarried woman with a high school degree and no children who has lived in her neighborhood for 10 years. \textbf{{[}Look into replacing this estimate with the AME{]}} In the complete model, only \maxdiffthree separated the least satisfied group (now whites) from the most satisfied (still Latinxs).

Not only are individual coefficients not large or statistically singificant, but including race does not improve the explanatory power of the model. The final column of Table @ref(tab:satisfaction) reports the results of model estimating satisfaction in which I did not include race. The bottom row of the table reports the Akaike information criterion (AIC), a measure that balances a model's parsimony with its goodness of fit to the data \textbf{{[}need cite{]}}. A lower AIC represents a better fit to the data, and we can observe from Table @ref(tab:satisfaction) that the model without race fits the data better than the model that includes race. I further affirmed the lack of explanatory power attributable to race by conducting Wald's F-test between the models with and without race. The mean value of the test across the imputed data sets was \satWaldF, and the p-values ranged from \satWaldpMin to \satWaldpMax.

These results establish that residents are satisfied living in multiethnic neighborhoods regardless of their own race. Two other factors, education and the perceived size of the neighborhood, are associated with neighborhood satisfaction rather than race. Education had a curvilinear relationship with satisfaction among multiethnic neighborhood residents. Residents of multiethnic neighborhoods with a high school degree were the most satisfied, while residents with less than a high school education and those with a postgraduate degree felt satisfaction the least often. The perceived size of the neighborhood also mattered, as those residents who thought their neighborhood comprised 10-50 blocks were \satnhdsize times more likely to be satisfied than those who perceived their neighborhood to be only one to nine blocks. \textbf{{[}Connect back to idea that this is a sufficient size to avoid the worst of microsegregation, so people are actually satisfied with their neighborhood with multiracial composition.{]}}

While seven in ten residents of multiethnic neighborhoods were satisfied, half that number (\meanimproved \%) thought that their neighborhood had become a ``much better'' or ``somewhat better'' place to live over the past five years. Also unlike satisfaction, the probability that residents perceived improvements varied by race. Thirty-eight percent of Asian and Latinx residents, and 35 percent of black residents, felt that their neighborhoods had improved, while only 27 percent of whites felt that way.

The first column of Table @ref(tab:improvement) shows that the racial differences were unlikely due to the randomness of the sample. The results show a strong association between perceived improvement and race: whites were much less likely than Asian, black, or Latinx residents to believe that their neighborhoods had improved. The left panel of Figure @ref(fig:improvement) shows the lower probability at which whites were predicted to report improvement in multiethnic neighborhoods than the probabilities at which people of color were predicted to report improvement. \textbf{{[}Look into replacing this estimate with the AME{]}}
{[}Table \ref{tab:improvement} about here{]}

The lower propensity of whites to report improvement in multiethnic neighborhoods persisted after including controls. The second column of Table @ref(tab:improvement) shows positive associations for all three non-white racial groups. Asians were \apibetter~times more likely to report improvement in their neighborhood, blacks were \nhbbetter~times more likely, and Latinos were \hspbetter~times more likely. The latter two differences were large enough to be confident that the difference was not due to sampling variation at the p=0.05 level. The right panel of Figure \ref(fig:improvement) shows the predicted probabilities of reporting neighborhood improvement across racial groups. \textbf{{[}Look into replacing this estimate with the AME{]}}

{[}Figure @ref(fig:improvement) about here{]}

In addition to the differences between whites and people of color in the outcome, the model that included race fit the data better than the model that did not. The third column of Table @ref(tab:improvement) reports coefficient estimates and standard errors of the model without race and shows that the AIC value increases (fits less well) between the full model and the model without race. Wald F-tests of including race versus not provided evidence in the same direction: the mean value of the tests across imputed data sets was \betWaldF~and the mean p-value was \betWaldp.

In summary, a lower proportion of whites perceived improvement in multiethnic neighborhoods than their neighbors of color. Whites, therefore, might be satisfied living in their multiethnic neighborhood, but not inclined to see their neighborhood being a better place to live in the future. Neighborhood racial diversity would have been growing in the time period whites perceived less improvement than their neighbors. The present study provides insufficient evidence to conclude that increasing diversity \emph{caused} whites to perceive less improvement than their neighbors, but it does indicate\textbf{{[}w.ch{]}} a temporal correlation between growing diversity and a lower white endorsement of neighborhood change.

Projecting forward, too, the less positive trajectory whites perceive could portend a future in which whites might be more inclined, on the margin, to move out of the neighborhood than their non-white neighbors. Such marginal differences combined with the high rates of satisfaction among whites would be unlikely to instigate white flight, but the higher probability of losing white households could increase the chances of long-term racial turnover. The chances of long-term turnover would increase further if multiethnic neighborhoods attracted people of color at higher rates than whites.

\section{Familiarity with and Preferences for Other Multiethnic Places}\label{familiarity-with-and-preferences-for-other-multiethnic-places}

The long-term stability of multiethnic neighborhoods requires that residents of all races consider moving to existing multiethnic neighborhoods. The subsequent analyses assess how likely residents are to consider neighborhoods and how likely they are to actively not consider multiethnic neighborhoods.

\subsection{Familiarity with Multiethnic Places}\label{familiarity-with-multiethnic-places}

Two measures of familiarity with other multiethnic places in the Washington, D.C. area:

\begin{enumerate} \def\labelenumi{\arabic{enumi}.} \tightlist \item   Whether the respondent reports knowing anything about the place \item   Whether respondent reports having friends or family in the place \end{enumerate}

Uses the same analytical strategy, except that we add two covariates to each model:

\begin{enumerate} \def\labelenumi{\arabic{enumi}.} \tightlist \item   Whether the respondent reports living in the place \end{enumerate}

\subsection{Where People Would Consider Moving}\label{where-people-would-consider-moving}

\subsection{Where Would People Never Consider Moving}\label{where-would-people-never-consider-moving}

\section{Notes}\label{notes}

\subsection{Notes from Writing Group April 4, 2019}\label{notes-from-writing-group-april-4-2019}

\subsubsection{Comments}\label{comments}

\begin{description} \item[David] \begin{itemize} \tightlist \item   Like the setup on the front end (but didn't have a whole lot of time   to read it) \item   The graphs with point estimates and confidence intervals were helpful   ``to prove the null'' \item   Examine whether a joint F-test for significance of the race variables   together provide significant improvement in fit; alternatively, use   the AIC \& BIC values to run the full model without race variable and   then with race variable \item   \textbf{Add fixed effects}

  \begin{itemize}   \tightlist   \item     Have fixed effects in the models the entire time   \item     Used fixed effects to frame the method as being about comparing     residents of different races \emph{in the same neighborhood}   \end{itemize} \end{itemize} \item[Alex] \begin{itemize} \tightlist \item   Was left with the question, so what? Bring out that this fits with my   research showing that white acceptance of non-white neighbors is   insufficient, they need to seek diverse neighborhoods for integration   to persist \item   I am selecting on people who stayed, so these would be the whites   \emph{most} inclined to be satisfied with their neighborhoods; bring   out that in most of these neighborhoods, whites are still the majority   -- so if they \emph{are} selected, there is still a large proportion   of whites satisfied living in multiethnic neighborhoods \item   Subset by parental status (link to Ann Owens' ASR paper) \item   Point out that this study is unique to some degree to DC, which is a   type of multiethnic metropolitan area that is often overlooked (it's   not a black/white city, which are now the anomolies, places like DC   are the future) \end{itemize} \end{description}

\subsubsection{Potential Journals}\label{potential-journals}

\begin{itemize} \tightlist \item   \emph{Demography} especially with Mark Hayward as the editor \item   Possibly worth trying at a general sociology journal like \emph{Social   Forces}; subject-matter also fits at \emph{Social Problems} and   \emph{Social Science Research} \item   \emph{Urban Affairs Review} \end{itemize}

\subsubsection{Future Directions}\label{future-directions}

\begin{itemize} \tightlist \item   Do white conservatives look different than white liberals? Maybe find   a political scientist interested in neighborhoods to collaborate \item   Look at paper by Ryan Enos about how the demolition of public housing   shifted whites' political views:   https://onlinelibrary.wiley.com/doi/epdf/10.1111/ajps.12156 \end{itemize}
>>>>>>> master


<<<<<<< HEAD
\begin{enumerate}
\def\labelenumi{\arabic{enumi}.}
\tightlist
\item
  Whether the respondent reports living in the place
\end{enumerate}

\subsection{Where People Would Consider
Moving}\label{where-people-would-consider-moving}

\subsection{Where Would People Never Consider
Moving}\label{where-would-people-never-consider-moving}

\section{\textless{}\textless{}\textgreater{}\textgreater{}}\label{section-4}

\section{Notes}\label{notes}

\subsection{Notes from Writing Group April 4,
2019}\label{notes-from-writing-group-april-4-2019}

\subsubsection{Comments}\label{comments}

\begin{description}
\item[David]
\begin{itemize}
\tightlist
\item
  Like the setup on the front end (but didn't have a whole lot of time
  to read it)
\item
  The graphs with point estimates and confidence intervals were helpful
  ``to prove the null''
\item
  Examine whether a joint F-test for significance of the race variables
  together provide significant improvement in fit; alternatively, use
  the AIC \& BIC values to run the full model without race variable and
  then with race variable
\item
  \textbf{Add fixed effects}

  \begin{itemize}
  \tightlist
  \item
    Have fixed effects in the models the entire time
  \item
    Used fixed effects to frame the method as being about comparing
    residents of different races \emph{in the same neighborhood}
  \end{itemize}
\end{itemize}
\item[Alex]
\begin{itemize}
\tightlist
\item
  Was left with the question, so what? Bring out that this fits with my
  research showing that white acceptance of non-white neighbors is
  insufficient, they need to seek diverse neighborhoods for integration
  to persist
\item
  I am selecting on people who stayed, so these would be the whites
  \emph{most} inclined to be satisfied with their neighborhoods; bring
  out that in most of these neighborhoods, whites are still the majority
  -- so if they \emph{are} selected, there is still a large proportion
  of whites satisfied living in multiethnic neighborhoods
\item
  Subset by parental status (link to Ann Owens' ASR paper)
\item
  Point out that this study is unique to some degree to DC, which is a
  type of multiethnic metropolitan area that is often overlooked (it's
  not a black/white city, which are now the anomolies, places like DC
  are the future)
\end{itemize}
\end{description}

\subsubsection{Potential Journals}\label{potential-journals}

\begin{itemize}
\tightlist
\item
  \emph{Demography} especially with Mark Hayward as the editor
\item
  Possibly worth trying at a general sociology journal like \emph{Social
  Forces}; subject-matter also fits at \emph{Social Problems} and
  \emph{Social Science Research}
\item
  \emph{Urban Affairs Review}
\end{itemize}

\subsubsection{Future Directions}\label{future-directions}

\begin{itemize}
\tightlist
\item
  Do white conservatives look different than white liberals? Maybe find
  a political scientist interested in neighborhoods to collaborate
\item
  Look at paper by Ryan Enos about how the demolition of public housing
  shifted whites' political views:
  https://onlinelibrary.wiley.com/doi/epdf/10.1111/ajps.12156
\end{itemize}


\end{document}

