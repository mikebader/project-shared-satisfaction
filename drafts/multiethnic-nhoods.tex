\documentclass{baderart}

\usepackage{hhline}
\usepackage{calc}
\usepackage{xifthen}

\providecommand{\tightlist}{%
  \setlength{\itemsep}{0pt}\setlength{\parskip}{0pt}}

\newcommand{\TK}[1][]{\strong{TK #1}}
\renewcommand{\needcite}[1][]{%
	\strong{[CITE: %
		\ifthenelse{\equal{#1}{}}{}{: #1}	%
	]}}



\title{Satisfaction and the Stability of Multiracial Neighborhoods}
\author{Michael D.\ M.\ Bader}

%% Values recorded from analysis source files
\newcommand{\N}{641}
\newcommand{\miinc}{43}
\newcommand{\miedu}{2}
\newcommand{\Rversion}{3.5.0}
\newcommand{\oneresp}{9}
\newcommand{\medNpertract}{5}
\newcommand{\maxNpertract}{19}
\newcommand{\meansatisfied}{71}
\newcommand{\apisatisfied}{72}
\newcommand{\nhbsatisfied}{67}
\newcommand{\hspsatisfied}{76}
\newcommand{\nhwsatisfied}{69}
\newcommand{\satWaldF}{0.54}
\newcommand{\satWaldpMin}{0.61}
\newcommand{\satWaldpMax}{0.69}
\newcommand{\satnhdsize}{3.16}
\newcommand{\maxdiffone}{8}
\newcommand{\maxdiffthree}{6}
\newcommand{\meanimproved}{34}
\newcommand{\betWaldF}{2.33}
\newcommand{\betWaldp}{0.07}
\newcommand{\apibetter}{2.36}
\newcommand{\nhbbetter}{2.91}
\newcommand{\hspbetter}{3.35}
\newcommand{\dkherndon}{46}
\newcommand{\dkgermantown}{31}
\newcommand{\dkwheaton}{37}
\newcommand{\ffherndon}{25}
\newcommand{\ffgermantown}{40}
\newcommand{\ffwheaton}{28}
\newcommand{\consherndon}{10}
\newcommand{\consgermantown}{10}
\newcommand{\conswheaton}{6}
\newcommand{\ncherndon}{21}
\newcommand{\ncgermantown}{17}
\newcommand{\ncwheaton}{27}



\begin{document}
\maketitle

\begin{abstract}
Multiracially integrated neighborhoods, which have emerged since the 1980s, disrupt conceptions of race and place in American metropolitan areas. What research exists suggests that increased racial diversity decreases neighborhood satisfaction and make multiracial integration difficult to sustain. Using data representative of residents living in multiracially integrated neighborhoods in the Washington, DC area, this article demonstrates that Asian, black, Latinx, and white residents all have high and, more importantly, \emph{equal} levels of satisfaction living in multiracially integrated neighborhoods. Some racial differences existed, however, in whether residents of multiracial neighborhoods would consider moving to other multiracial communities. Multiracial integration appears sustainable and should receive greater attention in theories of neighborhood racial change. 

\keywords{racial integration, neighborhood change, housing, regional migration, race}
\end{abstract}

\doublespace

\section{Introduction}\label{introduction}
Federal anti-discrimination laws, changing racial attitudes, and increased immigration have created new opportunities for racially integrated neighborhoods. Once exceptional, multiracial neighborhoods have emerged as a common feature in U.S. metropolitan areas during the past half century. Such growth has been surprising in a nation beset by its history of racial segregation. Multiracial neighborhoods provide an opportunity to evaluate the changing meaning of race and place in America and provide a beacon of hope for a more integrated future.

Academic research has not kept pace with the growing presence of multiracially integrated neighborhoods in American metropolitan areas. Research on the changing meaning of race and place has overwhelmingly focused on gentrification, resulting in a body of research on gentrification dwarfs that focusing on emergent integration. The relative frequency of each type of change cannot explain the under-representation of integrated neighborhoods in the literature: racial change related to gentrification occurs in a smaller number of neighborhoods than multiracial neighborhoods. 

A small body of qualitative studies has been the source of most research on the unique dynamics of integrated neighborhoods. These studies have found that residents of all races tend to enjoy living among racially diverse neighborhoods, contradicting longstanding models of neighborhoods that form around racial and ethnic solidarity.\needcite\ These qualitative studies point to evidence that long-term integration might be possible. But such long-term integration depends on a widely shared satisfaction among residents of all races living in diverse neighborhoods, a question better addressed by examining representative samples of residents living in integrated neighborhoods. Despite its necessity to assess the sustainability of multiracial integration, evidence based on statistical studies of multiracial neighborhoods does not exist. 

This article starts building statistical evidence on factors related to the long-term stability of integration. I first examined how satisfied Asian, black, Latinx, and white residents are living in multiracial neighborhoods. I then examined whether race was associated with dimensions that affect housing searches among \emph{other} multiracial communities. To study these questions, I used a representative sample of households in multiracial neighborhoods in the Washington, DC area. 

The results indicated that sustainable racial integration is possible, but not guaranteed. Evidence showed that a large majority of all racial groups are satisfied living in their current neighborhoods. Whites were less likely to report neighborhood improvement than their neighbors of color. Residents of multiracial communities were generally unfamiliar with integrated communities in the metropolitan area, but, again, levels of unfamiliarity did not vary by race. Some multiracial communities were more desirable to some groups than to others. The same was also try for rejection of neighborhoods, but those differences were largely explained by demographic differences across racial groups. The findings suggest a need to devote more attention to multiracial neighborhoods to understand the relationship between place and race in the twenty-first century. 


\section{Background}\label{background}
multiracial neighborhoods emerged as one of many types of neighborhood change that emerged since the 1970s. Other trends that emerged included gentrification and ``ethnoburbs.'' Changing metropolitan economies, family structures, and preferences fostered the gentrification of many urban--and even suburban--neighborhoods \citep{rose_rethinking_1984, slater_gentrification_2004, lung-amam_trespassers?:_2017}. A generation of whites who grew up after the Civil Rights Movement's successes in the 1960s started moving to urban neighborhoods that older generations avoided. People of color living in those neighborhoods have been displaced by economic, social, and legal means \citep{freeman_there_2006, hyra_back---city_2014, hyra_race_2017}. Simultaneously, the flow of immigrants and people of color into surrounding suburbs gave rise to ``ethnoburbs,'' ethnic enclaves that developed on the outskirts of major cities based on flows of capital and geopolitical crises \citep{li_anatomy_1998,  lung-amam_trespassers?:_2017, kye_rise_2018}. Taken together, these findings suggested a reversal of population flows that have led observers of metropolitan trends to reconceive longstanding theories of the relationship between race and place. 

A trend involving a growing number of multiracially integrated neighborhoods has passed largely unnoticed during the same period. Multiracial neighborhoods have come about as Asian, black, and Latinx newcomers moved into neighborhoods that had been all white, or nearly so \citep{logan_global_2010, bader_fragmented_2016} . The trend started in the 1980s and accelerated in the 1990s and 2000s. Unlike in the past, the entry of non-white residents did not spark violence or white flight. Instead, whites stayed put as their neighborhoods integrated around them. 

The result was a growing number of neighborhoods shared among white, Asian, black, and Latinx neighbors rose rapidly in the 1990s and 2000s. Logan and Zhang~\citeyearpar{logan_global_2011} found that 38~percent of neighborhoods in 2010 were what they termed ``global neighborhoods.'' Similarly, Bader and Warkentien~\citeyearpar{bader_fragmented_2016} found stable multiracial integration, what they called ``quadrivial neighborhoods,'' represented between eight and twenty percent of neighborhoods in the metropolitan areas of the four largest cities. Like ethnoburbs, these neighborhoods tended to be located in largely unknown suburbs than in better-known central city neighborhoods that once housed multiracial populations \citep{bader_fragmented_2016, parisi_remaking_2019}. 

Several currents made the growth of multiracial neighborhoods possible. Federal legislation, especially the Fair Housing Act, which was passed immediately after Rev.~Dr.~Martin Luther King,~Jr.'s assassination in 1968, made housing discrimination a federal crime. The impact of the Act was neither immediate nor universal. Racial discrimination persisted in more subtle ways such as real-estate agents steering clients to neighborhoods and targeted advertising \citep{ross_housing_2005, roscigno_complexities_2009}. While not a cure-all, the FHA provided provided a legal mechanism for people of color to enter white neighborhoods and challenge overt forms of discrimination. 

Second, \emph{multiracial} integration would be impossible without a multiracial population. Increased immigration from Asian and Latin American countries increased the number of Asian and Latinx people in the United States \citep{denton_patterns_1991, frey_latino_1996, singer_rise_2001}. Immigration unsettled the stark black-white divide in many American metropolitan areas \citep{alba_neighborhood_1995, logan_segregation_2004, friedman_declines_2008}. The change was most pronounced in metropolitan areas that became ``new immigrant'' destinations in the 1980s and 1990s \citep{singer_rise_2001, lichter_immigrant_2009}. The growing economies that attracted migrants also required new housing construction to accommodate the growing populations. Because they were built after the Fair Housing Act passed, the new neighborhoods were available to all racial groups from their initial construction. 

Third, white Americans have become more racially tolerant since the 1960s \citep{schuman_racial_1997}. For example, Gallup polls found that just four percent of whites agreed with interracial marriage in 1958, but 84~percent agreed in 2013 \citep{newport_u.s._2013}. 
%Attitudes have continued to become more tolerant, helped, in part, by Barack Obama's candidacy and election to the presidency \needcite. 
Some of the trend might reflect the growing awareness of socially desirable answers among whites that masks underlying racist sentiments \citep{jackman_education_1984, bonilla-silva_racism_2003}. Yet, even the idea that tolerant responses are now socially desirable represents progress from the 1970s, when outward expressions of racist sentiments were far from taboo even if they had become unpopular.

\subsection{Satisfaction and Stability of Multiracial Neighborhoods}\label{sec:satisfaction-stability}

\subsubsection{Satisfaction and Mobility}\label{satisfaction-and-mobility}

In his groundbreaking work, Rossi~\citeyearpar{rossi_why_1955} identified what he called ``residential stress'' as the reason families move. Stress comes from the dissatisfaction families have with their current living conditions. Families move to relieve their residential stress when dissatisfaction becomes unbearable, and they have the means to do so \citep{speare_residential_1974, speare_residential_1975}. The most common source of stress that leads families to move are those related to passing through the life-cycle, such as marriage and the birth of a child \citep{rossi_why_1955, clark_interpreting_2015, krysan_cycle_2017}. 

Neighborhood dissatisfaction can become another source of residential stress \citep{woldoff_effects_2002}. High crime rates or a lack of amenities might cause families to move. Racial composition has historically been a large source of dissatisfaction for whites \citep{crowder_racial_2000, crowder_spatial_2008}. The process of white flight can be described as whites having acted on the immediate and strong feelings of dissatisfaction that the entry of a black family prompted \citep{boustan_was_2010}. As whites vacated their houses, additional black families--who were dissatisfied being forced to live in overcrowded ghettos through law and custom--filled those vacancies \citep{sugrue_origins_1996, woldoff_white_2011}. The process resulted in the rapid ``succession'' of the neighborhood's composition from all-white to all-black \citep{duncan_negro_1957}.

But the distress of changing neighborhood composition seems to have waned since the 1980s. Rapid neighborhood transitions caused by white flight have all but disappeared \citep{ellen_sharing_2000, friedman_declines_2008, bader_fragmented_2016}. On the one hand, this evidence combined with more sanguine attitudes of whites toward people of color, suggests that its possible that whites have become more satisfied living in integrated neighborhoods. On the other hand, it is possible that the whites who remain in integrated neighborhoods could feel trapped by their circumstances, unsatisfied with the direction of their neighborhoods but unable to move \TK[SOUTH \& CROWDER?]. 

Research has generally shown a negative relationship between racial diversity and neighborhood satisfaction, especially among whites. Whites become more unsatisfied living in neighborhoods as the numbers of non-white neighbors increases \citep{lee_neighborhood_1994, crowder_racial_2000, swaroop_determinants_2011, greif_intersection_2015}. Whites also overestimate negative features of neighborhoods, such as crime and social disorder, when non-whites make up a larger share of neighbors \citep{quillian_black_2001, sampson_seeing_2004}, features which are negatively associated with neighborhood satisfaction \citep{woldoff_effects_2002, hipp_specifying_2009}. These studies have generally failed to consider the impact of \emph{multiracial} neighborhoods, though Hipp~\citeyearpar{hipp_specifying_2009} found that neighborhood racial diversity lowered levels of neighborhood satisfaction.  

This study examines multiracial neighborhoods exclusively. As a result, it differs from previous research in ways that help us predict the stability of multiracial neighborhoods. First, this study focuses on the satisfaction of whites already living in multiracial neighborhoods (satisfaction among all groups is necessary, but the historical propensity of whites to shun integration, however, warrants special attention). This study diverges from previous research that has only examined the context of white attitudes about neighborhoods in all different types of neighborhoods. Only the actions of white residents in multiracial neighborhoods immediately affect the racial composition of multiracial neighborhoods. 

Second, the exclusive focus on multiracial neighborhoods in this article considers how the unique \emph{combination} of all four racial groups affects satisfaction and perceptions. Previous studies have interpolated levels of satisfaction by assuming that integrated neighborhoods fall on a continuum between, for example, all-white and all-person-of-color. As I wrote earlier, multiracial neighborhoods do not fit on this spectrum, yet almost all existing studies interpolate from these continuous measures what outcomes are likely. 

Krysan and colleagues~\citeyearpar{krysan_diversity_2017}, however, provide an informative exception. Using a method where respondents could draw their own ideal neighborhood racial composition, they found that two-fifths of whites identified multiracial neighborhoods as their most preferred neighborhood.  Another 44~percent identified neighborhoods with a white majority, but substantial representation of blacks, Latinos, and Asians. A majorities of both blacks and Latinos also identified multiracial neighborhoods as being most desirable. Krysan and colleagues note that few respondents lived in the types of diverse neighborhoods respondents drew. The fact that a majority of respondents desired a mixed racial composition opens the possibility of interracial satisfaction among current residents of multiracial neighborhoods. 

\subsubsection{Long-Term Stability}\label{long-term-stability}

The maintenance of current neighborhood integration depends on all groups remaining in multiracial neighborhoods. The long-term stability of integration, however, depends on an interracial flow of newcomers into these existing communities. Even residents satisfied living in their neighborhood will find themselves exiting for reasons unrelated to the racial composition of the neighborhood. Residents will get married, retire, have children, get divorced, and, in the most extreme case, exit the neighborhood upon their death. Life-course transitions such as these cause most moves; and, as residents move out for those reasons, new residents will move in. Satisfaction will be insufficient to predict future integration because integration will survive only as long as the current cohort of residents does \citep[see, e.g., ][]{molotch_racial_1969, woldoff_white_2011}. To maintain the multiracial diversity of the neighborhood therefore requires that the racial composition of households \emph{leaving} multiracial neighborhoods approximates the racial composition of those \emph{entering} multiracial neighborhoods.

A second test of the sustainability of multiracial integration, therefore, is how likely current residents of multiracial neighborhoods are to consider moving \emph{to} integrated neighborhoods in a metropolitan area. If strong racial differences emerge among current residents, then the different flows into neighborhoods will make sustained integration less likely. Residents currently residing in multiracial neighborhoods provide an obvious clientele to fill vacancies in multiracial neighborhoods. Previous studies have not examined perceptions of multiracially integrated neighborhoods, which, as I mentioned above, do not fall on a continuum but represent a unique type of neighborhood composition. As a result, we have neither a sense whether current residents of multiracial neighborhoods would consider moving to another multiracial neighborhood, nor whether the proportion who would consider another multiracial neighborhood differs across racial groups.

Rather than being individually derived or collectively negotiated within households, housing preferences reflect a deeply social process. Households cannot feasibly engage all potential housing options across a metropolitan area \citep{brown_intra-urban_1970, maclennan_housing_1982, marsh_uncertainty_2011}. Krysan and Crowder \citeyearpar{krysan_cycle_2017} show that people approach housing searches based on word of mouth that shape what people think about different communities, and whether they even know anything about the community at all. Settling on which neighborhoods to consider---and which neighborhoods not to waste time considering---reflects social networks that are segregated by race.  Moves between multiracial communities, in the absence of racial differences in knowing about, considering, and rejecting other multiracial communities among current residents of multiracial neighborhoods, could help to sustain multiracial integration. 

Existing evidence shows that whites tend to know less about neighborhoods with larger shares of non-white residents, and tend to reject moving to those neighborhoods. The results hold whether researchers used hypothetical neighborhoods in experimental studies or actual communities in observational studies \citep{krysan_perceiving_2007, krysan_does_2009, lewis_who_2011, bader_community_2015}. At every stage of the search process, white movers tend to consider neighborhoods communities with whiter populations \citep{havekes_realizing_2016}: whites move to communities with lower shares of whites on average than communities that they search which, in turn, have lower shares of whites on average than communities whites report considering. All measure the influence of race using the percentage residents identifying with different racial groups; none considers the multiracial combination of residents that make multiracial neighborhoods unique \citep[for exception, see ][]{krysan_diversity_2017}.

Evidence of preferences and historical precedent suggest whites would find moving to other integrated neighborhoods less desirable then their non-white neighbors. Yet the white residents of multiracial neighborhoods already differ from most whites by living in multiracial neighborhoods. Multiracial neighborhoods might expose white residents to more diverse social networks and increase the probability that whites would consider another multiracial neighborhood. But, again, we have no evidence one way or the other among residents living in multiracial neighborhoods.

\subsubsection{Hypotheses}\label{hypotheses}

In light of existing evidence, this article examines two hypotheses:


\begin{enumerate} 

\item   Whites living in multiracial neighborhoods will be a) less satisfied  with their neighborhoods and b) less likely to perceive neighborhood  improvement compared to their Asian, black, and Latinx neighbors; and 

\item   Whites living in multiracial neighborhoods will be will be a) less familiar with, b) less likely to consider, and c) more likely to never consider moving to other multiracial neighborhoods in their metropolitan area compared to their Asian, black, and Latinx neighbors. 

\end{enumerate}

\section{Data \& Methods}\label{data}
\subsection{multiracial Neighborhoods in the Washington, D.C. Area}\label{multiracial-neighborhoods-in-the-washington-d.c.-area}

Data to test the hypotheses come from the 2016 DC Area Survey (DCAS). Washington, D.C. has historically been a segregated metropolitan area. Almost all blacks lived in the eastern part of the city and nearly all whites in the west. This east-west pattern spilled out to the Maryland suburbs. A few blacks lived across the Potomac River in Northern Virginia, though they were highly clustered in a small number of neighborhoods. In 1980, Asians and Latinxs each made up only 3~percent of the DC-area population.

The region's economy grew and became less based on the federal government in the 1980s and 1990. The expansion of finance, insurance, and real estate services, collectively known by the acronym FIRE, mirrored the sector's prominence in redeveloping metropolitan areas in the U.S. and abroad. During the same period, the region emerged as an immigrant destination: a quarter of residents in 2015 were born outside of the United States, with the largest shares of foreign-born residents having come from El~Salvador, India, and Korea. Latinos and Asians now comprise 16~and 11~percent, respectively, of DC-area residents while blacks make up 29~percent and whites 41~percent. As the nation's capital and importance in international relations, the region's foreign-born residents are more socioeconomically diverse than average. Among foreign-born residents ages 25 and older in the DC area, 21~percent have a bachelor's degree and 22~percent have a graduate degree, compared to 17~percent and 12~percent nationally. 

Real estate developers capitalized on the expanding population by developing large swaths of land in middle-ring suburbs in the counties surrounding the District. Having been built after the Fair Housing Act passed, the residences developed in these suburbs were not subject to the history of redlining restrictive covenants of previous developments. Residents of all races were attracted to these new suburban homes, especially in Montgomery and Fairfax Counties that boasted nationally renowned school systems. Additionally, Montgomery County has, since 1974, mandated that all housing developments larger than 20 units include a set percentage of units that qualify as affordable housing \citep{urban_institute_expanding_2012}.\footnote{Fairfax County passed an inclusionary   zoning ordinance in 1971, but it was struck down by the Virginia   Supreme Court. Fairfax County implemented a different mandatory   inclusionary zoning policy in 1990 \citep{silverstein_welcome_2017}.} As Figure~\ref{fig:map} shows, multiracial neighborhoods were especially likely to emerge along the hub-and-spoke system of the region's commuter rail lines in these counties.

\abouthere{Figure}{fig:map} 

The 2016 DCAS sought to represent residents of two types of neighborhoods--multiracial and disproportionately Latinx--in the Washington, D.C. area, which comprised Washington, D.C. and the surrounding jurisdictions of Montgomery and Prince George's Counties in Maryland, Arlington and Fairfax (including the cities of Falls Church and Fairfax) Counties Virginia, and the independent city of Alexandria, Virginia. Only responses from residents sampled in the multiracial neighborhoods were included in this analysis.\footnote{Disproportionately   Latinx neighborhoods were those in which Latinx residents made up at   least a quarter of the residents and were not already classified as a   multiracial neighborhood.}

multiracial neighborhoods included in the sampling frame met two criteria. First, Asians, blacks, Latinxs, and whites each comprised at least 10~percent of the population of the neighborhood. This cutoff was used to ensure that each of the four racial groups represented a distinguishable subset of a neighborhood's residents. Second, none of those four groups could represent a majority of residents. This criterion was included to eliminate neighborhoods where a single group formed the dominant identity of the neighborhood. These criteria resulted in a sample frame of 114 neighborhoods that represented just under 585,000 people. The majority of neighborhoods were in Montgomery County, Maryland, followed by Fairfax County, Virginia, then Prince George's County, Maryland. The neighborhoods included in the sampling frame are shown in Figure~\ref{fig:map}. An address-based sample of households was drawn from these eligible neighborhoods that included oversamples of households with Asian and Hispanic surnames and households located in from disproportionately black tracts (that still satisfied the criteria as multiracial neighborhoods).

Eighteen neighborhoods met the first inclusion criterion but not the second. Of those eighteen, fifteen were neighborhoods that had a white majority; six were located in Fairfax County, Virginia (including one in Fairfax city), five in Montgomery County, Maryland, and two each in D.C. and Arlington County, Virginia. Two neighborhoods had a black majority, one each in Montgomery County and Prince George's County, and only one neighborhood, in Montgomery County, had a Latinx majority.

Tabl~ \ref{tab:nhddescriptives} contains a comparison of multiracial neighborhoods to all neighborhoods in the DC area. Whites make up about a third of multiracial neighborhoods, on average, while Latinxs make up about a quarter. Blacks and Asians make up 22 and 18 percent of residents, respectively.

\abouthere{Table}{tab:nhddescriptives}

Immigration represented the largest deviation of multiracial neighborhoods from DC area neighborhoods overall. Immigrants make up nearly two of every five residents in multiracial neighborhoods, compared to just under one of every four residents in neighborhoods overall. Residents of multiracial neighborhoods had slightly lower educational attainment than neighborhoods overall. One in four residents of multiracial neighborhoods, however, had a bachelor's degree and one in five had a post-graduate degree. Married households and those with children comprised a larger share of households in multiracial neighborhoods compared to DC area neighborhoods overall.

%A 12-page survey booklet was mailed to 9,600 households in those two types of neighborhoods along with a one-page cover letter written in English and Spanish on opposite sides, a two-dollar bill as an incentive, and a postage-paid return envelope. A reminder was sent to all households that had not responded approximately two weeks after the original survey was sent. By the end of the 54-day field period, 1,222 households responded with complete surveys, 674 of which were from households in multiracial neighborhoods. The responses from the entire sample represented a 12.8 percent response rate (AAPOR RR4 standard).

\subsection{Dependent Variables}\label{dependent-variables}

\paragraph{Satisfaction and Improvement of Current Neighborhood.}
The analyses use dichotomous measures of satisfaction and neighborhood improvement as dependent variables. The DCAS asked respondents, ``How satisfied are you with your neighborhood as a place to live?,'' a question that has been used in previous research \citep[see, e.g.,][] {woldoff_effects_2002, greif_intersection_2015}. To measure current neighborhood \emph{satisfaction}, respondents who indicated that they were ``extremely'' or ``very'' satisfied living in their neighborhood were coded as being satisfied. Those who indicated that they were ``somewhat'' or ``not at all'' satisfied were coded as being unsatisfied.

The measure of neighborhood improvement was taken from the respondents' answers to the question, ``Looking back over the past five years or so, would you say that your neighborhood has\(\ldots\)'' I coded respondents as having indicated that their neighborhood improved if they answered that their neighborhood had become a ``much better'' or ``somewhat better'' place to live. Respondents coded as not perceiving neighborhood improvement were those who answered that their neighborhoods were ``about the same,'' ``somewhat worse,'' or ``much worse.''

\paragraph{Familiarity, Consideration, and Rejection of DC-Area Multiracial Communities.}

To test whether whites living in multiracial neighborhoods knew about and considered other integrated communities in the area, the DCAS asked respondents a series of questions about specific communities. The five questions asked whether respondents ``didn't know anything about'', ``have friends or family that live in'', ``would seriously consider moving to'', ``would never consider moving to'', and ``live in'' the community. Respondents were asked these questions about eleven communities in the DC area.

Asking about specific communities comes with a couple of costs. The first problem arises because the specific reputations of places can lead to ignorance and sentiments about those specific communities. In other words, the main benefit of asking about communities, the increased realism of the process, is also a disadvantage. The Potomac River likely divides knowledge and consideration since residents tend to have subregional identities that don't ``cross the river'' \citep{lacy_blue-chip_2007}.

Summarizing the results across multiple communities presents the second problem. Out of the eleven communities about which respondents were asked, I focused the analysis about knowledge and consideration on three communities: Herndon, Virginia; Germantown, Maryland; and Wheaton, Maryland. Among the eleven communities, these were the only three that met the same criteria I used to define multiracial neighborhoods: each racial group made up at least ten percent of the population and no group made up a majority. An outline of each community, based on Census Designated Place borders (which were also used to measure racial composition), are shown on Figure \ref{fig:map}.

I used questions matching those in previous work to analyze knowledge and preferences. To gauge knowledge, I used the question asking whether respondents ``didn't know anything about'' the community, the same question used by Krysan and Bader \citeyearpar{krysan_racial_2009}. Since the question asks about \emph{not} knowing, I will discuss the results in terms of \emph{unfamiliarity} of communities to avoid awkward double-negative constructions. To guage whether respondents would consider or reject the neighborhood, I used the same questions Krysan and Bader \citeyearpar[][\citealt{bader_community_2015}]{krysan_perceiving_2007, krysan_racial_2009} used that asked respondents whether they would ``seriously consider'' and ``never consider'' the community respectively. I used logistic regression including tract-level neighborhood fixed-effects to analyze the data, the same method that I used for the analysis reported in previous sections.\footnote{I originally included a variable indicating whether a respondent reported living in the target community. The near-perfect correlation of answers within neighborhoods, for which I included fixed effects, destabilized the model estimates. To preserve the within-neighborhood nature of the estimates, I opted to keep the fixed effects and remove whether the respondent indicated living in the community.}

\subsection{Independent Variable}\label{independent-variable}

A measure created from respondents' self-identified race and ethnicity is the independent variable in all analyses. I classified racial groups into four mutually exclusive categories based on the respondents' answers to questions about race (respondents were allowed to choose multiple racial groups) and Hispanic ethnicity. 
I classified those respondents who indicated Hispanic ethnicity as \emph{Latinx}, regardless of their race. Among the remaining non-Latinx respondents, I classified as \emph{white} those who selected white as their only race; as \emph{black} those identified as black, either alone or in combination with any other race; and as \emph{Asian} those who identified as Asian or Pacific Islander, either alone or in combination with any other race other than black. Respondents missing on either the race question or the ethnicity question and respondents who did not satisfy the criteria above were not included in the analysis. This left a final analytic dataset of~\N\ respondents.

\subsection{Control Variables}\label{control-variables}

In addition to racial variables, I also included variables to control for other demographic characteristics of residents. I calculated \emph{age} based on respondents' birth year and \emph{gender} based on respondents' gender identity, for which they were given the choices ``male,'' ``female,'' and ``other.'' Male was set as the reference. Only two respondents chose ``other'' and were coded as missing. I included an indicator for \emph{being partnered} based on those respondents who indicated that they were ``now married or in a marriage-style arrangement,'' with those not currently partnered as the reference.

I included whether the respondent had \emph{children in the household} because the presence of children would likely affect the outlook residents have of their neighborhoods. The question did not limit the response to respondents' own children. I also included an indicator to measure whether respondents were \emph{foreign born} since immigrant status would likely correlate with residential perceptions and race simultaneously.

I represented economic status with a measure of \emph{educational attainment}. I created a measure with five categories based on the highest level of education respondents reported: less than high school, high school (including GED), some college (including associates degrees), bachelors degree, and graduate degree. I chose to use educational attainment to measure socioeconomic status because it is a more stable measure than income, and space did not permit the DCAS to ascertain income longitudinally. Income also had more missing data  (N=\miinc) compared to educational attainment (N=\miedu). Among those respondents who answered both questions, however, education and income were highly correlated.

In addition to demographic characteristics, I also included controls for the respondent's experience in the neighborhood. First, I included the \emph{years respondents lived in their neighborhoods} since the length of time lived in neighborhoods could affect perceptions of quality and would certainly affect perceptions of neighborhood change. Second, I included a measure of \emph{neighborhood size}. This measure was self-reported by respondents and including it in models accounts for how subjective perceptions of neighborhood boundaries affect satisfaction and perceived change.

The 2016 DCAS data contain the necessary elements to test the hypotheses stated above. The data represent the whole of residents of multiracial neighborhoods in a metropolitan sample. The data also provide sufficient power to compare the responses of all four racial groups to one another due to the oversamples. The questions used to measure satisfaction and improvement come from previous surveys and allow for the direct comparison of responses to the existing literature\needcite. Together, these attributes overcome problems with previous data sources to investigate interracial differences in multiracial neighborhoods.

\section{Analytical Approach}\label{analytical-approach}

I used logistic regression analysis to test the hypotheses above. The general model estimated can be expressed as:

\[\mathbf{y} = \alpha + \mathbf{\beta^T X} + \mathbf{\gamma^T Z} + \mathbf{\delta_j} + \mathbf{\epsilon}\]

\noindent where \(\mathbf{y}\) is the vector of outcomes for respondents, \(\alpha\) is the intercept, \(\mathbf{\beta}\) is a vector of point estimates for racial groups (with whites omitted), \(\mathbf{X}\), and \(\mathbf{\epsilon}\) is a vector of individual-specific errors. Non-Latinx whites were the reference group since the hypotheses ask whether whites differ from other groups. The vector \(\mathbf{\gamma}\) contains point estimates of demographic and neighborhood experience controls, \(\mathbf{Z}\). I first estimated each outcome with only the intercept and race terms included. I then estimated each outcome with controls present. I conducted all analyses in R version \Rversion.

All models included a fixed effect, \(\mathbf{\delta_j}\), for the neighborhood of residence. Neighborhoods were defined as residents' census tracts. Including neighborhood fixed effects makes the estimates, \(\mathbf{\beta}\), reflect the differences between white outcomes and those of Asians, blacks, and Latinxs \emph{living in the same neighborhood}. Respondents who lived in neighborhoods without other respondents (N=\oneresp) were removed from the analysis. The remaining respondents lived in tracts with a median of \medNpertract and a maximum of \maxNpertract.

All analyses accounted for missing data and the complex survey design. I used the \texttt{Amelia} package~\citep{honaker_amelia_2011} to impute missing values in five datasets. I conducted all analyses using these five datasets weighting outcomes to account for the complex sample design using the \texttt{survey} library. I combined all results using Rubin's~\citeyearpar{rubin_multiple_2004} rules. All code for the models presented here are available at [redacted].

\section{Results}\label{results}

\subsection{Satisfication Living in Multiracial Neighborhoods}\label{satisfication-living-in-multiracial-neighborhoods}

Most residents of multiracial neighborhoods, \meansatisfied percent, are satisfied living in their neighborhood. Satisfaction was equally high among all four racial groups: \apisatisfied percent of Asians, \hspsatisfied percent of Latinxs, \nhbsatisfied percent of blacks, and \nhwsatisfied percent of whites in the sample were either ``extremely'' or ``very'' satisfied living in their multiracial neighborhoods.

The agreement across racial groups held when I statistically compared groups within the same neighborhoods by modeling satisfaction with neighborhood fixed effects. The first column of Table~\ref{tab:satisfaction} reports the results of a model including only respondent race and neighborhood fixed effects. The estimates for racial groups are small and not distinguishable from zero.

\abouthere{Table}{tab:satisfaction}


To further show the lack of difference between racial groups, I plotted the predicted probability from the model in the left panel of Figure~\ref{fig:satisfaction}. The plotted figures represent the predicted probability in the neighborhood with median satisfaction in the data, and reflect the estimates predicted across the five imputed data sets combined using Rubin's~\citep{rubin_multiple_2004} rules. \textbf{{[}Look into replacing this estimate with the AME{]}} Only \maxdiffone percent separated the most satisfied group (Latinxs) from the least satisfied (Asians).

\abouthere{Figure}{fig:satisfaction}

The second column of Table~\ref{tab:satisfaction} further confirmed that satisfaction was an interracial sentiment among residents of multiracial neighborhoods. The estimates were again small and could not be distinguished from a null effect. The right panel of Figure~\ref{fig:satisfaction} shows the predicted probabilities of satisfaction from the model with controls. I estimated the probabilities and standard errors at the reference level for all other variables in the model, meaning the figure represents the satisfaction of a 50-year-old, native-born, unmarried woman with a high school degree and no children who has lived in her neighborhood for 10 years. \textbf{{[}Look into replacing this estimate with the AME{]}} In the complete model, only \maxdiffthree separated the least satisfied group (now whites) from the most satisfied (still Latinxs).

Not only are individual coefficients not large or statistically significant, but including race does not improve the explanatory power of the model. The final column of Table \ref{tab:satisfaction} reports the results of model estimating satisfaction in which I did not include race. The bottom row of the table reports the Akaike information criterion (AIC), a measure that balances a model's parsimony with its goodness of fit to the data \citep{akaike_new_1974}. A lower AIC represents a better fit to the data, and we can observe from Table~\ref{tab:satisfaction} that the model without race fits the data better than the model that includes race. I further affirmed the lack of explanatory power attributable to race by conducting Wald's F-test between the models with and without race. The mean value of the test across the imputed data sets was \satWaldF, and the p-values ranged from \satWaldpMin to \satWaldpMax.

These results establish that residents are satisfied living in multiracial neighborhoods regardless of their own race. Two other factors, education and the perceived size of the neighborhood, are associated with neighborhood satisfaction rather than race. Education had a curvilinear relationship with satisfaction among multiracial neighborhood residents. Residents of multiracial neighborhoods with a high school degree were the most satisfied, while residents with less than a high school education and those with a postgraduate degree felt satisfaction the least often. The perceived size of the neighborhood also mattered, as those residents who thought their neighborhood comprised 10-50 blocks were \satnhdsize\ times more likely to be satisfied than those who perceived their neighborhood to be only one to nine blocks. 

While seven in ten residents of multiracial neighborhoods were satisfied, half that number (\meanimproved \%) thought that their neighborhood had become a ``much better'' or ``somewhat better'' place to live over the past five years. Also unlike satisfaction, the probability that residents perceived improvements varied by race. Thirty-eight percent of Asian and Latinx residents, and 35 percent of black residents, felt that their neighborhoods had improved, while only 27 percent of whites felt that way.

The first column of Table~\ref{tab:improvement} shows that the racial differences were unlikely due to the randomness of the sample. The results show a strong association between perceived improvement and race: whites were much less likely than Asian, black, or Latinx residents to believe that their neighborhoods had improved. The left panel of Figure~\ref{fig:improvement} shows the lower probability at which whites were predicted to report improvement in multiracial neighborhoods than the probabilities at which people of color were predicted to report improvement. \textbf{{[}Look into replacing this estimate with the AME{]}}

\abouthere{Table}{tab:improvement}

The lower propensity of whites to report improvement in multiracial neighborhoods persisted after including controls. The second column of Table~\ref{tab:improvement} shows positive associations for all three non-white racial groups. Asians were \apibetter~times more likely to report improvement in their neighborhood, blacks were \nhbbetter~times more likely, and Latinos were \hspbetter~times more likely. The latter two differences were large enough to be confident that the difference was not due to sampling variation at the p=0.05 level. The right panel of Figure~\ref{fig:improvement} shows the predicted probabilities of reporting neighborhood improvement across racial groups. \textbf{{[}Look into replacing this estimate with the AME{]}}

\abouthere{Figure}{fig:improvement}

In addition to the differences between whites and people of color in the outcome, the model that included race fit the data better than the model that did not. The third column of Table~\ref{tab:improvement} reports coefficient estimates and standard errors of the model without race and shows that the AIC value increases (fits less well) between the full model and the model without race. Wald F-tests of including race versus not provided evidence in the same direction: the mean value of the tests across imputed data sets was \betWaldF~and the mean p-value was \betWaldp.

In summary, a lower proportion of whites perceived improvement in multiracial neighborhoods than their neighbors of color. Whites, therefore, might be satisfied living in their multiracial neighborhood, but not inclined to see their neighborhood being a better place to live in the future. Neighborhood racial diversity would have been growing in the time period whites perceived less improvement than their neighbors. The present study provides insufficient evidence to conclude that increasing diversity \emph{caused} whites to perceive less improvement than their neighbors, but it does show a temporal correlation between growing diversity and a lower white endorsement of neighborhood change.

Projecting forward, too, the less positive trajectory whites perceive could portend a future in which whites might be more inclined, on the margin, to move out of the neighborhood than their non-white neighbors. Such marginal differences combined with the high rates of satisfaction among whites would be unlikely to instigate white flight, but the higher probability of losing white households could increase the chances of long-term racial turnover. The chances of long-term turnover would increase further if multiracial neighborhoods attracted people of color at higher rates than whites.

\section{Familiarity, Preferences, and the Future of Multiracial Neighborhoods}

The next section considers whether white residents of integrated neighborhoods--like whites generally--are more ignorant of and less likely to consider integrated neighborhoods in the Washington, DC area. On the one hand, living in a currently integrated neighborhood demonstrate ``revealed preferences'' for integration. These whites might be those most likely to know about and consider \emph{other} multiracial neighborhoods. On the other hand, whites living in multiracial neighborhoods might be those who desire moving to whiter neighborhoods but lack the resources necessary to move. They might be satisfied for now, but given the option would not live in another multiracial neighborhood.

\subsection{Unfamiliarity with Multiracial Communities}\label{unfamiliarity-with-multiracial-communities}

A substantial portion of residents living in multiracial neighborhoods were unfamiliar with (i.e., did not know anything about) other multiracial communities in the DC area. Almost half, \dkherndon~percent, were unfamiliar with Herndon while around a third were didn't know Germantown (\dkgermantown) and Wheaton (\dkwheaton). This was despite many people reporting friends or family living in the communities. As many as \ffgermantown~percent of respondents reported having friends or family in Germantown and approximately one in four reporting friends or family in Herndon and Wheaton.

Despite the relatively high levels of unfamiliarity, few differences existed between respondents identifying with different racial groups. Table \ref{tab:knowledge} reports estimated coefficients and standard errors of models predicting whether residents were unfamiliar with the three neighborhoods. The coefficients for all racial groups were generally small and none of the coefficients could be distinguished from there being no difference compared to whites. Directionally, whites were more likely to be unfamiliar with Wheaton than both blacks and Latinxs. Including race improved the model's fit to the data measuring unfamiliarity with Wheaton (results available in Supplement). Whites were also more likely to be unfamiliar with Germantown than Latinxs. Whites were somewhat less likely to be unfamiliar with Herndon than the other three groups. Including race did not improve the fit for either Germantown or Herndon. 

\abouthere{Table}{tab:knowledge}

\subsection{Considering Multiracial Communities in the DC Area}
Few respondents reported being willing to seriously consider the three neighborhoods. One in ten respondents reported seriously considering Herndon with few differences between races. Table \ref{tab:consider} shows small influences of race on willingness to consider Herndon, and adding race did not improve the model's fit. 

\abouthere{Table}{tab:consider}

Rates of consideration were similar in the Maryland suburbs: ten percent would consider Germantown and six percent would consider Wheaton.\footnote{These values fit into the middle of the distribution of values reported in Krysan and Bader (2007) and Bader and Krysan (2009).} Unlike in Herndon, however, race was associated with who considered the two Maryland communities. Table \ref{tab:consider} shows that Latinxs were the most likely to consider both communities; they were approximately fifteen times more likely than whites to consider Germantown and and nine times more likely than whites to consider Wheaton. Black respondents were 6.5 times more likely than whites to consider Wheaton. Adding controls strengthened the associations between consideration and race among blacks and Latinos. Controls also revealed more of a willingness among Asians to consider Germantown compared to whites, but the coefficient was not precise enough to be confident in a true underlying difference. 

The controls reveal some patterns that explain which current residents of integrated neighborhoods would consider moving to another integrated neighborhood. Compared to native-born residents, immigrants were about a third as likely to consider Germantown and a quarter as likely to consider Wheaton; though both estimates were sufficiently imprecise to know whether a true difference exists. Households with children were less likely to consider living in Germantown and Wheaton than those without children. The estimates of both coefficients were large, but sufficiently imprecise to rule out a difference in the population at large. 

Education did not have a straightforward relationship with the consideration of the communities. More education was generally associated with higher propensities to consider Herndon. Meanwhile, respondents with middling levels of education were those most likely to consider Germantown, and those with the least education were the most likely to consider Wheaton. Although some of these associations were large, especially those among the least educated, most were not significantly distinguishable from having no relationship at conventional levels of statistical significance. 

\subsection{Rejecting Multiracial Communities in the DC Area}\label{ssec:reject}

Respondents were overall more likely to reject all three multiracial communities than consider them. Among respondents, \ncherndon~percent rejected Herndon, \ncgermantown~percent rejected Germantown, and \ncwheaton~percent rejected Wheaton. While these rejection rates were two to four times higher than the rates at which respondents considered neighborhoods, the rates are on the low end of the distribution of rejection rates from previous research. 

Table~\ref{tab:notconsider} reports the results of models estimating the influence of individual characteristics on the propensity to reject the three communities. Whites were more likely than people identifying with the other racial groups to reject all three communities. The difference between whites and Asians was statistically distinguishable from zero. Adding controls, however, moderated the influence of race in all cases, though the difference between whites and Asians was still significant in Wheaton after adding controls. Models with nonracial controls fit the models better, according to AIC statistics, than those with race in Herndon and Germantown. The reduced influence of race after adding controls suggests that the demographic dissimilarity of whites from people of color generally explains whites' higher rejection rates of multiracial communities. 

\abouthere{Table}{tab:notconsider}

Among controls, immigration and education affected rejection. Immigrants were about a quarter as likely to reject Herndon as US-born residents were. The coefficients for foreign-born respondents were also negative in the Germantown and Wheaton models, but their magnitudes were small. Respondents with a high school education (the reference group) were the most likely to reject living in the three communities. They were much more likely than those without high school degrees and only modestly more likely than those with more education (though none of the coefficients for higher levels of attainment approached statistical significance). Surprisingly, having a child present in the household did not seem to affect rejecting the communities much. Although respondents with children were more likely to reject living in all three communities, the magnitudes of the coefficients were very small.  

\section{Discussion}
The preceding analyses explored the potential sustainability of multiracial integration in the Washington, DC area. The results lead to the conclusion that multiracial neighborhoods appear likely to be a persistent feature in the area. Three in four residents felt satisfied living in multiracial neighborhoods, a proportion that did not vary by race. The general satisfaction among residents precludes the likelihood of flight by any racial group, even whites who were hypothesized to be most likely to flee. With a cohort of satisfied residents, multiracial neighborhoods appear likely to be maintained for the duration of that cohort's residence in the neighborhood.

Finding that whites were less likely to perceive improvement tempers the confidence with which we should predict sustainable multiracial integration. On the one hand, the finding could be an artifact of the inability to measure baseline assessments of neighborhood quality due to the cross-sectional data used for this study. On the other, the finding could indicate that whites would sour on living in multiracial neighborhoods faster than their neighbors should changes, race-related or not, transpire. Lower perceptions of improvement over time might lead to future dissatisfaction among whites, an eventuality that could destabilize a neighborhood's integration.

Future work could examine the aspects of multiracial neighborhoods that residents like. Residents identifying with different racial groups might find different aspects of their neighborhoods appealing. Whites, for example, might draw on the diversity of multiracial neighborhoods to distinguish themselves from more exclusionary whites. People of color might derive greater satisfaction from the schools or commercial developments that come about near multiracial neighborhoods. Uncovering the different reasons multiracial neighborhoods satisfy residents could also help explain why whites report less improvement than Asians, blacks, and Latinxs. 

Even if the general satisfaction among current residents holds, however, satisfaction alone cannot guarantee the future sustainability of multiracial integration. Market perceptions introduce another source of potential instability. More residents than not were unfamiliar with other multiracial communities in the Washington, DC area. Unfamiliarity could reduce flows of migrants between multiracial communities when households find moving necessary, and it could be a factor that impedes stable integration. Race, however, was not associated with patterns of unfamiliarity. Similarly, rejection of multiracial communities appeared relatively consistent after controlling for group demographic differences. 

Racial groups did differ in their propensity to seriously consider metropolitan multiracial communities. Whites were less likely to consider moving two integrated communities compared to Latinxs and one integrated community compared to blacks. The different preferences portend larger flows of Latinx and black households than white (and, in some cases, Asian) households between multiracial communities, differences that could potentially change the racial balance of current multiracial neighborhoods. 

\subsection{Implications}
The general satisfaction with multiracial neighborhoods reveals a need to reconsider how we conceive of residential preferences. The overwhelming and interracial satisfaction living in multiracial neighborhoods renews hope for reducing residential segregation in American metropolitan areas. White apprehension has been the stumbling block to integration. While Asians, blacks, and Latinxs have all stated and revealed desires to live in integrated communities, whites have not \citep{charles_neighborhood_2000}. Therefore, the deviation of white perceptions from those of previous studies merit special attention and imply directions for future research and policy. 

The deviation from previous results offer the possibility that multiracial neighborhood integration occasions a larger focus on categorical variations in the perceptions and processes of neighborhood change. As I mentioned in the introduction, \emph{multiracial} integration has rarely been investigated as a separate category of neighborhood. Multiracial neighborhood integration gets represented by its component racial compositions that do not capture the unique combination that could affect individual perceptions. The results could instead reflect the fact that white residents might be categorically different from the other white residents living in the same metropolitan area. Or it might be possible both are true: multiracially integrated neighborhoods could elicit categorically different responses among a categorically distinct group of whites. 

Using models based on categorical distinctions of neighborhood composition would change our conceptions of race in American metropolitan areas. Existing models, even those that test the nonparametric dynamics of tipping points \citep[e.g.,][]{schelling_dynamic_1971, bruch_neighborhood_2006, xie_modeling_2012}, measure the outcomes we should expect from a percentage increase of a racial group or percentage change in preference for a group. A categorical conception would think more about \emph{combinations} of racial groups. Models that employ distinctive racial combinations reflect the underlying residential mobility dynamics better, at least in the case of multiracial neighborhoods, than four separate measures of racial composition. 

The results also lead to the question of what causes the perceptions and preferences of whites living in multiracial neighborhoods different from whites generally. The exposure of white residents to multiracial neighbors might lead them to perceive neighborhoods differently than other whites in the metropolitan area. In contrast, whites living in multiracial neighborhoods might be a subset of whites who seek to live among racially diverse neighbors. The former presages an expanding population of white residents whose moves would support racial integration; racial integration would, in other words, beget further racial integration. The latter, however, might suggest instead a group of finite size able to sustain integration. 

The lack of racial differences among \emph{current} residents of multiracial neighborhoods offers clear hope for perpetuating integration. The results imply that marketing multiracial communities to current residents of multiracial neighborhoods could sustain a flow of migrants into multiracial communities. Such affirmative marketing efforts \citep{haberle_accessing_2012}, especially to white residents, might lead residents to rank multiracial neighborhoods higher on their lists of preferred neighborhoods. This could lead to more searches in and moves into multiracial communities \citep{krysan_cycle_2017}. Affirmative marketing provides promise because efforts would not need to overcome racial differences in rejection rates, at least after controlling for demographic variation between racial groups. Local jurisdictions, in coordination with regional fair housing centers, could appropriate funds to support affirmative marketing campaigns. 

\subsection{Limitations}
The deviation of patterns among whites in this study could, however, be the result of two limitations in the study. First,the sample of respondents came from a different metropolitan area than the samples of previous studies. As a result, the differences could have arisen from the residential selection factors or interracial relationships particular to the Washington, DC area. Second, mailed surveys might be prone to selective response in which satisfied residents would be more likely to respond than unsatisfied residents. 

Future work representing residents of multiracial neighborhoods could overcome these limitations. Collecting data from residents of multiracial neighborhoods in other metropolitan areas could be used to replicate the analyses. Face-to-face surveys would provide the best data, but at substantial cost. Cluster-based designs that sample residents within multiracial neighborhoods would benefit future research, whether using mail-based or face-to-face interviews. Clustered designs would reduce standard errors due to fixed effects in addition to offering a wider array of analyses to examine within- and between-neighborhood variation.

While only replication can fully address these limitations, the findings jibe with existing research. The findings are consistent with recent research showing increases in the number of multiracial neighborhoods \citep{friedman_declines_2008, logan_global_2010, bader_fragmented_2016}. Even where turnover has occurred, the process has tended to occur slowly \citep{ellen_sharing_2000}. The slow pace of change fits the pattern of general satisfaction found in these analyses. Finally, the appearance of multiracial neighborhoods in middle-ring suburbs in the DC area conform to the geographic patterns seen in other major metropolitan areas \citep[e.g., ][]{bader_fragmented_2016}, lending credence to the similarity of multiracial neighborhoods in the DC area to other metropolitan areas. 

\section{Conclusion}
The interracial dynamics in multiracial neighborhoods deserve greater attention. They represent a new pattern of stable, long-term racial integration that has served as a model of progress for decades in this country. I have shown that residents of all races find satisfaction living in multiracial neighborhoods, and that many would consider moving to other multiracial communities. Studying which aspects of multiracial neighborhoods satisfy residents, and incorporating their existence into theories explaining metropolitan racial change represent an important frontier of research.  


\clearpage
\singlespace
\singlespace
 	\bibliographystyle{/Users/bader/work/Bibs/asr}
	\bibliography{bib/multiethnic-nhoods.bib}


\clearpage 

\section{Tables}

% latex table generated in R 3.5.0 by xtable 1.8-2 package
% 
\begin{table}[ht]
\centering
\caption{Means and standard deviations of tract-level variables in multiethnic and quadrivial neighborhoods in the DC Area} 
\label{tab:nhd_descriptives}
\begin{tabular}{p{2in}R{4em}R{4em}p{1em}R{4em}R{4em}}
  \toprule
&\multicolumn{2}{p{8em}}{\centering Multiethnic neighborhoods}& &\multicolumn{2}{p{8em}}{\centering All neighborhoods}\\ 
Variable & Mean & S.D. &  & Mean & S.D. \\ 
  \midrule
\emph{Racial composition}&&&\\Percent Asian & 17.6 & 6.1 &   & 10.0 & 9.2 \\ 
  Percent Hispanic & 24.3 & 9.2 &   & 14.4 & 13.5 \\ 
  Percent non-Hispanic black & 21.7 & 9.1 &   & 31.2 & 31.6 \\ 
  Percent non-Hispanic white & 32.6 & 9.6 &   & 41.2 & 27.4 \\ 
  \emph{Educational attainment}&&&\\Percent less than high school & 13.0 & 6.6 &   & 10.0 & 9.3 \\ 
  Percent high school & 18.2 & 5.2 &   & 17.6 & 11.2 \\ 
  Percent some college & 22.7 & 5.5 &   & 20.9 & 8.9 \\ 
  Percent bachelor's degree & 25.6 & 5.7 &   & 25.4 & 9.9 \\ 
  Percent professional degree & 20.5 & 6.0 &   & 26.1 & 15.6 \\ 
  \emph{Other demographic characteristics}&&&\\Percent foreign-born & 39.2 & 8.4 &   & 23.8 & 14.6 \\ 
  Percent of households with children present & 37.1 & 9.0 &   & 32.6 & 12.4 \\ 
  Percent married (not separated) & 48.9 & 7.8 &   & 44.7 & 16.0 \\ 
   \bottomrule
\end{tabular}
\end{table}


% latex table generated in R 3.5.0 by xtable 1.8-2 package
% Sun Nov 18 21:43:01 2018
\begin{table}[ht]
\centering
\caption{Un-weighted means and standard deviations of independent and control variables} 
\label{tab:descriptives}
\begin{tabular}{lp{.5in}p{.5in}}
  \toprule
Variable & Mean & S.D. \\ 
  \midrule
\emph{Race}&&\\Asian or Pacific Islander & 0.24 &  \\ 
  Black & 0.22 &  \\ 
  Latinx & 0.19 &  \\ 
  White & 0.36 &  \\ 
  \emph{Demographics}&&\\Age & 54.59 & 16.28 \\ 
  Foreign Born & 0.45 &  \\ 
  Man & 0.47 &  \\ 
  Children present & 0.31 &  \\ 
  Married & 0.58 &  \\ 
  \emph{Educational Attainment}&&\\<H.S. & 0.07 &  \\ 
  H.S. & 0.11 &  \\ 
  Some college, no B.A. & 0.23 &  \\ 
  B.A. & 0.31 &  \\ 
  M.A.+ & 0.29 &  \\ 
  \emph{Income}&&\\$<$\$40,000 & 0.25 &  \\ 
  \$40,000 to $<$\$75,000 & 0.24 &  \\ 
  \$75,000 to $<$\$150,000 & 0.36 &  \\ 
  \$150,000+ & 0.16 &  \\ 
  \emph{Neighborhood Experience}\\Years in Neighborhood & 14.74 & 12.82 \\ 
  1-9 blocks & 0.6 &  \\ 
  10-50 blocks & 0.33 &  \\ 
  >50 blocks & 0.05 &  \\ 
  Quadrivial Neighborhood & 0.55 &  \\ 
   \bottomrule
\end{tabular}
\end{table}


\begin{table}[h]
\centering\captionsetup{justification=centering,singlelinecheck=off}
\caption{Estimated coefficients predicting  neighborhood satisfaction}
\label{tab:satisfaction}

    \providecommand{\huxb}[2][0,0,0]{\arrayrulecolor[RGB]{#1}\global\arrayrulewidth=#2pt}
    \providecommand{\huxvb}[2][0,0,0]{\color[RGB]{#1}\vrule width #2pt}
    \providecommand{\huxtpad}[1]{\rule{0pt}{\baselineskip+#1}}
    \providecommand{\huxbpad}[1]{\rule[-#1]{0pt}{#1}}
  \begin{tabularx}{0.5\textwidth}{p{0.125\textwidth} p{0.125\textwidth} p{0.125\textwidth} p{0.125\textwidth}}


\hhline{>{\huxb{0.8}}->{\huxb{0.8}}->{\huxb{0.8}}->{\huxb{0.8}}-}
\arrayrulecolor{black}

\multicolumn{1}{!{\huxvb{0}}c!{\huxvb{0}}}{\huxtpad{4pt}\centering {\fontsize{9.5pt}{11.4pt}\selectfont }\huxbpad{4pt}} &
\multicolumn{1}{c!{\huxvb{0}}}{\huxtpad{4pt}\centering {\fontsize{9.5pt}{11.4pt}\selectfont (1)}\huxbpad{4pt}} &
\multicolumn{1}{c!{\huxvb{0}}}{\huxtpad{4pt}\centering {\fontsize{9.5pt}{11.4pt}\selectfont (2)}\huxbpad{4pt}} &
\multicolumn{1}{c!{\huxvb{0}}}{\huxtpad{4pt}\centering {\fontsize{9.5pt}{11.4pt}\selectfont (3)}\huxbpad{4pt}} \tabularnewline[-0.5pt]


\hhline{>{\huxb{1}}->{\huxb{1}}->{\huxb{1}}->{\huxb{1}}-}
\arrayrulecolor{black}

\multicolumn{1}{!{\huxvb{0}}l!{\huxvb{0}}}{\huxtpad{4pt}\raggedright {\fontsize{9.5pt}{11.4pt}\selectfont (Intercept)}\huxbpad{4pt}} &
\multicolumn{1}{r!{\huxvb{0}}}{\huxtpad{4pt}\raggedleft {\fontsize{9.5pt}{11.4pt}\selectfont 0.795 ***}\huxbpad{4pt}} &
\multicolumn{1}{r!{\huxvb{0}}}{\huxtpad{4pt}\raggedleft {\fontsize{9.5pt}{11.4pt}\selectfont 0.530~}\huxbpad{4pt}} &
\multicolumn{1}{r!{\huxvb{0}}}{\huxtpad{4pt}\raggedleft {\fontsize{9.5pt}{11.4pt}\selectfont 0.770~~}\huxbpad{4pt}} \tabularnewline[-0.5pt]


\hhline{}
\arrayrulecolor{black}

\multicolumn{1}{!{\huxvb{0}}l!{\huxvb{0}}}{\huxtpad{4pt}\raggedright {\fontsize{9.5pt}{11.4pt}\selectfont }\huxbpad{4pt}} &
\multicolumn{1}{r!{\huxvb{0}}}{\huxtpad{4pt}\raggedleft {\fontsize{9.5pt}{11.4pt}\selectfont (0.185)~~~}\huxbpad{4pt}} &
\multicolumn{1}{r!{\huxvb{0}}}{\huxtpad{4pt}\raggedleft {\fontsize{9.5pt}{11.4pt}\selectfont (0.511)}\huxbpad{4pt}} &
\multicolumn{1}{r!{\huxvb{0}}}{\huxtpad{4pt}\raggedleft {\fontsize{9.5pt}{11.4pt}\selectfont (0.549)~}\huxbpad{4pt}} \tabularnewline[-0.5pt]


\hhline{}
\arrayrulecolor{black}

\multicolumn{1}{!{\huxvb{0}}l!{\huxvb{0}}}{\huxtpad{4pt}\raggedright {\fontsize{9.5pt}{11.4pt}\selectfont Race}\huxbpad{4pt}} &
\multicolumn{1}{r!{\huxvb{0}}}{\huxtpad{4pt}\raggedleft {\fontsize{9.5pt}{11.4pt}\selectfont ~~~~~~~~}\huxbpad{4pt}} &
\multicolumn{1}{r!{\huxvb{0}}}{\huxtpad{4pt}\raggedleft {\fontsize{9.5pt}{11.4pt}\selectfont ~~~~~}\huxbpad{4pt}} &
\multicolumn{1}{r!{\huxvb{0}}}{\huxtpad{4pt}\raggedleft {\fontsize{9.5pt}{11.4pt}\selectfont ~~~~~~}\huxbpad{4pt}} \tabularnewline[-0.5pt]


\hhline{}
\arrayrulecolor{black}

\multicolumn{1}{!{\huxvb{0}}l!{\huxvb{0}}}{\huxtpad{4pt}\raggedright {\fontsize{9.5pt}{11.4pt}\selectfont Asian}\huxbpad{4pt}} &
\multicolumn{1}{r!{\huxvb{0}}}{\huxtpad{4pt}\raggedleft {\fontsize{9.5pt}{11.4pt}\selectfont 0.136~~~~}\huxbpad{4pt}} &
\multicolumn{1}{r!{\huxvb{0}}}{\huxtpad{4pt}\raggedleft {\fontsize{9.5pt}{11.4pt}\selectfont 0.071~}\huxbpad{4pt}} &
\multicolumn{1}{r!{\huxvb{0}}}{\huxtpad{4pt}\raggedleft {\fontsize{9.5pt}{11.4pt}\selectfont 0.143~~}\huxbpad{4pt}} \tabularnewline[-0.5pt]


\hhline{}
\arrayrulecolor{black}

\multicolumn{1}{!{\huxvb{0}}l!{\huxvb{0}}}{\huxtpad{4pt}\raggedright {\fontsize{9.5pt}{11.4pt}\selectfont }\huxbpad{4pt}} &
\multicolumn{1}{r!{\huxvb{0}}}{\huxtpad{4pt}\raggedleft {\fontsize{9.5pt}{11.4pt}\selectfont (0.305)~~~}\huxbpad{4pt}} &
\multicolumn{1}{r!{\huxvb{0}}}{\huxtpad{4pt}\raggedleft {\fontsize{9.5pt}{11.4pt}\selectfont (0.386)}\huxbpad{4pt}} &
\multicolumn{1}{r!{\huxvb{0}}}{\huxtpad{4pt}\raggedleft {\fontsize{9.5pt}{11.4pt}\selectfont (0.392)~}\huxbpad{4pt}} \tabularnewline[-0.5pt]


\hhline{}
\arrayrulecolor{black}

\multicolumn{1}{!{\huxvb{0}}l!{\huxvb{0}}}{\huxtpad{4pt}\raggedright {\fontsize{9.5pt}{11.4pt}\selectfont Black}\huxbpad{4pt}} &
\multicolumn{1}{r!{\huxvb{0}}}{\huxtpad{4pt}\raggedleft {\fontsize{9.5pt}{11.4pt}\selectfont -0.095~~~~}\huxbpad{4pt}} &
\multicolumn{1}{r!{\huxvb{0}}}{\huxtpad{4pt}\raggedleft {\fontsize{9.5pt}{11.4pt}\selectfont -0.107~}\huxbpad{4pt}} &
\multicolumn{1}{r!{\huxvb{0}}}{\huxtpad{4pt}\raggedleft {\fontsize{9.5pt}{11.4pt}\selectfont -0.168~~}\huxbpad{4pt}} \tabularnewline[-0.5pt]


\hhline{}
\arrayrulecolor{black}

\multicolumn{1}{!{\huxvb{0}}l!{\huxvb{0}}}{\huxtpad{4pt}\raggedright {\fontsize{9.5pt}{11.4pt}\selectfont }\huxbpad{4pt}} &
\multicolumn{1}{r!{\huxvb{0}}}{\huxtpad{4pt}\raggedleft {\fontsize{9.5pt}{11.4pt}\selectfont (0.326)~~~}\huxbpad{4pt}} &
\multicolumn{1}{r!{\huxvb{0}}}{\huxtpad{4pt}\raggedleft {\fontsize{9.5pt}{11.4pt}\selectfont (0.334)}\huxbpad{4pt}} &
\multicolumn{1}{r!{\huxvb{0}}}{\huxtpad{4pt}\raggedleft {\fontsize{9.5pt}{11.4pt}\selectfont (0.347)~}\huxbpad{4pt}} \tabularnewline[-0.5pt]


\hhline{}
\arrayrulecolor{black}

\multicolumn{1}{!{\huxvb{0}}l!{\huxvb{0}}}{\huxtpad{4pt}\raggedright {\fontsize{9.5pt}{11.4pt}\selectfont Latinx}\huxbpad{4pt}} &
\multicolumn{1}{r!{\huxvb{0}}}{\huxtpad{4pt}\raggedleft {\fontsize{9.5pt}{11.4pt}\selectfont 0.344~~~~}\huxbpad{4pt}} &
\multicolumn{1}{r!{\huxvb{0}}}{\huxtpad{4pt}\raggedleft {\fontsize{9.5pt}{11.4pt}\selectfont 0.396~}\huxbpad{4pt}} &
\multicolumn{1}{r!{\huxvb{0}}}{\huxtpad{4pt}\raggedleft {\fontsize{9.5pt}{11.4pt}\selectfont 0.361~~}\huxbpad{4pt}} \tabularnewline[-0.5pt]


\hhline{}
\arrayrulecolor{black}

\multicolumn{1}{!{\huxvb{0}}l!{\huxvb{0}}}{\huxtpad{4pt}\raggedright {\fontsize{9.5pt}{11.4pt}\selectfont }\huxbpad{4pt}} &
\multicolumn{1}{r!{\huxvb{0}}}{\huxtpad{4pt}\raggedleft {\fontsize{9.5pt}{11.4pt}\selectfont (0.356)~~~}\huxbpad{4pt}} &
\multicolumn{1}{r!{\huxvb{0}}}{\huxtpad{4pt}\raggedleft {\fontsize{9.5pt}{11.4pt}\selectfont (0.383)}\huxbpad{4pt}} &
\multicolumn{1}{r!{\huxvb{0}}}{\huxtpad{4pt}\raggedleft {\fontsize{9.5pt}{11.4pt}\selectfont (0.406)~}\huxbpad{4pt}} \tabularnewline[-0.5pt]


\hhline{}
\arrayrulecolor{black}

\multicolumn{1}{!{\huxvb{0}}l!{\huxvb{0}}}{\huxtpad{4pt}\raggedright {\fontsize{9.5pt}{11.4pt}\selectfont Demographics}\huxbpad{4pt}} &
\multicolumn{1}{r!{\huxvb{0}}}{\huxtpad{4pt}\raggedleft {\fontsize{9.5pt}{11.4pt}\selectfont ~~~~~~~~}\huxbpad{4pt}} &
\multicolumn{1}{r!{\huxvb{0}}}{\huxtpad{4pt}\raggedleft {\fontsize{9.5pt}{11.4pt}\selectfont ~~~~~}\huxbpad{4pt}} &
\multicolumn{1}{r!{\huxvb{0}}}{\huxtpad{4pt}\raggedleft {\fontsize{9.5pt}{11.4pt}\selectfont ~~~~~~}\huxbpad{4pt}} \tabularnewline[-0.5pt]


\hhline{}
\arrayrulecolor{black}

\multicolumn{1}{!{\huxvb{0}}l!{\huxvb{0}}}{\huxtpad{4pt}\raggedright {\fontsize{9.5pt}{11.4pt}\selectfont Age}\huxbpad{4pt}} &
\multicolumn{1}{r!{\huxvb{0}}}{\huxtpad{4pt}\raggedleft {\fontsize{9.5pt}{11.4pt}\selectfont ~~~~~~~~}\huxbpad{4pt}} &
\multicolumn{1}{r!{\huxvb{0}}}{\huxtpad{4pt}\raggedleft {\fontsize{9.5pt}{11.4pt}\selectfont 0.004~}\huxbpad{4pt}} &
\multicolumn{1}{r!{\huxvb{0}}}{\huxtpad{4pt}\raggedleft {\fontsize{9.5pt}{11.4pt}\selectfont 0.005~~}\huxbpad{4pt}} \tabularnewline[-0.5pt]


\hhline{}
\arrayrulecolor{black}

\multicolumn{1}{!{\huxvb{0}}l!{\huxvb{0}}}{\huxtpad{4pt}\raggedright {\fontsize{9.5pt}{11.4pt}\selectfont }\huxbpad{4pt}} &
\multicolumn{1}{r!{\huxvb{0}}}{\huxtpad{4pt}\raggedleft {\fontsize{9.5pt}{11.4pt}\selectfont ~~~~~~~~}\huxbpad{4pt}} &
\multicolumn{1}{r!{\huxvb{0}}}{\huxtpad{4pt}\raggedleft {\fontsize{9.5pt}{11.4pt}\selectfont (0.007)}\huxbpad{4pt}} &
\multicolumn{1}{r!{\huxvb{0}}}{\huxtpad{4pt}\raggedleft {\fontsize{9.5pt}{11.4pt}\selectfont (0.009)~}\huxbpad{4pt}} \tabularnewline[-0.5pt]


\hhline{}
\arrayrulecolor{black}

\multicolumn{1}{!{\huxvb{0}}l!{\huxvb{0}}}{\huxtpad{4pt}\raggedright {\fontsize{9.5pt}{11.4pt}\selectfont Foreign Born}\huxbpad{4pt}} &
\multicolumn{1}{r!{\huxvb{0}}}{\huxtpad{4pt}\raggedleft {\fontsize{9.5pt}{11.4pt}\selectfont ~~~~~~~~}\huxbpad{4pt}} &
\multicolumn{1}{r!{\huxvb{0}}}{\huxtpad{4pt}\raggedleft {\fontsize{9.5pt}{11.4pt}\selectfont 0.085~}\huxbpad{4pt}} &
\multicolumn{1}{r!{\huxvb{0}}}{\huxtpad{4pt}\raggedleft {\fontsize{9.5pt}{11.4pt}\selectfont -0.013~~}\huxbpad{4pt}} \tabularnewline[-0.5pt]


\hhline{}
\arrayrulecolor{black}

\multicolumn{1}{!{\huxvb{0}}l!{\huxvb{0}}}{\huxtpad{4pt}\raggedright {\fontsize{9.5pt}{11.4pt}\selectfont }\huxbpad{4pt}} &
\multicolumn{1}{r!{\huxvb{0}}}{\huxtpad{4pt}\raggedleft {\fontsize{9.5pt}{11.4pt}\selectfont ~~~~~~~~}\huxbpad{4pt}} &
\multicolumn{1}{r!{\huxvb{0}}}{\huxtpad{4pt}\raggedleft {\fontsize{9.5pt}{11.4pt}\selectfont (0.319)}\huxbpad{4pt}} &
\multicolumn{1}{r!{\huxvb{0}}}{\huxtpad{4pt}\raggedleft {\fontsize{9.5pt}{11.4pt}\selectfont (0.335)~}\huxbpad{4pt}} \tabularnewline[-0.5pt]


\hhline{}
\arrayrulecolor{black}

\multicolumn{1}{!{\huxvb{0}}l!{\huxvb{0}}}{\huxtpad{4pt}\raggedright {\fontsize{9.5pt}{11.4pt}\selectfont Male}\huxbpad{4pt}} &
\multicolumn{1}{r!{\huxvb{0}}}{\huxtpad{4pt}\raggedleft {\fontsize{9.5pt}{11.4pt}\selectfont ~~~~~~~~}\huxbpad{4pt}} &
\multicolumn{1}{r!{\huxvb{0}}}{\huxtpad{4pt}\raggedleft {\fontsize{9.5pt}{11.4pt}\selectfont -0.009~}\huxbpad{4pt}} &
\multicolumn{1}{r!{\huxvb{0}}}{\huxtpad{4pt}\raggedleft {\fontsize{9.5pt}{11.4pt}\selectfont 0.069~~}\huxbpad{4pt}} \tabularnewline[-0.5pt]


\hhline{}
\arrayrulecolor{black}

\multicolumn{1}{!{\huxvb{0}}l!{\huxvb{0}}}{\huxtpad{4pt}\raggedright {\fontsize{9.5pt}{11.4pt}\selectfont }\huxbpad{4pt}} &
\multicolumn{1}{r!{\huxvb{0}}}{\huxtpad{4pt}\raggedleft {\fontsize{9.5pt}{11.4pt}\selectfont ~~~~~~~~}\huxbpad{4pt}} &
\multicolumn{1}{r!{\huxvb{0}}}{\huxtpad{4pt}\raggedleft {\fontsize{9.5pt}{11.4pt}\selectfont (0.249)}\huxbpad{4pt}} &
\multicolumn{1}{r!{\huxvb{0}}}{\huxtpad{4pt}\raggedleft {\fontsize{9.5pt}{11.4pt}\selectfont (0.252)~}\huxbpad{4pt}} \tabularnewline[-0.5pt]


\hhline{}
\arrayrulecolor{black}

\multicolumn{1}{!{\huxvb{0}}l!{\huxvb{0}}}{\huxtpad{4pt}\raggedright {\fontsize{9.5pt}{11.4pt}\selectfont Children Present}\huxbpad{4pt}} &
\multicolumn{1}{r!{\huxvb{0}}}{\huxtpad{4pt}\raggedleft {\fontsize{9.5pt}{11.4pt}\selectfont ~~~~~~~~}\huxbpad{4pt}} &
\multicolumn{1}{r!{\huxvb{0}}}{\huxtpad{4pt}\raggedleft {\fontsize{9.5pt}{11.4pt}\selectfont -0.290~}\huxbpad{4pt}} &
\multicolumn{1}{r!{\huxvb{0}}}{\huxtpad{4pt}\raggedleft {\fontsize{9.5pt}{11.4pt}\selectfont -0.313~~}\huxbpad{4pt}} \tabularnewline[-0.5pt]


\hhline{}
\arrayrulecolor{black}

\multicolumn{1}{!{\huxvb{0}}l!{\huxvb{0}}}{\huxtpad{4pt}\raggedright {\fontsize{9.5pt}{11.4pt}\selectfont }\huxbpad{4pt}} &
\multicolumn{1}{r!{\huxvb{0}}}{\huxtpad{4pt}\raggedleft {\fontsize{9.5pt}{11.4pt}\selectfont ~~~~~~~~}\huxbpad{4pt}} &
\multicolumn{1}{r!{\huxvb{0}}}{\huxtpad{4pt}\raggedleft {\fontsize{9.5pt}{11.4pt}\selectfont (0.274)}\huxbpad{4pt}} &
\multicolumn{1}{r!{\huxvb{0}}}{\huxtpad{4pt}\raggedleft {\fontsize{9.5pt}{11.4pt}\selectfont (0.278)~}\huxbpad{4pt}} \tabularnewline[-0.5pt]


\hhline{}
\arrayrulecolor{black}

\multicolumn{1}{!{\huxvb{0}}l!{\huxvb{0}}}{\huxtpad{4pt}\raggedright {\fontsize{9.5pt}{11.4pt}\selectfont Married}\huxbpad{4pt}} &
\multicolumn{1}{r!{\huxvb{0}}}{\huxtpad{4pt}\raggedleft {\fontsize{9.5pt}{11.4pt}\selectfont ~~~~~~~~}\huxbpad{4pt}} &
\multicolumn{1}{r!{\huxvb{0}}}{\huxtpad{4pt}\raggedleft {\fontsize{9.5pt}{11.4pt}\selectfont 0.245~}\huxbpad{4pt}} &
\multicolumn{1}{r!{\huxvb{0}}}{\huxtpad{4pt}\raggedleft {\fontsize{9.5pt}{11.4pt}\selectfont 0.248~~}\huxbpad{4pt}} \tabularnewline[-0.5pt]


\hhline{}
\arrayrulecolor{black}

\multicolumn{1}{!{\huxvb{0}}l!{\huxvb{0}}}{\huxtpad{4pt}\raggedright {\fontsize{9.5pt}{11.4pt}\selectfont }\huxbpad{4pt}} &
\multicolumn{1}{r!{\huxvb{0}}}{\huxtpad{4pt}\raggedleft {\fontsize{9.5pt}{11.4pt}\selectfont ~~~~~~~~}\huxbpad{4pt}} &
\multicolumn{1}{r!{\huxvb{0}}}{\huxtpad{4pt}\raggedleft {\fontsize{9.5pt}{11.4pt}\selectfont (0.270)}\huxbpad{4pt}} &
\multicolumn{1}{r!{\huxvb{0}}}{\huxtpad{4pt}\raggedleft {\fontsize{9.5pt}{11.4pt}\selectfont (0.272)~}\huxbpad{4pt}} \tabularnewline[-0.5pt]


\hhline{}
\arrayrulecolor{black}

\multicolumn{1}{!{\huxvb{0}}l!{\huxvb{0}}}{\huxtpad{4pt}\raggedright {\fontsize{9.5pt}{11.4pt}\selectfont Socioeconomic}\huxbpad{4pt}} &
\multicolumn{1}{r!{\huxvb{0}}}{\huxtpad{4pt}\raggedleft {\fontsize{9.5pt}{11.4pt}\selectfont ~~~~~~~~}\huxbpad{4pt}} &
\multicolumn{1}{r!{\huxvb{0}}}{\huxtpad{4pt}\raggedleft {\fontsize{9.5pt}{11.4pt}\selectfont ~~~~~}\huxbpad{4pt}} &
\multicolumn{1}{r!{\huxvb{0}}}{\huxtpad{4pt}\raggedleft {\fontsize{9.5pt}{11.4pt}\selectfont ~~~~~~}\huxbpad{4pt}} \tabularnewline[-0.5pt]


\hhline{}
\arrayrulecolor{black}

\multicolumn{1}{!{\huxvb{0}}l!{\huxvb{0}}}{\huxtpad{4pt}\raggedright {\fontsize{9.5pt}{11.4pt}\selectfont $<$H.S.}\huxbpad{4pt}} &
\multicolumn{1}{r!{\huxvb{0}}}{\huxtpad{4pt}\raggedleft {\fontsize{9.5pt}{11.4pt}\selectfont ~~~~~~~~}\huxbpad{4pt}} &
\multicolumn{1}{r!{\huxvb{0}}}{\huxtpad{4pt}\raggedleft {\fontsize{9.5pt}{11.4pt}\selectfont 0.101~}\huxbpad{4pt}} &
\multicolumn{1}{r!{\huxvb{0}}}{\huxtpad{4pt}\raggedleft {\fontsize{9.5pt}{11.4pt}\selectfont 0.081~~}\huxbpad{4pt}} \tabularnewline[-0.5pt]


\hhline{}
\arrayrulecolor{black}

\multicolumn{1}{!{\huxvb{0}}l!{\huxvb{0}}}{\huxtpad{4pt}\raggedright {\fontsize{9.5pt}{11.4pt}\selectfont }\huxbpad{4pt}} &
\multicolumn{1}{r!{\huxvb{0}}}{\huxtpad{4pt}\raggedleft {\fontsize{9.5pt}{11.4pt}\selectfont ~~~~~~~~}\huxbpad{4pt}} &
\multicolumn{1}{r!{\huxvb{0}}}{\huxtpad{4pt}\raggedleft {\fontsize{9.5pt}{11.4pt}\selectfont (0.441)}\huxbpad{4pt}} &
\multicolumn{1}{r!{\huxvb{0}}}{\huxtpad{4pt}\raggedleft {\fontsize{9.5pt}{11.4pt}\selectfont (0.403)~}\huxbpad{4pt}} \tabularnewline[-0.5pt]


\hhline{}
\arrayrulecolor{black}

\multicolumn{1}{!{\huxvb{0}}l!{\huxvb{0}}}{\huxtpad{4pt}\raggedright {\fontsize{9.5pt}{11.4pt}\selectfont H.S.}\huxbpad{4pt}} &
\multicolumn{1}{r!{\huxvb{0}}}{\huxtpad{4pt}\raggedleft {\fontsize{9.5pt}{11.4pt}\selectfont ~~~~~~~~}\huxbpad{4pt}} &
\multicolumn{1}{r!{\huxvb{0}}}{\huxtpad{4pt}\raggedleft {\fontsize{9.5pt}{11.4pt}\selectfont -0.657~}\huxbpad{4pt}} &
\multicolumn{1}{r!{\huxvb{0}}}{\huxtpad{4pt}\raggedleft {\fontsize{9.5pt}{11.4pt}\selectfont -0.643~~}\huxbpad{4pt}} \tabularnewline[-0.5pt]


\hhline{}
\arrayrulecolor{black}

\multicolumn{1}{!{\huxvb{0}}l!{\huxvb{0}}}{\huxtpad{4pt}\raggedright {\fontsize{9.5pt}{11.4pt}\selectfont }\huxbpad{4pt}} &
\multicolumn{1}{r!{\huxvb{0}}}{\huxtpad{4pt}\raggedleft {\fontsize{9.5pt}{11.4pt}\selectfont ~~~~~~~~}\huxbpad{4pt}} &
\multicolumn{1}{r!{\huxvb{0}}}{\huxtpad{4pt}\raggedleft {\fontsize{9.5pt}{11.4pt}\selectfont (0.396)}\huxbpad{4pt}} &
\multicolumn{1}{r!{\huxvb{0}}}{\huxtpad{4pt}\raggedleft {\fontsize{9.5pt}{11.4pt}\selectfont (0.363)~}\huxbpad{4pt}} \tabularnewline[-0.5pt]


\hhline{}
\arrayrulecolor{black}

\multicolumn{1}{!{\huxvb{0}}l!{\huxvb{0}}}{\huxtpad{4pt}\raggedright {\fontsize{9.5pt}{11.4pt}\selectfont Some college, no B.A.}\huxbpad{4pt}} &
\multicolumn{1}{r!{\huxvb{0}}}{\huxtpad{4pt}\raggedleft {\fontsize{9.5pt}{11.4pt}\selectfont ~~~~~~~~}\huxbpad{4pt}} &
\multicolumn{1}{r!{\huxvb{0}}}{\huxtpad{4pt}\raggedleft {\fontsize{9.5pt}{11.4pt}\selectfont 0.137~}\huxbpad{4pt}} &
\multicolumn{1}{r!{\huxvb{0}}}{\huxtpad{4pt}\raggedleft {\fontsize{9.5pt}{11.4pt}\selectfont 0.156~~}\huxbpad{4pt}} \tabularnewline[-0.5pt]


\hhline{}
\arrayrulecolor{black}

\multicolumn{1}{!{\huxvb{0}}l!{\huxvb{0}}}{\huxtpad{4pt}\raggedright {\fontsize{9.5pt}{11.4pt}\selectfont }\huxbpad{4pt}} &
\multicolumn{1}{r!{\huxvb{0}}}{\huxtpad{4pt}\raggedleft {\fontsize{9.5pt}{11.4pt}\selectfont ~~~~~~~~}\huxbpad{4pt}} &
\multicolumn{1}{r!{\huxvb{0}}}{\huxtpad{4pt}\raggedleft {\fontsize{9.5pt}{11.4pt}\selectfont (0.361)}\huxbpad{4pt}} &
\multicolumn{1}{r!{\huxvb{0}}}{\huxtpad{4pt}\raggedleft {\fontsize{9.5pt}{11.4pt}\selectfont (0.344)~}\huxbpad{4pt}} \tabularnewline[-0.5pt]


\hhline{}
\arrayrulecolor{black}

\multicolumn{1}{!{\huxvb{0}}l!{\huxvb{0}}}{\huxtpad{4pt}\raggedright {\fontsize{9.5pt}{11.4pt}\selectfont B.A.}\huxbpad{4pt}} &
\multicolumn{1}{r!{\huxvb{0}}}{\huxtpad{4pt}\raggedleft {\fontsize{9.5pt}{11.4pt}\selectfont ~~~~~~~~}\huxbpad{4pt}} &
\multicolumn{1}{r!{\huxvb{0}}}{\huxtpad{4pt}\raggedleft {\fontsize{9.5pt}{11.4pt}\selectfont -0.398~}\huxbpad{4pt}} &
\multicolumn{1}{r!{\huxvb{0}}}{\huxtpad{4pt}\raggedleft {\fontsize{9.5pt}{11.4pt}\selectfont -0.446~~}\huxbpad{4pt}} \tabularnewline[-0.5pt]


\hhline{}
\arrayrulecolor{black}

\multicolumn{1}{!{\huxvb{0}}l!{\huxvb{0}}}{\huxtpad{4pt}\raggedright {\fontsize{9.5pt}{11.4pt}\selectfont }\huxbpad{4pt}} &
\multicolumn{1}{r!{\huxvb{0}}}{\huxtpad{4pt}\raggedleft {\fontsize{9.5pt}{11.4pt}\selectfont ~~~~~~~~}\huxbpad{4pt}} &
\multicolumn{1}{r!{\huxvb{0}}}{\huxtpad{4pt}\raggedleft {\fontsize{9.5pt}{11.4pt}\selectfont (0.312)}\huxbpad{4pt}} &
\multicolumn{1}{r!{\huxvb{0}}}{\huxtpad{4pt}\raggedleft {\fontsize{9.5pt}{11.4pt}\selectfont (0.305)~}\huxbpad{4pt}} \tabularnewline[-0.5pt]


\hhline{}
\arrayrulecolor{black}

\multicolumn{1}{!{\huxvb{0}}l!{\huxvb{0}}}{\huxtpad{4pt}\raggedright {\fontsize{9.5pt}{11.4pt}\selectfont Neighborhood experience}\huxbpad{4pt}} &
\multicolumn{1}{r!{\huxvb{0}}}{\huxtpad{4pt}\raggedleft {\fontsize{9.5pt}{11.4pt}\selectfont ~~~~~~~~}\huxbpad{4pt}} &
\multicolumn{1}{r!{\huxvb{0}}}{\huxtpad{4pt}\raggedleft {\fontsize{9.5pt}{11.4pt}\selectfont ~~~~~}\huxbpad{4pt}} &
\multicolumn{1}{r!{\huxvb{0}}}{\huxtpad{4pt}\raggedleft {\fontsize{9.5pt}{11.4pt}\selectfont ~~~~~~}\huxbpad{4pt}} \tabularnewline[-0.5pt]


\hhline{}
\arrayrulecolor{black}

\multicolumn{1}{!{\huxvb{0}}l!{\huxvb{0}}}{\huxtpad{4pt}\raggedright {\fontsize{9.5pt}{11.4pt}\selectfont Years in neighborhood}\huxbpad{4pt}} &
\multicolumn{1}{r!{\huxvb{0}}}{\huxtpad{4pt}\raggedleft {\fontsize{9.5pt}{11.4pt}\selectfont ~~~~~~~~}\huxbpad{4pt}} &
\multicolumn{1}{r!{\huxvb{0}}}{\huxtpad{4pt}\raggedleft {\fontsize{9.5pt}{11.4pt}\selectfont ~~~~~}\huxbpad{4pt}} &
\multicolumn{1}{r!{\huxvb{0}}}{\huxtpad{4pt}\raggedleft {\fontsize{9.5pt}{11.4pt}\selectfont -0.002~~}\huxbpad{4pt}} \tabularnewline[-0.5pt]


\hhline{}
\arrayrulecolor{black}

\multicolumn{1}{!{\huxvb{0}}l!{\huxvb{0}}}{\huxtpad{4pt}\raggedright {\fontsize{9.5pt}{11.4pt}\selectfont }\huxbpad{4pt}} &
\multicolumn{1}{r!{\huxvb{0}}}{\huxtpad{4pt}\raggedleft {\fontsize{9.5pt}{11.4pt}\selectfont ~~~~~~~~}\huxbpad{4pt}} &
\multicolumn{1}{r!{\huxvb{0}}}{\huxtpad{4pt}\raggedleft {\fontsize{9.5pt}{11.4pt}\selectfont ~~~~~}\huxbpad{4pt}} &
\multicolumn{1}{r!{\huxvb{0}}}{\huxtpad{4pt}\raggedleft {\fontsize{9.5pt}{11.4pt}\selectfont (0.013)~}\huxbpad{4pt}} \tabularnewline[-0.5pt]


\hhline{}
\arrayrulecolor{black}

\multicolumn{1}{!{\huxvb{0}}l!{\huxvb{0}}}{\huxtpad{4pt}\raggedright {\fontsize{9.5pt}{11.4pt}\selectfont 10-50 blocks}\huxbpad{4pt}} &
\multicolumn{1}{r!{\huxvb{0}}}{\huxtpad{4pt}\raggedleft {\fontsize{9.5pt}{11.4pt}\selectfont ~~~~~~~~}\huxbpad{4pt}} &
\multicolumn{1}{r!{\huxvb{0}}}{\huxtpad{4pt}\raggedleft {\fontsize{9.5pt}{11.4pt}\selectfont ~~~~~}\huxbpad{4pt}} &
\multicolumn{1}{r!{\huxvb{0}}}{\huxtpad{4pt}\raggedleft {\fontsize{9.5pt}{11.4pt}\selectfont 0.447~~}\huxbpad{4pt}} \tabularnewline[-0.5pt]


\hhline{}
\arrayrulecolor{black}

\multicolumn{1}{!{\huxvb{0}}l!{\huxvb{0}}}{\huxtpad{4pt}\raggedright {\fontsize{9.5pt}{11.4pt}\selectfont }\huxbpad{4pt}} &
\multicolumn{1}{r!{\huxvb{0}}}{\huxtpad{4pt}\raggedleft {\fontsize{9.5pt}{11.4pt}\selectfont ~~~~~~~~}\huxbpad{4pt}} &
\multicolumn{1}{r!{\huxvb{0}}}{\huxtpad{4pt}\raggedleft {\fontsize{9.5pt}{11.4pt}\selectfont ~~~~~}\huxbpad{4pt}} &
\multicolumn{1}{r!{\huxvb{0}}}{\huxtpad{4pt}\raggedleft {\fontsize{9.5pt}{11.4pt}\selectfont (0.379)~}\huxbpad{4pt}} \tabularnewline[-0.5pt]


\hhline{}
\arrayrulecolor{black}

\multicolumn{1}{!{\huxvb{0}}l!{\huxvb{0}}}{\huxtpad{4pt}\raggedright {\fontsize{9.5pt}{11.4pt}\selectfont \$$>$\$50 blocks}\huxbpad{4pt}} &
\multicolumn{1}{r!{\huxvb{0}}}{\huxtpad{4pt}\raggedleft {\fontsize{9.5pt}{11.4pt}\selectfont ~~~~~~~~}\huxbpad{4pt}} &
\multicolumn{1}{r!{\huxvb{0}}}{\huxtpad{4pt}\raggedleft {\fontsize{9.5pt}{11.4pt}\selectfont ~~~~~}\huxbpad{4pt}} &
\multicolumn{1}{r!{\huxvb{0}}}{\huxtpad{4pt}\raggedleft {\fontsize{9.5pt}{11.4pt}\selectfont -0.604 *}\huxbpad{4pt}} \tabularnewline[-0.5pt]


\hhline{}
\arrayrulecolor{black}

\multicolumn{1}{!{\huxvb{0}}l!{\huxvb{0}}}{\huxtpad{4pt}\raggedright {\fontsize{9.5pt}{11.4pt}\selectfont }\huxbpad{4pt}} &
\multicolumn{1}{r!{\huxvb{0}}}{\huxtpad{4pt}\raggedleft {\fontsize{9.5pt}{11.4pt}\selectfont ~~~~~~~~}\huxbpad{4pt}} &
\multicolumn{1}{r!{\huxvb{0}}}{\huxtpad{4pt}\raggedleft {\fontsize{9.5pt}{11.4pt}\selectfont ~~~~~}\huxbpad{4pt}} &
\multicolumn{1}{r!{\huxvb{0}}}{\huxtpad{4pt}\raggedleft {\fontsize{9.5pt}{11.4pt}\selectfont (0.299)~}\huxbpad{4pt}} \tabularnewline[-0.5pt]


\hhline{>{\huxb{1}}->{\huxb{1}}->{\huxb{1}}->{\huxb{1}}-}
\arrayrulecolor{black}

\multicolumn{1}{!{\huxvb{0}}l!{\huxvb{0}}}{\huxtpad{4pt}\raggedright {\fontsize{9.5pt}{11.4pt}\selectfont nobs}\huxbpad{4pt}} &
\multicolumn{1}{r!{\huxvb{0}}}{\huxtpad{4pt}\raggedleft {\fontsize{9.5pt}{11.4pt}\selectfont ~~~~~~~~}\huxbpad{4pt}} &
\multicolumn{1}{r!{\huxvb{0}}}{\huxtpad{4pt}\raggedleft {\fontsize{9.5pt}{11.4pt}\selectfont ~~~~~}\huxbpad{4pt}} &
\multicolumn{1}{r!{\huxvb{0}}}{\huxtpad{4pt}\raggedleft {\fontsize{9.5pt}{11.4pt}\selectfont ~~~~~~}\huxbpad{4pt}} \tabularnewline[-0.5pt]


\hhline{>{\huxb{0.8}}->{\huxb{0.8}}->{\huxb{0.8}}->{\huxb{0.8}}-}
\arrayrulecolor{black}

\multicolumn{4}{!{\huxvb{0}}p{0.5\textwidth+6\tabcolsep}!{\huxvb{0}}}{\parbox[b]{0.5\textwidth+6\tabcolsep-4pt-4pt}{\huxtpad{4pt}\raggedright {\fontsize{9.5pt}{11.4pt}\selectfont  *** p $<$ 0.001;  ** p $<$ 0.01;  * p $<$ 0.05.}\huxbpad{4pt}}} \tabularnewline[-0.5pt]


\hhline{}
\arrayrulecolor{black}
\end{tabularx}
\end{table}


\begin{table}[h]
\centering\captionsetup{justification=centering,singlelinecheck=off}
\caption{Estimated coefficients predicting neighborhood improvement}
\label{tab:improvement}

    \providecommand{\huxb}[2][0,0,0]{\arrayrulecolor[RGB]{#1}\global\arrayrulewidth=#2pt}
    \providecommand{\huxvb}[2][0,0,0]{\color[RGB]{#1}\vrule width #2pt}
    \providecommand{\huxtpad}[1]{\rule{0pt}{\baselineskip+#1}}
    \providecommand{\huxbpad}[1]{\rule[-#1]{0pt}{#1}}
  \begin{tabularx}{0.5\textwidth}{p{0.125\textwidth} p{0.125\textwidth} p{0.125\textwidth} p{0.125\textwidth}}


\hhline{>{\huxb{0.8}}->{\huxb{0.8}}->{\huxb{0.8}}->{\huxb{0.8}}-}
\arrayrulecolor{black}

\multicolumn{1}{!{\huxvb{0}}c!{\huxvb{0}}}{\huxtpad{4pt}\centering {\fontsize{9.5pt}{11.4pt}\selectfont }\huxbpad{4pt}} &
\multicolumn{1}{c!{\huxvb{0}}}{\huxtpad{4pt}\centering {\fontsize{9.5pt}{11.4pt}\selectfont (1)}\huxbpad{4pt}} &
\multicolumn{1}{c!{\huxvb{0}}}{\huxtpad{4pt}\centering {\fontsize{9.5pt}{11.4pt}\selectfont (2)}\huxbpad{4pt}} &
\multicolumn{1}{c!{\huxvb{0}}}{\huxtpad{4pt}\centering {\fontsize{9.5pt}{11.4pt}\selectfont (3)}\huxbpad{4pt}} \tabularnewline[-0.5pt]


\hhline{>{\huxb{1}}->{\huxb{1}}->{\huxb{1}}->{\huxb{1}}-}
\arrayrulecolor{black}

\multicolumn{1}{!{\huxvb{0}}l!{\huxvb{0}}}{\huxtpad{4pt}\raggedright {\fontsize{9.5pt}{11.4pt}\selectfont (Intercept)}\huxbpad{4pt}} &
\multicolumn{1}{r!{\huxvb{0}}}{\huxtpad{4pt}\raggedleft {\fontsize{9.5pt}{11.4pt}\selectfont -0.573 ***}\huxbpad{4pt}} &
\multicolumn{1}{r!{\huxvb{0}}}{\huxtpad{4pt}\raggedleft {\fontsize{9.5pt}{11.4pt}\selectfont -0.563 ***}\huxbpad{4pt}} &
\multicolumn{1}{r!{\huxvb{0}}}{\huxtpad{4pt}\raggedleft {\fontsize{9.5pt}{11.4pt}\selectfont -0.867 ***}\huxbpad{4pt}} \tabularnewline[-0.5pt]


\hhline{}
\arrayrulecolor{black}

\multicolumn{1}{!{\huxvb{0}}l!{\huxvb{0}}}{\huxtpad{4pt}\raggedright {\fontsize{9.5pt}{11.4pt}\selectfont }\huxbpad{4pt}} &
\multicolumn{1}{r!{\huxvb{0}}}{\huxtpad{4pt}\raggedleft {\fontsize{9.5pt}{11.4pt}\selectfont (0.157)~~~}\huxbpad{4pt}} &
\multicolumn{1}{r!{\huxvb{0}}}{\huxtpad{4pt}\raggedleft {\fontsize{9.5pt}{11.4pt}\selectfont (0.169)~~~}\huxbpad{4pt}} &
\multicolumn{1}{r!{\huxvb{0}}}{\huxtpad{4pt}\raggedleft {\fontsize{9.5pt}{11.4pt}\selectfont (0.184)~~~}\huxbpad{4pt}} \tabularnewline[-0.5pt]


\hhline{}
\arrayrulecolor{black}

\multicolumn{1}{!{\huxvb{0}}l!{\huxvb{0}}}{\huxtpad{4pt}\raggedright {\fontsize{9.5pt}{11.4pt}\selectfont Race}\huxbpad{4pt}} &
\multicolumn{1}{r!{\huxvb{0}}}{\huxtpad{4pt}\raggedleft {\fontsize{9.5pt}{11.4pt}\selectfont ~~~~~~~~}\huxbpad{4pt}} &
\multicolumn{1}{r!{\huxvb{0}}}{\huxtpad{4pt}\raggedleft {\fontsize{9.5pt}{11.4pt}\selectfont ~~~~~~~~}\huxbpad{4pt}} &
\multicolumn{1}{r!{\huxvb{0}}}{\huxtpad{4pt}\raggedleft {\fontsize{9.5pt}{11.4pt}\selectfont ~~~~~~~~}\huxbpad{4pt}} \tabularnewline[-0.5pt]


\hhline{}
\arrayrulecolor{black}

\multicolumn{1}{!{\huxvb{0}}l!{\huxvb{0}}}{\huxtpad{4pt}\raggedright {\fontsize{9.5pt}{11.4pt}\selectfont Asian}\huxbpad{4pt}} &
\multicolumn{1}{r!{\huxvb{0}}}{\huxtpad{4pt}\raggedleft {\fontsize{9.5pt}{11.4pt}\selectfont 0.331~~~~}\huxbpad{4pt}} &
\multicolumn{1}{r!{\huxvb{0}}}{\huxtpad{4pt}\raggedleft {\fontsize{9.5pt}{11.4pt}\selectfont 0.029~~~~}\huxbpad{4pt}} &
\multicolumn{1}{r!{\huxvb{0}}}{\huxtpad{4pt}\raggedleft {\fontsize{9.5pt}{11.4pt}\selectfont 0.309~~~~}\huxbpad{4pt}} \tabularnewline[-0.5pt]


\hhline{}
\arrayrulecolor{black}

\multicolumn{1}{!{\huxvb{0}}l!{\huxvb{0}}}{\huxtpad{4pt}\raggedright {\fontsize{9.5pt}{11.4pt}\selectfont }\huxbpad{4pt}} &
\multicolumn{1}{r!{\huxvb{0}}}{\huxtpad{4pt}\raggedleft {\fontsize{9.5pt}{11.4pt}\selectfont (0.263)~~~}\huxbpad{4pt}} &
\multicolumn{1}{r!{\huxvb{0}}}{\huxtpad{4pt}\raggedleft {\fontsize{9.5pt}{11.4pt}\selectfont (0.304)~~~}\huxbpad{4pt}} &
\multicolumn{1}{r!{\huxvb{0}}}{\huxtpad{4pt}\raggedleft {\fontsize{9.5pt}{11.4pt}\selectfont (0.318)~~~}\huxbpad{4pt}} \tabularnewline[-0.5pt]


\hhline{}
\arrayrulecolor{black}

\multicolumn{1}{!{\huxvb{0}}l!{\huxvb{0}}}{\huxtpad{4pt}\raggedright {\fontsize{9.5pt}{11.4pt}\selectfont Black}\huxbpad{4pt}} &
\multicolumn{1}{r!{\huxvb{0}}}{\huxtpad{4pt}\raggedleft {\fontsize{9.5pt}{11.4pt}\selectfont 0.078~~~~}\huxbpad{4pt}} &
\multicolumn{1}{r!{\huxvb{0}}}{\huxtpad{4pt}\raggedleft {\fontsize{9.5pt}{11.4pt}\selectfont 0.079~~~~}\huxbpad{4pt}} &
\multicolumn{1}{r!{\huxvb{0}}}{\huxtpad{4pt}\raggedleft {\fontsize{9.5pt}{11.4pt}\selectfont 0.066~~~~}\huxbpad{4pt}} \tabularnewline[-0.5pt]


\hhline{}
\arrayrulecolor{black}

\multicolumn{1}{!{\huxvb{0}}l!{\huxvb{0}}}{\huxtpad{4pt}\raggedright {\fontsize{9.5pt}{11.4pt}\selectfont }\huxbpad{4pt}} &
\multicolumn{1}{r!{\huxvb{0}}}{\huxtpad{4pt}\raggedleft {\fontsize{9.5pt}{11.4pt}\selectfont (0.234)~~~}\huxbpad{4pt}} &
\multicolumn{1}{r!{\huxvb{0}}}{\huxtpad{4pt}\raggedleft {\fontsize{9.5pt}{11.4pt}\selectfont (0.248)~~~}\huxbpad{4pt}} &
\multicolumn{1}{r!{\huxvb{0}}}{\huxtpad{4pt}\raggedleft {\fontsize{9.5pt}{11.4pt}\selectfont (0.265)~~~}\huxbpad{4pt}} \tabularnewline[-0.5pt]


\hhline{}
\arrayrulecolor{black}

\multicolumn{1}{!{\huxvb{0}}l!{\huxvb{0}}}{\huxtpad{4pt}\raggedright {\fontsize{9.5pt}{11.4pt}\selectfont Latinx}\huxbpad{4pt}} &
\multicolumn{1}{r!{\huxvb{0}}}{\huxtpad{4pt}\raggedleft {\fontsize{9.5pt}{11.4pt}\selectfont 0.331~~~~}\huxbpad{4pt}} &
\multicolumn{1}{r!{\huxvb{0}}}{\huxtpad{4pt}\raggedleft {\fontsize{9.5pt}{11.4pt}\selectfont 0.081~~~~}\huxbpad{4pt}} &
\multicolumn{1}{r!{\huxvb{0}}}{\huxtpad{4pt}\raggedleft {\fontsize{9.5pt}{11.4pt}\selectfont 0.125~~~~}\huxbpad{4pt}} \tabularnewline[-0.5pt]


\hhline{}
\arrayrulecolor{black}

\multicolumn{1}{!{\huxvb{0}}l!{\huxvb{0}}}{\huxtpad{4pt}\raggedright {\fontsize{9.5pt}{11.4pt}\selectfont }\huxbpad{4pt}} &
\multicolumn{1}{r!{\huxvb{0}}}{\huxtpad{4pt}\raggedleft {\fontsize{9.5pt}{11.4pt}\selectfont (0.228)~~~}\huxbpad{4pt}} &
\multicolumn{1}{r!{\huxvb{0}}}{\huxtpad{4pt}\raggedleft {\fontsize{9.5pt}{11.4pt}\selectfont (0.289)~~~}\huxbpad{4pt}} &
\multicolumn{1}{r!{\huxvb{0}}}{\huxtpad{4pt}\raggedleft {\fontsize{9.5pt}{11.4pt}\selectfont (0.290)~~~}\huxbpad{4pt}} \tabularnewline[-0.5pt]


\hhline{}
\arrayrulecolor{black}

\multicolumn{1}{!{\huxvb{0}}l!{\huxvb{0}}}{\huxtpad{4pt}\raggedright {\fontsize{9.5pt}{11.4pt}\selectfont Demographics}\huxbpad{4pt}} &
\multicolumn{1}{r!{\huxvb{0}}}{\huxtpad{4pt}\raggedleft {\fontsize{9.5pt}{11.4pt}\selectfont ~~~~~~~~}\huxbpad{4pt}} &
\multicolumn{1}{r!{\huxvb{0}}}{\huxtpad{4pt}\raggedleft {\fontsize{9.5pt}{11.4pt}\selectfont ~~~~~~~~}\huxbpad{4pt}} &
\multicolumn{1}{r!{\huxvb{0}}}{\huxtpad{4pt}\raggedleft {\fontsize{9.5pt}{11.4pt}\selectfont ~~~~~~~~}\huxbpad{4pt}} \tabularnewline[-0.5pt]


\hhline{}
\arrayrulecolor{black}

\multicolumn{1}{!{\huxvb{0}}l!{\huxvb{0}}}{\huxtpad{4pt}\raggedright {\fontsize{9.5pt}{11.4pt}\selectfont Age}\huxbpad{4pt}} &
\multicolumn{1}{r!{\huxvb{0}}}{\huxtpad{4pt}\raggedleft {\fontsize{9.5pt}{11.4pt}\selectfont ~~~~~~~~}\huxbpad{4pt}} &
\multicolumn{1}{r!{\huxvb{0}}}{\huxtpad{4pt}\raggedleft {\fontsize{9.5pt}{11.4pt}\selectfont -0.007~~~~}\huxbpad{4pt}} &
\multicolumn{1}{r!{\huxvb{0}}}{\huxtpad{4pt}\raggedleft {\fontsize{9.5pt}{11.4pt}\selectfont 0.004~~~~}\huxbpad{4pt}} \tabularnewline[-0.5pt]


\hhline{}
\arrayrulecolor{black}

\multicolumn{1}{!{\huxvb{0}}l!{\huxvb{0}}}{\huxtpad{4pt}\raggedright {\fontsize{9.5pt}{11.4pt}\selectfont }\huxbpad{4pt}} &
\multicolumn{1}{r!{\huxvb{0}}}{\huxtpad{4pt}\raggedleft {\fontsize{9.5pt}{11.4pt}\selectfont ~~~~~~~~}\huxbpad{4pt}} &
\multicolumn{1}{r!{\huxvb{0}}}{\huxtpad{4pt}\raggedleft {\fontsize{9.5pt}{11.4pt}\selectfont (0.006)~~~}\huxbpad{4pt}} &
\multicolumn{1}{r!{\huxvb{0}}}{\huxtpad{4pt}\raggedleft {\fontsize{9.5pt}{11.4pt}\selectfont (0.007)~~~}\huxbpad{4pt}} \tabularnewline[-0.5pt]


\hhline{}
\arrayrulecolor{black}

\multicolumn{1}{!{\huxvb{0}}l!{\huxvb{0}}}{\huxtpad{4pt}\raggedright {\fontsize{9.5pt}{11.4pt}\selectfont Foreign Born}\huxbpad{4pt}} &
\multicolumn{1}{r!{\huxvb{0}}}{\huxtpad{4pt}\raggedleft {\fontsize{9.5pt}{11.4pt}\selectfont ~~~~~~~~}\huxbpad{4pt}} &
\multicolumn{1}{r!{\huxvb{0}}}{\huxtpad{4pt}\raggedleft {\fontsize{9.5pt}{11.4pt}\selectfont 0.288~~~~}\huxbpad{4pt}} &
\multicolumn{1}{r!{\huxvb{0}}}{\huxtpad{4pt}\raggedleft {\fontsize{9.5pt}{11.4pt}\selectfont 0.166~~~~}\huxbpad{4pt}} \tabularnewline[-0.5pt]


\hhline{}
\arrayrulecolor{black}

\multicolumn{1}{!{\huxvb{0}}l!{\huxvb{0}}}{\huxtpad{4pt}\raggedright {\fontsize{9.5pt}{11.4pt}\selectfont }\huxbpad{4pt}} &
\multicolumn{1}{r!{\huxvb{0}}}{\huxtpad{4pt}\raggedleft {\fontsize{9.5pt}{11.4pt}\selectfont ~~~~~~~~}\huxbpad{4pt}} &
\multicolumn{1}{r!{\huxvb{0}}}{\huxtpad{4pt}\raggedleft {\fontsize{9.5pt}{11.4pt}\selectfont (0.238)~~~}\huxbpad{4pt}} &
\multicolumn{1}{r!{\huxvb{0}}}{\huxtpad{4pt}\raggedleft {\fontsize{9.5pt}{11.4pt}\selectfont (0.246)~~~}\huxbpad{4pt}} \tabularnewline[-0.5pt]


\hhline{}
\arrayrulecolor{black}

\multicolumn{1}{!{\huxvb{0}}l!{\huxvb{0}}}{\huxtpad{4pt}\raggedright {\fontsize{9.5pt}{11.4pt}\selectfont Male}\huxbpad{4pt}} &
\multicolumn{1}{r!{\huxvb{0}}}{\huxtpad{4pt}\raggedleft {\fontsize{9.5pt}{11.4pt}\selectfont ~~~~~~~~}\huxbpad{4pt}} &
\multicolumn{1}{r!{\huxvb{0}}}{\huxtpad{4pt}\raggedleft {\fontsize{9.5pt}{11.4pt}\selectfont 0.475 **~}\huxbpad{4pt}} &
\multicolumn{1}{r!{\huxvb{0}}}{\huxtpad{4pt}\raggedleft {\fontsize{9.5pt}{11.4pt}\selectfont 0.504 **~}\huxbpad{4pt}} \tabularnewline[-0.5pt]


\hhline{}
\arrayrulecolor{black}

\multicolumn{1}{!{\huxvb{0}}l!{\huxvb{0}}}{\huxtpad{4pt}\raggedright {\fontsize{9.5pt}{11.4pt}\selectfont }\huxbpad{4pt}} &
\multicolumn{1}{r!{\huxvb{0}}}{\huxtpad{4pt}\raggedleft {\fontsize{9.5pt}{11.4pt}\selectfont ~~~~~~~~}\huxbpad{4pt}} &
\multicolumn{1}{r!{\huxvb{0}}}{\huxtpad{4pt}\raggedleft {\fontsize{9.5pt}{11.4pt}\selectfont (0.183)~~~}\huxbpad{4pt}} &
\multicolumn{1}{r!{\huxvb{0}}}{\huxtpad{4pt}\raggedleft {\fontsize{9.5pt}{11.4pt}\selectfont (0.191)~~~}\huxbpad{4pt}} \tabularnewline[-0.5pt]


\hhline{}
\arrayrulecolor{black}

\multicolumn{1}{!{\huxvb{0}}l!{\huxvb{0}}}{\huxtpad{4pt}\raggedright {\fontsize{9.5pt}{11.4pt}\selectfont Children Present}\huxbpad{4pt}} &
\multicolumn{1}{r!{\huxvb{0}}}{\huxtpad{4pt}\raggedleft {\fontsize{9.5pt}{11.4pt}\selectfont ~~~~~~~~}\huxbpad{4pt}} &
\multicolumn{1}{r!{\huxvb{0}}}{\huxtpad{4pt}\raggedleft {\fontsize{9.5pt}{11.4pt}\selectfont 0.114~~~~}\huxbpad{4pt}} &
\multicolumn{1}{r!{\huxvb{0}}}{\huxtpad{4pt}\raggedleft {\fontsize{9.5pt}{11.4pt}\selectfont 0.156~~~~}\huxbpad{4pt}} \tabularnewline[-0.5pt]


\hhline{}
\arrayrulecolor{black}

\multicolumn{1}{!{\huxvb{0}}l!{\huxvb{0}}}{\huxtpad{4pt}\raggedright {\fontsize{9.5pt}{11.4pt}\selectfont }\huxbpad{4pt}} &
\multicolumn{1}{r!{\huxvb{0}}}{\huxtpad{4pt}\raggedleft {\fontsize{9.5pt}{11.4pt}\selectfont ~~~~~~~~}\huxbpad{4pt}} &
\multicolumn{1}{r!{\huxvb{0}}}{\huxtpad{4pt}\raggedleft {\fontsize{9.5pt}{11.4pt}\selectfont (0.207)~~~}\huxbpad{4pt}} &
\multicolumn{1}{r!{\huxvb{0}}}{\huxtpad{4pt}\raggedleft {\fontsize{9.5pt}{11.4pt}\selectfont (0.222)~~~}\huxbpad{4pt}} \tabularnewline[-0.5pt]


\hhline{}
\arrayrulecolor{black}

\multicolumn{1}{!{\huxvb{0}}l!{\huxvb{0}}}{\huxtpad{4pt}\raggedright {\fontsize{9.5pt}{11.4pt}\selectfont Married}\huxbpad{4pt}} &
\multicolumn{1}{r!{\huxvb{0}}}{\huxtpad{4pt}\raggedleft {\fontsize{9.5pt}{11.4pt}\selectfont ~~~~~~~~}\huxbpad{4pt}} &
\multicolumn{1}{r!{\huxvb{0}}}{\huxtpad{4pt}\raggedleft {\fontsize{9.5pt}{11.4pt}\selectfont -0.227~~~~}\huxbpad{4pt}} &
\multicolumn{1}{r!{\huxvb{0}}}{\huxtpad{4pt}\raggedleft {\fontsize{9.5pt}{11.4pt}\selectfont -0.302~~~~}\huxbpad{4pt}} \tabularnewline[-0.5pt]


\hhline{}
\arrayrulecolor{black}

\multicolumn{1}{!{\huxvb{0}}l!{\huxvb{0}}}{\huxtpad{4pt}\raggedright {\fontsize{9.5pt}{11.4pt}\selectfont }\huxbpad{4pt}} &
\multicolumn{1}{r!{\huxvb{0}}}{\huxtpad{4pt}\raggedleft {\fontsize{9.5pt}{11.4pt}\selectfont ~~~~~~~~}\huxbpad{4pt}} &
\multicolumn{1}{r!{\huxvb{0}}}{\huxtpad{4pt}\raggedleft {\fontsize{9.5pt}{11.4pt}\selectfont (0.208)~~~}\huxbpad{4pt}} &
\multicolumn{1}{r!{\huxvb{0}}}{\huxtpad{4pt}\raggedleft {\fontsize{9.5pt}{11.4pt}\selectfont (0.213)~~~}\huxbpad{4pt}} \tabularnewline[-0.5pt]


\hhline{}
\arrayrulecolor{black}

\multicolumn{1}{!{\huxvb{0}}l!{\huxvb{0}}}{\huxtpad{4pt}\raggedright {\fontsize{9.5pt}{11.4pt}\selectfont Socioeconomic}\huxbpad{4pt}} &
\multicolumn{1}{r!{\huxvb{0}}}{\huxtpad{4pt}\raggedleft {\fontsize{9.5pt}{11.4pt}\selectfont ~~~~~~~~}\huxbpad{4pt}} &
\multicolumn{1}{r!{\huxvb{0}}}{\huxtpad{4pt}\raggedleft {\fontsize{9.5pt}{11.4pt}\selectfont ~~~~~~~~}\huxbpad{4pt}} &
\multicolumn{1}{r!{\huxvb{0}}}{\huxtpad{4pt}\raggedleft {\fontsize{9.5pt}{11.4pt}\selectfont ~~~~~~~~}\huxbpad{4pt}} \tabularnewline[-0.5pt]


\hhline{}
\arrayrulecolor{black}

\multicolumn{1}{!{\huxvb{0}}l!{\huxvb{0}}}{\huxtpad{4pt}\raggedright {\fontsize{9.5pt}{11.4pt}\selectfont $<$H.S.}\huxbpad{4pt}} &
\multicolumn{1}{r!{\huxvb{0}}}{\huxtpad{4pt}\raggedleft {\fontsize{9.5pt}{11.4pt}\selectfont ~~~~~~~~}\huxbpad{4pt}} &
\multicolumn{1}{r!{\huxvb{0}}}{\huxtpad{4pt}\raggedleft {\fontsize{9.5pt}{11.4pt}\selectfont 0.276~~~~}\huxbpad{4pt}} &
\multicolumn{1}{r!{\huxvb{0}}}{\huxtpad{4pt}\raggedleft {\fontsize{9.5pt}{11.4pt}\selectfont 0.082~~~~}\huxbpad{4pt}} \tabularnewline[-0.5pt]


\hhline{}
\arrayrulecolor{black}

\multicolumn{1}{!{\huxvb{0}}l!{\huxvb{0}}}{\huxtpad{4pt}\raggedright {\fontsize{9.5pt}{11.4pt}\selectfont }\huxbpad{4pt}} &
\multicolumn{1}{r!{\huxvb{0}}}{\huxtpad{4pt}\raggedleft {\fontsize{9.5pt}{11.4pt}\selectfont ~~~~~~~~}\huxbpad{4pt}} &
\multicolumn{1}{r!{\huxvb{0}}}{\huxtpad{4pt}\raggedleft {\fontsize{9.5pt}{11.4pt}\selectfont (0.395)~~~}\huxbpad{4pt}} &
\multicolumn{1}{r!{\huxvb{0}}}{\huxtpad{4pt}\raggedleft {\fontsize{9.5pt}{11.4pt}\selectfont (0.423)~~~}\huxbpad{4pt}} \tabularnewline[-0.5pt]


\hhline{}
\arrayrulecolor{black}

\multicolumn{1}{!{\huxvb{0}}l!{\huxvb{0}}}{\huxtpad{4pt}\raggedright {\fontsize{9.5pt}{11.4pt}\selectfont H.S.}\huxbpad{4pt}} &
\multicolumn{1}{r!{\huxvb{0}}}{\huxtpad{4pt}\raggedleft {\fontsize{9.5pt}{11.4pt}\selectfont ~~~~~~~~}\huxbpad{4pt}} &
\multicolumn{1}{r!{\huxvb{0}}}{\huxtpad{4pt}\raggedleft {\fontsize{9.5pt}{11.4pt}\selectfont -0.025~~~~}\huxbpad{4pt}} &
\multicolumn{1}{r!{\huxvb{0}}}{\huxtpad{4pt}\raggedleft {\fontsize{9.5pt}{11.4pt}\selectfont -0.117~~~~}\huxbpad{4pt}} \tabularnewline[-0.5pt]


\hhline{}
\arrayrulecolor{black}

\multicolumn{1}{!{\huxvb{0}}l!{\huxvb{0}}}{\huxtpad{4pt}\raggedright {\fontsize{9.5pt}{11.4pt}\selectfont }\huxbpad{4pt}} &
\multicolumn{1}{r!{\huxvb{0}}}{\huxtpad{4pt}\raggedleft {\fontsize{9.5pt}{11.4pt}\selectfont ~~~~~~~~}\huxbpad{4pt}} &
\multicolumn{1}{r!{\huxvb{0}}}{\huxtpad{4pt}\raggedleft {\fontsize{9.5pt}{11.4pt}\selectfont (0.323)~~~}\huxbpad{4pt}} &
\multicolumn{1}{r!{\huxvb{0}}}{\huxtpad{4pt}\raggedleft {\fontsize{9.5pt}{11.4pt}\selectfont (0.304)~~~}\huxbpad{4pt}} \tabularnewline[-0.5pt]


\hhline{}
\arrayrulecolor{black}

\multicolumn{1}{!{\huxvb{0}}l!{\huxvb{0}}}{\huxtpad{4pt}\raggedright {\fontsize{9.5pt}{11.4pt}\selectfont Some college, no B.A.}\huxbpad{4pt}} &
\multicolumn{1}{r!{\huxvb{0}}}{\huxtpad{4pt}\raggedleft {\fontsize{9.5pt}{11.4pt}\selectfont ~~~~~~~~}\huxbpad{4pt}} &
\multicolumn{1}{r!{\huxvb{0}}}{\huxtpad{4pt}\raggedleft {\fontsize{9.5pt}{11.4pt}\selectfont -0.260~~~~}\huxbpad{4pt}} &
\multicolumn{1}{r!{\huxvb{0}}}{\huxtpad{4pt}\raggedleft {\fontsize{9.5pt}{11.4pt}\selectfont -0.334~~~~}\huxbpad{4pt}} \tabularnewline[-0.5pt]


\hhline{}
\arrayrulecolor{black}

\multicolumn{1}{!{\huxvb{0}}l!{\huxvb{0}}}{\huxtpad{4pt}\raggedright {\fontsize{9.5pt}{11.4pt}\selectfont }\huxbpad{4pt}} &
\multicolumn{1}{r!{\huxvb{0}}}{\huxtpad{4pt}\raggedleft {\fontsize{9.5pt}{11.4pt}\selectfont ~~~~~~~~}\huxbpad{4pt}} &
\multicolumn{1}{r!{\huxvb{0}}}{\huxtpad{4pt}\raggedleft {\fontsize{9.5pt}{11.4pt}\selectfont (0.255)~~~}\huxbpad{4pt}} &
\multicolumn{1}{r!{\huxvb{0}}}{\huxtpad{4pt}\raggedleft {\fontsize{9.5pt}{11.4pt}\selectfont (0.263)~~~}\huxbpad{4pt}} \tabularnewline[-0.5pt]


\hhline{}
\arrayrulecolor{black}

\multicolumn{1}{!{\huxvb{0}}l!{\huxvb{0}}}{\huxtpad{4pt}\raggedright {\fontsize{9.5pt}{11.4pt}\selectfont M.A.+}\huxbpad{4pt}} &
\multicolumn{1}{r!{\huxvb{0}}}{\huxtpad{4pt}\raggedleft {\fontsize{9.5pt}{11.4pt}\selectfont ~~~~~~~~}\huxbpad{4pt}} &
\multicolumn{1}{r!{\huxvb{0}}}{\huxtpad{4pt}\raggedleft {\fontsize{9.5pt}{11.4pt}\selectfont -0.590 *~~}\huxbpad{4pt}} &
\multicolumn{1}{r!{\huxvb{0}}}{\huxtpad{4pt}\raggedleft {\fontsize{9.5pt}{11.4pt}\selectfont -0.627 **~}\huxbpad{4pt}} \tabularnewline[-0.5pt]


\hhline{}
\arrayrulecolor{black}

\multicolumn{1}{!{\huxvb{0}}l!{\huxvb{0}}}{\huxtpad{4pt}\raggedright {\fontsize{9.5pt}{11.4pt}\selectfont }\huxbpad{4pt}} &
\multicolumn{1}{r!{\huxvb{0}}}{\huxtpad{4pt}\raggedleft {\fontsize{9.5pt}{11.4pt}\selectfont ~~~~~~~~}\huxbpad{4pt}} &
\multicolumn{1}{r!{\huxvb{0}}}{\huxtpad{4pt}\raggedleft {\fontsize{9.5pt}{11.4pt}\selectfont (0.232)~~~}\huxbpad{4pt}} &
\multicolumn{1}{r!{\huxvb{0}}}{\huxtpad{4pt}\raggedleft {\fontsize{9.5pt}{11.4pt}\selectfont (0.242)~~~}\huxbpad{4pt}} \tabularnewline[-0.5pt]


\hhline{}
\arrayrulecolor{black}

\multicolumn{1}{!{\huxvb{0}}l!{\huxvb{0}}}{\huxtpad{4pt}\raggedright {\fontsize{9.5pt}{11.4pt}\selectfont $<$\$40,000}\huxbpad{4pt}} &
\multicolumn{1}{r!{\huxvb{0}}}{\huxtpad{4pt}\raggedleft {\fontsize{9.5pt}{11.4pt}\selectfont ~~~~~~~~}\huxbpad{4pt}} &
\multicolumn{1}{r!{\huxvb{0}}}{\huxtpad{4pt}\raggedleft {\fontsize{9.5pt}{11.4pt}\selectfont -0.429~~~~}\huxbpad{4pt}} &
\multicolumn{1}{r!{\huxvb{0}}}{\huxtpad{4pt}\raggedleft {\fontsize{9.5pt}{11.4pt}\selectfont -0.393~~~~}\huxbpad{4pt}} \tabularnewline[-0.5pt]


\hhline{}
\arrayrulecolor{black}

\multicolumn{1}{!{\huxvb{0}}l!{\huxvb{0}}}{\huxtpad{4pt}\raggedright {\fontsize{9.5pt}{11.4pt}\selectfont }\huxbpad{4pt}} &
\multicolumn{1}{r!{\huxvb{0}}}{\huxtpad{4pt}\raggedleft {\fontsize{9.5pt}{11.4pt}\selectfont ~~~~~~~~}\huxbpad{4pt}} &
\multicolumn{1}{r!{\huxvb{0}}}{\huxtpad{4pt}\raggedleft {\fontsize{9.5pt}{11.4pt}\selectfont (0.274)~~~}\huxbpad{4pt}} &
\multicolumn{1}{r!{\huxvb{0}}}{\huxtpad{4pt}\raggedleft {\fontsize{9.5pt}{11.4pt}\selectfont (0.281)~~~}\huxbpad{4pt}} \tabularnewline[-0.5pt]


\hhline{}
\arrayrulecolor{black}

\multicolumn{1}{!{\huxvb{0}}l!{\huxvb{0}}}{\huxtpad{4pt}\raggedright {\fontsize{9.5pt}{11.4pt}\selectfont \$75,000 to $<$\$150,000}\huxbpad{4pt}} &
\multicolumn{1}{r!{\huxvb{0}}}{\huxtpad{4pt}\raggedleft {\fontsize{9.5pt}{11.4pt}\selectfont ~~~~~~~~}\huxbpad{4pt}} &
\multicolumn{1}{r!{\huxvb{0}}}{\huxtpad{4pt}\raggedleft {\fontsize{9.5pt}{11.4pt}\selectfont -0.163~~~~}\huxbpad{4pt}} &
\multicolumn{1}{r!{\huxvb{0}}}{\huxtpad{4pt}\raggedleft {\fontsize{9.5pt}{11.4pt}\selectfont -0.259~~~~}\huxbpad{4pt}} \tabularnewline[-0.5pt]


\hhline{}
\arrayrulecolor{black}

\multicolumn{1}{!{\huxvb{0}}l!{\huxvb{0}}}{\huxtpad{4pt}\raggedright {\fontsize{9.5pt}{11.4pt}\selectfont }\huxbpad{4pt}} &
\multicolumn{1}{r!{\huxvb{0}}}{\huxtpad{4pt}\raggedleft {\fontsize{9.5pt}{11.4pt}\selectfont ~~~~~~~~}\huxbpad{4pt}} &
\multicolumn{1}{r!{\huxvb{0}}}{\huxtpad{4pt}\raggedleft {\fontsize{9.5pt}{11.4pt}\selectfont (0.246)~~~}\huxbpad{4pt}} &
\multicolumn{1}{r!{\huxvb{0}}}{\huxtpad{4pt}\raggedleft {\fontsize{9.5pt}{11.4pt}\selectfont (0.248)~~~}\huxbpad{4pt}} \tabularnewline[-0.5pt]


\hhline{}
\arrayrulecolor{black}

\multicolumn{1}{!{\huxvb{0}}l!{\huxvb{0}}}{\huxtpad{4pt}\raggedright {\fontsize{9.5pt}{11.4pt}\selectfont \$150,000+}\huxbpad{4pt}} &
\multicolumn{1}{r!{\huxvb{0}}}{\huxtpad{4pt}\raggedleft {\fontsize{9.5pt}{11.4pt}\selectfont ~~~~~~~~}\huxbpad{4pt}} &
\multicolumn{1}{r!{\huxvb{0}}}{\huxtpad{4pt}\raggedleft {\fontsize{9.5pt}{11.4pt}\selectfont 0.175~~~~}\huxbpad{4pt}} &
\multicolumn{1}{r!{\huxvb{0}}}{\huxtpad{4pt}\raggedleft {\fontsize{9.5pt}{11.4pt}\selectfont 0.042~~~~}\huxbpad{4pt}} \tabularnewline[-0.5pt]


\hhline{}
\arrayrulecolor{black}

\multicolumn{1}{!{\huxvb{0}}l!{\huxvb{0}}}{\huxtpad{4pt}\raggedright {\fontsize{9.5pt}{11.4pt}\selectfont }\huxbpad{4pt}} &
\multicolumn{1}{r!{\huxvb{0}}}{\huxtpad{4pt}\raggedleft {\fontsize{9.5pt}{11.4pt}\selectfont ~~~~~~~~}\huxbpad{4pt}} &
\multicolumn{1}{r!{\huxvb{0}}}{\huxtpad{4pt}\raggedleft {\fontsize{9.5pt}{11.4pt}\selectfont (0.302)~~~}\huxbpad{4pt}} &
\multicolumn{1}{r!{\huxvb{0}}}{\huxtpad{4pt}\raggedleft {\fontsize{9.5pt}{11.4pt}\selectfont (0.315)~~~}\huxbpad{4pt}} \tabularnewline[-0.5pt]


\hhline{}
\arrayrulecolor{black}

\multicolumn{1}{!{\huxvb{0}}l!{\huxvb{0}}}{\huxtpad{4pt}\raggedright {\fontsize{9.5pt}{11.4pt}\selectfont Neighborhood experience}\huxbpad{4pt}} &
\multicolumn{1}{r!{\huxvb{0}}}{\huxtpad{4pt}\raggedleft {\fontsize{9.5pt}{11.4pt}\selectfont ~~~~~~~~}\huxbpad{4pt}} &
\multicolumn{1}{r!{\huxvb{0}}}{\huxtpad{4pt}\raggedleft {\fontsize{9.5pt}{11.4pt}\selectfont ~~~~~~~~}\huxbpad{4pt}} &
\multicolumn{1}{r!{\huxvb{0}}}{\huxtpad{4pt}\raggedleft {\fontsize{9.5pt}{11.4pt}\selectfont ~~~~~~~~}\huxbpad{4pt}} \tabularnewline[-0.5pt]


\hhline{}
\arrayrulecolor{black}

\multicolumn{1}{!{\huxvb{0}}l!{\huxvb{0}}}{\huxtpad{4pt}\raggedright {\fontsize{9.5pt}{11.4pt}\selectfont Years in neighborhood}\huxbpad{4pt}} &
\multicolumn{1}{r!{\huxvb{0}}}{\huxtpad{4pt}\raggedleft {\fontsize{9.5pt}{11.4pt}\selectfont ~~~~~~~~}\huxbpad{4pt}} &
\multicolumn{1}{r!{\huxvb{0}}}{\huxtpad{4pt}\raggedleft {\fontsize{9.5pt}{11.4pt}\selectfont ~~~~~~~~}\huxbpad{4pt}} &
\multicolumn{1}{r!{\huxvb{0}}}{\huxtpad{4pt}\raggedleft {\fontsize{9.5pt}{11.4pt}\selectfont -0.022 *~~}\huxbpad{4pt}} \tabularnewline[-0.5pt]


\hhline{}
\arrayrulecolor{black}

\multicolumn{1}{!{\huxvb{0}}l!{\huxvb{0}}}{\huxtpad{4pt}\raggedright {\fontsize{9.5pt}{11.4pt}\selectfont }\huxbpad{4pt}} &
\multicolumn{1}{r!{\huxvb{0}}}{\huxtpad{4pt}\raggedleft {\fontsize{9.5pt}{11.4pt}\selectfont ~~~~~~~~}\huxbpad{4pt}} &
\multicolumn{1}{r!{\huxvb{0}}}{\huxtpad{4pt}\raggedleft {\fontsize{9.5pt}{11.4pt}\selectfont ~~~~~~~~}\huxbpad{4pt}} &
\multicolumn{1}{r!{\huxvb{0}}}{\huxtpad{4pt}\raggedleft {\fontsize{9.5pt}{11.4pt}\selectfont (0.010)~~~}\huxbpad{4pt}} \tabularnewline[-0.5pt]


\hhline{}
\arrayrulecolor{black}

\multicolumn{1}{!{\huxvb{0}}l!{\huxvb{0}}}{\huxtpad{4pt}\raggedright {\fontsize{9.5pt}{11.4pt}\selectfont 1-9 blocks}\huxbpad{4pt}} &
\multicolumn{1}{r!{\huxvb{0}}}{\huxtpad{4pt}\raggedleft {\fontsize{9.5pt}{11.4pt}\selectfont ~~~~~~~~}\huxbpad{4pt}} &
\multicolumn{1}{r!{\huxvb{0}}}{\huxtpad{4pt}\raggedleft {\fontsize{9.5pt}{11.4pt}\selectfont ~~~~~~~~}\huxbpad{4pt}} &
\multicolumn{1}{r!{\huxvb{0}}}{\huxtpad{4pt}\raggedleft {\fontsize{9.5pt}{11.4pt}\selectfont -0.603 **~}\huxbpad{4pt}} \tabularnewline[-0.5pt]


\hhline{}
\arrayrulecolor{black}

\multicolumn{1}{!{\huxvb{0}}l!{\huxvb{0}}}{\huxtpad{4pt}\raggedright {\fontsize{9.5pt}{11.4pt}\selectfont }\huxbpad{4pt}} &
\multicolumn{1}{r!{\huxvb{0}}}{\huxtpad{4pt}\raggedleft {\fontsize{9.5pt}{11.4pt}\selectfont ~~~~~~~~}\huxbpad{4pt}} &
\multicolumn{1}{r!{\huxvb{0}}}{\huxtpad{4pt}\raggedleft {\fontsize{9.5pt}{11.4pt}\selectfont ~~~~~~~~}\huxbpad{4pt}} &
\multicolumn{1}{r!{\huxvb{0}}}{\huxtpad{4pt}\raggedleft {\fontsize{9.5pt}{11.4pt}\selectfont (0.198)~~~}\huxbpad{4pt}} \tabularnewline[-0.5pt]


\hhline{}
\arrayrulecolor{black}

\multicolumn{1}{!{\huxvb{0}}l!{\huxvb{0}}}{\huxtpad{4pt}\raggedright {\fontsize{9.5pt}{11.4pt}\selectfont \$$>$\$50 blocks}\huxbpad{4pt}} &
\multicolumn{1}{r!{\huxvb{0}}}{\huxtpad{4pt}\raggedleft {\fontsize{9.5pt}{11.4pt}\selectfont ~~~~~~~~}\huxbpad{4pt}} &
\multicolumn{1}{r!{\huxvb{0}}}{\huxtpad{4pt}\raggedleft {\fontsize{9.5pt}{11.4pt}\selectfont ~~~~~~~~}\huxbpad{4pt}} &
\multicolumn{1}{r!{\huxvb{0}}}{\huxtpad{4pt}\raggedleft {\fontsize{9.5pt}{11.4pt}\selectfont -0.049~~~~}\huxbpad{4pt}} \tabularnewline[-0.5pt]


\hhline{}
\arrayrulecolor{black}

\multicolumn{1}{!{\huxvb{0}}l!{\huxvb{0}}}{\huxtpad{4pt}\raggedright {\fontsize{9.5pt}{11.4pt}\selectfont }\huxbpad{4pt}} &
\multicolumn{1}{r!{\huxvb{0}}}{\huxtpad{4pt}\raggedleft {\fontsize{9.5pt}{11.4pt}\selectfont ~~~~~~~~}\huxbpad{4pt}} &
\multicolumn{1}{r!{\huxvb{0}}}{\huxtpad{4pt}\raggedleft {\fontsize{9.5pt}{11.4pt}\selectfont ~~~~~~~~}\huxbpad{4pt}} &
\multicolumn{1}{r!{\huxvb{0}}}{\huxtpad{4pt}\raggedleft {\fontsize{9.5pt}{11.4pt}\selectfont (0.397)~~~}\huxbpad{4pt}} \tabularnewline[-0.5pt]


\hhline{}
\arrayrulecolor{black}

\multicolumn{1}{!{\huxvb{0}}l!{\huxvb{0}}}{\huxtpad{4pt}\raggedright {\fontsize{9.5pt}{11.4pt}\selectfont Quadrivial Neighborhood}\huxbpad{4pt}} &
\multicolumn{1}{r!{\huxvb{0}}}{\huxtpad{4pt}\raggedleft {\fontsize{9.5pt}{11.4pt}\selectfont ~~~~~~~~}\huxbpad{4pt}} &
\multicolumn{1}{r!{\huxvb{0}}}{\huxtpad{4pt}\raggedleft {\fontsize{9.5pt}{11.4pt}\selectfont ~~~~~~~~}\huxbpad{4pt}} &
\multicolumn{1}{r!{\huxvb{0}}}{\huxtpad{4pt}\raggedleft {\fontsize{9.5pt}{11.4pt}\selectfont -0.586 **~}\huxbpad{4pt}} \tabularnewline[-0.5pt]


\hhline{}
\arrayrulecolor{black}

\multicolumn{1}{!{\huxvb{0}}l!{\huxvb{0}}}{\huxtpad{4pt}\raggedright {\fontsize{9.5pt}{11.4pt}\selectfont }\huxbpad{4pt}} &
\multicolumn{1}{r!{\huxvb{0}}}{\huxtpad{4pt}\raggedleft {\fontsize{9.5pt}{11.4pt}\selectfont ~~~~~~~~}\huxbpad{4pt}} &
\multicolumn{1}{r!{\huxvb{0}}}{\huxtpad{4pt}\raggedleft {\fontsize{9.5pt}{11.4pt}\selectfont ~~~~~~~~}\huxbpad{4pt}} &
\multicolumn{1}{r!{\huxvb{0}}}{\huxtpad{4pt}\raggedleft {\fontsize{9.5pt}{11.4pt}\selectfont (0.195)~~~}\huxbpad{4pt}} \tabularnewline[-0.5pt]


\hhline{}
\arrayrulecolor{black}

\multicolumn{1}{!{\huxvb{0}}l!{\huxvb{0}}}{\huxtpad{4pt}\raggedright {\fontsize{9.5pt}{11.4pt}\selectfont Extremely satisfied}\huxbpad{4pt}} &
\multicolumn{1}{r!{\huxvb{0}}}{\huxtpad{4pt}\raggedleft {\fontsize{9.5pt}{11.4pt}\selectfont ~~~~~~~~}\huxbpad{4pt}} &
\multicolumn{1}{r!{\huxvb{0}}}{\huxtpad{4pt}\raggedleft {\fontsize{9.5pt}{11.4pt}\selectfont ~~~~~~~~}\huxbpad{4pt}} &
\multicolumn{1}{r!{\huxvb{0}}}{\huxtpad{4pt}\raggedleft {\fontsize{9.5pt}{11.4pt}\selectfont 0.988 ***}\huxbpad{4pt}} \tabularnewline[-0.5pt]


\hhline{}
\arrayrulecolor{black}

\multicolumn{1}{!{\huxvb{0}}l!{\huxvb{0}}}{\huxtpad{4pt}\raggedright {\fontsize{9.5pt}{11.4pt}\selectfont }\huxbpad{4pt}} &
\multicolumn{1}{r!{\huxvb{0}}}{\huxtpad{4pt}\raggedleft {\fontsize{9.5pt}{11.4pt}\selectfont ~~~~~~~~}\huxbpad{4pt}} &
\multicolumn{1}{r!{\huxvb{0}}}{\huxtpad{4pt}\raggedleft {\fontsize{9.5pt}{11.4pt}\selectfont ~~~~~~~~}\huxbpad{4pt}} &
\multicolumn{1}{r!{\huxvb{0}}}{\huxtpad{4pt}\raggedleft {\fontsize{9.5pt}{11.4pt}\selectfont (0.230)~~~}\huxbpad{4pt}} \tabularnewline[-0.5pt]


\hhline{>{\huxb{1}}->{\huxb{1}}->{\huxb{1}}->{\huxb{1}}-}
\arrayrulecolor{black}

\multicolumn{1}{!{\huxvb{0}}l!{\huxvb{0}}}{\huxtpad{4pt}\raggedright {\fontsize{9.5pt}{11.4pt}\selectfont nobs}\huxbpad{4pt}} &
\multicolumn{1}{r!{\huxvb{0}}}{\huxtpad{4pt}\raggedleft {\fontsize{9.5pt}{11.4pt}\selectfont ~~~~~~~~}\huxbpad{4pt}} &
\multicolumn{1}{r!{\huxvb{0}}}{\huxtpad{4pt}\raggedleft {\fontsize{9.5pt}{11.4pt}\selectfont ~~~~~~~~}\huxbpad{4pt}} &
\multicolumn{1}{r!{\huxvb{0}}}{\huxtpad{4pt}\raggedleft {\fontsize{9.5pt}{11.4pt}\selectfont ~~~~~~~~}\huxbpad{4pt}} \tabularnewline[-0.5pt]


\hhline{>{\huxb{0.8}}->{\huxb{0.8}}->{\huxb{0.8}}->{\huxb{0.8}}-}
\arrayrulecolor{black}

\multicolumn{4}{!{\huxvb{0}}p{0.5\textwidth+6\tabcolsep}!{\huxvb{0}}}{\parbox[b]{0.5\textwidth+6\tabcolsep-4pt-4pt}{\huxtpad{4pt}\raggedright {\fontsize{9.5pt}{11.4pt}\selectfont  *** p $<$ 0.001;  ** p $<$ 0.01;  * p $<$ 0.05.}\huxbpad{4pt}}} \tabularnewline[-0.5pt]


\hhline{}
\arrayrulecolor{black}
\end{tabularx}
\end{table}


\begin{table}[h]
\centering\captionsetup{justification=centering,singlelinecheck=off}
\caption{Estimated coefficients predicting unfamiliarity with selected multiethnic communities}
\label{tab:knowledge}

    \providecommand{\huxb}[2][0,0,0]{\arrayrulecolor[RGB]{#1}\global\arrayrulewidth=#2pt}
    \providecommand{\huxvb}[2][0,0,0]{\color[RGB]{#1}\vrule width #2pt}
    \providecommand{\huxtpad}[1]{\rule{0pt}{\baselineskip+#1}}
    \providecommand{\huxbpad}[1]{\rule[-#1]{0pt}{#1}}
  \begin{tabularx}{0.5\textwidth}{p{0.125\textwidth} p{0.125\textwidth} p{0.125\textwidth} p{0.125\textwidth}}


\hhline{>{\huxb{0.8}}->{\huxb{0.8}}->{\huxb{0.8}}->{\huxb{0.8}}-}
\arrayrulecolor{black}

\multicolumn{1}{!{\huxvb{0}}c!{\huxvb{0}}}{\huxtpad{4pt}\centering {\fontsize{9.5pt}{11.4pt}\selectfont }\huxbpad{4pt}} &
\multicolumn{1}{c!{\huxvb{0}}}{\huxtpad{4pt}\centering {\fontsize{9.5pt}{11.4pt}\selectfont Herndon}\huxbpad{4pt}} &
\multicolumn{1}{c!{\huxvb{0}}}{\huxtpad{4pt}\centering {\fontsize{9.5pt}{11.4pt}\selectfont \~{}\~{}\~{}\~{}Germantown}\huxbpad{4pt}} &
\multicolumn{1}{c!{\huxvb{0}}}{\huxtpad{4pt}\centering {\fontsize{9.5pt}{11.4pt}\selectfont \~{}\~{}\~{}\~{}\~{}\~{}Wheaton}\huxbpad{4pt}} \tabularnewline[-0.5pt]


\hhline{>{\huxb{1}}->{\huxb{1}}->{\huxb{1}}->{\huxb{1}}-}
\arrayrulecolor{black}

\multicolumn{1}{!{\huxvb{0}}l!{\huxvb{0}}}{\huxtpad{4pt}\raggedright {\fontsize{9.5pt}{11.4pt}\selectfont (Intercept)}\huxbpad{4pt}} &
\multicolumn{1}{r!{\huxvb{0}}}{\huxtpad{4pt}\raggedleft {\fontsize{9.5pt}{11.4pt}\selectfont -2.296 *}\huxbpad{4pt}} &
\multicolumn{1}{r!{\huxvb{0}}}{\huxtpad{4pt}\raggedleft {\fontsize{9.5pt}{11.4pt}\selectfont -0.315~}\huxbpad{4pt}} &
\multicolumn{1}{r!{\huxvb{0}}}{\huxtpad{4pt}\raggedleft {\fontsize{9.5pt}{11.4pt}\selectfont 0.559~}\huxbpad{4pt}} \tabularnewline[-0.5pt]


\hhline{}
\arrayrulecolor{black}

\multicolumn{1}{!{\huxvb{0}}l!{\huxvb{0}}}{\huxtpad{4pt}\raggedright {\fontsize{9.5pt}{11.4pt}\selectfont }\huxbpad{4pt}} &
\multicolumn{1}{r!{\huxvb{0}}}{\huxtpad{4pt}\raggedleft {\fontsize{9.5pt}{11.4pt}\selectfont (1.011)~}\huxbpad{4pt}} &
\multicolumn{1}{r!{\huxvb{0}}}{\huxtpad{4pt}\raggedleft {\fontsize{9.5pt}{11.4pt}\selectfont (0.925)}\huxbpad{4pt}} &
\multicolumn{1}{r!{\huxvb{0}}}{\huxtpad{4pt}\raggedleft {\fontsize{9.5pt}{11.4pt}\selectfont (0.982)}\huxbpad{4pt}} \tabularnewline[-0.5pt]


\hhline{}
\arrayrulecolor{black}

\multicolumn{1}{!{\huxvb{0}}l!{\huxvb{0}}}{\huxtpad{4pt}\raggedright {\fontsize{9.5pt}{11.4pt}\selectfont Race}\huxbpad{4pt}} &
\multicolumn{1}{r!{\huxvb{0}}}{\huxtpad{4pt}\raggedleft {\fontsize{9.5pt}{11.4pt}\selectfont ~~~~~~}\huxbpad{4pt}} &
\multicolumn{1}{r!{\huxvb{0}}}{\huxtpad{4pt}\raggedleft {\fontsize{9.5pt}{11.4pt}\selectfont ~~~~~}\huxbpad{4pt}} &
\multicolumn{1}{r!{\huxvb{0}}}{\huxtpad{4pt}\raggedleft {\fontsize{9.5pt}{11.4pt}\selectfont ~~~~~}\huxbpad{4pt}} \tabularnewline[-0.5pt]


\hhline{}
\arrayrulecolor{black}

\multicolumn{1}{!{\huxvb{0}}l!{\huxvb{0}}}{\huxtpad{4pt}\raggedright {\fontsize{9.5pt}{11.4pt}\selectfont Asian}\huxbpad{4pt}} &
\multicolumn{1}{r!{\huxvb{0}}}{\huxtpad{4pt}\raggedleft {\fontsize{9.5pt}{11.4pt}\selectfont 0.151~~}\huxbpad{4pt}} &
\multicolumn{1}{r!{\huxvb{0}}}{\huxtpad{4pt}\raggedleft {\fontsize{9.5pt}{11.4pt}\selectfont 0.008~}\huxbpad{4pt}} &
\multicolumn{1}{r!{\huxvb{0}}}{\huxtpad{4pt}\raggedleft {\fontsize{9.5pt}{11.4pt}\selectfont 0.655~}\huxbpad{4pt}} \tabularnewline[-0.5pt]


\hhline{}
\arrayrulecolor{black}

\multicolumn{1}{!{\huxvb{0}}l!{\huxvb{0}}}{\huxtpad{4pt}\raggedright {\fontsize{9.5pt}{11.4pt}\selectfont }\huxbpad{4pt}} &
\multicolumn{1}{r!{\huxvb{0}}}{\huxtpad{4pt}\raggedleft {\fontsize{9.5pt}{11.4pt}\selectfont (0.408)~}\huxbpad{4pt}} &
\multicolumn{1}{r!{\huxvb{0}}}{\huxtpad{4pt}\raggedleft {\fontsize{9.5pt}{11.4pt}\selectfont (0.541)}\huxbpad{4pt}} &
\multicolumn{1}{r!{\huxvb{0}}}{\huxtpad{4pt}\raggedleft {\fontsize{9.5pt}{11.4pt}\selectfont (0.533)}\huxbpad{4pt}} \tabularnewline[-0.5pt]


\hhline{}
\arrayrulecolor{black}

\multicolumn{1}{!{\huxvb{0}}l!{\huxvb{0}}}{\huxtpad{4pt}\raggedright {\fontsize{9.5pt}{11.4pt}\selectfont Black}\huxbpad{4pt}} &
\multicolumn{1}{r!{\huxvb{0}}}{\huxtpad{4pt}\raggedleft {\fontsize{9.5pt}{11.4pt}\selectfont 0.561~~}\huxbpad{4pt}} &
\multicolumn{1}{r!{\huxvb{0}}}{\huxtpad{4pt}\raggedleft {\fontsize{9.5pt}{11.4pt}\selectfont 0.170~}\huxbpad{4pt}} &
\multicolumn{1}{r!{\huxvb{0}}}{\huxtpad{4pt}\raggedleft {\fontsize{9.5pt}{11.4pt}\selectfont -0.691~}\huxbpad{4pt}} \tabularnewline[-0.5pt]


\hhline{}
\arrayrulecolor{black}

\multicolumn{1}{!{\huxvb{0}}l!{\huxvb{0}}}{\huxtpad{4pt}\raggedright {\fontsize{9.5pt}{11.4pt}\selectfont }\huxbpad{4pt}} &
\multicolumn{1}{r!{\huxvb{0}}}{\huxtpad{4pt}\raggedleft {\fontsize{9.5pt}{11.4pt}\selectfont (0.378)~}\huxbpad{4pt}} &
\multicolumn{1}{r!{\huxvb{0}}}{\huxtpad{4pt}\raggedleft {\fontsize{9.5pt}{11.4pt}\selectfont (0.479)}\huxbpad{4pt}} &
\multicolumn{1}{r!{\huxvb{0}}}{\huxtpad{4pt}\raggedleft {\fontsize{9.5pt}{11.4pt}\selectfont (0.498)}\huxbpad{4pt}} \tabularnewline[-0.5pt]


\hhline{}
\arrayrulecolor{black}

\multicolumn{1}{!{\huxvb{0}}l!{\huxvb{0}}}{\huxtpad{4pt}\raggedright {\fontsize{9.5pt}{11.4pt}\selectfont Latinx}\huxbpad{4pt}} &
\multicolumn{1}{r!{\huxvb{0}}}{\huxtpad{4pt}\raggedleft {\fontsize{9.5pt}{11.4pt}\selectfont 0.219~~}\huxbpad{4pt}} &
\multicolumn{1}{r!{\huxvb{0}}}{\huxtpad{4pt}\raggedleft {\fontsize{9.5pt}{11.4pt}\selectfont -0.429~}\huxbpad{4pt}} &
\multicolumn{1}{r!{\huxvb{0}}}{\huxtpad{4pt}\raggedleft {\fontsize{9.5pt}{11.4pt}\selectfont -0.242~}\huxbpad{4pt}} \tabularnewline[-0.5pt]


\hhline{}
\arrayrulecolor{black}

\multicolumn{1}{!{\huxvb{0}}l!{\huxvb{0}}}{\huxtpad{4pt}\raggedright {\fontsize{9.5pt}{11.4pt}\selectfont }\huxbpad{4pt}} &
\multicolumn{1}{r!{\huxvb{0}}}{\huxtpad{4pt}\raggedleft {\fontsize{9.5pt}{11.4pt}\selectfont (0.419)~}\huxbpad{4pt}} &
\multicolumn{1}{r!{\huxvb{0}}}{\huxtpad{4pt}\raggedleft {\fontsize{9.5pt}{11.4pt}\selectfont (0.533)}\huxbpad{4pt}} &
\multicolumn{1}{r!{\huxvb{0}}}{\huxtpad{4pt}\raggedleft {\fontsize{9.5pt}{11.4pt}\selectfont (0.482)}\huxbpad{4pt}} \tabularnewline[-0.5pt]


\hhline{}
\arrayrulecolor{black}

\multicolumn{1}{!{\huxvb{0}}l!{\huxvb{0}}}{\huxtpad{4pt}\raggedright {\fontsize{9.5pt}{11.4pt}\selectfont Demographics}\huxbpad{4pt}} &
\multicolumn{1}{r!{\huxvb{0}}}{\huxtpad{4pt}\raggedleft {\fontsize{9.5pt}{11.4pt}\selectfont ~~~~~~}\huxbpad{4pt}} &
\multicolumn{1}{r!{\huxvb{0}}}{\huxtpad{4pt}\raggedleft {\fontsize{9.5pt}{11.4pt}\selectfont ~~~~~}\huxbpad{4pt}} &
\multicolumn{1}{r!{\huxvb{0}}}{\huxtpad{4pt}\raggedleft {\fontsize{9.5pt}{11.4pt}\selectfont ~~~~~}\huxbpad{4pt}} \tabularnewline[-0.5pt]


\hhline{}
\arrayrulecolor{black}

\multicolumn{1}{!{\huxvb{0}}l!{\huxvb{0}}}{\huxtpad{4pt}\raggedright {\fontsize{9.5pt}{11.4pt}\selectfont Age}\huxbpad{4pt}} &
\multicolumn{1}{r!{\huxvb{0}}}{\huxtpad{4pt}\raggedleft {\fontsize{9.5pt}{11.4pt}\selectfont -0.016~~}\huxbpad{4pt}} &
\multicolumn{1}{r!{\huxvb{0}}}{\huxtpad{4pt}\raggedleft {\fontsize{9.5pt}{11.4pt}\selectfont 0.020~}\huxbpad{4pt}} &
\multicolumn{1}{r!{\huxvb{0}}}{\huxtpad{4pt}\raggedleft {\fontsize{9.5pt}{11.4pt}\selectfont -0.014~}\huxbpad{4pt}} \tabularnewline[-0.5pt]


\hhline{}
\arrayrulecolor{black}

\multicolumn{1}{!{\huxvb{0}}l!{\huxvb{0}}}{\huxtpad{4pt}\raggedright {\fontsize{9.5pt}{11.4pt}\selectfont }\huxbpad{4pt}} &
\multicolumn{1}{r!{\huxvb{0}}}{\huxtpad{4pt}\raggedleft {\fontsize{9.5pt}{11.4pt}\selectfont (0.010)~}\huxbpad{4pt}} &
\multicolumn{1}{r!{\huxvb{0}}}{\huxtpad{4pt}\raggedleft {\fontsize{9.5pt}{11.4pt}\selectfont (0.012)}\huxbpad{4pt}} &
\multicolumn{1}{r!{\huxvb{0}}}{\huxtpad{4pt}\raggedleft {\fontsize{9.5pt}{11.4pt}\selectfont (0.011)}\huxbpad{4pt}} \tabularnewline[-0.5pt]


\hhline{}
\arrayrulecolor{black}

\multicolumn{1}{!{\huxvb{0}}l!{\huxvb{0}}}{\huxtpad{4pt}\raggedright {\fontsize{9.5pt}{11.4pt}\selectfont Foreign Born}\huxbpad{4pt}} &
\multicolumn{1}{r!{\huxvb{0}}}{\huxtpad{4pt}\raggedleft {\fontsize{9.5pt}{11.4pt}\selectfont 0.361~~}\huxbpad{4pt}} &
\multicolumn{1}{r!{\huxvb{0}}}{\huxtpad{4pt}\raggedleft {\fontsize{9.5pt}{11.4pt}\selectfont 0.209~}\huxbpad{4pt}} &
\multicolumn{1}{r!{\huxvb{0}}}{\huxtpad{4pt}\raggedleft {\fontsize{9.5pt}{11.4pt}\selectfont -0.147~}\huxbpad{4pt}} \tabularnewline[-0.5pt]


\hhline{}
\arrayrulecolor{black}

\multicolumn{1}{!{\huxvb{0}}l!{\huxvb{0}}}{\huxtpad{4pt}\raggedright {\fontsize{9.5pt}{11.4pt}\selectfont }\huxbpad{4pt}} &
\multicolumn{1}{r!{\huxvb{0}}}{\huxtpad{4pt}\raggedleft {\fontsize{9.5pt}{11.4pt}\selectfont (0.357)~}\huxbpad{4pt}} &
\multicolumn{1}{r!{\huxvb{0}}}{\huxtpad{4pt}\raggedleft {\fontsize{9.5pt}{11.4pt}\selectfont (0.424)}\huxbpad{4pt}} &
\multicolumn{1}{r!{\huxvb{0}}}{\huxtpad{4pt}\raggedleft {\fontsize{9.5pt}{11.4pt}\selectfont (0.455)}\huxbpad{4pt}} \tabularnewline[-0.5pt]


\hhline{}
\arrayrulecolor{black}

\multicolumn{1}{!{\huxvb{0}}l!{\huxvb{0}}}{\huxtpad{4pt}\raggedright {\fontsize{9.5pt}{11.4pt}\selectfont Male}\huxbpad{4pt}} &
\multicolumn{1}{r!{\huxvb{0}}}{\huxtpad{4pt}\raggedleft {\fontsize{9.5pt}{11.4pt}\selectfont 0.087~~}\huxbpad{4pt}} &
\multicolumn{1}{r!{\huxvb{0}}}{\huxtpad{4pt}\raggedleft {\fontsize{9.5pt}{11.4pt}\selectfont 0.313~}\huxbpad{4pt}} &
\multicolumn{1}{r!{\huxvb{0}}}{\huxtpad{4pt}\raggedleft {\fontsize{9.5pt}{11.4pt}\selectfont -0.053~}\huxbpad{4pt}} \tabularnewline[-0.5pt]


\hhline{}
\arrayrulecolor{black}

\multicolumn{1}{!{\huxvb{0}}l!{\huxvb{0}}}{\huxtpad{4pt}\raggedright {\fontsize{9.5pt}{11.4pt}\selectfont }\huxbpad{4pt}} &
\multicolumn{1}{r!{\huxvb{0}}}{\huxtpad{4pt}\raggedleft {\fontsize{9.5pt}{11.4pt}\selectfont (0.280)~}\huxbpad{4pt}} &
\multicolumn{1}{r!{\huxvb{0}}}{\huxtpad{4pt}\raggedleft {\fontsize{9.5pt}{11.4pt}\selectfont (0.332)}\huxbpad{4pt}} &
\multicolumn{1}{r!{\huxvb{0}}}{\huxtpad{4pt}\raggedleft {\fontsize{9.5pt}{11.4pt}\selectfont (0.337)}\huxbpad{4pt}} \tabularnewline[-0.5pt]


\hhline{}
\arrayrulecolor{black}

\multicolumn{1}{!{\huxvb{0}}l!{\huxvb{0}}}{\huxtpad{4pt}\raggedright {\fontsize{9.5pt}{11.4pt}\selectfont Children Present}\huxbpad{4pt}} &
\multicolumn{1}{r!{\huxvb{0}}}{\huxtpad{4pt}\raggedleft {\fontsize{9.5pt}{11.4pt}\selectfont -0.124~~}\huxbpad{4pt}} &
\multicolumn{1}{r!{\huxvb{0}}}{\huxtpad{4pt}\raggedleft {\fontsize{9.5pt}{11.4pt}\selectfont -0.096~}\huxbpad{4pt}} &
\multicolumn{1}{r!{\huxvb{0}}}{\huxtpad{4pt}\raggedleft {\fontsize{9.5pt}{11.4pt}\selectfont -0.027~}\huxbpad{4pt}} \tabularnewline[-0.5pt]


\hhline{}
\arrayrulecolor{black}

\multicolumn{1}{!{\huxvb{0}}l!{\huxvb{0}}}{\huxtpad{4pt}\raggedright {\fontsize{9.5pt}{11.4pt}\selectfont }\huxbpad{4pt}} &
\multicolumn{1}{r!{\huxvb{0}}}{\huxtpad{4pt}\raggedleft {\fontsize{9.5pt}{11.4pt}\selectfont (0.301)~}\huxbpad{4pt}} &
\multicolumn{1}{r!{\huxvb{0}}}{\huxtpad{4pt}\raggedleft {\fontsize{9.5pt}{11.4pt}\selectfont (0.390)}\huxbpad{4pt}} &
\multicolumn{1}{r!{\huxvb{0}}}{\huxtpad{4pt}\raggedleft {\fontsize{9.5pt}{11.4pt}\selectfont (0.399)}\huxbpad{4pt}} \tabularnewline[-0.5pt]


\hhline{}
\arrayrulecolor{black}

\multicolumn{1}{!{\huxvb{0}}l!{\huxvb{0}}}{\huxtpad{4pt}\raggedright {\fontsize{9.5pt}{11.4pt}\selectfont Married}\huxbpad{4pt}} &
\multicolumn{1}{r!{\huxvb{0}}}{\huxtpad{4pt}\raggedleft {\fontsize{9.5pt}{11.4pt}\selectfont -0.394~~}\huxbpad{4pt}} &
\multicolumn{1}{r!{\huxvb{0}}}{\huxtpad{4pt}\raggedleft {\fontsize{9.5pt}{11.4pt}\selectfont 0.319~}\huxbpad{4pt}} &
\multicolumn{1}{r!{\huxvb{0}}}{\huxtpad{4pt}\raggedleft {\fontsize{9.5pt}{11.4pt}\selectfont 0.606~}\huxbpad{4pt}} \tabularnewline[-0.5pt]


\hhline{}
\arrayrulecolor{black}

\multicolumn{1}{!{\huxvb{0}}l!{\huxvb{0}}}{\huxtpad{4pt}\raggedright {\fontsize{9.5pt}{11.4pt}\selectfont }\huxbpad{4pt}} &
\multicolumn{1}{r!{\huxvb{0}}}{\huxtpad{4pt}\raggedleft {\fontsize{9.5pt}{11.4pt}\selectfont (0.301)~}\huxbpad{4pt}} &
\multicolumn{1}{r!{\huxvb{0}}}{\huxtpad{4pt}\raggedleft {\fontsize{9.5pt}{11.4pt}\selectfont (0.344)}\huxbpad{4pt}} &
\multicolumn{1}{r!{\huxvb{0}}}{\huxtpad{4pt}\raggedleft {\fontsize{9.5pt}{11.4pt}\selectfont (0.346)}\huxbpad{4pt}} \tabularnewline[-0.5pt]


\hhline{}
\arrayrulecolor{black}

\multicolumn{1}{!{\huxvb{0}}l!{\huxvb{0}}}{\huxtpad{4pt}\raggedright {\fontsize{9.5pt}{11.4pt}\selectfont Socioeconomic}\huxbpad{4pt}} &
\multicolumn{1}{r!{\huxvb{0}}}{\huxtpad{4pt}\raggedleft {\fontsize{9.5pt}{11.4pt}\selectfont ~~~~~~}\huxbpad{4pt}} &
\multicolumn{1}{r!{\huxvb{0}}}{\huxtpad{4pt}\raggedleft {\fontsize{9.5pt}{11.4pt}\selectfont ~~~~~}\huxbpad{4pt}} &
\multicolumn{1}{r!{\huxvb{0}}}{\huxtpad{4pt}\raggedleft {\fontsize{9.5pt}{11.4pt}\selectfont ~~~~~}\huxbpad{4pt}} \tabularnewline[-0.5pt]


\hhline{}
\arrayrulecolor{black}

\multicolumn{1}{!{\huxvb{0}}l!{\huxvb{0}}}{\huxtpad{4pt}\raggedright {\fontsize{9.5pt}{11.4pt}\selectfont $<$H.S.}\huxbpad{4pt}} &
\multicolumn{1}{r!{\huxvb{0}}}{\huxtpad{4pt}\raggedleft {\fontsize{9.5pt}{11.4pt}\selectfont 2.714 *}\huxbpad{4pt}} &
\multicolumn{1}{r!{\huxvb{0}}}{\huxtpad{4pt}\raggedleft {\fontsize{9.5pt}{11.4pt}\selectfont -0.336~}\huxbpad{4pt}} &
\multicolumn{1}{r!{\huxvb{0}}}{\huxtpad{4pt}\raggedleft {\fontsize{9.5pt}{11.4pt}\selectfont -1.387~}\huxbpad{4pt}} \tabularnewline[-0.5pt]


\hhline{}
\arrayrulecolor{black}

\multicolumn{1}{!{\huxvb{0}}l!{\huxvb{0}}}{\huxtpad{4pt}\raggedright {\fontsize{9.5pt}{11.4pt}\selectfont }\huxbpad{4pt}} &
\multicolumn{1}{r!{\huxvb{0}}}{\huxtpad{4pt}\raggedleft {\fontsize{9.5pt}{11.4pt}\selectfont (1.058)~}\huxbpad{4pt}} &
\multicolumn{1}{r!{\huxvb{0}}}{\huxtpad{4pt}\raggedleft {\fontsize{9.5pt}{11.4pt}\selectfont (1.041)}\huxbpad{4pt}} &
\multicolumn{1}{r!{\huxvb{0}}}{\huxtpad{4pt}\raggedleft {\fontsize{9.5pt}{11.4pt}\selectfont (1.092)}\huxbpad{4pt}} \tabularnewline[-0.5pt]


\hhline{}
\arrayrulecolor{black}

\multicolumn{1}{!{\huxvb{0}}l!{\huxvb{0}}}{\huxtpad{4pt}\raggedright {\fontsize{9.5pt}{11.4pt}\selectfont Some college, no B.A.}\huxbpad{4pt}} &
\multicolumn{1}{r!{\huxvb{0}}}{\huxtpad{4pt}\raggedleft {\fontsize{9.5pt}{11.4pt}\selectfont 1.344 *}\huxbpad{4pt}} &
\multicolumn{1}{r!{\huxvb{0}}}{\huxtpad{4pt}\raggedleft {\fontsize{9.5pt}{11.4pt}\selectfont 0.303~}\huxbpad{4pt}} &
\multicolumn{1}{r!{\huxvb{0}}}{\huxtpad{4pt}\raggedleft {\fontsize{9.5pt}{11.4pt}\selectfont -0.228~}\huxbpad{4pt}} \tabularnewline[-0.5pt]


\hhline{}
\arrayrulecolor{black}

\multicolumn{1}{!{\huxvb{0}}l!{\huxvb{0}}}{\huxtpad{4pt}\raggedright {\fontsize{9.5pt}{11.4pt}\selectfont }\huxbpad{4pt}} &
\multicolumn{1}{r!{\huxvb{0}}}{\huxtpad{4pt}\raggedleft {\fontsize{9.5pt}{11.4pt}\selectfont (0.567)~}\huxbpad{4pt}} &
\multicolumn{1}{r!{\huxvb{0}}}{\huxtpad{4pt}\raggedleft {\fontsize{9.5pt}{11.4pt}\selectfont (0.654)}\huxbpad{4pt}} &
\multicolumn{1}{r!{\huxvb{0}}}{\huxtpad{4pt}\raggedleft {\fontsize{9.5pt}{11.4pt}\selectfont (0.671)}\huxbpad{4pt}} \tabularnewline[-0.5pt]


\hhline{}
\arrayrulecolor{black}

\multicolumn{1}{!{\huxvb{0}}l!{\huxvb{0}}}{\huxtpad{4pt}\raggedright {\fontsize{9.5pt}{11.4pt}\selectfont B.A.}\huxbpad{4pt}} &
\multicolumn{1}{r!{\huxvb{0}}}{\huxtpad{4pt}\raggedleft {\fontsize{9.5pt}{11.4pt}\selectfont 0.935~~}\huxbpad{4pt}} &
\multicolumn{1}{r!{\huxvb{0}}}{\huxtpad{4pt}\raggedleft {\fontsize{9.5pt}{11.4pt}\selectfont 0.157~}\huxbpad{4pt}} &
\multicolumn{1}{r!{\huxvb{0}}}{\huxtpad{4pt}\raggedleft {\fontsize{9.5pt}{11.4pt}\selectfont -0.301~}\huxbpad{4pt}} \tabularnewline[-0.5pt]


\hhline{}
\arrayrulecolor{black}

\multicolumn{1}{!{\huxvb{0}}l!{\huxvb{0}}}{\huxtpad{4pt}\raggedright {\fontsize{9.5pt}{11.4pt}\selectfont }\huxbpad{4pt}} &
\multicolumn{1}{r!{\huxvb{0}}}{\huxtpad{4pt}\raggedleft {\fontsize{9.5pt}{11.4pt}\selectfont (0.550)~}\huxbpad{4pt}} &
\multicolumn{1}{r!{\huxvb{0}}}{\huxtpad{4pt}\raggedleft {\fontsize{9.5pt}{11.4pt}\selectfont (0.628)}\huxbpad{4pt}} &
\multicolumn{1}{r!{\huxvb{0}}}{\huxtpad{4pt}\raggedleft {\fontsize{9.5pt}{11.4pt}\selectfont (0.640)}\huxbpad{4pt}} \tabularnewline[-0.5pt]


\hhline{}
\arrayrulecolor{black}

\multicolumn{1}{!{\huxvb{0}}l!{\huxvb{0}}}{\huxtpad{4pt}\raggedright {\fontsize{9.5pt}{11.4pt}\selectfont M.A.+}\huxbpad{4pt}} &
\multicolumn{1}{r!{\huxvb{0}}}{\huxtpad{4pt}\raggedleft {\fontsize{9.5pt}{11.4pt}\selectfont 1.283 *}\huxbpad{4pt}} &
\multicolumn{1}{r!{\huxvb{0}}}{\huxtpad{4pt}\raggedleft {\fontsize{9.5pt}{11.4pt}\selectfont 0.692~}\huxbpad{4pt}} &
\multicolumn{1}{r!{\huxvb{0}}}{\huxtpad{4pt}\raggedleft {\fontsize{9.5pt}{11.4pt}\selectfont -0.442~}\huxbpad{4pt}} \tabularnewline[-0.5pt]


\hhline{}
\arrayrulecolor{black}

\multicolumn{1}{!{\huxvb{0}}l!{\huxvb{0}}}{\huxtpad{4pt}\raggedright {\fontsize{9.5pt}{11.4pt}\selectfont }\huxbpad{4pt}} &
\multicolumn{1}{r!{\huxvb{0}}}{\huxtpad{4pt}\raggedleft {\fontsize{9.5pt}{11.4pt}\selectfont (0.549)~}\huxbpad{4pt}} &
\multicolumn{1}{r!{\huxvb{0}}}{\huxtpad{4pt}\raggedleft {\fontsize{9.5pt}{11.4pt}\selectfont (0.640)}\huxbpad{4pt}} &
\multicolumn{1}{r!{\huxvb{0}}}{\huxtpad{4pt}\raggedleft {\fontsize{9.5pt}{11.4pt}\selectfont (0.653)}\huxbpad{4pt}} \tabularnewline[-0.5pt]


\hhline{>{\huxb{1}}->{\huxb{1}}->{\huxb{1}}->{\huxb{1}}-}
\arrayrulecolor{black}
\end{tabularx}
\end{table}


% latex table generated in R 3.5.0 by xtable 1.8-2 package
% Tue Nov 20 14:10:28 2018
\begin{sidewaystable}[ht]
\centering
\caption{Estimated race coefficients for willingness to consider communities} 
\label{tab:consider}
\begin{tabular}{lrlrlrlrlrlrlrlrl}
  \toprule
  & \multicolumn{8}{c}{Race only} & \multicolumn{8}{c}{With controls}\\
 & Intercept && Asian && Black && Latino&& Intercept && Asian && Black && Latino&\\
 \midrule
Columbia Heights & -2.212 & *** & -1.607 & ** & -0.976 & * & -0.911 &  & -3.275 & *** & -1.083 &  & -0.390 &  & -0.183 &  \\ 
  Brightwood & -4.762 & *** & -14.829 & *** & 1.886 & * & 0.879 &  & -7.127 & *** & -14.582 & *** & 2.271 & * & 1.778 &  \\ 
  Langley Park & -4.658 & *** & 1.282 &  & 1.066 &  & 1.093 &  & -6.652 & *** & 1.606 &  & 1.151 &  & 1.377 &  \\ 
  Hyattsville & -3.526 & *** & -0.001 &  & 0.128 &  & 0.182 &  & -4.272 & *** & 0.292 &  & 0.224 &  & 0.446 &  \\ 
  Greenbelt & -3.520 & *** & -1.045 &  & 1.307 & *** & 0.739 &  & -4.082 & *** & -0.412 &  & 1.496 & ** & 1.234 & * \\ 
  Wheaton & -3.706 & *** & 0.387 &  & 1.346 & ** & 1.691 & *** & -4.135 & *** & 0.614 &  & 1.329 & ** & 1.630 & ** \\ 
  Germantown & -2.634 & *** & -0.045 &  & -0.174 &  & 0.670 &  & -3.113 & *** & 0.167 &  & -0.130 &  & 0.920 &  \\ 
  Arlington & -1.307 & *** & -0.510 &  & -0.301 &  & -0.083 &  & -1.989 & *** & -0.194 &  & 0.044 &  & 0.369 &  \\ 
  Annandale & -2.278 & *** & -0.731 &  & -0.230 &  & -0.791 &  & -2.464 & *** & -0.706 &  & -0.340 &  & -0.807 &  \\ 
  Huntington & -3.592 & *** & -0.454 &  & 0.848 &  & 0.141 &  & -4.956 & *** & 0.385 &  & 1.401 & * & 0.859 &  \\ 
  Herndon & -2.125 & *** & -0.572 &  & -0.317 &  & -0.178 &  & -2.367 & *** & -0.502 &  & -0.161 &  & -0.079 &  \\ 
   \bottomrule
\end{tabular}
\end{sidewaystable}


% latex table generated in R 3.5.0 by xtable 1.8-2 package
% Wed Nov 21 14:01:40 2018
\begin{sidewaystable}[ht]
\centering
\caption{Estimated race coefficients for not considering communities} 
\label{tab:notconsider}
\begin{tabular}{lrlrlrlrlrlrlrlrl}
  \toprule
  & \multicolumn{8}{c}{Race only} & \multicolumn{8}{c}{With controls}\\
 & Intercept && Asian && Black && Latino&& Intercept && Asian && Black && Latino&\\
 \midrule
Columbia Heights & -0.977 & *** & -0.978 & *** & -0.731 & ** & -0.279 &  & -1.201 & *** & -0.707 & * & -0.469 &  & 0.245 &  \\ 
  Brightwood & -1.414 & *** & -0.650 & * & -0.616 & * & -0.098 &  & -1.461 & *** & -0.548 &  & -0.528 &  & 0.186 &  \\ 
  Langley Park & -0.470 & ** & -0.895 & ** & -0.280 &  & 0.392 &  & -0.581 & *** & -0.841 & * & -0.087 &  & 0.738 & ** \\ 
  Hyattsville & -0.580 & *** & -0.800 & ** & -0.659 & ** & -0.096 &  & -0.798 & *** & -0.816 & * & -0.377 &  & 0.258 &  \\ 
  Greenbelt & -0.544 & *** & -0.867 & ** & -0.932 & *** & -0.172 &  & -0.839 & *** & -0.764 & * & -0.646 & * & 0.301 &  \\ 
  Wheaton & -0.640 & *** & -1.240 & *** & -0.548 & * & -0.440 &  & -0.921 & *** & -1.232 & *** & -0.235 &  & -0.070 &  \\ 
  Germantown & -0.868 & *** & -1.259 & *** & -0.782 & ** & -0.557 & * & -1.043 & *** & -1.100 & ** & -0.631 & * & -0.282 &  \\ 
  Arlington & -1.507 & *** & -1.096 & ** & -0.284 &  & -0.487 &  & -1.459 & *** & -1.167 & ** & -0.346 &  & -0.454 &  \\ 
  Annandale & -1.141 & *** & -0.539 &  & -0.571 & * & -0.526 &  & -1.279 & *** & -0.601 &  & -0.471 &  & -0.296 &  \\ 
  Huntington & -0.823 & *** & -1.298 & *** & -0.795 & ** & -0.338 &  & -1.006 & *** & -1.121 & ** & -0.659 & * & -0.045 &  \\ 
  Herndon & -0.877 & *** & -0.946 & ** & -0.533 & * & -0.597 & * & -1.170 & *** & -0.641 &  & -0.318 &  & -0.176 &  \\ 
   \bottomrule
\end{tabular}
\end{sidewaystable}



\clearpage
\section{Figures}

\begin{figure}[h!]
\caption{Multiracial and disproportionately Latino neighborhoods in the DC Area and communities asked about on the DCAS 2016}
\label{fig:map}
\centering
\includegraphics[scale=.45]{../analysis/images/DCASCommunityMap.png}
\end{figure}

\begin{sidewaysfigure}
\caption{Predicted probabilities of being satisfied in current neighborhood, by race}
\label{fig:satisfaction}
\centering
\includegraphics[scale=0.3]{../analysis/images/satisfied.png}
\end{sidewaysfigure}

\begin{sidewaysfigure}
\caption{Predicted probabilities of perceiving improvement in neighborhood, by race}
\label{fig:improvement}
\centering
\includegraphics[scale=0.3]{../analysis/images/improvement.png}
\end{sidewaysfigure}

\end{document}

