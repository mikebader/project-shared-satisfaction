\documentclass[11pt]{baderart}
\usepackage{array}
\usepackage{caption}
\usepackage{graphicx}
\usepackage{siunitx}
\usepackage{multirow}
\usepackage{hhline}
\usepackage{calc}
\usepackage{tabularx}

\newcommand{\TK}{\textbf{\{TK\}}}



\title{Multiethnic Neighborhoods}
\author{Michael D.M.~Bader}

\begin{document}
\maketitle

\noindent Racially integrated neighborhoods have become more numerous in the landscape of American metropolitan areas. While they were once unstable rarities, integrated neighborhoods have become both more common and more stable \needcite. Previous studies estimate \TK\ to \TK\ percent of metropolitan neighborhoods to be racially integrated, depending on the method used to classify integration. More tolerant racial attitudes among whites, preferences for racially integrated neighborhoods among people of color, and federal policies outlawing discrimination have helped the number of multiethnic neighborhoods grow \needcite. Integration has been bolstered by the expanding diversity of U.S.\ population, especially from the immigration and natural growth of Latinos and Asian-Americans, have increased. 

Maintaining integration over time depends on residents with different racial identities being satisfied living in multiracial neighborhoods. Differences in neighborhood satisfaction across racial groups would lead to some groups being likely to leave the neighborhood, thereby imperiling the racial composition and could set neighborhoods on a trajectory toward integration. Despite the relevance of this question for predicting the long-term stability of racial integration, we do not know whether satisfaction with integrated neighborhoods varies across racial groups. 

This article addresses the shortcoming of previous research by studying levels of satisfaction among residents of integrated neighborhoods. Unlike previous research that focuses on a handful of integrated neighborhoods, this article examines neighborhood satisfaction using data from a large representative sample of residents living in integrated neighborhoods in a metropolitan area. The results reveal few racial differences in how satisfied residents feel living in integrated neighborhoods and in how they perceive the direction of the neighborhood. Yet, whites are less likely to consider moving to one of the other integrated neighborhoods in the metropolitan area, possibly reducing the chances that integration can be maintained over time. 

\section{Rise of Multiethnic Neighborhoods}
The human ecological model of neighborhood change developed by Park and colleagues (\needcite) applied for most of the twentieth century. One or two households of a racial or ethnic group would ``invade'' a neighborhood where the majority of residents identified as another group. After this initial invasion, change would proceed rapidly as the incumbents would leave and the more members of the new racial group would enter. The model explained racial change associated with white flight particularly well as neighborhoods would change rapidly once the mobility pattern took hold \needcite. 

The pattern of rapid succession, however, slowed during the 1980s and 1990s. White residents tended not to flee neighborhoods as people of color moved into their neighborhoods. Ellen \needcite documented the rise of durable racial integration and the possibility that racial segregation should not be a foregone conclusion. Contrary to the hypothesis of invasion and succession, she showed that racial change occurred very slowly in integrated neighborhoods. Since Ellen's work, several studies have documented the fact that integrated neighborhoods represent a plurality of American neighborhoods. Integration became probable a few decades after many questioned whether it was even possible. 

Several factors led to the growth of integrated neighborhoods. First, federal, state, and local civil rights legislation provided a legal basis for people of color, especially blacks, to enter previously all-white neighborhoods. Second, the American population became more diverse as immigration led to more people of color living in the United States. Third, the rising economic position among people of color (supported by civil rights legislation against discrimination) allowed more people of color to enter neighborhoods that had been previously unavailable. Finally, whites have become more racially tolerant of being neighbors with people of color. 

These forces led to the growth of two specific types of racially integrated neighborhoods. The first are ``quadrivial neighborhoods'' in which white, black, Latino, and Asian residents live in proportion to their representation in the overall population.
\footnote{Logan and Zhang \needcite\ called these neighborhoods ``global neighborhoods.'' I prefer to follow Bader and Warkentien \needcite\ who used the term ``quadrivial'' neighborhoods, who used the Latin term meaning four roads coming to an intersection. The term ``global'' implies non-white residents of the neighborhood are foreign born rather than being as American as white residents.}
Estimates regarding the number of quadrivial neighborhoods range from an estimate of fifteen percent to \TK\ based on Logan and Zhang \needcite. 

The type comprises neighborhoods in which the Latinx population has grown but a substantial number of whites still live in the neighborhood. Evidence shows that this type of neighborhood makes up the majority or a substantial plurality of integrated neighborhoods. These neighborhoods might include other racial groups, more often Asian than black, at smaller percentages of the composition \needcite. While integrated, evidence suggests integration as a temporary phenomenon. Farrell and Lee \needcite\ show that Latinos are simultaneously the group most likely to integrate a racially homogenous neighborhood \emph{and} experience increasing isolation. Bader and Warkentien \needcite\ estimate the rate of change in neighborhoods in which whites and Latinos make up the two largest groups in the population, and find that the neighborhoods are on pace to resegregate as all-Latino neighborhoods.

\section{Satisfaction Living in Multiethnic Neighborhoods}
Research on life in multiethnic neighborhoods has not kept pace with the increasing prevalence of multiethnic neighborhoods in American metropolitan areas. Among the studies that exist, most find that residents are satisfied living in multiracial neighborhoods \needcite. Most of the research on integrated neighborhoods comes from case-studies of residents integrated neighborhoods. 

Some case studies focus on multiethnic neighborhoods in central cities like Roger's Park in Chicago or Jackson Heights in Queens. \textbf{TK: more on specific studies about integrated neighborhoods}. Another source of case studies come from inner ring suburbs that have taken proactive steps to maintain explicit integration. Places like Oak Park, Illinois, Shaker Heights, Ohio, and Mount Airy, Pennsylvania offer case studies for how to actively manage and maintain integration. 

While useful, the cases presented in this research do not reflect where multiethnic neighborhoods exist today. The most diverse neighborhoods, the quadrivial neighborhoods, tend to be found in the middle-ring suburbs of American metropolitan areas. These were places where a majority of the housing stock tended to be built after the Fair Housing Act passed in 1968 and located among major transportation corridors. Integration is more often found in places like Naperville, Illnois (outside of Chicago), Sugar Land, Texas (outside of Houston), Germantown, Maryland (outside of Washington, D.C.) than in the well-known central city and activist suburban neighborhoods. 

The same is true of disproportionately Latino neighborhoods. While popular imagination and research alike tend to focus on central city ethnic enclaves, immigrants now often bypass those neighborhoods to move to suburban enclaves. Native-born Latinos, especially those who have experienced upward mobility, have dispersed throughout metropolitan areas. 

While case studies of specific neighborhoods provide valuable insights, they do not capture the contemporary conditions of racially integrated neighborhoods. We need data from residents of to understand the future of integration in U.S.\ society. Knowing whether racial groups derive different levels of satisfaction from living in integrated neighborhoods will help ascertain whether groups feel socially integrated and predict how stable integration will be in multiracial neighborhoods. This study uses data from a representative sample of residents living in multiracial neighborhoods in the Washington, D.C.\ metropolitan area. It investigates whether neighborhood satisfaction perceptions of neighborhood change differ across racial groups. 

The study also investigates the willingness to live in \emph{other} integrated neighborhoods in the Washington, D.C.\ metropolitan area among people living in integrated neighborhoods. This latter research aim provides a glimpse into the durability of integration. All neighborhoods undergo change through people moving out of and into the neighborhood. The racial composition of people who move into the neighborhood will have to roughly represent the racial composition of the metropolitan area as a whole to maintain integration into the future. 

Previous research would suggest that white preferences play a disproportionate role in the process of maintaining integration. While people of color have expressed a desire to live in integrated neighborhoods and have moved in substantial numbers to previous all-white neighborhoods, whites neither express nor move to non-white neighborhoods in more than token numbers. Previous research has not, to date, focused either on the residents of multiethnic neighborhoods nor preferences for multiethnic neighborhoods. This study will provide the first systematic evidence regarding the long-term viability of multiracial neighborhoods across an entire metropolitan area. 

\section{Data and Methods}
\subsection{Setting}
This study uses data from the Washington, D.C. area. It is a large city with high levels of segregation, like many other cities. 

It differs, however, in its overall wealth, the size of its black middle class and presence of immigrants with high and low levels of education. 

Description of quadrivial and disproportionately Latiino neighborhoods in the DC area

\subsection{Sample}
The 2016 D.C.\ Area Study was conducted in March-May 2016. Multi-stage design that selected households in Census tracts in the District of Columbia and neighboring jurisdictions that fell into one of two categories: quadrivial neighborhoods or disproportionately Latino neighborhoods. (Defns) Household PAPI survey mailed to households in equal proportion across the two neighborhoods. 

Survey weights were created so that the survey would represent all residents of quadrivial and disproportionately Latino neighborhoods in the D.C.\ area.

\subsection{Dependent Variables} 


\subsection{Independent and Control Variables}

\subsection{Method}

\section{Results}










\section{Tables}
% latex table generated in R 3.5.0 by xtable 1.8-2 package
% Sun Nov 18 21:43:01 2018
\begin{table}[ht]
\centering
\caption{Un-weighted means and standard deviations of independent and control variables} 
\label{tab:descriptives}
\begin{tabular}{lp{.5in}p{.5in}}
  \toprule
Variable & Mean & S.D. \\ 
  \midrule
\emph{Race}&&\\Asian or Pacific Islander & 0.24 &  \\ 
  Black & 0.22 &  \\ 
  Latinx & 0.19 &  \\ 
  White & 0.36 &  \\ 
  \emph{Demographics}&&\\Age & 54.59 & 16.28 \\ 
  Foreign Born & 0.45 &  \\ 
  Man & 0.47 &  \\ 
  Children present & 0.31 &  \\ 
  Married & 0.58 &  \\ 
  \emph{Educational Attainment}&&\\<H.S. & 0.07 &  \\ 
  H.S. & 0.11 &  \\ 
  Some college, no B.A. & 0.23 &  \\ 
  B.A. & 0.31 &  \\ 
  M.A.+ & 0.29 &  \\ 
  \emph{Income}&&\\$<$\$40,000 & 0.25 &  \\ 
  \$40,000 to $<$\$75,000 & 0.24 &  \\ 
  \$75,000 to $<$\$150,000 & 0.36 &  \\ 
  \$150,000+ & 0.16 &  \\ 
  \emph{Neighborhood Experience}\\Years in Neighborhood & 14.74 & 12.82 \\ 
  1-9 blocks & 0.6 &  \\ 
  10-50 blocks & 0.33 &  \\ 
  >50 blocks & 0.05 &  \\ 
  Quadrivial Neighborhood & 0.55 &  \\ 
   \bottomrule
\end{tabular}
\end{table}

\begin{table}[h]
\centering\captionsetup{justification=centering,singlelinecheck=off}
\caption{Estimated coefficients predicting  neighborhood satisfaction}
\label{tab:satisfaction}

    \providecommand{\huxb}[2][0,0,0]{\arrayrulecolor[RGB]{#1}\global\arrayrulewidth=#2pt}
    \providecommand{\huxvb}[2][0,0,0]{\color[RGB]{#1}\vrule width #2pt}
    \providecommand{\huxtpad}[1]{\rule{0pt}{\baselineskip+#1}}
    \providecommand{\huxbpad}[1]{\rule[-#1]{0pt}{#1}}
  \begin{tabularx}{0.5\textwidth}{p{0.125\textwidth} p{0.125\textwidth} p{0.125\textwidth} p{0.125\textwidth}}


\hhline{>{\huxb{0.8}}->{\huxb{0.8}}->{\huxb{0.8}}->{\huxb{0.8}}-}
\arrayrulecolor{black}

\multicolumn{1}{!{\huxvb{0}}c!{\huxvb{0}}}{\huxtpad{4pt}\centering {\fontsize{9.5pt}{11.4pt}\selectfont }\huxbpad{4pt}} &
\multicolumn{1}{c!{\huxvb{0}}}{\huxtpad{4pt}\centering {\fontsize{9.5pt}{11.4pt}\selectfont (1)}\huxbpad{4pt}} &
\multicolumn{1}{c!{\huxvb{0}}}{\huxtpad{4pt}\centering {\fontsize{9.5pt}{11.4pt}\selectfont (2)}\huxbpad{4pt}} &
\multicolumn{1}{c!{\huxvb{0}}}{\huxtpad{4pt}\centering {\fontsize{9.5pt}{11.4pt}\selectfont (3)}\huxbpad{4pt}} \tabularnewline[-0.5pt]


\hhline{>{\huxb{1}}->{\huxb{1}}->{\huxb{1}}->{\huxb{1}}-}
\arrayrulecolor{black}

\multicolumn{1}{!{\huxvb{0}}l!{\huxvb{0}}}{\huxtpad{4pt}\raggedright {\fontsize{9.5pt}{11.4pt}\selectfont (Intercept)}\huxbpad{4pt}} &
\multicolumn{1}{r!{\huxvb{0}}}{\huxtpad{4pt}\raggedleft {\fontsize{9.5pt}{11.4pt}\selectfont 0.795 ***}\huxbpad{4pt}} &
\multicolumn{1}{r!{\huxvb{0}}}{\huxtpad{4pt}\raggedleft {\fontsize{9.5pt}{11.4pt}\selectfont 0.530~}\huxbpad{4pt}} &
\multicolumn{1}{r!{\huxvb{0}}}{\huxtpad{4pt}\raggedleft {\fontsize{9.5pt}{11.4pt}\selectfont 0.770~~}\huxbpad{4pt}} \tabularnewline[-0.5pt]


\hhline{}
\arrayrulecolor{black}

\multicolumn{1}{!{\huxvb{0}}l!{\huxvb{0}}}{\huxtpad{4pt}\raggedright {\fontsize{9.5pt}{11.4pt}\selectfont }\huxbpad{4pt}} &
\multicolumn{1}{r!{\huxvb{0}}}{\huxtpad{4pt}\raggedleft {\fontsize{9.5pt}{11.4pt}\selectfont (0.185)~~~}\huxbpad{4pt}} &
\multicolumn{1}{r!{\huxvb{0}}}{\huxtpad{4pt}\raggedleft {\fontsize{9.5pt}{11.4pt}\selectfont (0.511)}\huxbpad{4pt}} &
\multicolumn{1}{r!{\huxvb{0}}}{\huxtpad{4pt}\raggedleft {\fontsize{9.5pt}{11.4pt}\selectfont (0.549)~}\huxbpad{4pt}} \tabularnewline[-0.5pt]


\hhline{}
\arrayrulecolor{black}

\multicolumn{1}{!{\huxvb{0}}l!{\huxvb{0}}}{\huxtpad{4pt}\raggedright {\fontsize{9.5pt}{11.4pt}\selectfont Race}\huxbpad{4pt}} &
\multicolumn{1}{r!{\huxvb{0}}}{\huxtpad{4pt}\raggedleft {\fontsize{9.5pt}{11.4pt}\selectfont ~~~~~~~~}\huxbpad{4pt}} &
\multicolumn{1}{r!{\huxvb{0}}}{\huxtpad{4pt}\raggedleft {\fontsize{9.5pt}{11.4pt}\selectfont ~~~~~}\huxbpad{4pt}} &
\multicolumn{1}{r!{\huxvb{0}}}{\huxtpad{4pt}\raggedleft {\fontsize{9.5pt}{11.4pt}\selectfont ~~~~~~}\huxbpad{4pt}} \tabularnewline[-0.5pt]


\hhline{}
\arrayrulecolor{black}

\multicolumn{1}{!{\huxvb{0}}l!{\huxvb{0}}}{\huxtpad{4pt}\raggedright {\fontsize{9.5pt}{11.4pt}\selectfont Asian}\huxbpad{4pt}} &
\multicolumn{1}{r!{\huxvb{0}}}{\huxtpad{4pt}\raggedleft {\fontsize{9.5pt}{11.4pt}\selectfont 0.136~~~~}\huxbpad{4pt}} &
\multicolumn{1}{r!{\huxvb{0}}}{\huxtpad{4pt}\raggedleft {\fontsize{9.5pt}{11.4pt}\selectfont 0.071~}\huxbpad{4pt}} &
\multicolumn{1}{r!{\huxvb{0}}}{\huxtpad{4pt}\raggedleft {\fontsize{9.5pt}{11.4pt}\selectfont 0.143~~}\huxbpad{4pt}} \tabularnewline[-0.5pt]


\hhline{}
\arrayrulecolor{black}

\multicolumn{1}{!{\huxvb{0}}l!{\huxvb{0}}}{\huxtpad{4pt}\raggedright {\fontsize{9.5pt}{11.4pt}\selectfont }\huxbpad{4pt}} &
\multicolumn{1}{r!{\huxvb{0}}}{\huxtpad{4pt}\raggedleft {\fontsize{9.5pt}{11.4pt}\selectfont (0.305)~~~}\huxbpad{4pt}} &
\multicolumn{1}{r!{\huxvb{0}}}{\huxtpad{4pt}\raggedleft {\fontsize{9.5pt}{11.4pt}\selectfont (0.386)}\huxbpad{4pt}} &
\multicolumn{1}{r!{\huxvb{0}}}{\huxtpad{4pt}\raggedleft {\fontsize{9.5pt}{11.4pt}\selectfont (0.392)~}\huxbpad{4pt}} \tabularnewline[-0.5pt]


\hhline{}
\arrayrulecolor{black}

\multicolumn{1}{!{\huxvb{0}}l!{\huxvb{0}}}{\huxtpad{4pt}\raggedright {\fontsize{9.5pt}{11.4pt}\selectfont Black}\huxbpad{4pt}} &
\multicolumn{1}{r!{\huxvb{0}}}{\huxtpad{4pt}\raggedleft {\fontsize{9.5pt}{11.4pt}\selectfont -0.095~~~~}\huxbpad{4pt}} &
\multicolumn{1}{r!{\huxvb{0}}}{\huxtpad{4pt}\raggedleft {\fontsize{9.5pt}{11.4pt}\selectfont -0.107~}\huxbpad{4pt}} &
\multicolumn{1}{r!{\huxvb{0}}}{\huxtpad{4pt}\raggedleft {\fontsize{9.5pt}{11.4pt}\selectfont -0.168~~}\huxbpad{4pt}} \tabularnewline[-0.5pt]


\hhline{}
\arrayrulecolor{black}

\multicolumn{1}{!{\huxvb{0}}l!{\huxvb{0}}}{\huxtpad{4pt}\raggedright {\fontsize{9.5pt}{11.4pt}\selectfont }\huxbpad{4pt}} &
\multicolumn{1}{r!{\huxvb{0}}}{\huxtpad{4pt}\raggedleft {\fontsize{9.5pt}{11.4pt}\selectfont (0.326)~~~}\huxbpad{4pt}} &
\multicolumn{1}{r!{\huxvb{0}}}{\huxtpad{4pt}\raggedleft {\fontsize{9.5pt}{11.4pt}\selectfont (0.334)}\huxbpad{4pt}} &
\multicolumn{1}{r!{\huxvb{0}}}{\huxtpad{4pt}\raggedleft {\fontsize{9.5pt}{11.4pt}\selectfont (0.347)~}\huxbpad{4pt}} \tabularnewline[-0.5pt]


\hhline{}
\arrayrulecolor{black}

\multicolumn{1}{!{\huxvb{0}}l!{\huxvb{0}}}{\huxtpad{4pt}\raggedright {\fontsize{9.5pt}{11.4pt}\selectfont Latinx}\huxbpad{4pt}} &
\multicolumn{1}{r!{\huxvb{0}}}{\huxtpad{4pt}\raggedleft {\fontsize{9.5pt}{11.4pt}\selectfont 0.344~~~~}\huxbpad{4pt}} &
\multicolumn{1}{r!{\huxvb{0}}}{\huxtpad{4pt}\raggedleft {\fontsize{9.5pt}{11.4pt}\selectfont 0.396~}\huxbpad{4pt}} &
\multicolumn{1}{r!{\huxvb{0}}}{\huxtpad{4pt}\raggedleft {\fontsize{9.5pt}{11.4pt}\selectfont 0.361~~}\huxbpad{4pt}} \tabularnewline[-0.5pt]


\hhline{}
\arrayrulecolor{black}

\multicolumn{1}{!{\huxvb{0}}l!{\huxvb{0}}}{\huxtpad{4pt}\raggedright {\fontsize{9.5pt}{11.4pt}\selectfont }\huxbpad{4pt}} &
\multicolumn{1}{r!{\huxvb{0}}}{\huxtpad{4pt}\raggedleft {\fontsize{9.5pt}{11.4pt}\selectfont (0.356)~~~}\huxbpad{4pt}} &
\multicolumn{1}{r!{\huxvb{0}}}{\huxtpad{4pt}\raggedleft {\fontsize{9.5pt}{11.4pt}\selectfont (0.383)}\huxbpad{4pt}} &
\multicolumn{1}{r!{\huxvb{0}}}{\huxtpad{4pt}\raggedleft {\fontsize{9.5pt}{11.4pt}\selectfont (0.406)~}\huxbpad{4pt}} \tabularnewline[-0.5pt]


\hhline{}
\arrayrulecolor{black}

\multicolumn{1}{!{\huxvb{0}}l!{\huxvb{0}}}{\huxtpad{4pt}\raggedright {\fontsize{9.5pt}{11.4pt}\selectfont Demographics}\huxbpad{4pt}} &
\multicolumn{1}{r!{\huxvb{0}}}{\huxtpad{4pt}\raggedleft {\fontsize{9.5pt}{11.4pt}\selectfont ~~~~~~~~}\huxbpad{4pt}} &
\multicolumn{1}{r!{\huxvb{0}}}{\huxtpad{4pt}\raggedleft {\fontsize{9.5pt}{11.4pt}\selectfont ~~~~~}\huxbpad{4pt}} &
\multicolumn{1}{r!{\huxvb{0}}}{\huxtpad{4pt}\raggedleft {\fontsize{9.5pt}{11.4pt}\selectfont ~~~~~~}\huxbpad{4pt}} \tabularnewline[-0.5pt]


\hhline{}
\arrayrulecolor{black}

\multicolumn{1}{!{\huxvb{0}}l!{\huxvb{0}}}{\huxtpad{4pt}\raggedright {\fontsize{9.5pt}{11.4pt}\selectfont Age}\huxbpad{4pt}} &
\multicolumn{1}{r!{\huxvb{0}}}{\huxtpad{4pt}\raggedleft {\fontsize{9.5pt}{11.4pt}\selectfont ~~~~~~~~}\huxbpad{4pt}} &
\multicolumn{1}{r!{\huxvb{0}}}{\huxtpad{4pt}\raggedleft {\fontsize{9.5pt}{11.4pt}\selectfont 0.004~}\huxbpad{4pt}} &
\multicolumn{1}{r!{\huxvb{0}}}{\huxtpad{4pt}\raggedleft {\fontsize{9.5pt}{11.4pt}\selectfont 0.005~~}\huxbpad{4pt}} \tabularnewline[-0.5pt]


\hhline{}
\arrayrulecolor{black}

\multicolumn{1}{!{\huxvb{0}}l!{\huxvb{0}}}{\huxtpad{4pt}\raggedright {\fontsize{9.5pt}{11.4pt}\selectfont }\huxbpad{4pt}} &
\multicolumn{1}{r!{\huxvb{0}}}{\huxtpad{4pt}\raggedleft {\fontsize{9.5pt}{11.4pt}\selectfont ~~~~~~~~}\huxbpad{4pt}} &
\multicolumn{1}{r!{\huxvb{0}}}{\huxtpad{4pt}\raggedleft {\fontsize{9.5pt}{11.4pt}\selectfont (0.007)}\huxbpad{4pt}} &
\multicolumn{1}{r!{\huxvb{0}}}{\huxtpad{4pt}\raggedleft {\fontsize{9.5pt}{11.4pt}\selectfont (0.009)~}\huxbpad{4pt}} \tabularnewline[-0.5pt]


\hhline{}
\arrayrulecolor{black}

\multicolumn{1}{!{\huxvb{0}}l!{\huxvb{0}}}{\huxtpad{4pt}\raggedright {\fontsize{9.5pt}{11.4pt}\selectfont Foreign Born}\huxbpad{4pt}} &
\multicolumn{1}{r!{\huxvb{0}}}{\huxtpad{4pt}\raggedleft {\fontsize{9.5pt}{11.4pt}\selectfont ~~~~~~~~}\huxbpad{4pt}} &
\multicolumn{1}{r!{\huxvb{0}}}{\huxtpad{4pt}\raggedleft {\fontsize{9.5pt}{11.4pt}\selectfont 0.085~}\huxbpad{4pt}} &
\multicolumn{1}{r!{\huxvb{0}}}{\huxtpad{4pt}\raggedleft {\fontsize{9.5pt}{11.4pt}\selectfont -0.013~~}\huxbpad{4pt}} \tabularnewline[-0.5pt]


\hhline{}
\arrayrulecolor{black}

\multicolumn{1}{!{\huxvb{0}}l!{\huxvb{0}}}{\huxtpad{4pt}\raggedright {\fontsize{9.5pt}{11.4pt}\selectfont }\huxbpad{4pt}} &
\multicolumn{1}{r!{\huxvb{0}}}{\huxtpad{4pt}\raggedleft {\fontsize{9.5pt}{11.4pt}\selectfont ~~~~~~~~}\huxbpad{4pt}} &
\multicolumn{1}{r!{\huxvb{0}}}{\huxtpad{4pt}\raggedleft {\fontsize{9.5pt}{11.4pt}\selectfont (0.319)}\huxbpad{4pt}} &
\multicolumn{1}{r!{\huxvb{0}}}{\huxtpad{4pt}\raggedleft {\fontsize{9.5pt}{11.4pt}\selectfont (0.335)~}\huxbpad{4pt}} \tabularnewline[-0.5pt]


\hhline{}
\arrayrulecolor{black}

\multicolumn{1}{!{\huxvb{0}}l!{\huxvb{0}}}{\huxtpad{4pt}\raggedright {\fontsize{9.5pt}{11.4pt}\selectfont Male}\huxbpad{4pt}} &
\multicolumn{1}{r!{\huxvb{0}}}{\huxtpad{4pt}\raggedleft {\fontsize{9.5pt}{11.4pt}\selectfont ~~~~~~~~}\huxbpad{4pt}} &
\multicolumn{1}{r!{\huxvb{0}}}{\huxtpad{4pt}\raggedleft {\fontsize{9.5pt}{11.4pt}\selectfont -0.009~}\huxbpad{4pt}} &
\multicolumn{1}{r!{\huxvb{0}}}{\huxtpad{4pt}\raggedleft {\fontsize{9.5pt}{11.4pt}\selectfont 0.069~~}\huxbpad{4pt}} \tabularnewline[-0.5pt]


\hhline{}
\arrayrulecolor{black}

\multicolumn{1}{!{\huxvb{0}}l!{\huxvb{0}}}{\huxtpad{4pt}\raggedright {\fontsize{9.5pt}{11.4pt}\selectfont }\huxbpad{4pt}} &
\multicolumn{1}{r!{\huxvb{0}}}{\huxtpad{4pt}\raggedleft {\fontsize{9.5pt}{11.4pt}\selectfont ~~~~~~~~}\huxbpad{4pt}} &
\multicolumn{1}{r!{\huxvb{0}}}{\huxtpad{4pt}\raggedleft {\fontsize{9.5pt}{11.4pt}\selectfont (0.249)}\huxbpad{4pt}} &
\multicolumn{1}{r!{\huxvb{0}}}{\huxtpad{4pt}\raggedleft {\fontsize{9.5pt}{11.4pt}\selectfont (0.252)~}\huxbpad{4pt}} \tabularnewline[-0.5pt]


\hhline{}
\arrayrulecolor{black}

\multicolumn{1}{!{\huxvb{0}}l!{\huxvb{0}}}{\huxtpad{4pt}\raggedright {\fontsize{9.5pt}{11.4pt}\selectfont Children Present}\huxbpad{4pt}} &
\multicolumn{1}{r!{\huxvb{0}}}{\huxtpad{4pt}\raggedleft {\fontsize{9.5pt}{11.4pt}\selectfont ~~~~~~~~}\huxbpad{4pt}} &
\multicolumn{1}{r!{\huxvb{0}}}{\huxtpad{4pt}\raggedleft {\fontsize{9.5pt}{11.4pt}\selectfont -0.290~}\huxbpad{4pt}} &
\multicolumn{1}{r!{\huxvb{0}}}{\huxtpad{4pt}\raggedleft {\fontsize{9.5pt}{11.4pt}\selectfont -0.313~~}\huxbpad{4pt}} \tabularnewline[-0.5pt]


\hhline{}
\arrayrulecolor{black}

\multicolumn{1}{!{\huxvb{0}}l!{\huxvb{0}}}{\huxtpad{4pt}\raggedright {\fontsize{9.5pt}{11.4pt}\selectfont }\huxbpad{4pt}} &
\multicolumn{1}{r!{\huxvb{0}}}{\huxtpad{4pt}\raggedleft {\fontsize{9.5pt}{11.4pt}\selectfont ~~~~~~~~}\huxbpad{4pt}} &
\multicolumn{1}{r!{\huxvb{0}}}{\huxtpad{4pt}\raggedleft {\fontsize{9.5pt}{11.4pt}\selectfont (0.274)}\huxbpad{4pt}} &
\multicolumn{1}{r!{\huxvb{0}}}{\huxtpad{4pt}\raggedleft {\fontsize{9.5pt}{11.4pt}\selectfont (0.278)~}\huxbpad{4pt}} \tabularnewline[-0.5pt]


\hhline{}
\arrayrulecolor{black}

\multicolumn{1}{!{\huxvb{0}}l!{\huxvb{0}}}{\huxtpad{4pt}\raggedright {\fontsize{9.5pt}{11.4pt}\selectfont Married}\huxbpad{4pt}} &
\multicolumn{1}{r!{\huxvb{0}}}{\huxtpad{4pt}\raggedleft {\fontsize{9.5pt}{11.4pt}\selectfont ~~~~~~~~}\huxbpad{4pt}} &
\multicolumn{1}{r!{\huxvb{0}}}{\huxtpad{4pt}\raggedleft {\fontsize{9.5pt}{11.4pt}\selectfont 0.245~}\huxbpad{4pt}} &
\multicolumn{1}{r!{\huxvb{0}}}{\huxtpad{4pt}\raggedleft {\fontsize{9.5pt}{11.4pt}\selectfont 0.248~~}\huxbpad{4pt}} \tabularnewline[-0.5pt]


\hhline{}
\arrayrulecolor{black}

\multicolumn{1}{!{\huxvb{0}}l!{\huxvb{0}}}{\huxtpad{4pt}\raggedright {\fontsize{9.5pt}{11.4pt}\selectfont }\huxbpad{4pt}} &
\multicolumn{1}{r!{\huxvb{0}}}{\huxtpad{4pt}\raggedleft {\fontsize{9.5pt}{11.4pt}\selectfont ~~~~~~~~}\huxbpad{4pt}} &
\multicolumn{1}{r!{\huxvb{0}}}{\huxtpad{4pt}\raggedleft {\fontsize{9.5pt}{11.4pt}\selectfont (0.270)}\huxbpad{4pt}} &
\multicolumn{1}{r!{\huxvb{0}}}{\huxtpad{4pt}\raggedleft {\fontsize{9.5pt}{11.4pt}\selectfont (0.272)~}\huxbpad{4pt}} \tabularnewline[-0.5pt]


\hhline{}
\arrayrulecolor{black}

\multicolumn{1}{!{\huxvb{0}}l!{\huxvb{0}}}{\huxtpad{4pt}\raggedright {\fontsize{9.5pt}{11.4pt}\selectfont Socioeconomic}\huxbpad{4pt}} &
\multicolumn{1}{r!{\huxvb{0}}}{\huxtpad{4pt}\raggedleft {\fontsize{9.5pt}{11.4pt}\selectfont ~~~~~~~~}\huxbpad{4pt}} &
\multicolumn{1}{r!{\huxvb{0}}}{\huxtpad{4pt}\raggedleft {\fontsize{9.5pt}{11.4pt}\selectfont ~~~~~}\huxbpad{4pt}} &
\multicolumn{1}{r!{\huxvb{0}}}{\huxtpad{4pt}\raggedleft {\fontsize{9.5pt}{11.4pt}\selectfont ~~~~~~}\huxbpad{4pt}} \tabularnewline[-0.5pt]


\hhline{}
\arrayrulecolor{black}

\multicolumn{1}{!{\huxvb{0}}l!{\huxvb{0}}}{\huxtpad{4pt}\raggedright {\fontsize{9.5pt}{11.4pt}\selectfont $<$H.S.}\huxbpad{4pt}} &
\multicolumn{1}{r!{\huxvb{0}}}{\huxtpad{4pt}\raggedleft {\fontsize{9.5pt}{11.4pt}\selectfont ~~~~~~~~}\huxbpad{4pt}} &
\multicolumn{1}{r!{\huxvb{0}}}{\huxtpad{4pt}\raggedleft {\fontsize{9.5pt}{11.4pt}\selectfont 0.101~}\huxbpad{4pt}} &
\multicolumn{1}{r!{\huxvb{0}}}{\huxtpad{4pt}\raggedleft {\fontsize{9.5pt}{11.4pt}\selectfont 0.081~~}\huxbpad{4pt}} \tabularnewline[-0.5pt]


\hhline{}
\arrayrulecolor{black}

\multicolumn{1}{!{\huxvb{0}}l!{\huxvb{0}}}{\huxtpad{4pt}\raggedright {\fontsize{9.5pt}{11.4pt}\selectfont }\huxbpad{4pt}} &
\multicolumn{1}{r!{\huxvb{0}}}{\huxtpad{4pt}\raggedleft {\fontsize{9.5pt}{11.4pt}\selectfont ~~~~~~~~}\huxbpad{4pt}} &
\multicolumn{1}{r!{\huxvb{0}}}{\huxtpad{4pt}\raggedleft {\fontsize{9.5pt}{11.4pt}\selectfont (0.441)}\huxbpad{4pt}} &
\multicolumn{1}{r!{\huxvb{0}}}{\huxtpad{4pt}\raggedleft {\fontsize{9.5pt}{11.4pt}\selectfont (0.403)~}\huxbpad{4pt}} \tabularnewline[-0.5pt]


\hhline{}
\arrayrulecolor{black}

\multicolumn{1}{!{\huxvb{0}}l!{\huxvb{0}}}{\huxtpad{4pt}\raggedright {\fontsize{9.5pt}{11.4pt}\selectfont H.S.}\huxbpad{4pt}} &
\multicolumn{1}{r!{\huxvb{0}}}{\huxtpad{4pt}\raggedleft {\fontsize{9.5pt}{11.4pt}\selectfont ~~~~~~~~}\huxbpad{4pt}} &
\multicolumn{1}{r!{\huxvb{0}}}{\huxtpad{4pt}\raggedleft {\fontsize{9.5pt}{11.4pt}\selectfont -0.657~}\huxbpad{4pt}} &
\multicolumn{1}{r!{\huxvb{0}}}{\huxtpad{4pt}\raggedleft {\fontsize{9.5pt}{11.4pt}\selectfont -0.643~~}\huxbpad{4pt}} \tabularnewline[-0.5pt]


\hhline{}
\arrayrulecolor{black}

\multicolumn{1}{!{\huxvb{0}}l!{\huxvb{0}}}{\huxtpad{4pt}\raggedright {\fontsize{9.5pt}{11.4pt}\selectfont }\huxbpad{4pt}} &
\multicolumn{1}{r!{\huxvb{0}}}{\huxtpad{4pt}\raggedleft {\fontsize{9.5pt}{11.4pt}\selectfont ~~~~~~~~}\huxbpad{4pt}} &
\multicolumn{1}{r!{\huxvb{0}}}{\huxtpad{4pt}\raggedleft {\fontsize{9.5pt}{11.4pt}\selectfont (0.396)}\huxbpad{4pt}} &
\multicolumn{1}{r!{\huxvb{0}}}{\huxtpad{4pt}\raggedleft {\fontsize{9.5pt}{11.4pt}\selectfont (0.363)~}\huxbpad{4pt}} \tabularnewline[-0.5pt]


\hhline{}
\arrayrulecolor{black}

\multicolumn{1}{!{\huxvb{0}}l!{\huxvb{0}}}{\huxtpad{4pt}\raggedright {\fontsize{9.5pt}{11.4pt}\selectfont Some college, no B.A.}\huxbpad{4pt}} &
\multicolumn{1}{r!{\huxvb{0}}}{\huxtpad{4pt}\raggedleft {\fontsize{9.5pt}{11.4pt}\selectfont ~~~~~~~~}\huxbpad{4pt}} &
\multicolumn{1}{r!{\huxvb{0}}}{\huxtpad{4pt}\raggedleft {\fontsize{9.5pt}{11.4pt}\selectfont 0.137~}\huxbpad{4pt}} &
\multicolumn{1}{r!{\huxvb{0}}}{\huxtpad{4pt}\raggedleft {\fontsize{9.5pt}{11.4pt}\selectfont 0.156~~}\huxbpad{4pt}} \tabularnewline[-0.5pt]


\hhline{}
\arrayrulecolor{black}

\multicolumn{1}{!{\huxvb{0}}l!{\huxvb{0}}}{\huxtpad{4pt}\raggedright {\fontsize{9.5pt}{11.4pt}\selectfont }\huxbpad{4pt}} &
\multicolumn{1}{r!{\huxvb{0}}}{\huxtpad{4pt}\raggedleft {\fontsize{9.5pt}{11.4pt}\selectfont ~~~~~~~~}\huxbpad{4pt}} &
\multicolumn{1}{r!{\huxvb{0}}}{\huxtpad{4pt}\raggedleft {\fontsize{9.5pt}{11.4pt}\selectfont (0.361)}\huxbpad{4pt}} &
\multicolumn{1}{r!{\huxvb{0}}}{\huxtpad{4pt}\raggedleft {\fontsize{9.5pt}{11.4pt}\selectfont (0.344)~}\huxbpad{4pt}} \tabularnewline[-0.5pt]


\hhline{}
\arrayrulecolor{black}

\multicolumn{1}{!{\huxvb{0}}l!{\huxvb{0}}}{\huxtpad{4pt}\raggedright {\fontsize{9.5pt}{11.4pt}\selectfont B.A.}\huxbpad{4pt}} &
\multicolumn{1}{r!{\huxvb{0}}}{\huxtpad{4pt}\raggedleft {\fontsize{9.5pt}{11.4pt}\selectfont ~~~~~~~~}\huxbpad{4pt}} &
\multicolumn{1}{r!{\huxvb{0}}}{\huxtpad{4pt}\raggedleft {\fontsize{9.5pt}{11.4pt}\selectfont -0.398~}\huxbpad{4pt}} &
\multicolumn{1}{r!{\huxvb{0}}}{\huxtpad{4pt}\raggedleft {\fontsize{9.5pt}{11.4pt}\selectfont -0.446~~}\huxbpad{4pt}} \tabularnewline[-0.5pt]


\hhline{}
\arrayrulecolor{black}

\multicolumn{1}{!{\huxvb{0}}l!{\huxvb{0}}}{\huxtpad{4pt}\raggedright {\fontsize{9.5pt}{11.4pt}\selectfont }\huxbpad{4pt}} &
\multicolumn{1}{r!{\huxvb{0}}}{\huxtpad{4pt}\raggedleft {\fontsize{9.5pt}{11.4pt}\selectfont ~~~~~~~~}\huxbpad{4pt}} &
\multicolumn{1}{r!{\huxvb{0}}}{\huxtpad{4pt}\raggedleft {\fontsize{9.5pt}{11.4pt}\selectfont (0.312)}\huxbpad{4pt}} &
\multicolumn{1}{r!{\huxvb{0}}}{\huxtpad{4pt}\raggedleft {\fontsize{9.5pt}{11.4pt}\selectfont (0.305)~}\huxbpad{4pt}} \tabularnewline[-0.5pt]


\hhline{}
\arrayrulecolor{black}

\multicolumn{1}{!{\huxvb{0}}l!{\huxvb{0}}}{\huxtpad{4pt}\raggedright {\fontsize{9.5pt}{11.4pt}\selectfont Neighborhood experience}\huxbpad{4pt}} &
\multicolumn{1}{r!{\huxvb{0}}}{\huxtpad{4pt}\raggedleft {\fontsize{9.5pt}{11.4pt}\selectfont ~~~~~~~~}\huxbpad{4pt}} &
\multicolumn{1}{r!{\huxvb{0}}}{\huxtpad{4pt}\raggedleft {\fontsize{9.5pt}{11.4pt}\selectfont ~~~~~}\huxbpad{4pt}} &
\multicolumn{1}{r!{\huxvb{0}}}{\huxtpad{4pt}\raggedleft {\fontsize{9.5pt}{11.4pt}\selectfont ~~~~~~}\huxbpad{4pt}} \tabularnewline[-0.5pt]


\hhline{}
\arrayrulecolor{black}

\multicolumn{1}{!{\huxvb{0}}l!{\huxvb{0}}}{\huxtpad{4pt}\raggedright {\fontsize{9.5pt}{11.4pt}\selectfont Years in neighborhood}\huxbpad{4pt}} &
\multicolumn{1}{r!{\huxvb{0}}}{\huxtpad{4pt}\raggedleft {\fontsize{9.5pt}{11.4pt}\selectfont ~~~~~~~~}\huxbpad{4pt}} &
\multicolumn{1}{r!{\huxvb{0}}}{\huxtpad{4pt}\raggedleft {\fontsize{9.5pt}{11.4pt}\selectfont ~~~~~}\huxbpad{4pt}} &
\multicolumn{1}{r!{\huxvb{0}}}{\huxtpad{4pt}\raggedleft {\fontsize{9.5pt}{11.4pt}\selectfont -0.002~~}\huxbpad{4pt}} \tabularnewline[-0.5pt]


\hhline{}
\arrayrulecolor{black}

\multicolumn{1}{!{\huxvb{0}}l!{\huxvb{0}}}{\huxtpad{4pt}\raggedright {\fontsize{9.5pt}{11.4pt}\selectfont }\huxbpad{4pt}} &
\multicolumn{1}{r!{\huxvb{0}}}{\huxtpad{4pt}\raggedleft {\fontsize{9.5pt}{11.4pt}\selectfont ~~~~~~~~}\huxbpad{4pt}} &
\multicolumn{1}{r!{\huxvb{0}}}{\huxtpad{4pt}\raggedleft {\fontsize{9.5pt}{11.4pt}\selectfont ~~~~~}\huxbpad{4pt}} &
\multicolumn{1}{r!{\huxvb{0}}}{\huxtpad{4pt}\raggedleft {\fontsize{9.5pt}{11.4pt}\selectfont (0.013)~}\huxbpad{4pt}} \tabularnewline[-0.5pt]


\hhline{}
\arrayrulecolor{black}

\multicolumn{1}{!{\huxvb{0}}l!{\huxvb{0}}}{\huxtpad{4pt}\raggedright {\fontsize{9.5pt}{11.4pt}\selectfont 10-50 blocks}\huxbpad{4pt}} &
\multicolumn{1}{r!{\huxvb{0}}}{\huxtpad{4pt}\raggedleft {\fontsize{9.5pt}{11.4pt}\selectfont ~~~~~~~~}\huxbpad{4pt}} &
\multicolumn{1}{r!{\huxvb{0}}}{\huxtpad{4pt}\raggedleft {\fontsize{9.5pt}{11.4pt}\selectfont ~~~~~}\huxbpad{4pt}} &
\multicolumn{1}{r!{\huxvb{0}}}{\huxtpad{4pt}\raggedleft {\fontsize{9.5pt}{11.4pt}\selectfont 0.447~~}\huxbpad{4pt}} \tabularnewline[-0.5pt]


\hhline{}
\arrayrulecolor{black}

\multicolumn{1}{!{\huxvb{0}}l!{\huxvb{0}}}{\huxtpad{4pt}\raggedright {\fontsize{9.5pt}{11.4pt}\selectfont }\huxbpad{4pt}} &
\multicolumn{1}{r!{\huxvb{0}}}{\huxtpad{4pt}\raggedleft {\fontsize{9.5pt}{11.4pt}\selectfont ~~~~~~~~}\huxbpad{4pt}} &
\multicolumn{1}{r!{\huxvb{0}}}{\huxtpad{4pt}\raggedleft {\fontsize{9.5pt}{11.4pt}\selectfont ~~~~~}\huxbpad{4pt}} &
\multicolumn{1}{r!{\huxvb{0}}}{\huxtpad{4pt}\raggedleft {\fontsize{9.5pt}{11.4pt}\selectfont (0.379)~}\huxbpad{4pt}} \tabularnewline[-0.5pt]


\hhline{}
\arrayrulecolor{black}

\multicolumn{1}{!{\huxvb{0}}l!{\huxvb{0}}}{\huxtpad{4pt}\raggedright {\fontsize{9.5pt}{11.4pt}\selectfont \$$>$\$50 blocks}\huxbpad{4pt}} &
\multicolumn{1}{r!{\huxvb{0}}}{\huxtpad{4pt}\raggedleft {\fontsize{9.5pt}{11.4pt}\selectfont ~~~~~~~~}\huxbpad{4pt}} &
\multicolumn{1}{r!{\huxvb{0}}}{\huxtpad{4pt}\raggedleft {\fontsize{9.5pt}{11.4pt}\selectfont ~~~~~}\huxbpad{4pt}} &
\multicolumn{1}{r!{\huxvb{0}}}{\huxtpad{4pt}\raggedleft {\fontsize{9.5pt}{11.4pt}\selectfont -0.604 *}\huxbpad{4pt}} \tabularnewline[-0.5pt]


\hhline{}
\arrayrulecolor{black}

\multicolumn{1}{!{\huxvb{0}}l!{\huxvb{0}}}{\huxtpad{4pt}\raggedright {\fontsize{9.5pt}{11.4pt}\selectfont }\huxbpad{4pt}} &
\multicolumn{1}{r!{\huxvb{0}}}{\huxtpad{4pt}\raggedleft {\fontsize{9.5pt}{11.4pt}\selectfont ~~~~~~~~}\huxbpad{4pt}} &
\multicolumn{1}{r!{\huxvb{0}}}{\huxtpad{4pt}\raggedleft {\fontsize{9.5pt}{11.4pt}\selectfont ~~~~~}\huxbpad{4pt}} &
\multicolumn{1}{r!{\huxvb{0}}}{\huxtpad{4pt}\raggedleft {\fontsize{9.5pt}{11.4pt}\selectfont (0.299)~}\huxbpad{4pt}} \tabularnewline[-0.5pt]


\hhline{>{\huxb{1}}->{\huxb{1}}->{\huxb{1}}->{\huxb{1}}-}
\arrayrulecolor{black}

\multicolumn{1}{!{\huxvb{0}}l!{\huxvb{0}}}{\huxtpad{4pt}\raggedright {\fontsize{9.5pt}{11.4pt}\selectfont nobs}\huxbpad{4pt}} &
\multicolumn{1}{r!{\huxvb{0}}}{\huxtpad{4pt}\raggedleft {\fontsize{9.5pt}{11.4pt}\selectfont ~~~~~~~~}\huxbpad{4pt}} &
\multicolumn{1}{r!{\huxvb{0}}}{\huxtpad{4pt}\raggedleft {\fontsize{9.5pt}{11.4pt}\selectfont ~~~~~}\huxbpad{4pt}} &
\multicolumn{1}{r!{\huxvb{0}}}{\huxtpad{4pt}\raggedleft {\fontsize{9.5pt}{11.4pt}\selectfont ~~~~~~}\huxbpad{4pt}} \tabularnewline[-0.5pt]


\hhline{>{\huxb{0.8}}->{\huxb{0.8}}->{\huxb{0.8}}->{\huxb{0.8}}-}
\arrayrulecolor{black}

\multicolumn{4}{!{\huxvb{0}}p{0.5\textwidth+6\tabcolsep}!{\huxvb{0}}}{\parbox[b]{0.5\textwidth+6\tabcolsep-4pt-4pt}{\huxtpad{4pt}\raggedright {\fontsize{9.5pt}{11.4pt}\selectfont  *** p $<$ 0.001;  ** p $<$ 0.01;  * p $<$ 0.05.}\huxbpad{4pt}}} \tabularnewline[-0.5pt]


\hhline{}
\arrayrulecolor{black}
\end{tabularx}
\end{table}


% Table created by stargazer v.5.2.2 by Marek Hlavac, Harvard University. E-mail: hlavac at fas.harvard.edu
% Date and time: Fri, Oct 26, 2018 - 13:53:08
% Requires LaTeX packages: dcolumn 
\begin{table}[!htbp] \centering 
  \caption{Estimated coefficients for logistic regression of seriously considering communities} 
  \label{tab:consider} 
\begin{tabular}{@{\extracolsep{5pt}}lD{.}{.}{-3} D{.}{.}{-3} } 
\\[-1.8ex]\hline 
\hline \\[-1.8ex] 
\\[-1.8ex] & \multicolumn{1}{c}{(1)} & \multicolumn{1}{c}{(2)}\\ 
\hline \\[-1.8ex] 
 Entropy & 0.010 & 0.011 \\ 
  & (0.016) & (0.017) \\ 
  \quad $\times$ Asian & -0.021 & -0.023 \\ 
  & (0.024) & (0.025) \\ 
  \quad $\times$ Black & -0.020 & -0.024 \\ 
  & (0.022) & (0.022) \\ 
  \quad $\times$ Latino & -0.023 & -0.027 \\ 
  & (0.023) & (0.023) \\ 
  Median Home Value &  & -0.0003 \\ 
  &  & (0.001) \\ 
  \quad $\times$ High income &  & 0.001 \\ 
  &  & (0.001) \\ 
  Asian & 1.330 & 1.498 \\ 
  & (2.130) & (2.143) \\ 
  Black & 1.891 & 2.089 \\ 
  & (1.885) & (1.930) \\ 
  Latino & 2.083 & 2.301 \\ 
  & (1.973) & (1.972) \\ 
  High income &  & -0.949^{*} \\ 
  &  & (0.518) \\ 
  Lives in the community & 0.290 & 0.291 \\ 
  & (0.246) & (0.245) \\ 
  Constant & -3.573^{***} & -3.412^{**} \\ 
  & (1.377) & (1.544) \\ 
 \hline \\[-1.8ex] 
Observations & \multicolumn{1}{c}{11,620} & \multicolumn{1}{c}{11,620} \\ 
Log Likelihood & \multicolumn{1}{c}{-2,602.205} & \multicolumn{1}{c}{-2,595.259} \\ 
Akaike Inf. Crit. & \multicolumn{1}{c}{5,222.411} & \multicolumn{1}{c}{5,214.519} \\ 
\hline 
\hline \\[-1.8ex] 
\textit{Note:}  & \multicolumn{2}{r}{$^{*}$p$<$0.1; $^{**}$p$<$0.05; $^{***}$p$<$0.01} \\ 
\end{tabular} 
\end{table} 


% Table created by stargazer v.5.2.2 by Marek Hlavac, Harvard University. E-mail: hlavac at fas.harvard.edu
% Date and time: Fri, Oct 26, 2018 - 13:53:08
% Requires LaTeX packages: dcolumn 
\begin{table}[!htbp] \centering 
  \caption{Estimated coefficients for logistic regression of never considering communities} 
  \label{tab:neverconsider} 
\begin{tabular}{@{\extracolsep{5pt}}lD{.}{.}{-3} D{.}{.}{-3} } 
\\[-1.8ex]\hline 
\hline \\[-1.8ex] 
\\[-1.8ex] & \multicolumn{1}{c}{(1)} & \multicolumn{1}{c}{(2)}\\ 
\hline \\[-1.8ex] 
 Entropy & 0.003 & 0.002 \\ 
  & (0.008) & (0.009) \\ 
  \quad $\times$ Asian & 0.002 & 0.004 \\ 
  & (0.016) & (0.016) \\ 
  \quad $\times$ Black & 0.005 & 0.007 \\ 
  & (0.012) & (0.012) \\ 
  \quad $\times$ Latino & -0.006 & -0.004 \\ 
  & (0.012) & (0.012) \\ 
  Median Home Value &  & 0.0001 \\ 
  &  & (0.0003) \\ 
  \quad $\times$ High income &  & -0.001 \\ 
  &  & (0.001) \\ 
  Asian & -1.152 & -1.207 \\ 
  & (1.375) & (1.383) \\ 
  Black & -1.008 & -1.090 \\ 
  & (1.069) & (1.078) \\ 
  Latino & 0.320 & 0.219 \\ 
  & (1.067) & (1.080) \\ 
  High income &  & 0.681^{***} \\ 
  &  & (0.260) \\ 
  Lives in the community & -1.755^{***} & -1.760^{***} \\ 
  & (0.297) & (0.299) \\ 
  Constant & -1.133 & -1.156 \\ 
  & (0.690) & (0.807) \\ 
 \hline \\[-1.8ex] 
Observations & \multicolumn{1}{c}{11,620} & \multicolumn{1}{c}{11,620} \\ 
Log Likelihood & \multicolumn{1}{c}{-5,591.998} & \multicolumn{1}{c}{-5,568.267} \\ 
Akaike Inf. Crit. & \multicolumn{1}{c}{11,202.000} & \multicolumn{1}{c}{11,160.530} \\ 
\hline 
\hline \\[-1.8ex] 
\textit{Note:}  & \multicolumn{2}{r}{$^{*}$p$<$0.1; $^{**}$p$<$0.05; $^{***}$p$<$0.01} \\ 
\end{tabular} 
\end{table} 



\section{Figures}
\begin{figure}
\centering
\includegraphics[scale=.35]{../analysis/images/satisfied_descriptive.png}
\caption{Proportion of respondents reporting that they are extremely or very satisfied with their neighborhood, by race}
\label{fig:satisfied}
\end{figure}

\end{document}