\documentclass[11pt]{baderart}
\usepackage{array}
\usepackage{caption}
\usepackage{graphicx}
\usepackage{siunitx}
\usepackage{multirow}
\usepackage{hhline}
\usepackage{calc}
\usepackage{tabularx}

\newcommand{\TK}{\textbf{\{TK\}}}



\title{Multiethnic Neighborhoods}
\author{Michael D.M.~Bader}

\begin{document}
\maketitle

\noindent Racially integrated neighborhoods have become more numerous in the landscape of American metropolitan areas. While they were once unstable rarities, integrated neighborhoods have become both more common and more stable \needcite. Previous studies estimate \TK\ to \TK\ percent of metropolitan neighborhoods to be racially integrated, depending on the method used to classify integration. More tolerant racial attitudes among whites, preferences for racially integrated neighborhoods among people of color, and federal policies outlawing discrimination have helped the number of multiethnic neighborhoods grow \needcite. Integration has been bolstered by the expanding diversity of U.S.\ population, especially from the immigration and natural growth of Latinos and Asian-Americans, have increased. 

Maintaining integration over time depends on residents with different racial identities being satisfied living in multiracial neighborhoods. Differences in neighborhood satisfaction across racial groups would lead to some groups being likely to leave the neighborhood, thereby imperiling the racial composition and could set neighborhoods on a trajectory toward integration. Despite the relevance of this question for predicting the long-term stability of racial integration, we do not know whether satisfaction with integrated neighborhoods varies across racial groups. 

This article addresses the shortcoming of previous research by studying levels of satisfaction among residents of integrated neighborhoods. Unlike previous research that focuses on a handful of integrated neighborhoods, this article examines neighborhood satisfaction using data from a large representative sample of residents living in integrated neighborhoods in a metropolitan area. The results reveal few racial differences in how satisfied residents feel living in integrated neighborhoods and in how they perceive the direction of the neighborhood. Yet, whites are less likely to consider moving to one of the other integrated neighborhoods in the metropolitan area, possibly reducing the chances that integration can be maintained over time. 

\section{Rise of Multiethnic Neighborhoods}
The human ecological model of neighborhood change developed by Park and colleagues (\needcite) applied for most of the twentieth century. One or two households of a racial or ethnic group would ``invade'' a neighborhood where the majority of residents identified as another group. After this initial invasion, change would proceed rapidly as the incumbents would leave and the more members of the new racial group would enter. The model explained racial change associated with white flight particularly well as neighborhoods would change rapidly once the mobility pattern took hold \needcite. 

The pattern of rapid succession, however, slowed during the 1980s and 1990s. White residents tended not to flee neighborhoods as people of color moved into their neighborhoods. Ellen \needcite documented the rise of durable racial integration and the possibility that racial segregation should not be a foregone conclusion. Contrary to the hypothesis of invasion and succession, she showed that racial change occurred very slowly in integrated neighborhoods. Since Ellen's work, several studies have documented the fact that integrated neighborhoods represent a plurality of American neighborhoods. Integration became probable a few decades after many questioned whether it was even possible. 

Several factors led to the growth of integrated neighborhoods. First, federal, state, and local civil rights legislation provided a legal basis for people of color, especially blacks, to enter previously all-white neighborhoods. Second, the American population became more diverse as immigration led to more people of color living in the United States. Third, the rising economic position among people of color (supported by civil rights legislation against discrimination) allowed more people of color to enter neighborhoods that had been previously unavailable. Finally, whites have become more racially tolerant of being neighbors with people of color. 

These forces led to the growth of two specific types of racially integrated neighborhoods. The first are ``quadrivial neighborhoods'' in which white, black, Latino, and Asian residents live in proportion to their representation in the overall population.
\footnote{Logan and Zhang \needcite\ called these neighborhoods ``global neighborhoods.'' I prefer to follow Bader and Warkentien \needcite\ who used the term ``quadrivial'' neighborhoods, who used the Latin term meaning four roads coming to an intersection. The term ``global'' implies non-white residents of the neighborhood are foreign born rather than being as American as white residents.}
Estimates regarding the number of quadrivial neighborhoods range from an estimate of fifteen percent to \TK\ based on Logan and Zhang \needcite. 

The type comprises neighborhoods in which the Latinx population has grown but a substantial number of whites still live in the neighborhood. Evidence shows that this type of neighborhood makes up the majority or a substantial plurality of integrated neighborhoods. These neighborhoods might include other racial groups, more often Asian than black, at smaller percentages of the composition \needcite. While integrated, evidence suggests integration as a temporary phenomenon. Farrell and Lee \needcite\ show that Latinos are simultaneously the group most likely to integrate a racially homogenous neighborhood \emph{and} experience increasing isolation. Bader and Warkentien \needcite\ estimate the rate of change in neighborhoods in which whites and Latinos make up the two largest groups in the population, and find that the neighborhoods are on pace to resegregate as all-Latino neighborhoods.

\section{Satisfaction Living in Multiethnic Neighborhoods}
Research on life in multiethnic neighborhoods has not kept pace with the increasing prevalence of multiethnic neighborhoods in American metropolitan areas. Among the studies that exist, most find that residents are satisfied living in multiracial neighborhoods \needcite. Most of the research on integrated neighborhoods comes from case-studies of residents integrated neighborhoods. 

Some case studies focus on multiethnic neighborhoods in central cities like Roger's Park in Chicago or Jackson Heights in Queens. \textbf{TK: more on specific studies about integrated neighborhoods}. Another source of case studies come from inner ring suburbs that have taken proactive steps to maintain explicit integration. Places like Oak Park, Illinois, Shaker Heights, Ohio, and Mount Airy, Pennsylvania offer case studies for how to actively manage and maintain integration. 

While useful, the cases presented in this research do not reflect where multiethnic neighborhoods exist today. The most diverse neighborhoods, the quadrivial neighborhoods, tend to be found in the middle-ring suburbs of American metropolitan areas. These were places where a majority of the housing stock tended to be built after the Fair Housing Act passed in 1968 and located among major transportation corridors. Integration is more often found in places like Naperville, Illnois (outside of Chicago), Sugar Land, Texas (outside of Houston), Germantown, Maryland (outside of Washington, D.C.) than in the well-known central city and activist suburban neighborhoods. 

The same is true of disproportionately Latino neighborhoods. While popular imagination and research alike tend to focus on central city ethnic enclaves, immigrants now often bypass those neighborhoods to move to suburban enclaves. Native-born Latinos, especially those who have experienced upward mobility, have dispersed throughout metropolitan areas. 

While case studies of specific neighborhoods provide valuable insights, they do not capture the contemporary conditions of racially integrated neighborhoods. We need data from residents of to understand the future of integration in U.S.\ society. Knowing whether racial groups derive different levels of satisfaction from living in integrated neighborhoods will help ascertain whether groups feel socially integrated and predict how stable integration will be in multiracial neighborhoods. This study uses data from a representative sample of residents living in multiracial neighborhoods in the Washington, D.C.\ metropolitan area. It investigates whether neighborhood satisfaction perceptions of neighborhood change differ across racial groups. 

The study also investigates the willingness to live in \emph{other} integrated neighborhoods in the Washington, D.C.\ metropolitan area among people living in integrated neighborhoods. This latter research aim provides a glimpse into the durability of integration. All neighborhoods undergo change through people moving out of and into the neighborhood. The racial composition of people who move into the neighborhood will have to roughly represent the racial composition of the metropolitan area as a whole to maintain integration into the future. 

Previous research would suggest that white preferences play a disproportionate role in the process of maintaining integration. While people of color have expressed a desire to live in integrated neighborhoods and have moved in substantial numbers to previous all-white neighborhoods, whites neither express nor move to non-white neighborhoods in more than token numbers. Previous research has not, to date, focused either on the residents of multiethnic neighborhoods nor preferences for multiethnic neighborhoods. This study will provide the first systematic evidence regarding the long-term viability of multiracial neighborhoods across an entire metropolitan area. 

\section{Data and Methods}
\subsection{Sample}
Data for this study come from the 2016 D.C.\ Area Survey. The Washington, D.C.\ metropolitan areas, like most others in the U.S., is very segregated. Blacks and whites are groups most segregated from one another. The western side of the city is overwhelmingly white, as are the surrounding jurisdictions of Montgomery County in Maryland, and Fairfax and Arlington Counties in Virginia. Black residents in the metropolitan area disproportionately live in the eastern half of the city and in neighboring Prince George's County. 

D.C.\ area residents have attained high levels of education and wealth compared to other metropolitan areas. \TK percent of residents aged 25 and older have a four year degree and \TK have a professional degree. This is true across racial groups. The D.C. area has the highest percentage of black residents, \TK\% with a four year degree of all metropolitan areas. Immigrants in the D.C.\ area are also more educated than the average metropolitan area, with \TK\ holding a four year degree and \TK\ holding a professional degree (among other large metropolitan areas \TK\ hold a four year degree and \TK\ hold a professional degree). These factors reduce the level of inequality across racial and immigration status that typify many U.S.\ metropolitan areas and produce conditions conducive to study the potential for stable integration.

The D.C.\ Area Survey sampled from a subset of the D.C.\ metropolitan area. The sample included the District of Columbia and its adjacent jurisdictions. Those are Montgomery and Prince George's Counties in Maryland, Arlington and Fairfax Counties in Virginia, and the independent city of Alexandria, Virginia.\footnote{The independent cities of Fairfax and Falls Church were included as part of Fairfax County.}  Consistent with national trends, the majority of quadrivial and disproportionately Latino neighborhoods were in the five suburban jurisdictions, not in the District of Columbia. Figure~\ref{fig:nhoodmap} shows the location of eligible neighborhoods in the D.C.\ area. 

Households were sampled for the survey using a stratified address-based sample of residents living in either a quadrivial or disproportionately Latino neighborhood. Quadrivial neighborhoods were defined as any Census tract in which blacks, Latinos, Asians, and whites each made up at least ten percent of the population based on the 2010 Census, and in which no racial group was a majority. Disproportionately Latino neighborhoods were neighborhoods that was not already classified as a quadrivial neighborhood and in which at least 25~percent of the residents identified as Latino.

Packets were mailed to 9,600 addresses in Census tracts defined as either quadrivial or disproportionately Latino neighborhoods. Strata were created within each type of neighborhood to oversample households from the address registry with Asian and Hispanic surnames as well as neighborhoods with high percentages of black residents. These strata helped ensure sufficient representation of all four racial groups to model findings by race. The sample generalizes to the population of residents living in quadrivial and disproportionately Latino neighborhoods. 

The packet mailed to households contained a two-sided cover letter written in English on one side and Spanish on the other, a questionnaire booklet, a postage-paid return envelope and a two dollar bill. Packets sent to households in the disproportionately Latino neighborhoods also included a Spanish language questionnaire. Packets were sent at the beginning of March 2016, and a reminder letter was sent two weeks later. Surveys were collected for approximately seven weeks after the initial mailing. We received 1,222 completed surveys. The response rate using AAPOR Response Rate 4 was 12.8~percent

\subsection{Dependent Variables} 
\paragraph{Neighborhood satisfaction.} I first examined how satisfied respondents report being in their neighborhood. The questionnaire asked respondents, ``How satisfied are you with your neighborhood as a place to live?'' and respondents could respond that they were extremely, very, somewhat, or not at all satisfied. For the primary analysis, I coded a dichotomous satisfaction variable with respondents who said that they were extremely or very satisfied coded as ``yes'' and those who said somewhat or not at all coded as ``no.'' 

\paragraph{Neighborhood improvement.} I examined whether respondents considered their neighborhood to be improving. Perceiving improvement in neighborhood conditions would make residents likelier to stay in the neighborhood. Measuring perceptions of improvement helps ascertain whether one racial group might view integration less favorably than others, and thus more prone to moving. 

To measure whether a respondent perceived the neighborhood improvement, the questionnaire asked, ``Looking back over the past five years or so, would you say that your neighborhood has:'' and offered the following responses: ``Become a much better place to live,'' ``Become a somewhat better place to live,'' ``Stayed about the same,'' ``Become a somewhat worse place to live,'' and ``Become a much worse place to live.'' I created a variable measuring whether a respondent perceived neighborhood improvement that was coded as ``yes'' if the respondent indicated that their neighborhood had become a much better or somewhat better place to live, and ``no'' otherwise. 

\paragraph{Preferences for other multiethnic neighborhoods.}





\subsection{Independent and Control Variables}

\subsection{Method}

\section{Results}










\section{Tables}
% latex table generated in R 3.5.0 by xtable 1.8-2 package
% 
\begin{sidewaystable}[ht]
\centering
\caption{Means and standard deviations of independent and control variables} 
\label{tab:descriptives}
\begin{tabular}{lR{.4in}R{.4in}R{.4in}R{.4in}R{.4in}R{.4in}R{.4in}R{.4in}R{.4in}R{.4in}}
  \toprule
& \multicolumn{2}{p{.8in}}{\centering \strong{Total Sample}} & \multicolumn{2}{p{.8in}}{\centering \strong{Asians}} & \multicolumn{2}{p{.8in}}{\centering \strong{Blacks}} & \multicolumn{2}{p{.8in}}{\centering \strong{Latinxs}} & \multicolumn{2}{p{.8in}}{\centering \strong{Whites}} \\
Variable & Mean & S.D. & Mean & S.D. & Mean & S.D. & Mean & S.D. & Mean & S.D. \\ 
  \midrule
\emph{Race}&&\\White &  0.32 &  &  &  &  &  &  &  &  &  \\ 
  Asian &  0.21 &  &  &  &  &  &  &  &  &  \\ 
  Black &  0.22 &  &  &  &  &  &  &  &  &  \\ 
  Latinx\vspace{1em} &  0.25 &  &  &  &  &  &  &  &  &  \\ 
  \emph{Demographics}&&\\Age & 47.11 & 0.82 & 45.04 & 1.57 & 46.78 & 1.78 & 47.25 & 1.89 & 48.63 & 1.39 \\ 
  Foreign Born &  0.46 &  & 0.79 &  & 0.37 &  & 0.65 &  & 0.14 &  \\ 
  Man &  0.49 &  & 0.51 &  & 0.43 &  & 0.52 &  & 0.48 &  \\ 
  Children present &  0.40 &  & 0.36 &  & 0.45 &  & 0.45 &  & 0.35 &  \\ 
  Married\vspace{1em} &  0.65 &  & 0.7 &  & 0.61 &  & 0.55 &  & 0.72 &  \\ 
  \emph{Education}&&\\Less than H.S. &  0.04 &  & 0.07 &  & 0 &  & 0.09 &  & 0.01 &  \\ 
  H.S. or G.E.D. &  0.09 &  & 0.08 &  & 0.14 &  & 0.08 &  & 0.07 &  \\ 
  Some college &  0.21 &  & 0.13 &  & 0.21 &  & 0.3 &  & 0.2 &  \\ 
  Bachelor's degree &  0.31 &  & 0.42 &  & 0.3 &  & 0.24 &  & 0.3 &  \\ 
  Professional degree\vspace{1em} &  0.34 &  & 0.3 &  & 0.34 &  & 0.28 &  & 0.42 &  \\ 
  \emph{Neighborhood Experience}\\Years in Neighborhood & 11.89 & 0.52 & 10.44 & 0.92 & 10.24 & 1.03 & 12.59 & 1.12 & 13.47 & 0.96 \\ 
  Perceived neighborhood size\\1-9 blocks &  0.60 &  & 0.66 &  & 0.58 &  & 0.52 &  & 0.65 &  \\ 
  10-50 blocks &  0.33 &  & 0.26 &  & 0.38 &  & 0.35 &  & 0.32 &  \\ 
  >50 blocks &  0.07 &  & 0.08 &  & 0.04 &  & 0.14 &  & 0.03 &  \\ 
   \bottomrule
\end{tabular}
\end{sidewaystable}

\begin{table}[h]
\centering\captionsetup{justification=centering,singlelinecheck=off}
\caption{Estimated coefficients predicting  neighborhood satisfaction}
\label{tab:satisfaction}

    \providecommand{\huxb}[2][0,0,0]{\arrayrulecolor[RGB]{#1}\global\arrayrulewidth=#2pt}
    \providecommand{\huxvb}[2][0,0,0]{\color[RGB]{#1}\vrule width #2pt}
    \providecommand{\huxtpad}[1]{\rule{0pt}{\baselineskip+#1}}
    \providecommand{\huxbpad}[1]{\rule[-#1]{0pt}{#1}}
  \begin{tabularx}{0.5\textwidth}{p{0.1\textwidth} p{0.1\textwidth} p{0.1\textwidth} p{0.1\textwidth} p{0.1\textwidth}}


\hhline{>{\huxb{0.8}}->{\huxb{0.8}}->{\huxb{0.8}}->{\huxb{0.8}}->{\huxb{0.8}}-}
\arrayrulecolor{black}

\multicolumn{1}{!{\huxvb{0}}c!{\huxvb{0}}}{\huxtpad{4pt}\centering {\fontsize{9.5pt}{11.4pt}\selectfont }\huxbpad{4pt}} &
\multicolumn{1}{c!{\huxvb{0}}}{\huxtpad{4pt}\centering {\fontsize{9.5pt}{11.4pt}\selectfont (1)}\huxbpad{4pt}} &
\multicolumn{1}{c!{\huxvb{0}}}{\huxtpad{4pt}\centering {\fontsize{9.5pt}{11.4pt}\selectfont (2)}\huxbpad{4pt}} &
\multicolumn{1}{c!{\huxvb{0}}}{\huxtpad{4pt}\centering {\fontsize{9.5pt}{11.4pt}\selectfont (3)}\huxbpad{4pt}} &
\multicolumn{1}{c!{\huxvb{0}}}{\huxtpad{4pt}\centering {\fontsize{9.5pt}{11.4pt}\selectfont (4)}\huxbpad{4pt}} \tabularnewline[-0.5pt]


\hhline{>{\huxb{1}}->{\huxb{1}}->{\huxb{1}}->{\huxb{1}}->{\huxb{1}}-}
\arrayrulecolor{black}

\multicolumn{1}{!{\huxvb{0}}l!{\huxvb{0}}}{\huxtpad{4pt}\raggedright {\fontsize{9.5pt}{11.4pt}\selectfont (Intercept)}\huxbpad{4pt}} &
\multicolumn{1}{r!{\huxvb{0}}}{\huxtpad{4pt}\raggedleft {\fontsize{9.5pt}{11.4pt}\selectfont 0.944~}\huxbpad{4pt}} &
\multicolumn{1}{r!{\huxvb{0}}}{\huxtpad{4pt}\raggedleft {\fontsize{9.5pt}{11.4pt}\selectfont 1.714~~}\huxbpad{4pt}} &
\multicolumn{1}{r!{\huxvb{0}}}{\huxtpad{4pt}\raggedleft {\fontsize{9.5pt}{11.4pt}\selectfont 1.606~~~}\huxbpad{4pt}} &
\multicolumn{1}{r!{\huxvb{0}}}{\huxtpad{4pt}\raggedleft {\fontsize{9.5pt}{11.4pt}\selectfont 1.900~~~}\huxbpad{4pt}} \tabularnewline[-0.5pt]


\hhline{}
\arrayrulecolor{black}

\multicolumn{1}{!{\huxvb{0}}l!{\huxvb{0}}}{\huxtpad{4pt}\raggedright {\fontsize{9.5pt}{11.4pt}\selectfont }\huxbpad{4pt}} &
\multicolumn{1}{r!{\huxvb{0}}}{\huxtpad{4pt}\raggedleft {\fontsize{9.5pt}{11.4pt}\selectfont (0.732)}\huxbpad{4pt}} &
\multicolumn{1}{r!{\huxvb{0}}}{\huxtpad{4pt}\raggedleft {\fontsize{9.5pt}{11.4pt}\selectfont (1.089)~}\huxbpad{4pt}} &
\multicolumn{1}{r!{\huxvb{0}}}{\huxtpad{4pt}\raggedleft {\fontsize{9.5pt}{11.4pt}\selectfont (1.164)~~}\huxbpad{4pt}} &
\multicolumn{1}{r!{\huxvb{0}}}{\huxtpad{4pt}\raggedleft {\fontsize{9.5pt}{11.4pt}\selectfont (1.142)~~}\huxbpad{4pt}} \tabularnewline[-0.5pt]


\hhline{}
\arrayrulecolor{black}

\multicolumn{1}{!{\huxvb{0}}l!{\huxvb{0}}}{\huxtpad{4pt}\raggedright {\fontsize{9.5pt}{11.4pt}\selectfont Race}\huxbpad{4pt}} &
\multicolumn{1}{r!{\huxvb{0}}}{\huxtpad{4pt}\raggedleft {\fontsize{9.5pt}{11.4pt}\selectfont ~~~~~}\huxbpad{4pt}} &
\multicolumn{1}{r!{\huxvb{0}}}{\huxtpad{4pt}\raggedleft {\fontsize{9.5pt}{11.4pt}\selectfont ~~~~~~}\huxbpad{4pt}} &
\multicolumn{1}{r!{\huxvb{0}}}{\huxtpad{4pt}\raggedleft {\fontsize{9.5pt}{11.4pt}\selectfont ~~~~~~~}\huxbpad{4pt}} &
\multicolumn{1}{r!{\huxvb{0}}}{\huxtpad{4pt}\raggedleft {\fontsize{9.5pt}{11.4pt}\selectfont ~~~~~~~}\huxbpad{4pt}} \tabularnewline[-0.5pt]


\hhline{}
\arrayrulecolor{black}

\multicolumn{1}{!{\huxvb{0}}l!{\huxvb{0}}}{\huxtpad{4pt}\raggedright {\fontsize{9.5pt}{11.4pt}\selectfont Asian}\huxbpad{4pt}} &
\multicolumn{1}{r!{\huxvb{0}}}{\huxtpad{4pt}\raggedleft {\fontsize{9.5pt}{11.4pt}\selectfont -0.050~}\huxbpad{4pt}} &
\multicolumn{1}{r!{\huxvb{0}}}{\huxtpad{4pt}\raggedleft {\fontsize{9.5pt}{11.4pt}\selectfont -0.072~~}\huxbpad{4pt}} &
\multicolumn{1}{r!{\huxvb{0}}}{\huxtpad{4pt}\raggedleft {\fontsize{9.5pt}{11.4pt}\selectfont 0.061~~~}\huxbpad{4pt}} &
\multicolumn{1}{r!{\huxvb{0}}}{\huxtpad{4pt}\raggedleft {\fontsize{9.5pt}{11.4pt}\selectfont ~~~~~~~}\huxbpad{4pt}} \tabularnewline[-0.5pt]


\hhline{}
\arrayrulecolor{black}

\multicolumn{1}{!{\huxvb{0}}l!{\huxvb{0}}}{\huxtpad{4pt}\raggedright {\fontsize{9.5pt}{11.4pt}\selectfont }\huxbpad{4pt}} &
\multicolumn{1}{r!{\huxvb{0}}}{\huxtpad{4pt}\raggedleft {\fontsize{9.5pt}{11.4pt}\selectfont (0.384)}\huxbpad{4pt}} &
\multicolumn{1}{r!{\huxvb{0}}}{\huxtpad{4pt}\raggedleft {\fontsize{9.5pt}{11.4pt}\selectfont (0.484)~}\huxbpad{4pt}} &
\multicolumn{1}{r!{\huxvb{0}}}{\huxtpad{4pt}\raggedleft {\fontsize{9.5pt}{11.4pt}\selectfont (0.494)~~}\huxbpad{4pt}} &
\multicolumn{1}{r!{\huxvb{0}}}{\huxtpad{4pt}\raggedleft {\fontsize{9.5pt}{11.4pt}\selectfont ~~~~~~~}\huxbpad{4pt}} \tabularnewline[-0.5pt]


\hhline{}
\arrayrulecolor{black}

\multicolumn{1}{!{\huxvb{0}}l!{\huxvb{0}}}{\huxtpad{4pt}\raggedright {\fontsize{9.5pt}{11.4pt}\selectfont Black}\huxbpad{4pt}} &
\multicolumn{1}{r!{\huxvb{0}}}{\huxtpad{4pt}\raggedleft {\fontsize{9.5pt}{11.4pt}\selectfont 0.364~}\huxbpad{4pt}} &
\multicolumn{1}{r!{\huxvb{0}}}{\huxtpad{4pt}\raggedleft {\fontsize{9.5pt}{11.4pt}\selectfont 0.464~~}\huxbpad{4pt}} &
\multicolumn{1}{r!{\huxvb{0}}}{\huxtpad{4pt}\raggedleft {\fontsize{9.5pt}{11.4pt}\selectfont 0.363~~~}\huxbpad{4pt}} &
\multicolumn{1}{r!{\huxvb{0}}}{\huxtpad{4pt}\raggedleft {\fontsize{9.5pt}{11.4pt}\selectfont ~~~~~~~}\huxbpad{4pt}} \tabularnewline[-0.5pt]


\hhline{}
\arrayrulecolor{black}

\multicolumn{1}{!{\huxvb{0}}l!{\huxvb{0}}}{\huxtpad{4pt}\raggedright {\fontsize{9.5pt}{11.4pt}\selectfont }\huxbpad{4pt}} &
\multicolumn{1}{r!{\huxvb{0}}}{\huxtpad{4pt}\raggedleft {\fontsize{9.5pt}{11.4pt}\selectfont (0.400)}\huxbpad{4pt}} &
\multicolumn{1}{r!{\huxvb{0}}}{\huxtpad{4pt}\raggedleft {\fontsize{9.5pt}{11.4pt}\selectfont (0.422)~}\huxbpad{4pt}} &
\multicolumn{1}{r!{\huxvb{0}}}{\huxtpad{4pt}\raggedleft {\fontsize{9.5pt}{11.4pt}\selectfont (0.441)~~}\huxbpad{4pt}} &
\multicolumn{1}{r!{\huxvb{0}}}{\huxtpad{4pt}\raggedleft {\fontsize{9.5pt}{11.4pt}\selectfont ~~~~~~~}\huxbpad{4pt}} \tabularnewline[-0.5pt]


\hhline{}
\arrayrulecolor{black}

\multicolumn{1}{!{\huxvb{0}}l!{\huxvb{0}}}{\huxtpad{4pt}\raggedright {\fontsize{9.5pt}{11.4pt}\selectfont Latinx}\huxbpad{4pt}} &
\multicolumn{1}{r!{\huxvb{0}}}{\huxtpad{4pt}\raggedleft {\fontsize{9.5pt}{11.4pt}\selectfont 0.207~}\huxbpad{4pt}} &
\multicolumn{1}{r!{\huxvb{0}}}{\huxtpad{4pt}\raggedleft {\fontsize{9.5pt}{11.4pt}\selectfont 0.505~~}\huxbpad{4pt}} &
\multicolumn{1}{r!{\huxvb{0}}}{\huxtpad{4pt}\raggedleft {\fontsize{9.5pt}{11.4pt}\selectfont 0.564~~~}\huxbpad{4pt}} &
\multicolumn{1}{r!{\huxvb{0}}}{\huxtpad{4pt}\raggedleft {\fontsize{9.5pt}{11.4pt}\selectfont ~~~~~~~}\huxbpad{4pt}} \tabularnewline[-0.5pt]


\hhline{}
\arrayrulecolor{black}

\multicolumn{1}{!{\huxvb{0}}l!{\huxvb{0}}}{\huxtpad{4pt}\raggedright {\fontsize{9.5pt}{11.4pt}\selectfont }\huxbpad{4pt}} &
\multicolumn{1}{r!{\huxvb{0}}}{\huxtpad{4pt}\raggedleft {\fontsize{9.5pt}{11.4pt}\selectfont (0.424)}\huxbpad{4pt}} &
\multicolumn{1}{r!{\huxvb{0}}}{\huxtpad{4pt}\raggedleft {\fontsize{9.5pt}{11.4pt}\selectfont (0.495)~}\huxbpad{4pt}} &
\multicolumn{1}{r!{\huxvb{0}}}{\huxtpad{4pt}\raggedleft {\fontsize{9.5pt}{11.4pt}\selectfont (0.538)~~}\huxbpad{4pt}} &
\multicolumn{1}{r!{\huxvb{0}}}{\huxtpad{4pt}\raggedleft {\fontsize{9.5pt}{11.4pt}\selectfont ~~~~~~~}\huxbpad{4pt}} \tabularnewline[-0.5pt]


\hhline{}
\arrayrulecolor{black}

\multicolumn{1}{!{\huxvb{0}}l!{\huxvb{0}}}{\huxtpad{4pt}\raggedright {\fontsize{9.5pt}{11.4pt}\selectfont Demographics}\huxbpad{4pt}} &
\multicolumn{1}{r!{\huxvb{0}}}{\huxtpad{4pt}\raggedleft {\fontsize{9.5pt}{11.4pt}\selectfont ~~~~~}\huxbpad{4pt}} &
\multicolumn{1}{r!{\huxvb{0}}}{\huxtpad{4pt}\raggedleft {\fontsize{9.5pt}{11.4pt}\selectfont ~~~~~~}\huxbpad{4pt}} &
\multicolumn{1}{r!{\huxvb{0}}}{\huxtpad{4pt}\raggedleft {\fontsize{9.5pt}{11.4pt}\selectfont ~~~~~~~}\huxbpad{4pt}} &
\multicolumn{1}{r!{\huxvb{0}}}{\huxtpad{4pt}\raggedleft {\fontsize{9.5pt}{11.4pt}\selectfont ~~~~~~~}\huxbpad{4pt}} \tabularnewline[-0.5pt]


\hhline{}
\arrayrulecolor{black}

\multicolumn{1}{!{\huxvb{0}}l!{\huxvb{0}}}{\huxtpad{4pt}\raggedright {\fontsize{9.5pt}{11.4pt}\selectfont Age}\huxbpad{4pt}} &
\multicolumn{1}{r!{\huxvb{0}}}{\huxtpad{4pt}\raggedleft {\fontsize{9.5pt}{11.4pt}\selectfont ~~~~~}\huxbpad{4pt}} &
\multicolumn{1}{r!{\huxvb{0}}}{\huxtpad{4pt}\raggedleft {\fontsize{9.5pt}{11.4pt}\selectfont 0.002~~}\huxbpad{4pt}} &
\multicolumn{1}{r!{\huxvb{0}}}{\huxtpad{4pt}\raggedleft {\fontsize{9.5pt}{11.4pt}\selectfont 0.004~~~}\huxbpad{4pt}} &
\multicolumn{1}{r!{\huxvb{0}}}{\huxtpad{4pt}\raggedleft {\fontsize{9.5pt}{11.4pt}\selectfont 0.004~~~}\huxbpad{4pt}} \tabularnewline[-0.5pt]


\hhline{}
\arrayrulecolor{black}

\multicolumn{1}{!{\huxvb{0}}l!{\huxvb{0}}}{\huxtpad{4pt}\raggedright {\fontsize{9.5pt}{11.4pt}\selectfont }\huxbpad{4pt}} &
\multicolumn{1}{r!{\huxvb{0}}}{\huxtpad{4pt}\raggedleft {\fontsize{9.5pt}{11.4pt}\selectfont ~~~~~}\huxbpad{4pt}} &
\multicolumn{1}{r!{\huxvb{0}}}{\huxtpad{4pt}\raggedleft {\fontsize{9.5pt}{11.4pt}\selectfont (0.010)~}\huxbpad{4pt}} &
\multicolumn{1}{r!{\huxvb{0}}}{\huxtpad{4pt}\raggedleft {\fontsize{9.5pt}{11.4pt}\selectfont (0.012)~~}\huxbpad{4pt}} &
\multicolumn{1}{r!{\huxvb{0}}}{\huxtpad{4pt}\raggedleft {\fontsize{9.5pt}{11.4pt}\selectfont (0.012)~~}\huxbpad{4pt}} \tabularnewline[-0.5pt]


\hhline{}
\arrayrulecolor{black}

\multicolumn{1}{!{\huxvb{0}}l!{\huxvb{0}}}{\huxtpad{4pt}\raggedright {\fontsize{9.5pt}{11.4pt}\selectfont Foreign Born}\huxbpad{4pt}} &
\multicolumn{1}{r!{\huxvb{0}}}{\huxtpad{4pt}\raggedleft {\fontsize{9.5pt}{11.4pt}\selectfont ~~~~~}\huxbpad{4pt}} &
\multicolumn{1}{r!{\huxvb{0}}}{\huxtpad{4pt}\raggedleft {\fontsize{9.5pt}{11.4pt}\selectfont -0.075~~}\huxbpad{4pt}} &
\multicolumn{1}{r!{\huxvb{0}}}{\huxtpad{4pt}\raggedleft {\fontsize{9.5pt}{11.4pt}\selectfont -0.084~~~}\huxbpad{4pt}} &
\multicolumn{1}{r!{\huxvb{0}}}{\huxtpad{4pt}\raggedleft {\fontsize{9.5pt}{11.4pt}\selectfont -0.016~~~}\huxbpad{4pt}} \tabularnewline[-0.5pt]


\hhline{}
\arrayrulecolor{black}

\multicolumn{1}{!{\huxvb{0}}l!{\huxvb{0}}}{\huxtpad{4pt}\raggedright {\fontsize{9.5pt}{11.4pt}\selectfont }\huxbpad{4pt}} &
\multicolumn{1}{r!{\huxvb{0}}}{\huxtpad{4pt}\raggedleft {\fontsize{9.5pt}{11.4pt}\selectfont ~~~~~}\huxbpad{4pt}} &
\multicolumn{1}{r!{\huxvb{0}}}{\huxtpad{4pt}\raggedleft {\fontsize{9.5pt}{11.4pt}\selectfont (0.399)~}\huxbpad{4pt}} &
\multicolumn{1}{r!{\huxvb{0}}}{\huxtpad{4pt}\raggedleft {\fontsize{9.5pt}{11.4pt}\selectfont (0.428)~~}\huxbpad{4pt}} &
\multicolumn{1}{r!{\huxvb{0}}}{\huxtpad{4pt}\raggedleft {\fontsize{9.5pt}{11.4pt}\selectfont (0.330)~~}\huxbpad{4pt}} \tabularnewline[-0.5pt]


\hhline{}
\arrayrulecolor{black}

\multicolumn{1}{!{\huxvb{0}}l!{\huxvb{0}}}{\huxtpad{4pt}\raggedright {\fontsize{9.5pt}{11.4pt}\selectfont Male}\huxbpad{4pt}} &
\multicolumn{1}{r!{\huxvb{0}}}{\huxtpad{4pt}\raggedleft {\fontsize{9.5pt}{11.4pt}\selectfont ~~~~~}\huxbpad{4pt}} &
\multicolumn{1}{r!{\huxvb{0}}}{\huxtpad{4pt}\raggedleft {\fontsize{9.5pt}{11.4pt}\selectfont 0.228~~}\huxbpad{4pt}} &
\multicolumn{1}{r!{\huxvb{0}}}{\huxtpad{4pt}\raggedleft {\fontsize{9.5pt}{11.4pt}\selectfont 0.294~~~}\huxbpad{4pt}} &
\multicolumn{1}{r!{\huxvb{0}}}{\huxtpad{4pt}\raggedleft {\fontsize{9.5pt}{11.4pt}\selectfont 0.264~~~}\huxbpad{4pt}} \tabularnewline[-0.5pt]


\hhline{}
\arrayrulecolor{black}

\multicolumn{1}{!{\huxvb{0}}l!{\huxvb{0}}}{\huxtpad{4pt}\raggedright {\fontsize{9.5pt}{11.4pt}\selectfont }\huxbpad{4pt}} &
\multicolumn{1}{r!{\huxvb{0}}}{\huxtpad{4pt}\raggedleft {\fontsize{9.5pt}{11.4pt}\selectfont ~~~~~}\huxbpad{4pt}} &
\multicolumn{1}{r!{\huxvb{0}}}{\huxtpad{4pt}\raggedleft {\fontsize{9.5pt}{11.4pt}\selectfont (0.290)~}\huxbpad{4pt}} &
\multicolumn{1}{r!{\huxvb{0}}}{\huxtpad{4pt}\raggedleft {\fontsize{9.5pt}{11.4pt}\selectfont (0.294)~~}\huxbpad{4pt}} &
\multicolumn{1}{r!{\huxvb{0}}}{\huxtpad{4pt}\raggedleft {\fontsize{9.5pt}{11.4pt}\selectfont (0.295)~~}\huxbpad{4pt}} \tabularnewline[-0.5pt]


\hhline{}
\arrayrulecolor{black}

\multicolumn{1}{!{\huxvb{0}}l!{\huxvb{0}}}{\huxtpad{4pt}\raggedright {\fontsize{9.5pt}{11.4pt}\selectfont Children Present}\huxbpad{4pt}} &
\multicolumn{1}{r!{\huxvb{0}}}{\huxtpad{4pt}\raggedleft {\fontsize{9.5pt}{11.4pt}\selectfont ~~~~~}\huxbpad{4pt}} &
\multicolumn{1}{r!{\huxvb{0}}}{\huxtpad{4pt}\raggedleft {\fontsize{9.5pt}{11.4pt}\selectfont -0.678 *}\huxbpad{4pt}} &
\multicolumn{1}{r!{\huxvb{0}}}{\huxtpad{4pt}\raggedleft {\fontsize{9.5pt}{11.4pt}\selectfont -0.665~~~}\huxbpad{4pt}} &
\multicolumn{1}{r!{\huxvb{0}}}{\huxtpad{4pt}\raggedleft {\fontsize{9.5pt}{11.4pt}\selectfont -0.573~~~}\huxbpad{4pt}} \tabularnewline[-0.5pt]


\hhline{}
\arrayrulecolor{black}

\multicolumn{1}{!{\huxvb{0}}l!{\huxvb{0}}}{\huxtpad{4pt}\raggedright {\fontsize{9.5pt}{11.4pt}\selectfont }\huxbpad{4pt}} &
\multicolumn{1}{r!{\huxvb{0}}}{\huxtpad{4pt}\raggedleft {\fontsize{9.5pt}{11.4pt}\selectfont ~~~~~}\huxbpad{4pt}} &
\multicolumn{1}{r!{\huxvb{0}}}{\huxtpad{4pt}\raggedleft {\fontsize{9.5pt}{11.4pt}\selectfont (0.345)~}\huxbpad{4pt}} &
\multicolumn{1}{r!{\huxvb{0}}}{\huxtpad{4pt}\raggedleft {\fontsize{9.5pt}{11.4pt}\selectfont (0.367)~~}\huxbpad{4pt}} &
\multicolumn{1}{r!{\huxvb{0}}}{\huxtpad{4pt}\raggedleft {\fontsize{9.5pt}{11.4pt}\selectfont (0.360)~~}\huxbpad{4pt}} \tabularnewline[-0.5pt]


\hhline{}
\arrayrulecolor{black}

\multicolumn{1}{!{\huxvb{0}}l!{\huxvb{0}}}{\huxtpad{4pt}\raggedright {\fontsize{9.5pt}{11.4pt}\selectfont Married}\huxbpad{4pt}} &
\multicolumn{1}{r!{\huxvb{0}}}{\huxtpad{4pt}\raggedleft {\fontsize{9.5pt}{11.4pt}\selectfont ~~~~~}\huxbpad{4pt}} &
\multicolumn{1}{r!{\huxvb{0}}}{\huxtpad{4pt}\raggedleft {\fontsize{9.5pt}{11.4pt}\selectfont 0.617 *}\huxbpad{4pt}} &
\multicolumn{1}{r!{\huxvb{0}}}{\huxtpad{4pt}\raggedleft {\fontsize{9.5pt}{11.4pt}\selectfont 0.575~~~}\huxbpad{4pt}} &
\multicolumn{1}{r!{\huxvb{0}}}{\huxtpad{4pt}\raggedleft {\fontsize{9.5pt}{11.4pt}\selectfont 0.483~~~}\huxbpad{4pt}} \tabularnewline[-0.5pt]


\hhline{}
\arrayrulecolor{black}

\multicolumn{1}{!{\huxvb{0}}l!{\huxvb{0}}}{\huxtpad{4pt}\raggedright {\fontsize{9.5pt}{11.4pt}\selectfont }\huxbpad{4pt}} &
\multicolumn{1}{r!{\huxvb{0}}}{\huxtpad{4pt}\raggedleft {\fontsize{9.5pt}{11.4pt}\selectfont ~~~~~}\huxbpad{4pt}} &
\multicolumn{1}{r!{\huxvb{0}}}{\huxtpad{4pt}\raggedleft {\fontsize{9.5pt}{11.4pt}\selectfont (0.312)~}\huxbpad{4pt}} &
\multicolumn{1}{r!{\huxvb{0}}}{\huxtpad{4pt}\raggedleft {\fontsize{9.5pt}{11.4pt}\selectfont (0.319)~~}\huxbpad{4pt}} &
\multicolumn{1}{r!{\huxvb{0}}}{\huxtpad{4pt}\raggedleft {\fontsize{9.5pt}{11.4pt}\selectfont (0.307)~~}\huxbpad{4pt}} \tabularnewline[-0.5pt]


\hhline{}
\arrayrulecolor{black}

\multicolumn{1}{!{\huxvb{0}}l!{\huxvb{0}}}{\huxtpad{4pt}\raggedright {\fontsize{9.5pt}{11.4pt}\selectfont Socioeconomic}\huxbpad{4pt}} &
\multicolumn{1}{r!{\huxvb{0}}}{\huxtpad{4pt}\raggedleft {\fontsize{9.5pt}{11.4pt}\selectfont ~~~~~}\huxbpad{4pt}} &
\multicolumn{1}{r!{\huxvb{0}}}{\huxtpad{4pt}\raggedleft {\fontsize{9.5pt}{11.4pt}\selectfont ~~~~~~}\huxbpad{4pt}} &
\multicolumn{1}{r!{\huxvb{0}}}{\huxtpad{4pt}\raggedleft {\fontsize{9.5pt}{11.4pt}\selectfont ~~~~~~~}\huxbpad{4pt}} &
\multicolumn{1}{r!{\huxvb{0}}}{\huxtpad{4pt}\raggedleft {\fontsize{9.5pt}{11.4pt}\selectfont ~~~~~~~}\huxbpad{4pt}} \tabularnewline[-0.5pt]


\hhline{}
\arrayrulecolor{black}

\multicolumn{1}{!{\huxvb{0}}l!{\huxvb{0}}}{\huxtpad{4pt}\raggedright {\fontsize{9.5pt}{11.4pt}\selectfont $<$H.S.}\huxbpad{4pt}} &
\multicolumn{1}{r!{\huxvb{0}}}{\huxtpad{4pt}\raggedleft {\fontsize{9.5pt}{11.4pt}\selectfont ~~~~~}\huxbpad{4pt}} &
\multicolumn{1}{r!{\huxvb{0}}}{\huxtpad{4pt}\raggedleft {\fontsize{9.5pt}{11.4pt}\selectfont -1.951 *}\huxbpad{4pt}} &
\multicolumn{1}{r!{\huxvb{0}}}{\huxtpad{4pt}\raggedleft {\fontsize{9.5pt}{11.4pt}\selectfont -1.932 *~}\huxbpad{4pt}} &
\multicolumn{1}{r!{\huxvb{0}}}{\huxtpad{4pt}\raggedleft {\fontsize{9.5pt}{11.4pt}\selectfont -1.885 *~}\huxbpad{4pt}} \tabularnewline[-0.5pt]


\hhline{}
\arrayrulecolor{black}

\multicolumn{1}{!{\huxvb{0}}l!{\huxvb{0}}}{\huxtpad{4pt}\raggedright {\fontsize{9.5pt}{11.4pt}\selectfont }\huxbpad{4pt}} &
\multicolumn{1}{r!{\huxvb{0}}}{\huxtpad{4pt}\raggedleft {\fontsize{9.5pt}{11.4pt}\selectfont ~~~~~}\huxbpad{4pt}} &
\multicolumn{1}{r!{\huxvb{0}}}{\huxtpad{4pt}\raggedleft {\fontsize{9.5pt}{11.4pt}\selectfont (0.889)~}\huxbpad{4pt}} &
\multicolumn{1}{r!{\huxvb{0}}}{\huxtpad{4pt}\raggedleft {\fontsize{9.5pt}{11.4pt}\selectfont (0.867)~~}\huxbpad{4pt}} &
\multicolumn{1}{r!{\huxvb{0}}}{\huxtpad{4pt}\raggedleft {\fontsize{9.5pt}{11.4pt}\selectfont (0.896)~~}\huxbpad{4pt}} \tabularnewline[-0.5pt]


\hhline{}
\arrayrulecolor{black}

\multicolumn{1}{!{\huxvb{0}}l!{\huxvb{0}}}{\huxtpad{4pt}\raggedright {\fontsize{9.5pt}{11.4pt}\selectfont Some college, no B.A.}\huxbpad{4pt}} &
\multicolumn{1}{r!{\huxvb{0}}}{\huxtpad{4pt}\raggedleft {\fontsize{9.5pt}{11.4pt}\selectfont ~~~~~}\huxbpad{4pt}} &
\multicolumn{1}{r!{\huxvb{0}}}{\huxtpad{4pt}\raggedleft {\fontsize{9.5pt}{11.4pt}\selectfont -1.122~~}\huxbpad{4pt}} &
\multicolumn{1}{r!{\huxvb{0}}}{\huxtpad{4pt}\raggedleft {\fontsize{9.5pt}{11.4pt}\selectfont -1.119~~~}\huxbpad{4pt}} &
\multicolumn{1}{r!{\huxvb{0}}}{\huxtpad{4pt}\raggedleft {\fontsize{9.5pt}{11.4pt}\selectfont -1.108~~~}\huxbpad{4pt}} \tabularnewline[-0.5pt]


\hhline{}
\arrayrulecolor{black}

\multicolumn{1}{!{\huxvb{0}}l!{\huxvb{0}}}{\huxtpad{4pt}\raggedright {\fontsize{9.5pt}{11.4pt}\selectfont }\huxbpad{4pt}} &
\multicolumn{1}{r!{\huxvb{0}}}{\huxtpad{4pt}\raggedleft {\fontsize{9.5pt}{11.4pt}\selectfont ~~~~~}\huxbpad{4pt}} &
\multicolumn{1}{r!{\huxvb{0}}}{\huxtpad{4pt}\raggedleft {\fontsize{9.5pt}{11.4pt}\selectfont (0.607)~}\huxbpad{4pt}} &
\multicolumn{1}{r!{\huxvb{0}}}{\huxtpad{4pt}\raggedleft {\fontsize{9.5pt}{11.4pt}\selectfont (0.631)~~}\huxbpad{4pt}} &
\multicolumn{1}{r!{\huxvb{0}}}{\huxtpad{4pt}\raggedleft {\fontsize{9.5pt}{11.4pt}\selectfont (0.625)~~}\huxbpad{4pt}} \tabularnewline[-0.5pt]


\hhline{}
\arrayrulecolor{black}

\multicolumn{1}{!{\huxvb{0}}l!{\huxvb{0}}}{\huxtpad{4pt}\raggedright {\fontsize{9.5pt}{11.4pt}\selectfont B.A.}\huxbpad{4pt}} &
\multicolumn{1}{r!{\huxvb{0}}}{\huxtpad{4pt}\raggedleft {\fontsize{9.5pt}{11.4pt}\selectfont ~~~~~}\huxbpad{4pt}} &
\multicolumn{1}{r!{\huxvb{0}}}{\huxtpad{4pt}\raggedleft {\fontsize{9.5pt}{11.4pt}\selectfont -0.933~~}\huxbpad{4pt}} &
\multicolumn{1}{r!{\huxvb{0}}}{\huxtpad{4pt}\raggedleft {\fontsize{9.5pt}{11.4pt}\selectfont -0.922~~~}\huxbpad{4pt}} &
\multicolumn{1}{r!{\huxvb{0}}}{\huxtpad{4pt}\raggedleft {\fontsize{9.5pt}{11.4pt}\selectfont -0.991~~~}\huxbpad{4pt}} \tabularnewline[-0.5pt]


\hhline{}
\arrayrulecolor{black}

\multicolumn{1}{!{\huxvb{0}}l!{\huxvb{0}}}{\huxtpad{4pt}\raggedright {\fontsize{9.5pt}{11.4pt}\selectfont }\huxbpad{4pt}} &
\multicolumn{1}{r!{\huxvb{0}}}{\huxtpad{4pt}\raggedleft {\fontsize{9.5pt}{11.4pt}\selectfont ~~~~~}\huxbpad{4pt}} &
\multicolumn{1}{r!{\huxvb{0}}}{\huxtpad{4pt}\raggedleft {\fontsize{9.5pt}{11.4pt}\selectfont (0.578)~}\huxbpad{4pt}} &
\multicolumn{1}{r!{\huxvb{0}}}{\huxtpad{4pt}\raggedleft {\fontsize{9.5pt}{11.4pt}\selectfont (0.598)~~}\huxbpad{4pt}} &
\multicolumn{1}{r!{\huxvb{0}}}{\huxtpad{4pt}\raggedleft {\fontsize{9.5pt}{11.4pt}\selectfont (0.594)~~}\huxbpad{4pt}} \tabularnewline[-0.5pt]


\hhline{}
\arrayrulecolor{black}

\multicolumn{1}{!{\huxvb{0}}l!{\huxvb{0}}}{\huxtpad{4pt}\raggedright {\fontsize{9.5pt}{11.4pt}\selectfont Neighborhood perceptions}\huxbpad{4pt}} &
\multicolumn{1}{r!{\huxvb{0}}}{\huxtpad{4pt}\raggedleft {\fontsize{9.5pt}{11.4pt}\selectfont ~~~~~}\huxbpad{4pt}} &
\multicolumn{1}{r!{\huxvb{0}}}{\huxtpad{4pt}\raggedleft {\fontsize{9.5pt}{11.4pt}\selectfont ~~~~~~}\huxbpad{4pt}} &
\multicolumn{1}{r!{\huxvb{0}}}{\huxtpad{4pt}\raggedleft {\fontsize{9.5pt}{11.4pt}\selectfont ~~~~~~~}\huxbpad{4pt}} &
\multicolumn{1}{r!{\huxvb{0}}}{\huxtpad{4pt}\raggedleft {\fontsize{9.5pt}{11.4pt}\selectfont ~~~~~~~}\huxbpad{4pt}} \tabularnewline[-0.5pt]


\hhline{}
\arrayrulecolor{black}

\multicolumn{1}{!{\huxvb{0}}l!{\huxvb{0}}}{\huxtpad{4pt}\raggedright {\fontsize{9.5pt}{11.4pt}\selectfont M.A.+}\huxbpad{4pt}} &
\multicolumn{1}{r!{\huxvb{0}}}{\huxtpad{4pt}\raggedleft {\fontsize{9.5pt}{11.4pt}\selectfont ~~~~~}\huxbpad{4pt}} &
\multicolumn{1}{r!{\huxvb{0}}}{\huxtpad{4pt}\raggedleft {\fontsize{9.5pt}{11.4pt}\selectfont -1.367 *}\huxbpad{4pt}} &
\multicolumn{1}{r!{\huxvb{0}}}{\huxtpad{4pt}\raggedleft {\fontsize{9.5pt}{11.4pt}\selectfont -1.353 *~}\huxbpad{4pt}} &
\multicolumn{1}{r!{\huxvb{0}}}{\huxtpad{4pt}\raggedleft {\fontsize{9.5pt}{11.4pt}\selectfont -1.415 *~}\huxbpad{4pt}} \tabularnewline[-0.5pt]


\hhline{}
\arrayrulecolor{black}

\multicolumn{1}{!{\huxvb{0}}l!{\huxvb{0}}}{\huxtpad{4pt}\raggedright {\fontsize{9.5pt}{11.4pt}\selectfont }\huxbpad{4pt}} &
\multicolumn{1}{r!{\huxvb{0}}}{\huxtpad{4pt}\raggedleft {\fontsize{9.5pt}{11.4pt}\selectfont ~~~~~}\huxbpad{4pt}} &
\multicolumn{1}{r!{\huxvb{0}}}{\huxtpad{4pt}\raggedleft {\fontsize{9.5pt}{11.4pt}\selectfont (0.562)~}\huxbpad{4pt}} &
\multicolumn{1}{r!{\huxvb{0}}}{\huxtpad{4pt}\raggedleft {\fontsize{9.5pt}{11.4pt}\selectfont (0.568)~~}\huxbpad{4pt}} &
\multicolumn{1}{r!{\huxvb{0}}}{\huxtpad{4pt}\raggedleft {\fontsize{9.5pt}{11.4pt}\selectfont (0.564)~~}\huxbpad{4pt}} \tabularnewline[-0.5pt]


\hhline{}
\arrayrulecolor{black}

\multicolumn{1}{!{\huxvb{0}}l!{\huxvb{0}}}{\huxtpad{4pt}\raggedright {\fontsize{9.5pt}{11.4pt}\selectfont Years in neighborhood}\huxbpad{4pt}} &
\multicolumn{1}{r!{\huxvb{0}}}{\huxtpad{4pt}\raggedleft {\fontsize{9.5pt}{11.4pt}\selectfont ~~~~~}\huxbpad{4pt}} &
\multicolumn{1}{r!{\huxvb{0}}}{\huxtpad{4pt}\raggedleft {\fontsize{9.5pt}{11.4pt}\selectfont ~~~~~~}\huxbpad{4pt}} &
\multicolumn{1}{r!{\huxvb{0}}}{\huxtpad{4pt}\raggedleft {\fontsize{9.5pt}{11.4pt}\selectfont -0.009~~~}\huxbpad{4pt}} &
\multicolumn{1}{r!{\huxvb{0}}}{\huxtpad{4pt}\raggedleft {\fontsize{9.5pt}{11.4pt}\selectfont -0.010~~~}\huxbpad{4pt}} \tabularnewline[-0.5pt]


\hhline{}
\arrayrulecolor{black}

\multicolumn{1}{!{\huxvb{0}}l!{\huxvb{0}}}{\huxtpad{4pt}\raggedright {\fontsize{9.5pt}{11.4pt}\selectfont }\huxbpad{4pt}} &
\multicolumn{1}{r!{\huxvb{0}}}{\huxtpad{4pt}\raggedleft {\fontsize{9.5pt}{11.4pt}\selectfont ~~~~~}\huxbpad{4pt}} &
\multicolumn{1}{r!{\huxvb{0}}}{\huxtpad{4pt}\raggedleft {\fontsize{9.5pt}{11.4pt}\selectfont ~~~~~~}\huxbpad{4pt}} &
\multicolumn{1}{r!{\huxvb{0}}}{\huxtpad{4pt}\raggedleft {\fontsize{9.5pt}{11.4pt}\selectfont (0.016)~~}\huxbpad{4pt}} &
\multicolumn{1}{r!{\huxvb{0}}}{\huxtpad{4pt}\raggedleft {\fontsize{9.5pt}{11.4pt}\selectfont (0.016)~~}\huxbpad{4pt}} \tabularnewline[-0.5pt]


\hhline{}
\arrayrulecolor{black}

\multicolumn{1}{!{\huxvb{0}}l!{\huxvb{0}}}{\huxtpad{4pt}\raggedright {\fontsize{9.5pt}{11.4pt}\selectfont 10-50 blocks}\huxbpad{4pt}} &
\multicolumn{1}{r!{\huxvb{0}}}{\huxtpad{4pt}\raggedleft {\fontsize{9.5pt}{11.4pt}\selectfont ~~~~~}\huxbpad{4pt}} &
\multicolumn{1}{r!{\huxvb{0}}}{\huxtpad{4pt}\raggedleft {\fontsize{9.5pt}{11.4pt}\selectfont ~~~~~~}\huxbpad{4pt}} &
\multicolumn{1}{r!{\huxvb{0}}}{\huxtpad{4pt}\raggedleft {\fontsize{9.5pt}{11.4pt}\selectfont 1.100 **}\huxbpad{4pt}} &
\multicolumn{1}{r!{\huxvb{0}}}{\huxtpad{4pt}\raggedleft {\fontsize{9.5pt}{11.4pt}\selectfont 1.125 **}\huxbpad{4pt}} \tabularnewline[-0.5pt]


\hhline{}
\arrayrulecolor{black}

\multicolumn{1}{!{\huxvb{0}}l!{\huxvb{0}}}{\huxtpad{4pt}\raggedright {\fontsize{9.5pt}{11.4pt}\selectfont }\huxbpad{4pt}} &
\multicolumn{1}{r!{\huxvb{0}}}{\huxtpad{4pt}\raggedleft {\fontsize{9.5pt}{11.4pt}\selectfont ~~~~~}\huxbpad{4pt}} &
\multicolumn{1}{r!{\huxvb{0}}}{\huxtpad{4pt}\raggedleft {\fontsize{9.5pt}{11.4pt}\selectfont ~~~~~~}\huxbpad{4pt}} &
\multicolumn{1}{r!{\huxvb{0}}}{\huxtpad{4pt}\raggedleft {\fontsize{9.5pt}{11.4pt}\selectfont (0.384)~~}\huxbpad{4pt}} &
\multicolumn{1}{r!{\huxvb{0}}}{\huxtpad{4pt}\raggedleft {\fontsize{9.5pt}{11.4pt}\selectfont (0.368)~~}\huxbpad{4pt}} \tabularnewline[-0.5pt]


\hhline{}
\arrayrulecolor{black}

\multicolumn{1}{!{\huxvb{0}}l!{\huxvb{0}}}{\huxtpad{4pt}\raggedright {\fontsize{9.5pt}{11.4pt}\selectfont $>$50 blocks}\huxbpad{4pt}} &
\multicolumn{1}{r!{\huxvb{0}}}{\huxtpad{4pt}\raggedleft {\fontsize{9.5pt}{11.4pt}\selectfont ~~~~~}\huxbpad{4pt}} &
\multicolumn{1}{r!{\huxvb{0}}}{\huxtpad{4pt}\raggedleft {\fontsize{9.5pt}{11.4pt}\selectfont ~~~~~~}\huxbpad{4pt}} &
\multicolumn{1}{r!{\huxvb{0}}}{\huxtpad{4pt}\raggedleft {\fontsize{9.5pt}{11.4pt}\selectfont 0.267~~~}\huxbpad{4pt}} &
\multicolumn{1}{r!{\huxvb{0}}}{\huxtpad{4pt}\raggedleft {\fontsize{9.5pt}{11.4pt}\selectfont 0.421~~~}\huxbpad{4pt}} \tabularnewline[-0.5pt]


\hhline{}
\arrayrulecolor{black}

\multicolumn{1}{!{\huxvb{0}}l!{\huxvb{0}}}{\huxtpad{4pt}\raggedright {\fontsize{9.5pt}{11.4pt}\selectfont }\huxbpad{4pt}} &
\multicolumn{1}{r!{\huxvb{0}}}{\huxtpad{4pt}\raggedleft {\fontsize{9.5pt}{11.4pt}\selectfont ~~~~~}\huxbpad{4pt}} &
\multicolumn{1}{r!{\huxvb{0}}}{\huxtpad{4pt}\raggedleft {\fontsize{9.5pt}{11.4pt}\selectfont ~~~~~~}\huxbpad{4pt}} &
\multicolumn{1}{r!{\huxvb{0}}}{\huxtpad{4pt}\raggedleft {\fontsize{9.5pt}{11.4pt}\selectfont (0.601)~~}\huxbpad{4pt}} &
\multicolumn{1}{r!{\huxvb{0}}}{\huxtpad{4pt}\raggedleft {\fontsize{9.5pt}{11.4pt}\selectfont (0.578)~~}\huxbpad{4pt}} \tabularnewline[-0.5pt]


\hhline{>{\huxb{1}}->{\huxb{1}}->{\huxb{1}}->{\huxb{1}}->{\huxb{1}}-}
\arrayrulecolor{black}

\multicolumn{1}{!{\huxvb{0}}l!{\huxvb{0}}}{\huxtpad{4pt}\raggedright {\fontsize{9.5pt}{11.4pt}\selectfont AIC}\huxbpad{4pt}} &
\multicolumn{1}{r!{\huxvb{0}}}{\huxtpad{4pt}\raggedleft {\fontsize{9.5pt}{11.4pt}\selectfont 673.618~}\huxbpad{4pt}} &
\multicolumn{1}{r!{\huxvb{0}}}{\huxtpad{4pt}\raggedleft {\fontsize{9.5pt}{11.4pt}\selectfont 663.346~~}\huxbpad{4pt}} &
\multicolumn{1}{r!{\huxvb{0}}}{\huxtpad{4pt}\raggedleft {\fontsize{9.5pt}{11.4pt}\selectfont 659.927~~~}\huxbpad{4pt}} &
\multicolumn{1}{r!{\huxvb{0}}}{\huxtpad{4pt}\raggedleft {\fontsize{9.5pt}{11.4pt}\selectfont 655.029~~~}\huxbpad{4pt}} \tabularnewline[-0.5pt]


\hhline{>{\huxb{1}}->{\huxb{1}}->{\huxb{1}}->{\huxb{1}}->{\huxb{1}}-}
\arrayrulecolor{black}

\multicolumn{5}{!{\huxvb{0}}p{0.5\textwidth+8\tabcolsep}!{\huxvb{0}}}{\parbox[b]{0.5\textwidth+8\tabcolsep-4pt-4pt}{\huxtpad{4pt}\raggedright {\fontsize{9.5pt}{11.4pt}\selectfont  *** p $<$ 0.001;  ** p $<$ 0.01;  * p $<$ 0.05.}\huxbpad{4pt}}} \tabularnewline[-0.5pt]


\hhline{}
\arrayrulecolor{black}
\end{tabularx}
\end{table}

\begin{table}[h]
\centering\captionsetup{justification=centering,singlelinecheck=off}
\caption{Estimated coefficients predicting neighborhood improvement}
\label{tab:improvement}

    \providecommand{\huxb}[2][0,0,0]{\arrayrulecolor[RGB]{#1}\global\arrayrulewidth=#2pt}
    \providecommand{\huxvb}[2][0,0,0]{\color[RGB]{#1}\vrule width #2pt}
    \providecommand{\huxtpad}[1]{\rule{0pt}{\baselineskip+#1}}
    \providecommand{\huxbpad}[1]{\rule[-#1]{0pt}{#1}}
  \begin{tabularx}{0.5\textwidth}{p{0.125\textwidth} p{0.125\textwidth} p{0.125\textwidth} p{0.125\textwidth}}


\hhline{>{\huxb{0.8}}->{\huxb{0.8}}->{\huxb{0.8}}->{\huxb{0.8}}-}
\arrayrulecolor{black}

\multicolumn{1}{!{\huxvb{0}}c!{\huxvb{0}}}{\huxtpad{4pt}\centering {\fontsize{9.5pt}{11.4pt}\selectfont }\huxbpad{4pt}} &
\multicolumn{1}{c!{\huxvb{0}}}{\huxtpad{4pt}\centering {\fontsize{9.5pt}{11.4pt}\selectfont (1)}\huxbpad{4pt}} &
\multicolumn{1}{c!{\huxvb{0}}}{\huxtpad{4pt}\centering {\fontsize{9.5pt}{11.4pt}\selectfont (2)}\huxbpad{4pt}} &
\multicolumn{1}{c!{\huxvb{0}}}{\huxtpad{4pt}\centering {\fontsize{9.5pt}{11.4pt}\selectfont (3)}\huxbpad{4pt}} \tabularnewline[-0.5pt]


\hhline{>{\huxb{1}}->{\huxb{1}}->{\huxb{1}}->{\huxb{1}}-}
\arrayrulecolor{black}

\multicolumn{1}{!{\huxvb{0}}l!{\huxvb{0}}}{\huxtpad{4pt}\raggedright {\fontsize{9.5pt}{11.4pt}\selectfont (Intercept)}\huxbpad{4pt}} &
\multicolumn{1}{r!{\huxvb{0}}}{\huxtpad{4pt}\raggedleft {\fontsize{9.5pt}{11.4pt}\selectfont -20.595 ***}\huxbpad{4pt}} &
\multicolumn{1}{r!{\huxvb{0}}}{\huxtpad{4pt}\raggedleft {\fontsize{9.5pt}{11.4pt}\selectfont -20.210 ***}\huxbpad{4pt}} &
\multicolumn{1}{r!{\huxvb{0}}}{\huxtpad{4pt}\raggedleft {\fontsize{9.5pt}{11.4pt}\selectfont -21.999 ***}\huxbpad{4pt}} \tabularnewline[-0.5pt]


\hhline{}
\arrayrulecolor{black}

\multicolumn{1}{!{\huxvb{0}}l!{\huxvb{0}}}{\huxtpad{4pt}\raggedright {\fontsize{9.5pt}{11.4pt}\selectfont }\huxbpad{4pt}} &
\multicolumn{1}{r!{\huxvb{0}}}{\huxtpad{4pt}\raggedleft {\fontsize{9.5pt}{11.4pt}\selectfont (0.948)~~~}\huxbpad{4pt}} &
\multicolumn{1}{r!{\huxvb{0}}}{\huxtpad{4pt}\raggedleft {\fontsize{9.5pt}{11.4pt}\selectfont (1.173)~~~}\huxbpad{4pt}} &
\multicolumn{1}{r!{\huxvb{0}}}{\huxtpad{4pt}\raggedleft {\fontsize{9.5pt}{11.4pt}\selectfont (1.091)~~~}\huxbpad{4pt}} \tabularnewline[-0.5pt]


\hhline{}
\arrayrulecolor{black}

\multicolumn{1}{!{\huxvb{0}}l!{\huxvb{0}}}{\huxtpad{4pt}\raggedright {\fontsize{9.5pt}{11.4pt}\selectfont Race}\huxbpad{4pt}} &
\multicolumn{1}{r!{\huxvb{0}}}{\huxtpad{4pt}\raggedleft {\fontsize{9.5pt}{11.4pt}\selectfont ~~~~~~~~}\huxbpad{4pt}} &
\multicolumn{1}{r!{\huxvb{0}}}{\huxtpad{4pt}\raggedleft {\fontsize{9.5pt}{11.4pt}\selectfont ~~~~~~~~}\huxbpad{4pt}} &
\multicolumn{1}{r!{\huxvb{0}}}{\huxtpad{4pt}\raggedleft {\fontsize{9.5pt}{11.4pt}\selectfont ~~~~~~~~}\huxbpad{4pt}} \tabularnewline[-0.5pt]


\hhline{}
\arrayrulecolor{black}

\multicolumn{1}{!{\huxvb{0}}l!{\huxvb{0}}}{\huxtpad{4pt}\raggedright {\fontsize{9.5pt}{11.4pt}\selectfont Asian}\huxbpad{4pt}} &
\multicolumn{1}{r!{\huxvb{0}}}{\huxtpad{4pt}\raggedleft {\fontsize{9.5pt}{11.4pt}\selectfont 1.325 **~}\huxbpad{4pt}} &
\multicolumn{1}{r!{\huxvb{0}}}{\huxtpad{4pt}\raggedleft {\fontsize{9.5pt}{11.4pt}\selectfont 0.909~~~~}\huxbpad{4pt}} &
\multicolumn{1}{r!{\huxvb{0}}}{\huxtpad{4pt}\raggedleft {\fontsize{9.5pt}{11.4pt}\selectfont 1.055 *~~}\huxbpad{4pt}} \tabularnewline[-0.5pt]


\hhline{}
\arrayrulecolor{black}

\multicolumn{1}{!{\huxvb{0}}l!{\huxvb{0}}}{\huxtpad{4pt}\raggedright {\fontsize{9.5pt}{11.4pt}\selectfont }\huxbpad{4pt}} &
\multicolumn{1}{r!{\huxvb{0}}}{\huxtpad{4pt}\raggedleft {\fontsize{9.5pt}{11.4pt}\selectfont (0.423)~~~}\huxbpad{4pt}} &
\multicolumn{1}{r!{\huxvb{0}}}{\huxtpad{4pt}\raggedleft {\fontsize{9.5pt}{11.4pt}\selectfont (0.519)~~~}\huxbpad{4pt}} &
\multicolumn{1}{r!{\huxvb{0}}}{\huxtpad{4pt}\raggedleft {\fontsize{9.5pt}{11.4pt}\selectfont (0.509)~~~}\huxbpad{4pt}} \tabularnewline[-0.5pt]


\hhline{}
\arrayrulecolor{black}

\multicolumn{1}{!{\huxvb{0}}l!{\huxvb{0}}}{\huxtpad{4pt}\raggedright {\fontsize{9.5pt}{11.4pt}\selectfont Black}\huxbpad{4pt}} &
\multicolumn{1}{r!{\huxvb{0}}}{\huxtpad{4pt}\raggedleft {\fontsize{9.5pt}{11.4pt}\selectfont 1.435 **~}\huxbpad{4pt}} &
\multicolumn{1}{r!{\huxvb{0}}}{\huxtpad{4pt}\raggedleft {\fontsize{9.5pt}{11.4pt}\selectfont 1.401 **~}\huxbpad{4pt}} &
\multicolumn{1}{r!{\huxvb{0}}}{\huxtpad{4pt}\raggedleft {\fontsize{9.5pt}{11.4pt}\selectfont 0.937 *~~}\huxbpad{4pt}} \tabularnewline[-0.5pt]


\hhline{}
\arrayrulecolor{black}

\multicolumn{1}{!{\huxvb{0}}l!{\huxvb{0}}}{\huxtpad{4pt}\raggedright {\fontsize{9.5pt}{11.4pt}\selectfont }\huxbpad{4pt}} &
\multicolumn{1}{r!{\huxvb{0}}}{\huxtpad{4pt}\raggedleft {\fontsize{9.5pt}{11.4pt}\selectfont (0.457)~~~}\huxbpad{4pt}} &
\multicolumn{1}{r!{\huxvb{0}}}{\huxtpad{4pt}\raggedleft {\fontsize{9.5pt}{11.4pt}\selectfont (0.474)~~~}\huxbpad{4pt}} &
\multicolumn{1}{r!{\huxvb{0}}}{\huxtpad{4pt}\raggedleft {\fontsize{9.5pt}{11.4pt}\selectfont (0.465)~~~}\huxbpad{4pt}} \tabularnewline[-0.5pt]


\hhline{}
\arrayrulecolor{black}

\multicolumn{1}{!{\huxvb{0}}l!{\huxvb{0}}}{\huxtpad{4pt}\raggedright {\fontsize{9.5pt}{11.4pt}\selectfont Latinx}\huxbpad{4pt}} &
\multicolumn{1}{r!{\huxvb{0}}}{\huxtpad{4pt}\raggedleft {\fontsize{9.5pt}{11.4pt}\selectfont 1.641 ***}\huxbpad{4pt}} &
\multicolumn{1}{r!{\huxvb{0}}}{\huxtpad{4pt}\raggedleft {\fontsize{9.5pt}{11.4pt}\selectfont 1.280 *~~}\huxbpad{4pt}} &
\multicolumn{1}{r!{\huxvb{0}}}{\huxtpad{4pt}\raggedleft {\fontsize{9.5pt}{11.4pt}\selectfont 1.341 *~~}\huxbpad{4pt}} \tabularnewline[-0.5pt]


\hhline{}
\arrayrulecolor{black}

\multicolumn{1}{!{\huxvb{0}}l!{\huxvb{0}}}{\huxtpad{4pt}\raggedright {\fontsize{9.5pt}{11.4pt}\selectfont }\huxbpad{4pt}} &
\multicolumn{1}{r!{\huxvb{0}}}{\huxtpad{4pt}\raggedleft {\fontsize{9.5pt}{11.4pt}\selectfont (0.463)~~~}\huxbpad{4pt}} &
\multicolumn{1}{r!{\huxvb{0}}}{\huxtpad{4pt}\raggedleft {\fontsize{9.5pt}{11.4pt}\selectfont (0.515)~~~}\huxbpad{4pt}} &
\multicolumn{1}{r!{\huxvb{0}}}{\huxtpad{4pt}\raggedleft {\fontsize{9.5pt}{11.4pt}\selectfont (0.548)~~~}\huxbpad{4pt}} \tabularnewline[-0.5pt]


\hhline{}
\arrayrulecolor{black}

\multicolumn{1}{!{\huxvb{0}}l!{\huxvb{0}}}{\huxtpad{4pt}\raggedright {\fontsize{9.5pt}{11.4pt}\selectfont Demographics}\huxbpad{4pt}} &
\multicolumn{1}{r!{\huxvb{0}}}{\huxtpad{4pt}\raggedleft {\fontsize{9.5pt}{11.4pt}\selectfont ~~~~~~~~}\huxbpad{4pt}} &
\multicolumn{1}{r!{\huxvb{0}}}{\huxtpad{4pt}\raggedleft {\fontsize{9.5pt}{11.4pt}\selectfont ~~~~~~~~}\huxbpad{4pt}} &
\multicolumn{1}{r!{\huxvb{0}}}{\huxtpad{4pt}\raggedleft {\fontsize{9.5pt}{11.4pt}\selectfont ~~~~~~~~}\huxbpad{4pt}} \tabularnewline[-0.5pt]


\hhline{}
\arrayrulecolor{black}

\multicolumn{1}{!{\huxvb{0}}l!{\huxvb{0}}}{\huxtpad{4pt}\raggedright {\fontsize{9.5pt}{11.4pt}\selectfont Age}\huxbpad{4pt}} &
\multicolumn{1}{r!{\huxvb{0}}}{\huxtpad{4pt}\raggedleft {\fontsize{9.5pt}{11.4pt}\selectfont 20.548 ***}\huxbpad{4pt}} &
\multicolumn{1}{r!{\huxvb{0}}}{\huxtpad{4pt}\raggedleft {\fontsize{9.5pt}{11.4pt}\selectfont 20.168 ***}\huxbpad{4pt}} &
\multicolumn{1}{r!{\huxvb{0}}}{\huxtpad{4pt}\raggedleft {\fontsize{9.5pt}{11.4pt}\selectfont 19.561 ***}\huxbpad{4pt}} \tabularnewline[-0.5pt]


\hhline{}
\arrayrulecolor{black}

\multicolumn{1}{!{\huxvb{0}}l!{\huxvb{0}}}{\huxtpad{4pt}\raggedright {\fontsize{9.5pt}{11.4pt}\selectfont }\huxbpad{4pt}} &
\multicolumn{1}{r!{\huxvb{0}}}{\huxtpad{4pt}\raggedleft {\fontsize{9.5pt}{11.4pt}\selectfont (1.401)~~~}\huxbpad{4pt}} &
\multicolumn{1}{r!{\huxvb{0}}}{\huxtpad{4pt}\raggedleft {\fontsize{9.5pt}{11.4pt}\selectfont (1.574)~~~}\huxbpad{4pt}} &
\multicolumn{1}{r!{\huxvb{0}}}{\huxtpad{4pt}\raggedleft {\fontsize{9.5pt}{11.4pt}\selectfont (1.393)~~~}\huxbpad{4pt}} \tabularnewline[-0.5pt]


\hhline{}
\arrayrulecolor{black}

\multicolumn{1}{!{\huxvb{0}}l!{\huxvb{0}}}{\huxtpad{4pt}\raggedright {\fontsize{9.5pt}{11.4pt}\selectfont Foreign Born}\huxbpad{4pt}} &
\multicolumn{1}{r!{\huxvb{0}}}{\huxtpad{4pt}\raggedleft {\fontsize{9.5pt}{11.4pt}\selectfont 17.919 ***}\huxbpad{4pt}} &
\multicolumn{1}{r!{\huxvb{0}}}{\huxtpad{4pt}\raggedleft {\fontsize{9.5pt}{11.4pt}\selectfont 17.708 ***}\huxbpad{4pt}} &
\multicolumn{1}{r!{\huxvb{0}}}{\huxtpad{4pt}\raggedleft {\fontsize{9.5pt}{11.4pt}\selectfont 18.553 ***}\huxbpad{4pt}} \tabularnewline[-0.5pt]


\hhline{}
\arrayrulecolor{black}

\multicolumn{1}{!{\huxvb{0}}l!{\huxvb{0}}}{\huxtpad{4pt}\raggedright {\fontsize{9.5pt}{11.4pt}\selectfont }\huxbpad{4pt}} &
\multicolumn{1}{r!{\huxvb{0}}}{\huxtpad{4pt}\raggedleft {\fontsize{9.5pt}{11.4pt}\selectfont (1.468)~~~}\huxbpad{4pt}} &
\multicolumn{1}{r!{\huxvb{0}}}{\huxtpad{4pt}\raggedleft {\fontsize{9.5pt}{11.4pt}\selectfont (1.576)~~~}\huxbpad{4pt}} &
\multicolumn{1}{r!{\huxvb{0}}}{\huxtpad{4pt}\raggedleft {\fontsize{9.5pt}{11.4pt}\selectfont (1.364)~~~}\huxbpad{4pt}} \tabularnewline[-0.5pt]


\hhline{}
\arrayrulecolor{black}

\multicolumn{1}{!{\huxvb{0}}l!{\huxvb{0}}}{\huxtpad{4pt}\raggedright {\fontsize{9.5pt}{11.4pt}\selectfont Male}\huxbpad{4pt}} &
\multicolumn{1}{r!{\huxvb{0}}}{\huxtpad{4pt}\raggedleft {\fontsize{9.5pt}{11.4pt}\selectfont 1.139~~~~}\huxbpad{4pt}} &
\multicolumn{1}{r!{\huxvb{0}}}{\huxtpad{4pt}\raggedleft {\fontsize{9.5pt}{11.4pt}\selectfont 0.192~~~~}\huxbpad{4pt}} &
\multicolumn{1}{r!{\huxvb{0}}}{\huxtpad{4pt}\raggedleft {\fontsize{9.5pt}{11.4pt}\selectfont 0.445~~~~}\huxbpad{4pt}} \tabularnewline[-0.5pt]


\hhline{}
\arrayrulecolor{black}

\multicolumn{1}{!{\huxvb{0}}l!{\huxvb{0}}}{\huxtpad{4pt}\raggedright {\fontsize{9.5pt}{11.4pt}\selectfont }\huxbpad{4pt}} &
\multicolumn{1}{r!{\huxvb{0}}}{\huxtpad{4pt}\raggedleft {\fontsize{9.5pt}{11.4pt}\selectfont (1.112)~~~}\huxbpad{4pt}} &
\multicolumn{1}{r!{\huxvb{0}}}{\huxtpad{4pt}\raggedleft {\fontsize{9.5pt}{11.4pt}\selectfont (1.395)~~~}\huxbpad{4pt}} &
\multicolumn{1}{r!{\huxvb{0}}}{\huxtpad{4pt}\raggedleft {\fontsize{9.5pt}{11.4pt}\selectfont (1.509)~~~}\huxbpad{4pt}} \tabularnewline[-0.5pt]


\hhline{}
\arrayrulecolor{black}

\multicolumn{1}{!{\huxvb{0}}l!{\huxvb{0}}}{\huxtpad{4pt}\raggedright {\fontsize{9.5pt}{11.4pt}\selectfont Children Present}\huxbpad{4pt}} &
\multicolumn{1}{r!{\huxvb{0}}}{\huxtpad{4pt}\raggedleft {\fontsize{9.5pt}{11.4pt}\selectfont 0.780~~~~}\huxbpad{4pt}} &
\multicolumn{1}{r!{\huxvb{0}}}{\huxtpad{4pt}\raggedleft {\fontsize{9.5pt}{11.4pt}\selectfont 0.782~~~~}\huxbpad{4pt}} &
\multicolumn{1}{r!{\huxvb{0}}}{\huxtpad{4pt}\raggedleft {\fontsize{9.5pt}{11.4pt}\selectfont -0.031~~~~}\huxbpad{4pt}} \tabularnewline[-0.5pt]


\hhline{}
\arrayrulecolor{black}

\multicolumn{1}{!{\huxvb{0}}l!{\huxvb{0}}}{\huxtpad{4pt}\raggedright {\fontsize{9.5pt}{11.4pt}\selectfont }\huxbpad{4pt}} &
\multicolumn{1}{r!{\huxvb{0}}}{\huxtpad{4pt}\raggedleft {\fontsize{9.5pt}{11.4pt}\selectfont (1.293)~~~}\huxbpad{4pt}} &
\multicolumn{1}{r!{\huxvb{0}}}{\huxtpad{4pt}\raggedleft {\fontsize{9.5pt}{11.4pt}\selectfont (1.392)~~~}\huxbpad{4pt}} &
\multicolumn{1}{r!{\huxvb{0}}}{\huxtpad{4pt}\raggedleft {\fontsize{9.5pt}{11.4pt}\selectfont (1.469)~~~}\huxbpad{4pt}} \tabularnewline[-0.5pt]


\hhline{}
\arrayrulecolor{black}

\multicolumn{1}{!{\huxvb{0}}l!{\huxvb{0}}}{\huxtpad{4pt}\raggedright {\fontsize{9.5pt}{11.4pt}\selectfont Married}\huxbpad{4pt}} &
\multicolumn{1}{r!{\huxvb{0}}}{\huxtpad{4pt}\raggedleft {\fontsize{9.5pt}{11.4pt}\selectfont 16.796 ***}\huxbpad{4pt}} &
\multicolumn{1}{r!{\huxvb{0}}}{\huxtpad{4pt}\raggedleft {\fontsize{9.5pt}{11.4pt}\selectfont 16.408 ***}\huxbpad{4pt}} &
\multicolumn{1}{r!{\huxvb{0}}}{\huxtpad{4pt}\raggedleft {\fontsize{9.5pt}{11.4pt}\selectfont 18.279 ***}\huxbpad{4pt}} \tabularnewline[-0.5pt]


\hhline{}
\arrayrulecolor{black}

\multicolumn{1}{!{\huxvb{0}}l!{\huxvb{0}}}{\huxtpad{4pt}\raggedright {\fontsize{9.5pt}{11.4pt}\selectfont }\huxbpad{4pt}} &
\multicolumn{1}{r!{\huxvb{0}}}{\huxtpad{4pt}\raggedleft {\fontsize{9.5pt}{11.4pt}\selectfont (1.279)~~~}\huxbpad{4pt}} &
\multicolumn{1}{r!{\huxvb{0}}}{\huxtpad{4pt}\raggedleft {\fontsize{9.5pt}{11.4pt}\selectfont (1.427)~~~}\huxbpad{4pt}} &
\multicolumn{1}{r!{\huxvb{0}}}{\huxtpad{4pt}\raggedleft {\fontsize{9.5pt}{11.4pt}\selectfont (1.352)~~~}\huxbpad{4pt}} \tabularnewline[-0.5pt]


\hhline{}
\arrayrulecolor{black}

\multicolumn{1}{!{\huxvb{0}}l!{\huxvb{0}}}{\huxtpad{4pt}\raggedright {\fontsize{9.5pt}{11.4pt}\selectfont Socioeconomic}\huxbpad{4pt}} &
\multicolumn{1}{r!{\huxvb{0}}}{\huxtpad{4pt}\raggedleft {\fontsize{9.5pt}{11.4pt}\selectfont ~~~~~~~~}\huxbpad{4pt}} &
\multicolumn{1}{r!{\huxvb{0}}}{\huxtpad{4pt}\raggedleft {\fontsize{9.5pt}{11.4pt}\selectfont ~~~~~~~~}\huxbpad{4pt}} &
\multicolumn{1}{r!{\huxvb{0}}}{\huxtpad{4pt}\raggedleft {\fontsize{9.5pt}{11.4pt}\selectfont ~~~~~~~~}\huxbpad{4pt}} \tabularnewline[-0.5pt]


\hhline{}
\arrayrulecolor{black}

\multicolumn{1}{!{\huxvb{0}}l!{\huxvb{0}}}{\huxtpad{4pt}\raggedright {\fontsize{9.5pt}{11.4pt}\selectfont $<$H.S.}\huxbpad{4pt}} &
\multicolumn{1}{r!{\huxvb{0}}}{\huxtpad{4pt}\raggedleft {\fontsize{9.5pt}{11.4pt}\selectfont 17.245 ***}\huxbpad{4pt}} &
\multicolumn{1}{r!{\huxvb{0}}}{\huxtpad{4pt}\raggedleft {\fontsize{9.5pt}{11.4pt}\selectfont 16.970 ***}\huxbpad{4pt}} &
\multicolumn{1}{r!{\huxvb{0}}}{\huxtpad{4pt}\raggedleft {\fontsize{9.5pt}{11.4pt}\selectfont 17.851 ***}\huxbpad{4pt}} \tabularnewline[-0.5pt]


\hhline{}
\arrayrulecolor{black}

\multicolumn{1}{!{\huxvb{0}}l!{\huxvb{0}}}{\huxtpad{4pt}\raggedright {\fontsize{9.5pt}{11.4pt}\selectfont }\huxbpad{4pt}} &
\multicolumn{1}{r!{\huxvb{0}}}{\huxtpad{4pt}\raggedleft {\fontsize{9.5pt}{11.4pt}\selectfont (1.277)~~~}\huxbpad{4pt}} &
\multicolumn{1}{r!{\huxvb{0}}}{\huxtpad{4pt}\raggedleft {\fontsize{9.5pt}{11.4pt}\selectfont (1.429)~~~}\huxbpad{4pt}} &
\multicolumn{1}{r!{\huxvb{0}}}{\huxtpad{4pt}\raggedleft {\fontsize{9.5pt}{11.4pt}\selectfont (1.518)~~~}\huxbpad{4pt}} \tabularnewline[-0.5pt]


\hhline{}
\arrayrulecolor{black}

\multicolumn{1}{!{\huxvb{0}}l!{\huxvb{0}}}{\huxtpad{4pt}\raggedright {\fontsize{9.5pt}{11.4pt}\selectfont H.S.}\huxbpad{4pt}} &
\multicolumn{1}{r!{\huxvb{0}}}{\huxtpad{4pt}\raggedleft {\fontsize{9.5pt}{11.4pt}\selectfont 20.921 ***}\huxbpad{4pt}} &
\multicolumn{1}{r!{\huxvb{0}}}{\huxtpad{4pt}\raggedleft {\fontsize{9.5pt}{11.4pt}\selectfont 20.529 ***}\huxbpad{4pt}} &
\multicolumn{1}{r!{\huxvb{0}}}{\huxtpad{4pt}\raggedleft {\fontsize{9.5pt}{11.4pt}\selectfont 21.776 ***}\huxbpad{4pt}} \tabularnewline[-0.5pt]


\hhline{}
\arrayrulecolor{black}

\multicolumn{1}{!{\huxvb{0}}l!{\huxvb{0}}}{\huxtpad{4pt}\raggedright {\fontsize{9.5pt}{11.4pt}\selectfont }\huxbpad{4pt}} &
\multicolumn{1}{r!{\huxvb{0}}}{\huxtpad{4pt}\raggedleft {\fontsize{9.5pt}{11.4pt}\selectfont (1.109)~~~}\huxbpad{4pt}} &
\multicolumn{1}{r!{\huxvb{0}}}{\huxtpad{4pt}\raggedleft {\fontsize{9.5pt}{11.4pt}\selectfont (1.223)~~~}\huxbpad{4pt}} &
\multicolumn{1}{r!{\huxvb{0}}}{\huxtpad{4pt}\raggedleft {\fontsize{9.5pt}{11.4pt}\selectfont (1.055)~~~}\huxbpad{4pt}} \tabularnewline[-0.5pt]


\hhline{}
\arrayrulecolor{black}

\multicolumn{1}{!{\huxvb{0}}l!{\huxvb{0}}}{\huxtpad{4pt}\raggedright {\fontsize{9.5pt}{11.4pt}\selectfont Some college, no B.A.}\huxbpad{4pt}} &
\multicolumn{1}{r!{\huxvb{0}}}{\huxtpad{4pt}\raggedleft {\fontsize{9.5pt}{11.4pt}\selectfont 20.173 ***}\huxbpad{4pt}} &
\multicolumn{1}{r!{\huxvb{0}}}{\huxtpad{4pt}\raggedleft {\fontsize{9.5pt}{11.4pt}\selectfont 19.785 ***}\huxbpad{4pt}} &
\multicolumn{1}{r!{\huxvb{0}}}{\huxtpad{4pt}\raggedleft {\fontsize{9.5pt}{11.4pt}\selectfont 21.555 ***}\huxbpad{4pt}} \tabularnewline[-0.5pt]


\hhline{}
\arrayrulecolor{black}

\multicolumn{1}{!{\huxvb{0}}l!{\huxvb{0}}}{\huxtpad{4pt}\raggedright {\fontsize{9.5pt}{11.4pt}\selectfont }\huxbpad{4pt}} &
\multicolumn{1}{r!{\huxvb{0}}}{\huxtpad{4pt}\raggedleft {\fontsize{9.5pt}{11.4pt}\selectfont (1.155)~~~}\huxbpad{4pt}} &
\multicolumn{1}{r!{\huxvb{0}}}{\huxtpad{4pt}\raggedleft {\fontsize{9.5pt}{11.4pt}\selectfont (1.292)~~~}\huxbpad{4pt}} &
\multicolumn{1}{r!{\huxvb{0}}}{\huxtpad{4pt}\raggedleft {\fontsize{9.5pt}{11.4pt}\selectfont (1.238)~~~}\huxbpad{4pt}} \tabularnewline[-0.5pt]


\hhline{}
\arrayrulecolor{black}

\multicolumn{1}{!{\huxvb{0}}l!{\huxvb{0}}}{\huxtpad{4pt}\raggedright {\fontsize{9.5pt}{11.4pt}\selectfont B.A.}\huxbpad{4pt}} &
\multicolumn{1}{r!{\huxvb{0}}}{\huxtpad{4pt}\raggedleft {\fontsize{9.5pt}{11.4pt}\selectfont 37.543 ***}\huxbpad{4pt}} &
\multicolumn{1}{r!{\huxvb{0}}}{\huxtpad{4pt}\raggedleft {\fontsize{9.5pt}{11.4pt}\selectfont 37.053 ***}\huxbpad{4pt}} &
\multicolumn{1}{r!{\huxvb{0}}}{\huxtpad{4pt}\raggedleft {\fontsize{9.5pt}{11.4pt}\selectfont 38.731 ***}\huxbpad{4pt}} \tabularnewline[-0.5pt]


\hhline{}
\arrayrulecolor{black}

\multicolumn{1}{!{\huxvb{0}}l!{\huxvb{0}}}{\huxtpad{4pt}\raggedright {\fontsize{9.5pt}{11.4pt}\selectfont }\huxbpad{4pt}} &
\multicolumn{1}{r!{\huxvb{0}}}{\huxtpad{4pt}\raggedleft {\fontsize{9.5pt}{11.4pt}\selectfont (1.371)~~~}\huxbpad{4pt}} &
\multicolumn{1}{r!{\huxvb{0}}}{\huxtpad{4pt}\raggedleft {\fontsize{9.5pt}{11.4pt}\selectfont (1.473)~~~}\huxbpad{4pt}} &
\multicolumn{1}{r!{\huxvb{0}}}{\huxtpad{4pt}\raggedleft {\fontsize{9.5pt}{11.4pt}\selectfont (1.408)~~~}\huxbpad{4pt}} \tabularnewline[-0.5pt]


\hhline{}
\arrayrulecolor{black}

\multicolumn{1}{!{\huxvb{0}}l!{\huxvb{0}}}{\huxtpad{4pt}\raggedright {\fontsize{9.5pt}{11.4pt}\selectfont Neighborhood experience}\huxbpad{4pt}} &
\multicolumn{1}{r!{\huxvb{0}}}{\huxtpad{4pt}\raggedleft {\fontsize{9.5pt}{11.4pt}\selectfont ~~~~~~~~}\huxbpad{4pt}} &
\multicolumn{1}{r!{\huxvb{0}}}{\huxtpad{4pt}\raggedleft {\fontsize{9.5pt}{11.4pt}\selectfont ~~~~~~~~}\huxbpad{4pt}} &
\multicolumn{1}{r!{\huxvb{0}}}{\huxtpad{4pt}\raggedleft {\fontsize{9.5pt}{11.4pt}\selectfont ~~~~~~~~}\huxbpad{4pt}} \tabularnewline[-0.5pt]


\hhline{}
\arrayrulecolor{black}

\multicolumn{1}{!{\huxvb{0}}l!{\huxvb{0}}}{\huxtpad{4pt}\raggedright {\fontsize{9.5pt}{11.4pt}\selectfont Years in neighborhood}\huxbpad{4pt}} &
\multicolumn{1}{r!{\huxvb{0}}}{\huxtpad{4pt}\raggedleft {\fontsize{9.5pt}{11.4pt}\selectfont 21.063 ***}\huxbpad{4pt}} &
\multicolumn{1}{r!{\huxvb{0}}}{\huxtpad{4pt}\raggedleft {\fontsize{9.5pt}{11.4pt}\selectfont 20.587 ***}\huxbpad{4pt}} &
\multicolumn{1}{r!{\huxvb{0}}}{\huxtpad{4pt}\raggedleft {\fontsize{9.5pt}{11.4pt}\selectfont 21.965 ***}\huxbpad{4pt}} \tabularnewline[-0.5pt]


\hhline{}
\arrayrulecolor{black}

\multicolumn{1}{!{\huxvb{0}}l!{\huxvb{0}}}{\huxtpad{4pt}\raggedright {\fontsize{9.5pt}{11.4pt}\selectfont }\huxbpad{4pt}} &
\multicolumn{1}{r!{\huxvb{0}}}{\huxtpad{4pt}\raggedleft {\fontsize{9.5pt}{11.4pt}\selectfont (1.461)~~~}\huxbpad{4pt}} &
\multicolumn{1}{r!{\huxvb{0}}}{\huxtpad{4pt}\raggedleft {\fontsize{9.5pt}{11.4pt}\selectfont (1.598)~~~}\huxbpad{4pt}} &
\multicolumn{1}{r!{\huxvb{0}}}{\huxtpad{4pt}\raggedleft {\fontsize{9.5pt}{11.4pt}\selectfont (1.641)~~~}\huxbpad{4pt}} \tabularnewline[-0.5pt]


\hhline{}
\arrayrulecolor{black}

\multicolumn{1}{!{\huxvb{0}}l!{\huxvb{0}}}{\huxtpad{4pt}\raggedright {\fontsize{9.5pt}{11.4pt}\selectfont 1-9 blocks}\huxbpad{4pt}} &
\multicolumn{1}{r!{\huxvb{0}}}{\huxtpad{4pt}\raggedleft {\fontsize{9.5pt}{11.4pt}\selectfont 17.296 ***}\huxbpad{4pt}} &
\multicolumn{1}{r!{\huxvb{0}}}{\huxtpad{4pt}\raggedleft {\fontsize{9.5pt}{11.4pt}\selectfont 17.031 ***}\huxbpad{4pt}} &
\multicolumn{1}{r!{\huxvb{0}}}{\huxtpad{4pt}\raggedleft {\fontsize{9.5pt}{11.4pt}\selectfont 19.466 ***}\huxbpad{4pt}} \tabularnewline[-0.5pt]


\hhline{}
\arrayrulecolor{black}

\multicolumn{1}{!{\huxvb{0}}l!{\huxvb{0}}}{\huxtpad{4pt}\raggedright {\fontsize{9.5pt}{11.4pt}\selectfont }\huxbpad{4pt}} &
\multicolumn{1}{r!{\huxvb{0}}}{\huxtpad{4pt}\raggedleft {\fontsize{9.5pt}{11.4pt}\selectfont (1.596)~~~}\huxbpad{4pt}} &
\multicolumn{1}{r!{\huxvb{0}}}{\huxtpad{4pt}\raggedleft {\fontsize{9.5pt}{11.4pt}\selectfont (1.508)~~~}\huxbpad{4pt}} &
\multicolumn{1}{r!{\huxvb{0}}}{\huxtpad{4pt}\raggedleft {\fontsize{9.5pt}{11.4pt}\selectfont (1.323)~~~}\huxbpad{4pt}} \tabularnewline[-0.5pt]


\hhline{}
\arrayrulecolor{black}

\multicolumn{1}{!{\huxvb{0}}l!{\huxvb{0}}}{\huxtpad{4pt}\raggedright {\fontsize{9.5pt}{11.4pt}\selectfont 10-50 blocks}\huxbpad{4pt}} &
\multicolumn{1}{r!{\huxvb{0}}}{\huxtpad{4pt}\raggedleft {\fontsize{9.5pt}{11.4pt}\selectfont 18.377 ***}\huxbpad{4pt}} &
\multicolumn{1}{r!{\huxvb{0}}}{\huxtpad{4pt}\raggedleft {\fontsize{9.5pt}{11.4pt}\selectfont 17.968 ***}\huxbpad{4pt}} &
\multicolumn{1}{r!{\huxvb{0}}}{\huxtpad{4pt}\raggedleft {\fontsize{9.5pt}{11.4pt}\selectfont 19.748 ***}\huxbpad{4pt}} \tabularnewline[-0.5pt]


\hhline{}
\arrayrulecolor{black}

\multicolumn{1}{!{\huxvb{0}}l!{\huxvb{0}}}{\huxtpad{4pt}\raggedright {\fontsize{9.5pt}{11.4pt}\selectfont }\huxbpad{4pt}} &
\multicolumn{1}{r!{\huxvb{0}}}{\huxtpad{4pt}\raggedleft {\fontsize{9.5pt}{11.4pt}\selectfont (1.529)~~~}\huxbpad{4pt}} &
\multicolumn{1}{r!{\huxvb{0}}}{\huxtpad{4pt}\raggedleft {\fontsize{9.5pt}{11.4pt}\selectfont (1.548)~~~}\huxbpad{4pt}} &
\multicolumn{1}{r!{\huxvb{0}}}{\huxtpad{4pt}\raggedleft {\fontsize{9.5pt}{11.4pt}\selectfont (1.331)~~~}\huxbpad{4pt}} \tabularnewline[-0.5pt]


\hhline{}
\arrayrulecolor{black}

\multicolumn{1}{!{\huxvb{0}}l!{\huxvb{0}}}{\huxtpad{4pt}\raggedright {\fontsize{9.5pt}{11.4pt}\selectfont Extremely satisfied}\huxbpad{4pt}} &
\multicolumn{1}{r!{\huxvb{0}}}{\huxtpad{4pt}\raggedleft {\fontsize{9.5pt}{11.4pt}\selectfont 19.556 ***}\huxbpad{4pt}} &
\multicolumn{1}{r!{\huxvb{0}}}{\huxtpad{4pt}\raggedleft {\fontsize{9.5pt}{11.4pt}\selectfont 18.566 ***}\huxbpad{4pt}} &
\multicolumn{1}{r!{\huxvb{0}}}{\huxtpad{4pt}\raggedleft {\fontsize{9.5pt}{11.4pt}\selectfont 20.032 ***}\huxbpad{4pt}} \tabularnewline[-0.5pt]


\hhline{}
\arrayrulecolor{black}

\multicolumn{1}{!{\huxvb{0}}l!{\huxvb{0}}}{\huxtpad{4pt}\raggedright {\fontsize{9.5pt}{11.4pt}\selectfont }\huxbpad{4pt}} &
\multicolumn{1}{r!{\huxvb{0}}}{\huxtpad{4pt}\raggedleft {\fontsize{9.5pt}{11.4pt}\selectfont (1.807)~~~}\huxbpad{4pt}} &
\multicolumn{1}{r!{\huxvb{0}}}{\huxtpad{4pt}\raggedleft {\fontsize{9.5pt}{11.4pt}\selectfont (1.773)~~~}\huxbpad{4pt}} &
\multicolumn{1}{r!{\huxvb{0}}}{\huxtpad{4pt}\raggedleft {\fontsize{9.5pt}{11.4pt}\selectfont (1.637)~~~}\huxbpad{4pt}} \tabularnewline[-0.5pt]


\hhline{}
\arrayrulecolor{black}

\multicolumn{1}{!{\huxvb{0}}l!{\huxvb{0}}}{\huxtpad{4pt}\raggedright {\fontsize{9.5pt}{11.4pt}\selectfont sample\_tract24031700715}\huxbpad{4pt}} &
\multicolumn{1}{r!{\huxvb{0}}}{\huxtpad{4pt}\raggedleft {\fontsize{9.5pt}{11.4pt}\selectfont 18.550 ***}\huxbpad{4pt}} &
\multicolumn{1}{r!{\huxvb{0}}}{\huxtpad{4pt}\raggedleft {\fontsize{9.5pt}{11.4pt}\selectfont 17.856 ***}\huxbpad{4pt}} &
\multicolumn{1}{r!{\huxvb{0}}}{\huxtpad{4pt}\raggedleft {\fontsize{9.5pt}{11.4pt}\selectfont 18.422 ***}\huxbpad{4pt}} \tabularnewline[-0.5pt]


\hhline{}
\arrayrulecolor{black}

\multicolumn{1}{!{\huxvb{0}}l!{\huxvb{0}}}{\huxtpad{4pt}\raggedright {\fontsize{9.5pt}{11.4pt}\selectfont }\huxbpad{4pt}} &
\multicolumn{1}{r!{\huxvb{0}}}{\huxtpad{4pt}\raggedleft {\fontsize{9.5pt}{11.4pt}\selectfont (1.418)~~~}\huxbpad{4pt}} &
\multicolumn{1}{r!{\huxvb{0}}}{\huxtpad{4pt}\raggedleft {\fontsize{9.5pt}{11.4pt}\selectfont (1.532)~~~}\huxbpad{4pt}} &
\multicolumn{1}{r!{\huxvb{0}}}{\huxtpad{4pt}\raggedleft {\fontsize{9.5pt}{11.4pt}\selectfont (2.010)~~~}\huxbpad{4pt}} \tabularnewline[-0.5pt]


\hhline{}
\arrayrulecolor{black}

\multicolumn{1}{!{\huxvb{0}}l!{\huxvb{0}}}{\huxtpad{4pt}\raggedright {\fontsize{9.5pt}{11.4pt}\selectfont sample\_tract24031700716}\huxbpad{4pt}} &
\multicolumn{1}{r!{\huxvb{0}}}{\huxtpad{4pt}\raggedleft {\fontsize{9.5pt}{11.4pt}\selectfont 0.667~~~~}\huxbpad{4pt}} &
\multicolumn{1}{r!{\huxvb{0}}}{\huxtpad{4pt}\raggedleft {\fontsize{9.5pt}{11.4pt}\selectfont -0.099~~~~}\huxbpad{4pt}} &
\multicolumn{1}{r!{\huxvb{0}}}{\huxtpad{4pt}\raggedleft {\fontsize{9.5pt}{11.4pt}\selectfont 0.583~~~~}\huxbpad{4pt}} \tabularnewline[-0.5pt]


\hhline{}
\arrayrulecolor{black}

\multicolumn{1}{!{\huxvb{0}}l!{\huxvb{0}}}{\huxtpad{4pt}\raggedright {\fontsize{9.5pt}{11.4pt}\selectfont }\huxbpad{4pt}} &
\multicolumn{1}{r!{\huxvb{0}}}{\huxtpad{4pt}\raggedleft {\fontsize{9.5pt}{11.4pt}\selectfont (1.022)~~~}\huxbpad{4pt}} &
\multicolumn{1}{r!{\huxvb{0}}}{\huxtpad{4pt}\raggedleft {\fontsize{9.5pt}{11.4pt}\selectfont (1.156)~~~}\huxbpad{4pt}} &
\multicolumn{1}{r!{\huxvb{0}}}{\huxtpad{4pt}\raggedleft {\fontsize{9.5pt}{11.4pt}\selectfont (1.192)~~~}\huxbpad{4pt}} \tabularnewline[-0.5pt]


\hhline{}
\arrayrulecolor{black}

\multicolumn{1}{!{\huxvb{0}}l!{\huxvb{0}}}{\huxtpad{4pt}\raggedright {\fontsize{9.5pt}{11.4pt}\selectfont sample\_tract24031700720}\huxbpad{4pt}} &
\multicolumn{1}{r!{\huxvb{0}}}{\huxtpad{4pt}\raggedleft {\fontsize{9.5pt}{11.4pt}\selectfont 0.283~~~~}\huxbpad{4pt}} &
\multicolumn{1}{r!{\huxvb{0}}}{\huxtpad{4pt}\raggedleft {\fontsize{9.5pt}{11.4pt}\selectfont -0.072~~~~}\huxbpad{4pt}} &
\multicolumn{1}{r!{\huxvb{0}}}{\huxtpad{4pt}\raggedleft {\fontsize{9.5pt}{11.4pt}\selectfont -0.029~~~~}\huxbpad{4pt}} \tabularnewline[-0.5pt]


\hhline{}
\arrayrulecolor{black}

\multicolumn{1}{!{\huxvb{0}}l!{\huxvb{0}}}{\huxtpad{4pt}\raggedright {\fontsize{9.5pt}{11.4pt}\selectfont }\huxbpad{4pt}} &
\multicolumn{1}{r!{\huxvb{0}}}{\huxtpad{4pt}\raggedleft {\fontsize{9.5pt}{11.4pt}\selectfont (1.109)~~~}\huxbpad{4pt}} &
\multicolumn{1}{r!{\huxvb{0}}}{\huxtpad{4pt}\raggedleft {\fontsize{9.5pt}{11.4pt}\selectfont (1.187)~~~}\huxbpad{4pt}} &
\multicolumn{1}{r!{\huxvb{0}}}{\huxtpad{4pt}\raggedleft {\fontsize{9.5pt}{11.4pt}\selectfont (1.287)~~~}\huxbpad{4pt}} \tabularnewline[-0.5pt]


\hhline{}
\arrayrulecolor{black}

\multicolumn{1}{!{\huxvb{0}}l!{\huxvb{0}}}{\huxtpad{4pt}\raggedright {\fontsize{9.5pt}{11.4pt}\selectfont sample\_tract24031700721}\huxbpad{4pt}} &
\multicolumn{1}{r!{\huxvb{0}}}{\huxtpad{4pt}\raggedleft {\fontsize{9.5pt}{11.4pt}\selectfont 21.972 ***}\huxbpad{4pt}} &
\multicolumn{1}{r!{\huxvb{0}}}{\huxtpad{4pt}\raggedleft {\fontsize{9.5pt}{11.4pt}\selectfont 21.815 ***}\huxbpad{4pt}} &
\multicolumn{1}{r!{\huxvb{0}}}{\huxtpad{4pt}\raggedleft {\fontsize{9.5pt}{11.4pt}\selectfont 23.752 ***}\huxbpad{4pt}} \tabularnewline[-0.5pt]


\hhline{}
\arrayrulecolor{black}

\multicolumn{1}{!{\huxvb{0}}l!{\huxvb{0}}}{\huxtpad{4pt}\raggedright {\fontsize{9.5pt}{11.4pt}\selectfont }\huxbpad{4pt}} &
\multicolumn{1}{r!{\huxvb{0}}}{\huxtpad{4pt}\raggedleft {\fontsize{9.5pt}{11.4pt}\selectfont (1.704)~~~}\huxbpad{4pt}} &
\multicolumn{1}{r!{\huxvb{0}}}{\huxtpad{4pt}\raggedleft {\fontsize{9.5pt}{11.4pt}\selectfont (1.729)~~~}\huxbpad{4pt}} &
\multicolumn{1}{r!{\huxvb{0}}}{\huxtpad{4pt}\raggedleft {\fontsize{9.5pt}{11.4pt}\selectfont (1.584)~~~}\huxbpad{4pt}} \tabularnewline[-0.5pt]


\hhline{}
\arrayrulecolor{black}

\multicolumn{1}{!{\huxvb{0}}l!{\huxvb{0}}}{\huxtpad{4pt}\raggedright {\fontsize{9.5pt}{11.4pt}\selectfont sample\_tract24031700722}\huxbpad{4pt}} &
\multicolumn{1}{r!{\huxvb{0}}}{\huxtpad{4pt}\raggedleft {\fontsize{9.5pt}{11.4pt}\selectfont 19.637 ***}\huxbpad{4pt}} &
\multicolumn{1}{r!{\huxvb{0}}}{\huxtpad{4pt}\raggedleft {\fontsize{9.5pt}{11.4pt}\selectfont 19.742 ***}\huxbpad{4pt}} &
\multicolumn{1}{r!{\huxvb{0}}}{\huxtpad{4pt}\raggedleft {\fontsize{9.5pt}{11.4pt}\selectfont 21.787 ***}\huxbpad{4pt}} \tabularnewline[-0.5pt]


\hhline{}
\arrayrulecolor{black}

\multicolumn{1}{!{\huxvb{0}}l!{\huxvb{0}}}{\huxtpad{4pt}\raggedright {\fontsize{9.5pt}{11.4pt}\selectfont }\huxbpad{4pt}} &
\multicolumn{1}{r!{\huxvb{0}}}{\huxtpad{4pt}\raggedleft {\fontsize{9.5pt}{11.4pt}\selectfont (1.216)~~~}\huxbpad{4pt}} &
\multicolumn{1}{r!{\huxvb{0}}}{\huxtpad{4pt}\raggedleft {\fontsize{9.5pt}{11.4pt}\selectfont (1.405)~~~}\huxbpad{4pt}} &
\multicolumn{1}{r!{\huxvb{0}}}{\huxtpad{4pt}\raggedleft {\fontsize{9.5pt}{11.4pt}\selectfont (1.971)~~~}\huxbpad{4pt}} \tabularnewline[-0.5pt]


\hhline{}
\arrayrulecolor{black}

\multicolumn{1}{!{\huxvb{0}}l!{\huxvb{0}}}{\huxtpad{4pt}\raggedright {\fontsize{9.5pt}{11.4pt}\selectfont sample\_tract24031700724}\huxbpad{4pt}} &
\multicolumn{1}{r!{\huxvb{0}}}{\huxtpad{4pt}\raggedleft {\fontsize{9.5pt}{11.4pt}\selectfont 18.432 ***}\huxbpad{4pt}} &
\multicolumn{1}{r!{\huxvb{0}}}{\huxtpad{4pt}\raggedleft {\fontsize{9.5pt}{11.4pt}\selectfont 17.724 ***}\huxbpad{4pt}} &
\multicolumn{1}{r!{\huxvb{0}}}{\huxtpad{4pt}\raggedleft {\fontsize{9.5pt}{11.4pt}\selectfont 19.578 ***}\huxbpad{4pt}} \tabularnewline[-0.5pt]


\hhline{}
\arrayrulecolor{black}

\multicolumn{1}{!{\huxvb{0}}l!{\huxvb{0}}}{\huxtpad{4pt}\raggedright {\fontsize{9.5pt}{11.4pt}\selectfont }\huxbpad{4pt}} &
\multicolumn{1}{r!{\huxvb{0}}}{\huxtpad{4pt}\raggedleft {\fontsize{9.5pt}{11.4pt}\selectfont (1.424)~~~}\huxbpad{4pt}} &
\multicolumn{1}{r!{\huxvb{0}}}{\huxtpad{4pt}\raggedleft {\fontsize{9.5pt}{11.4pt}\selectfont (1.515)~~~}\huxbpad{4pt}} &
\multicolumn{1}{r!{\huxvb{0}}}{\huxtpad{4pt}\raggedleft {\fontsize{9.5pt}{11.4pt}\selectfont (1.308)~~~}\huxbpad{4pt}} \tabularnewline[-0.5pt]


\hhline{}
\arrayrulecolor{black}

\multicolumn{1}{!{\huxvb{0}}l!{\huxvb{0}}}{\huxtpad{4pt}\raggedright {\fontsize{9.5pt}{11.4pt}\selectfont sample\_tract24031700810}\huxbpad{4pt}} &
\multicolumn{1}{r!{\huxvb{0}}}{\huxtpad{4pt}\raggedleft {\fontsize{9.5pt}{11.4pt}\selectfont 0.823~~~~}\huxbpad{4pt}} &
\multicolumn{1}{r!{\huxvb{0}}}{\huxtpad{4pt}\raggedleft {\fontsize{9.5pt}{11.4pt}\selectfont 0.358~~~~}\huxbpad{4pt}} &
\multicolumn{1}{r!{\huxvb{0}}}{\huxtpad{4pt}\raggedleft {\fontsize{9.5pt}{11.4pt}\selectfont 0.183~~~~}\huxbpad{4pt}} \tabularnewline[-0.5pt]


\hhline{}
\arrayrulecolor{black}

\multicolumn{1}{!{\huxvb{0}}l!{\huxvb{0}}}{\huxtpad{4pt}\raggedright {\fontsize{9.5pt}{11.4pt}\selectfont }\huxbpad{4pt}} &
\multicolumn{1}{r!{\huxvb{0}}}{\huxtpad{4pt}\raggedleft {\fontsize{9.5pt}{11.4pt}\selectfont (1.122)~~~}\huxbpad{4pt}} &
\multicolumn{1}{r!{\huxvb{0}}}{\huxtpad{4pt}\raggedleft {\fontsize{9.5pt}{11.4pt}\selectfont (1.234)~~~}\huxbpad{4pt}} &
\multicolumn{1}{r!{\huxvb{0}}}{\huxtpad{4pt}\raggedleft {\fontsize{9.5pt}{11.4pt}\selectfont (1.326)~~~}\huxbpad{4pt}} \tabularnewline[-0.5pt]


\hhline{}
\arrayrulecolor{black}

\multicolumn{1}{!{\huxvb{0}}l!{\huxvb{0}}}{\huxtpad{4pt}\raggedright {\fontsize{9.5pt}{11.4pt}\selectfont sample\_tract24031700815}\huxbpad{4pt}} &
\multicolumn{1}{r!{\huxvb{0}}}{\huxtpad{4pt}\raggedleft {\fontsize{9.5pt}{11.4pt}\selectfont 18.418 ***}\huxbpad{4pt}} &
\multicolumn{1}{r!{\huxvb{0}}}{\huxtpad{4pt}\raggedleft {\fontsize{9.5pt}{11.4pt}\selectfont 17.938 ***}\huxbpad{4pt}} &
\multicolumn{1}{r!{\huxvb{0}}}{\huxtpad{4pt}\raggedleft {\fontsize{9.5pt}{11.4pt}\selectfont 19.561 ***}\huxbpad{4pt}} \tabularnewline[-0.5pt]


\hhline{}
\arrayrulecolor{black}

\multicolumn{1}{!{\huxvb{0}}l!{\huxvb{0}}}{\huxtpad{4pt}\raggedright {\fontsize{9.5pt}{11.4pt}\selectfont }\huxbpad{4pt}} &
\multicolumn{1}{r!{\huxvb{0}}}{\huxtpad{4pt}\raggedleft {\fontsize{9.5pt}{11.4pt}\selectfont (1.397)~~~}\huxbpad{4pt}} &
\multicolumn{1}{r!{\huxvb{0}}}{\huxtpad{4pt}\raggedleft {\fontsize{9.5pt}{11.4pt}\selectfont (1.538)~~~}\huxbpad{4pt}} &
\multicolumn{1}{r!{\huxvb{0}}}{\huxtpad{4pt}\raggedleft {\fontsize{9.5pt}{11.4pt}\selectfont (1.593)~~~}\huxbpad{4pt}} \tabularnewline[-0.5pt]


\hhline{}
\arrayrulecolor{black}

\multicolumn{1}{!{\huxvb{0}}l!{\huxvb{0}}}{\huxtpad{4pt}\raggedright {\fontsize{9.5pt}{11.4pt}\selectfont sample\_tract24031700816}\huxbpad{4pt}} &
\multicolumn{1}{r!{\huxvb{0}}}{\huxtpad{4pt}\raggedleft {\fontsize{9.5pt}{11.4pt}\selectfont 0.410~~~~}\huxbpad{4pt}} &
\multicolumn{1}{r!{\huxvb{0}}}{\huxtpad{4pt}\raggedleft {\fontsize{9.5pt}{11.4pt}\selectfont 0.546~~~~}\huxbpad{4pt}} &
\multicolumn{1}{r!{\huxvb{0}}}{\huxtpad{4pt}\raggedleft {\fontsize{9.5pt}{11.4pt}\selectfont 0.973~~~~}\huxbpad{4pt}} \tabularnewline[-0.5pt]


\hhline{}
\arrayrulecolor{black}

\multicolumn{1}{!{\huxvb{0}}l!{\huxvb{0}}}{\huxtpad{4pt}\raggedright {\fontsize{9.5pt}{11.4pt}\selectfont }\huxbpad{4pt}} &
\multicolumn{1}{r!{\huxvb{0}}}{\huxtpad{4pt}\raggedleft {\fontsize{9.5pt}{11.4pt}\selectfont (1.169)~~~}\huxbpad{4pt}} &
\multicolumn{1}{r!{\huxvb{0}}}{\huxtpad{4pt}\raggedleft {\fontsize{9.5pt}{11.4pt}\selectfont (1.246)~~~}\huxbpad{4pt}} &
\multicolumn{1}{r!{\huxvb{0}}}{\huxtpad{4pt}\raggedleft {\fontsize{9.5pt}{11.4pt}\selectfont (1.191)~~~}\huxbpad{4pt}} \tabularnewline[-0.5pt]


\hhline{}
\arrayrulecolor{black}

\multicolumn{1}{!{\huxvb{0}}l!{\huxvb{0}}}{\huxtpad{4pt}\raggedright {\fontsize{9.5pt}{11.4pt}\selectfont sample\_tract24031700817}\huxbpad{4pt}} &
\multicolumn{1}{r!{\huxvb{0}}}{\huxtpad{4pt}\raggedleft {\fontsize{9.5pt}{11.4pt}\selectfont 21.765 ***}\huxbpad{4pt}} &
\multicolumn{1}{r!{\huxvb{0}}}{\huxtpad{4pt}\raggedleft {\fontsize{9.5pt}{11.4pt}\selectfont 21.478 ***}\huxbpad{4pt}} &
\multicolumn{1}{r!{\huxvb{0}}}{\huxtpad{4pt}\raggedleft {\fontsize{9.5pt}{11.4pt}\selectfont 23.061 ***}\huxbpad{4pt}} \tabularnewline[-0.5pt]


\hhline{}
\arrayrulecolor{black}

\multicolumn{1}{!{\huxvb{0}}l!{\huxvb{0}}}{\huxtpad{4pt}\raggedright {\fontsize{9.5pt}{11.4pt}\selectfont }\huxbpad{4pt}} &
\multicolumn{1}{r!{\huxvb{0}}}{\huxtpad{4pt}\raggedleft {\fontsize{9.5pt}{11.4pt}\selectfont (1.367)~~~}\huxbpad{4pt}} &
\multicolumn{1}{r!{\huxvb{0}}}{\huxtpad{4pt}\raggedleft {\fontsize{9.5pt}{11.4pt}\selectfont (1.423)~~~}\huxbpad{4pt}} &
\multicolumn{1}{r!{\huxvb{0}}}{\huxtpad{4pt}\raggedleft {\fontsize{9.5pt}{11.4pt}\selectfont (1.409)~~~}\huxbpad{4pt}} \tabularnewline[-0.5pt]


\hhline{}
\arrayrulecolor{black}

\multicolumn{1}{!{\huxvb{0}}l!{\huxvb{0}}}{\huxtpad{4pt}\raggedright {\fontsize{9.5pt}{11.4pt}\selectfont sample\_tract24031700818}\huxbpad{4pt}} &
\multicolumn{1}{r!{\huxvb{0}}}{\huxtpad{4pt}\raggedleft {\fontsize{9.5pt}{11.4pt}\selectfont 20.159 ***}\huxbpad{4pt}} &
\multicolumn{1}{r!{\huxvb{0}}}{\huxtpad{4pt}\raggedleft {\fontsize{9.5pt}{11.4pt}\selectfont 19.639 ***}\huxbpad{4pt}} &
\multicolumn{1}{r!{\huxvb{0}}}{\huxtpad{4pt}\raggedleft {\fontsize{9.5pt}{11.4pt}\selectfont 20.634 ***}\huxbpad{4pt}} \tabularnewline[-0.5pt]


\hhline{}
\arrayrulecolor{black}

\multicolumn{1}{!{\huxvb{0}}l!{\huxvb{0}}}{\huxtpad{4pt}\raggedright {\fontsize{9.5pt}{11.4pt}\selectfont }\huxbpad{4pt}} &
\multicolumn{1}{r!{\huxvb{0}}}{\huxtpad{4pt}\raggedleft {\fontsize{9.5pt}{11.4pt}\selectfont (1.278)~~~}\huxbpad{4pt}} &
\multicolumn{1}{r!{\huxvb{0}}}{\huxtpad{4pt}\raggedleft {\fontsize{9.5pt}{11.4pt}\selectfont (1.407)~~~}\huxbpad{4pt}} &
\multicolumn{1}{r!{\huxvb{0}}}{\huxtpad{4pt}\raggedleft {\fontsize{9.5pt}{11.4pt}\selectfont (1.282)~~~}\huxbpad{4pt}} \tabularnewline[-0.5pt]


\hhline{}
\arrayrulecolor{black}

\multicolumn{1}{!{\huxvb{0}}l!{\huxvb{0}}}{\huxtpad{4pt}\raggedright {\fontsize{9.5pt}{11.4pt}\selectfont sample\_tract24031700819}\huxbpad{4pt}} &
\multicolumn{1}{r!{\huxvb{0}}}{\huxtpad{4pt}\raggedleft {\fontsize{9.5pt}{11.4pt}\selectfont 19.301 ***}\huxbpad{4pt}} &
\multicolumn{1}{r!{\huxvb{0}}}{\huxtpad{4pt}\raggedleft {\fontsize{9.5pt}{11.4pt}\selectfont 18.867 ***}\huxbpad{4pt}} &
\multicolumn{1}{r!{\huxvb{0}}}{\huxtpad{4pt}\raggedleft {\fontsize{9.5pt}{11.4pt}\selectfont 20.342 ***}\huxbpad{4pt}} \tabularnewline[-0.5pt]


\hhline{}
\arrayrulecolor{black}

\multicolumn{1}{!{\huxvb{0}}l!{\huxvb{0}}}{\huxtpad{4pt}\raggedright {\fontsize{9.5pt}{11.4pt}\selectfont }\huxbpad{4pt}} &
\multicolumn{1}{r!{\huxvb{0}}}{\huxtpad{4pt}\raggedleft {\fontsize{9.5pt}{11.4pt}\selectfont (1.241)~~~}\huxbpad{4pt}} &
\multicolumn{1}{r!{\huxvb{0}}}{\huxtpad{4pt}\raggedleft {\fontsize{9.5pt}{11.4pt}\selectfont (1.331)~~~}\huxbpad{4pt}} &
\multicolumn{1}{r!{\huxvb{0}}}{\huxtpad{4pt}\raggedleft {\fontsize{9.5pt}{11.4pt}\selectfont (1.222)~~~}\huxbpad{4pt}} \tabularnewline[-0.5pt]


\hhline{}
\arrayrulecolor{black}

\multicolumn{1}{!{\huxvb{0}}l!{\huxvb{0}}}{\huxtpad{4pt}\raggedright {\fontsize{9.5pt}{11.4pt}\selectfont sample\_tract24031700820}\huxbpad{4pt}} &
\multicolumn{1}{r!{\huxvb{0}}}{\huxtpad{4pt}\raggedleft {\fontsize{9.5pt}{11.4pt}\selectfont 17.540 ***}\huxbpad{4pt}} &
\multicolumn{1}{r!{\huxvb{0}}}{\huxtpad{4pt}\raggedleft {\fontsize{9.5pt}{11.4pt}\selectfont 16.994 ***}\huxbpad{4pt}} &
\multicolumn{1}{r!{\huxvb{0}}}{\huxtpad{4pt}\raggedleft {\fontsize{9.5pt}{11.4pt}\selectfont 17.871 ***}\huxbpad{4pt}} \tabularnewline[-0.5pt]


\hhline{}
\arrayrulecolor{black}

\multicolumn{1}{!{\huxvb{0}}l!{\huxvb{0}}}{\huxtpad{4pt}\raggedright {\fontsize{9.5pt}{11.4pt}\selectfont }\huxbpad{4pt}} &
\multicolumn{1}{r!{\huxvb{0}}}{\huxtpad{4pt}\raggedleft {\fontsize{9.5pt}{11.4pt}\selectfont (1.470)~~~}\huxbpad{4pt}} &
\multicolumn{1}{r!{\huxvb{0}}}{\huxtpad{4pt}\raggedleft {\fontsize{9.5pt}{11.4pt}\selectfont (1.645)~~~}\huxbpad{4pt}} &
\multicolumn{1}{r!{\huxvb{0}}}{\huxtpad{4pt}\raggedleft {\fontsize{9.5pt}{11.4pt}\selectfont (1.423)~~~}\huxbpad{4pt}} \tabularnewline[-0.5pt]


\hhline{}
\arrayrulecolor{black}

\multicolumn{1}{!{\huxvb{0}}l!{\huxvb{0}}}{\huxtpad{4pt}\raggedright {\fontsize{9.5pt}{11.4pt}\selectfont sample\_tract24031700822}\huxbpad{4pt}} &
\multicolumn{1}{r!{\huxvb{0}}}{\huxtpad{4pt}\raggedleft {\fontsize{9.5pt}{11.4pt}\selectfont 37.533 ***}\huxbpad{4pt}} &
\multicolumn{1}{r!{\huxvb{0}}}{\huxtpad{4pt}\raggedleft {\fontsize{9.5pt}{11.4pt}\selectfont 37.137 ***}\huxbpad{4pt}} &
\multicolumn{1}{r!{\huxvb{0}}}{\huxtpad{4pt}\raggedleft {\fontsize{9.5pt}{11.4pt}\selectfont 39.490 ***}\huxbpad{4pt}} \tabularnewline[-0.5pt]


\hhline{}
\arrayrulecolor{black}

\multicolumn{1}{!{\huxvb{0}}l!{\huxvb{0}}}{\huxtpad{4pt}\raggedright {\fontsize{9.5pt}{11.4pt}\selectfont }\huxbpad{4pt}} &
\multicolumn{1}{r!{\huxvb{0}}}{\huxtpad{4pt}\raggedleft {\fontsize{9.5pt}{11.4pt}\selectfont (1.371)~~~}\huxbpad{4pt}} &
\multicolumn{1}{r!{\huxvb{0}}}{\huxtpad{4pt}\raggedleft {\fontsize{9.5pt}{11.4pt}\selectfont (1.536)~~~}\huxbpad{4pt}} &
\multicolumn{1}{r!{\huxvb{0}}}{\huxtpad{4pt}\raggedleft {\fontsize{9.5pt}{11.4pt}\selectfont (1.733)~~~}\huxbpad{4pt}} \tabularnewline[-0.5pt]


\hhline{}
\arrayrulecolor{black}

\multicolumn{1}{!{\huxvb{0}}l!{\huxvb{0}}}{\huxtpad{4pt}\raggedright {\fontsize{9.5pt}{11.4pt}\selectfont sample\_tract24031700830}\huxbpad{4pt}} &
\multicolumn{1}{r!{\huxvb{0}}}{\huxtpad{4pt}\raggedleft {\fontsize{9.5pt}{11.4pt}\selectfont 1.230~~~~}\huxbpad{4pt}} &
\multicolumn{1}{r!{\huxvb{0}}}{\huxtpad{4pt}\raggedleft {\fontsize{9.5pt}{11.4pt}\selectfont 1.603~~~~}\huxbpad{4pt}} &
\multicolumn{1}{r!{\huxvb{0}}}{\huxtpad{4pt}\raggedleft {\fontsize{9.5pt}{11.4pt}\selectfont 1.344~~~~}\huxbpad{4pt}} \tabularnewline[-0.5pt]


\hhline{}
\arrayrulecolor{black}

\multicolumn{1}{!{\huxvb{0}}l!{\huxvb{0}}}{\huxtpad{4pt}\raggedright {\fontsize{9.5pt}{11.4pt}\selectfont }\huxbpad{4pt}} &
\multicolumn{1}{r!{\huxvb{0}}}{\huxtpad{4pt}\raggedleft {\fontsize{9.5pt}{11.4pt}\selectfont (1.112)~~~}\huxbpad{4pt}} &
\multicolumn{1}{r!{\huxvb{0}}}{\huxtpad{4pt}\raggedleft {\fontsize{9.5pt}{11.4pt}\selectfont (1.191)~~~}\huxbpad{4pt}} &
\multicolumn{1}{r!{\huxvb{0}}}{\huxtpad{4pt}\raggedleft {\fontsize{9.5pt}{11.4pt}\selectfont (1.261)~~~}\huxbpad{4pt}} \tabularnewline[-0.5pt]


\hhline{}
\arrayrulecolor{black}

\multicolumn{1}{!{\huxvb{0}}l!{\huxvb{0}}}{\huxtpad{4pt}\raggedright {\fontsize{9.5pt}{11.4pt}\selectfont sample\_tract24031700833}\huxbpad{4pt}} &
\multicolumn{1}{r!{\huxvb{0}}}{\huxtpad{4pt}\raggedleft {\fontsize{9.5pt}{11.4pt}\selectfont 0.814~~~~}\huxbpad{4pt}} &
\multicolumn{1}{r!{\huxvb{0}}}{\huxtpad{4pt}\raggedleft {\fontsize{9.5pt}{11.4pt}\selectfont 0.514~~~~}\huxbpad{4pt}} &
\multicolumn{1}{r!{\huxvb{0}}}{\huxtpad{4pt}\raggedleft {\fontsize{9.5pt}{11.4pt}\selectfont 0.503~~~~}\huxbpad{4pt}} \tabularnewline[-0.5pt]


\hhline{}
\arrayrulecolor{black}

\multicolumn{1}{!{\huxvb{0}}l!{\huxvb{0}}}{\huxtpad{4pt}\raggedright {\fontsize{9.5pt}{11.4pt}\selectfont }\huxbpad{4pt}} &
\multicolumn{1}{r!{\huxvb{0}}}{\huxtpad{4pt}\raggedleft {\fontsize{9.5pt}{11.4pt}\selectfont (1.080)~~~}\huxbpad{4pt}} &
\multicolumn{1}{r!{\huxvb{0}}}{\huxtpad{4pt}\raggedleft {\fontsize{9.5pt}{11.4pt}\selectfont (1.215)~~~}\huxbpad{4pt}} &
\multicolumn{1}{r!{\huxvb{0}}}{\huxtpad{4pt}\raggedleft {\fontsize{9.5pt}{11.4pt}\selectfont (1.109)~~~}\huxbpad{4pt}} \tabularnewline[-0.5pt]


\hhline{}
\arrayrulecolor{black}

\multicolumn{1}{!{\huxvb{0}}l!{\huxvb{0}}}{\huxtpad{4pt}\raggedright {\fontsize{9.5pt}{11.4pt}\selectfont sample\_tract24031700834}\huxbpad{4pt}} &
\multicolumn{1}{r!{\huxvb{0}}}{\huxtpad{4pt}\raggedleft {\fontsize{9.5pt}{11.4pt}\selectfont 15.567 ***}\huxbpad{4pt}} &
\multicolumn{1}{r!{\huxvb{0}}}{\huxtpad{4pt}\raggedleft {\fontsize{9.5pt}{11.4pt}\selectfont 14.542 ***}\huxbpad{4pt}} &
\multicolumn{1}{r!{\huxvb{0}}}{\huxtpad{4pt}\raggedleft {\fontsize{9.5pt}{11.4pt}\selectfont 16.325 ***}\huxbpad{4pt}} \tabularnewline[-0.5pt]


\hhline{}
\arrayrulecolor{black}

\multicolumn{1}{!{\huxvb{0}}l!{\huxvb{0}}}{\huxtpad{4pt}\raggedright {\fontsize{9.5pt}{11.4pt}\selectfont }\huxbpad{4pt}} &
\multicolumn{1}{r!{\huxvb{0}}}{\huxtpad{4pt}\raggedleft {\fontsize{9.5pt}{11.4pt}\selectfont (1.415)~~~}\huxbpad{4pt}} &
\multicolumn{1}{r!{\huxvb{0}}}{\huxtpad{4pt}\raggedleft {\fontsize{9.5pt}{11.4pt}\selectfont (1.572)~~~}\huxbpad{4pt}} &
\multicolumn{1}{r!{\huxvb{0}}}{\huxtpad{4pt}\raggedleft {\fontsize{9.5pt}{11.4pt}\selectfont (1.555)~~~}\huxbpad{4pt}} \tabularnewline[-0.5pt]


\hhline{}
\arrayrulecolor{black}

\multicolumn{1}{!{\huxvb{0}}l!{\huxvb{0}}}{\huxtpad{4pt}\raggedright {\fontsize{9.5pt}{11.4pt}\selectfont sample\_tract24031700835}\huxbpad{4pt}} &
\multicolumn{1}{r!{\huxvb{0}}}{\huxtpad{4pt}\raggedleft {\fontsize{9.5pt}{11.4pt}\selectfont 0.461~~~~}\huxbpad{4pt}} &
\multicolumn{1}{r!{\huxvb{0}}}{\huxtpad{4pt}\raggedleft {\fontsize{9.5pt}{11.4pt}\selectfont 0.147~~~~}\huxbpad{4pt}} &
\multicolumn{1}{r!{\huxvb{0}}}{\huxtpad{4pt}\raggedleft {\fontsize{9.5pt}{11.4pt}\selectfont 0.039~~~~}\huxbpad{4pt}} \tabularnewline[-0.5pt]


\hhline{}
\arrayrulecolor{black}

\multicolumn{1}{!{\huxvb{0}}l!{\huxvb{0}}}{\huxtpad{4pt}\raggedright {\fontsize{9.5pt}{11.4pt}\selectfont }\huxbpad{4pt}} &
\multicolumn{1}{r!{\huxvb{0}}}{\huxtpad{4pt}\raggedleft {\fontsize{9.5pt}{11.4pt}\selectfont (1.119)~~~}\huxbpad{4pt}} &
\multicolumn{1}{r!{\huxvb{0}}}{\huxtpad{4pt}\raggedleft {\fontsize{9.5pt}{11.4pt}\selectfont (1.219)~~~}\huxbpad{4pt}} &
\multicolumn{1}{r!{\huxvb{0}}}{\huxtpad{4pt}\raggedleft {\fontsize{9.5pt}{11.4pt}\selectfont (1.260)~~~}\huxbpad{4pt}} \tabularnewline[-0.5pt]


\hhline{}
\arrayrulecolor{black}

\multicolumn{1}{!{\huxvb{0}}l!{\huxvb{0}}}{\huxtpad{4pt}\raggedright {\fontsize{9.5pt}{11.4pt}\selectfont sample\_tract24031700902}\huxbpad{4pt}} &
\multicolumn{1}{r!{\huxvb{0}}}{\huxtpad{4pt}\raggedleft {\fontsize{9.5pt}{11.4pt}\selectfont 0.410~~~~}\huxbpad{4pt}} &
\multicolumn{1}{r!{\huxvb{0}}}{\huxtpad{4pt}\raggedleft {\fontsize{9.5pt}{11.4pt}\selectfont -1.263~~~~}\huxbpad{4pt}} &
\multicolumn{1}{r!{\huxvb{0}}}{\huxtpad{4pt}\raggedleft {\fontsize{9.5pt}{11.4pt}\selectfont 0.827~~~~}\huxbpad{4pt}} \tabularnewline[-0.5pt]


\hhline{}
\arrayrulecolor{black}

\multicolumn{1}{!{\huxvb{0}}l!{\huxvb{0}}}{\huxtpad{4pt}\raggedright {\fontsize{9.5pt}{11.4pt}\selectfont }\huxbpad{4pt}} &
\multicolumn{1}{r!{\huxvb{0}}}{\huxtpad{4pt}\raggedleft {\fontsize{9.5pt}{11.4pt}\selectfont (1.266)~~~}\huxbpad{4pt}} &
\multicolumn{1}{r!{\huxvb{0}}}{\huxtpad{4pt}\raggedleft {\fontsize{9.5pt}{11.4pt}\selectfont (1.687)~~~}\huxbpad{4pt}} &
\multicolumn{1}{r!{\huxvb{0}}}{\huxtpad{4pt}\raggedleft {\fontsize{9.5pt}{11.4pt}\selectfont (1.577)~~~}\huxbpad{4pt}} \tabularnewline[-0.5pt]


\hhline{}
\arrayrulecolor{black}

\multicolumn{1}{!{\huxvb{0}}l!{\huxvb{0}}}{\huxtpad{4pt}\raggedright {\fontsize{9.5pt}{11.4pt}\selectfont sample\_tract24031700903}\huxbpad{4pt}} &
\multicolumn{1}{r!{\huxvb{0}}}{\huxtpad{4pt}\raggedleft {\fontsize{9.5pt}{11.4pt}\selectfont 20.218 ***}\huxbpad{4pt}} &
\multicolumn{1}{r!{\huxvb{0}}}{\huxtpad{4pt}\raggedleft {\fontsize{9.5pt}{11.4pt}\selectfont 19.522 ***}\huxbpad{4pt}} &
\multicolumn{1}{r!{\huxvb{0}}}{\huxtpad{4pt}\raggedleft {\fontsize{9.5pt}{11.4pt}\selectfont 20.324 ***}\huxbpad{4pt}} \tabularnewline[-0.5pt]


\hhline{}
\arrayrulecolor{black}

\multicolumn{1}{!{\huxvb{0}}l!{\huxvb{0}}}{\huxtpad{4pt}\raggedright {\fontsize{9.5pt}{11.4pt}\selectfont }\huxbpad{4pt}} &
\multicolumn{1}{r!{\huxvb{0}}}{\huxtpad{4pt}\raggedleft {\fontsize{9.5pt}{11.4pt}\selectfont (1.388)~~~}\huxbpad{4pt}} &
\multicolumn{1}{r!{\huxvb{0}}}{\huxtpad{4pt}\raggedleft {\fontsize{9.5pt}{11.4pt}\selectfont (1.485)~~~}\huxbpad{4pt}} &
\multicolumn{1}{r!{\huxvb{0}}}{\huxtpad{4pt}\raggedleft {\fontsize{9.5pt}{11.4pt}\selectfont (1.910)~~~}\huxbpad{4pt}} \tabularnewline[-0.5pt]


\hhline{}
\arrayrulecolor{black}

\multicolumn{1}{!{\huxvb{0}}l!{\huxvb{0}}}{\huxtpad{4pt}\raggedright {\fontsize{9.5pt}{11.4pt}\selectfont sample\_tract24031701211}\huxbpad{4pt}} &
\multicolumn{1}{r!{\huxvb{0}}}{\huxtpad{4pt}\raggedleft {\fontsize{9.5pt}{11.4pt}\selectfont 18.442 ***}\huxbpad{4pt}} &
\multicolumn{1}{r!{\huxvb{0}}}{\huxtpad{4pt}\raggedleft {\fontsize{9.5pt}{11.4pt}\selectfont 18.004 ***}\huxbpad{4pt}} &
\multicolumn{1}{r!{\huxvb{0}}}{\huxtpad{4pt}\raggedleft {\fontsize{9.5pt}{11.4pt}\selectfont 18.458 ***}\huxbpad{4pt}} \tabularnewline[-0.5pt]


\hhline{}
\arrayrulecolor{black}

\multicolumn{1}{!{\huxvb{0}}l!{\huxvb{0}}}{\huxtpad{4pt}\raggedright {\fontsize{9.5pt}{11.4pt}\selectfont }\huxbpad{4pt}} &
\multicolumn{1}{r!{\huxvb{0}}}{\huxtpad{4pt}\raggedleft {\fontsize{9.5pt}{11.4pt}\selectfont (1.170)~~~}\huxbpad{4pt}} &
\multicolumn{1}{r!{\huxvb{0}}}{\huxtpad{4pt}\raggedleft {\fontsize{9.5pt}{11.4pt}\selectfont (1.279)~~~}\huxbpad{4pt}} &
\multicolumn{1}{r!{\huxvb{0}}}{\huxtpad{4pt}\raggedleft {\fontsize{9.5pt}{11.4pt}\selectfont (1.437)~~~}\huxbpad{4pt}} \tabularnewline[-0.5pt]


\hhline{}
\arrayrulecolor{black}

\multicolumn{1}{!{\huxvb{0}}l!{\huxvb{0}}}{\huxtpad{4pt}\raggedright {\fontsize{9.5pt}{11.4pt}\selectfont sample\_tract24031701219}\huxbpad{4pt}} &
\multicolumn{1}{r!{\huxvb{0}}}{\huxtpad{4pt}\raggedleft {\fontsize{9.5pt}{11.4pt}\selectfont 39.175 ***}\huxbpad{4pt}} &
\multicolumn{1}{r!{\huxvb{0}}}{\huxtpad{4pt}\raggedleft {\fontsize{9.5pt}{11.4pt}\selectfont 38.258 ***}\huxbpad{4pt}} &
\multicolumn{1}{r!{\huxvb{0}}}{\huxtpad{4pt}\raggedleft {\fontsize{9.5pt}{11.4pt}\selectfont 40.630 ***}\huxbpad{4pt}} \tabularnewline[-0.5pt]


\hhline{}
\arrayrulecolor{black}

\multicolumn{1}{!{\huxvb{0}}l!{\huxvb{0}}}{\huxtpad{4pt}\raggedright {\fontsize{9.5pt}{11.4pt}\selectfont }\huxbpad{4pt}} &
\multicolumn{1}{r!{\huxvb{0}}}{\huxtpad{4pt}\raggedleft {\fontsize{9.5pt}{11.4pt}\selectfont (1.372)~~~}\huxbpad{4pt}} &
\multicolumn{1}{r!{\huxvb{0}}}{\huxtpad{4pt}\raggedleft {\fontsize{9.5pt}{11.4pt}\selectfont (1.534)~~~}\huxbpad{4pt}} &
\multicolumn{1}{r!{\huxvb{0}}}{\huxtpad{4pt}\raggedleft {\fontsize{9.5pt}{11.4pt}\selectfont (1.482)~~~}\huxbpad{4pt}} \tabularnewline[-0.5pt]


\hhline{}
\arrayrulecolor{black}

\multicolumn{1}{!{\huxvb{0}}l!{\huxvb{0}}}{\huxtpad{4pt}\raggedright {\fontsize{9.5pt}{11.4pt}\selectfont sample\_tract24031701407}\huxbpad{4pt}} &
\multicolumn{1}{r!{\huxvb{0}}}{\huxtpad{4pt}\raggedleft {\fontsize{9.5pt}{11.4pt}\selectfont 17.740 ***}\huxbpad{4pt}} &
\multicolumn{1}{r!{\huxvb{0}}}{\huxtpad{4pt}\raggedleft {\fontsize{9.5pt}{11.4pt}\selectfont 17.588 ***}\huxbpad{4pt}} &
\multicolumn{1}{r!{\huxvb{0}}}{\huxtpad{4pt}\raggedleft {\fontsize{9.5pt}{11.4pt}\selectfont 19.833 ***}\huxbpad{4pt}} \tabularnewline[-0.5pt]


\hhline{}
\arrayrulecolor{black}

\multicolumn{1}{!{\huxvb{0}}l!{\huxvb{0}}}{\huxtpad{4pt}\raggedright {\fontsize{9.5pt}{11.4pt}\selectfont }\huxbpad{4pt}} &
\multicolumn{1}{r!{\huxvb{0}}}{\huxtpad{4pt}\raggedleft {\fontsize{9.5pt}{11.4pt}\selectfont (1.227)~~~}\huxbpad{4pt}} &
\multicolumn{1}{r!{\huxvb{0}}}{\huxtpad{4pt}\raggedleft {\fontsize{9.5pt}{11.4pt}\selectfont (1.296)~~~}\huxbpad{4pt}} &
\multicolumn{1}{r!{\huxvb{0}}}{\huxtpad{4pt}\raggedleft {\fontsize{9.5pt}{11.4pt}\selectfont (1.276)~~~}\huxbpad{4pt}} \tabularnewline[-0.5pt]


\hhline{}
\arrayrulecolor{black}

\multicolumn{1}{!{\huxvb{0}}l!{\huxvb{0}}}{\huxtpad{4pt}\raggedright {\fontsize{9.5pt}{11.4pt}\selectfont sample\_tract24031701408}\huxbpad{4pt}} &
\multicolumn{1}{r!{\huxvb{0}}}{\huxtpad{4pt}\raggedleft {\fontsize{9.5pt}{11.4pt}\selectfont 18.255 ***}\huxbpad{4pt}} &
\multicolumn{1}{r!{\huxvb{0}}}{\huxtpad{4pt}\raggedleft {\fontsize{9.5pt}{11.4pt}\selectfont 17.418 ***}\huxbpad{4pt}} &
\multicolumn{1}{r!{\huxvb{0}}}{\huxtpad{4pt}\raggedleft {\fontsize{9.5pt}{11.4pt}\selectfont 18.331 ***}\huxbpad{4pt}} \tabularnewline[-0.5pt]


\hhline{}
\arrayrulecolor{black}

\multicolumn{1}{!{\huxvb{0}}l!{\huxvb{0}}}{\huxtpad{4pt}\raggedright {\fontsize{9.5pt}{11.4pt}\selectfont }\huxbpad{4pt}} &
\multicolumn{1}{r!{\huxvb{0}}}{\huxtpad{4pt}\raggedleft {\fontsize{9.5pt}{11.4pt}\selectfont (1.701)~~~}\huxbpad{4pt}} &
\multicolumn{1}{r!{\huxvb{0}}}{\huxtpad{4pt}\raggedleft {\fontsize{9.5pt}{11.4pt}\selectfont (1.845)~~~}\huxbpad{4pt}} &
\multicolumn{1}{r!{\huxvb{0}}}{\huxtpad{4pt}\raggedleft {\fontsize{9.5pt}{11.4pt}\selectfont (1.469)~~~}\huxbpad{4pt}} \tabularnewline[-0.5pt]


\hhline{}
\arrayrulecolor{black}

\multicolumn{1}{!{\huxvb{0}}l!{\huxvb{0}}}{\huxtpad{4pt}\raggedright {\fontsize{9.5pt}{11.4pt}\selectfont sample\_tract24031701414}\huxbpad{4pt}} &
\multicolumn{1}{r!{\huxvb{0}}}{\huxtpad{4pt}\raggedleft {\fontsize{9.5pt}{11.4pt}\selectfont 17.459 ***}\huxbpad{4pt}} &
\multicolumn{1}{r!{\huxvb{0}}}{\huxtpad{4pt}\raggedleft {\fontsize{9.5pt}{11.4pt}\selectfont 17.522 ***}\huxbpad{4pt}} &
\multicolumn{1}{r!{\huxvb{0}}}{\huxtpad{4pt}\raggedleft {\fontsize{9.5pt}{11.4pt}\selectfont 18.667 ***}\huxbpad{4pt}} \tabularnewline[-0.5pt]


\hhline{}
\arrayrulecolor{black}

\multicolumn{1}{!{\huxvb{0}}l!{\huxvb{0}}}{\huxtpad{4pt}\raggedright {\fontsize{9.5pt}{11.4pt}\selectfont }\huxbpad{4pt}} &
\multicolumn{1}{r!{\huxvb{0}}}{\huxtpad{4pt}\raggedleft {\fontsize{9.5pt}{11.4pt}\selectfont (1.139)~~~}\huxbpad{4pt}} &
\multicolumn{1}{r!{\huxvb{0}}}{\huxtpad{4pt}\raggedleft {\fontsize{9.5pt}{11.4pt}\selectfont (1.179)~~~}\huxbpad{4pt}} &
\multicolumn{1}{r!{\huxvb{0}}}{\huxtpad{4pt}\raggedleft {\fontsize{9.5pt}{11.4pt}\selectfont (1.094)~~~}\huxbpad{4pt}} \tabularnewline[-0.5pt]


\hhline{}
\arrayrulecolor{black}

\multicolumn{1}{!{\huxvb{0}}l!{\huxvb{0}}}{\huxtpad{4pt}\raggedright {\fontsize{9.5pt}{11.4pt}\selectfont sample\_tract24031701415}\huxbpad{4pt}} &
\multicolumn{1}{r!{\huxvb{0}}}{\huxtpad{4pt}\raggedleft {\fontsize{9.5pt}{11.4pt}\selectfont 17.444 ***}\huxbpad{4pt}} &
\multicolumn{1}{r!{\huxvb{0}}}{\huxtpad{4pt}\raggedleft {\fontsize{9.5pt}{11.4pt}\selectfont 16.980 ***}\huxbpad{4pt}} &
\multicolumn{1}{r!{\huxvb{0}}}{\huxtpad{4pt}\raggedleft {\fontsize{9.5pt}{11.4pt}\selectfont 18.411 ***}\huxbpad{4pt}} \tabularnewline[-0.5pt]


\hhline{}
\arrayrulecolor{black}

\multicolumn{1}{!{\huxvb{0}}l!{\huxvb{0}}}{\huxtpad{4pt}\raggedright {\fontsize{9.5pt}{11.4pt}\selectfont }\huxbpad{4pt}} &
\multicolumn{1}{r!{\huxvb{0}}}{\huxtpad{4pt}\raggedleft {\fontsize{9.5pt}{11.4pt}\selectfont (1.152)~~~}\huxbpad{4pt}} &
\multicolumn{1}{r!{\huxvb{0}}}{\huxtpad{4pt}\raggedleft {\fontsize{9.5pt}{11.4pt}\selectfont (1.235)~~~}\huxbpad{4pt}} &
\multicolumn{1}{r!{\huxvb{0}}}{\huxtpad{4pt}\raggedleft {\fontsize{9.5pt}{11.4pt}\selectfont (1.193)~~~}\huxbpad{4pt}} \tabularnewline[-0.5pt]


\hhline{}
\arrayrulecolor{black}

\multicolumn{1}{!{\huxvb{0}}l!{\huxvb{0}}}{\huxtpad{4pt}\raggedright {\fontsize{9.5pt}{11.4pt}\selectfont sample\_tract24031701503}\huxbpad{4pt}} &
\multicolumn{1}{r!{\huxvb{0}}}{\huxtpad{4pt}\raggedleft {\fontsize{9.5pt}{11.4pt}\selectfont 19.358 ***}\huxbpad{4pt}} &
\multicolumn{1}{r!{\huxvb{0}}}{\huxtpad{4pt}\raggedleft {\fontsize{9.5pt}{11.4pt}\selectfont 18.703 ***}\huxbpad{4pt}} &
\multicolumn{1}{r!{\huxvb{0}}}{\huxtpad{4pt}\raggedleft {\fontsize{9.5pt}{11.4pt}\selectfont 19.953 ***}\huxbpad{4pt}} \tabularnewline[-0.5pt]


\hhline{}
\arrayrulecolor{black}

\multicolumn{1}{!{\huxvb{0}}l!{\huxvb{0}}}{\huxtpad{4pt}\raggedright {\fontsize{9.5pt}{11.4pt}\selectfont }\huxbpad{4pt}} &
\multicolumn{1}{r!{\huxvb{0}}}{\huxtpad{4pt}\raggedleft {\fontsize{9.5pt}{11.4pt}\selectfont (1.196)~~~}\huxbpad{4pt}} &
\multicolumn{1}{r!{\huxvb{0}}}{\huxtpad{4pt}\raggedleft {\fontsize{9.5pt}{11.4pt}\selectfont (1.265)~~~}\huxbpad{4pt}} &
\multicolumn{1}{r!{\huxvb{0}}}{\huxtpad{4pt}\raggedleft {\fontsize{9.5pt}{11.4pt}\selectfont (1.168)~~~}\huxbpad{4pt}} \tabularnewline[-0.5pt]


\hhline{}
\arrayrulecolor{black}

\multicolumn{1}{!{\huxvb{0}}l!{\huxvb{0}}}{\huxtpad{4pt}\raggedright {\fontsize{9.5pt}{11.4pt}\selectfont sample\_tract24031701507}\huxbpad{4pt}} &
\multicolumn{1}{r!{\huxvb{0}}}{\huxtpad{4pt}\raggedleft {\fontsize{9.5pt}{11.4pt}\selectfont 17.914 ***}\huxbpad{4pt}} &
\multicolumn{1}{r!{\huxvb{0}}}{\huxtpad{4pt}\raggedleft {\fontsize{9.5pt}{11.4pt}\selectfont 17.163 ***}\huxbpad{4pt}} &
\multicolumn{1}{r!{\huxvb{0}}}{\huxtpad{4pt}\raggedleft {\fontsize{9.5pt}{11.4pt}\selectfont 18.847 ***}\huxbpad{4pt}} \tabularnewline[-0.5pt]


\hhline{}
\arrayrulecolor{black}

\multicolumn{1}{!{\huxvb{0}}l!{\huxvb{0}}}{\huxtpad{4pt}\raggedright {\fontsize{9.5pt}{11.4pt}\selectfont }\huxbpad{4pt}} &
\multicolumn{1}{r!{\huxvb{0}}}{\huxtpad{4pt}\raggedleft {\fontsize{9.5pt}{11.4pt}\selectfont (1.323)~~~}\huxbpad{4pt}} &
\multicolumn{1}{r!{\huxvb{0}}}{\huxtpad{4pt}\raggedleft {\fontsize{9.5pt}{11.4pt}\selectfont (1.247)~~~}\huxbpad{4pt}} &
\multicolumn{1}{r!{\huxvb{0}}}{\huxtpad{4pt}\raggedleft {\fontsize{9.5pt}{11.4pt}\selectfont (1.238)~~~}\huxbpad{4pt}} \tabularnewline[-0.5pt]


\hhline{}
\arrayrulecolor{black}

\multicolumn{1}{!{\huxvb{0}}l!{\huxvb{0}}}{\huxtpad{4pt}\raggedright {\fontsize{9.5pt}{11.4pt}\selectfont sample\_tract24031703206}\huxbpad{4pt}} &
\multicolumn{1}{r!{\huxvb{0}}}{\huxtpad{4pt}\raggedleft {\fontsize{9.5pt}{11.4pt}\selectfont 18.264 ***}\huxbpad{4pt}} &
\multicolumn{1}{r!{\huxvb{0}}}{\huxtpad{4pt}\raggedleft {\fontsize{9.5pt}{11.4pt}\selectfont 17.717 ***}\huxbpad{4pt}} &
\multicolumn{1}{r!{\huxvb{0}}}{\huxtpad{4pt}\raggedleft {\fontsize{9.5pt}{11.4pt}\selectfont 18.547 ***}\huxbpad{4pt}} \tabularnewline[-0.5pt]


\hhline{}
\arrayrulecolor{black}

\multicolumn{1}{!{\huxvb{0}}l!{\huxvb{0}}}{\huxtpad{4pt}\raggedright {\fontsize{9.5pt}{11.4pt}\selectfont }\huxbpad{4pt}} &
\multicolumn{1}{r!{\huxvb{0}}}{\huxtpad{4pt}\raggedleft {\fontsize{9.5pt}{11.4pt}\selectfont (1.212)~~~}\huxbpad{4pt}} &
\multicolumn{1}{r!{\huxvb{0}}}{\huxtpad{4pt}\raggedleft {\fontsize{9.5pt}{11.4pt}\selectfont (1.569)~~~}\huxbpad{4pt}} &
\multicolumn{1}{r!{\huxvb{0}}}{\huxtpad{4pt}\raggedleft {\fontsize{9.5pt}{11.4pt}\selectfont (2.462)~~~}\huxbpad{4pt}} \tabularnewline[-0.5pt]


\hhline{}
\arrayrulecolor{black}

\multicolumn{1}{!{\huxvb{0}}l!{\huxvb{0}}}{\huxtpad{4pt}\raggedright {\fontsize{9.5pt}{11.4pt}\selectfont sample\_tract24031703209}\huxbpad{4pt}} &
\multicolumn{1}{r!{\huxvb{0}}}{\huxtpad{4pt}\raggedleft {\fontsize{9.5pt}{11.4pt}\selectfont 1.048~~~~}\huxbpad{4pt}} &
\multicolumn{1}{r!{\huxvb{0}}}{\huxtpad{4pt}\raggedleft {\fontsize{9.5pt}{11.4pt}\selectfont 0.820~~~~}\huxbpad{4pt}} &
\multicolumn{1}{r!{\huxvb{0}}}{\huxtpad{4pt}\raggedleft {\fontsize{9.5pt}{11.4pt}\selectfont 0.630~~~~}\huxbpad{4pt}} \tabularnewline[-0.5pt]


\hhline{}
\arrayrulecolor{black}

\multicolumn{1}{!{\huxvb{0}}l!{\huxvb{0}}}{\huxtpad{4pt}\raggedright {\fontsize{9.5pt}{11.4pt}\selectfont }\huxbpad{4pt}} &
\multicolumn{1}{r!{\huxvb{0}}}{\huxtpad{4pt}\raggedleft {\fontsize{9.5pt}{11.4pt}\selectfont (1.147)~~~}\huxbpad{4pt}} &
\multicolumn{1}{r!{\huxvb{0}}}{\huxtpad{4pt}\raggedleft {\fontsize{9.5pt}{11.4pt}\selectfont (1.312)~~~}\huxbpad{4pt}} &
\multicolumn{1}{r!{\huxvb{0}}}{\huxtpad{4pt}\raggedleft {\fontsize{9.5pt}{11.4pt}\selectfont (1.130)~~~}\huxbpad{4pt}} \tabularnewline[-0.5pt]


\hhline{}
\arrayrulecolor{black}

\multicolumn{1}{!{\huxvb{0}}l!{\huxvb{0}}}{\huxtpad{4pt}\raggedright {\fontsize{9.5pt}{11.4pt}\selectfont sample\_tract24031703210}\huxbpad{4pt}} &
\multicolumn{1}{r!{\huxvb{0}}}{\huxtpad{4pt}\raggedleft {\fontsize{9.5pt}{11.4pt}\selectfont 0.363~~~~}\huxbpad{4pt}} &
\multicolumn{1}{r!{\huxvb{0}}}{\huxtpad{4pt}\raggedleft {\fontsize{9.5pt}{11.4pt}\selectfont -0.512~~~~}\huxbpad{4pt}} &
\multicolumn{1}{r!{\huxvb{0}}}{\huxtpad{4pt}\raggedleft {\fontsize{9.5pt}{11.4pt}\selectfont -0.208~~~~}\huxbpad{4pt}} \tabularnewline[-0.5pt]


\hhline{}
\arrayrulecolor{black}

\multicolumn{1}{!{\huxvb{0}}l!{\huxvb{0}}}{\huxtpad{4pt}\raggedright {\fontsize{9.5pt}{11.4pt}\selectfont }\huxbpad{4pt}} &
\multicolumn{1}{r!{\huxvb{0}}}{\huxtpad{4pt}\raggedleft {\fontsize{9.5pt}{11.4pt}\selectfont (1.157)~~~}\huxbpad{4pt}} &
\multicolumn{1}{r!{\huxvb{0}}}{\huxtpad{4pt}\raggedleft {\fontsize{9.5pt}{11.4pt}\selectfont (1.399)~~~}\huxbpad{4pt}} &
\multicolumn{1}{r!{\huxvb{0}}}{\huxtpad{4pt}\raggedleft {\fontsize{9.5pt}{11.4pt}\selectfont (1.352)~~~}\huxbpad{4pt}} \tabularnewline[-0.5pt]


\hhline{}
\arrayrulecolor{black}

\multicolumn{1}{!{\huxvb{0}}l!{\huxvb{0}}}{\huxtpad{4pt}\raggedright {\fontsize{9.5pt}{11.4pt}\selectfont sample\_tract24031703212}\huxbpad{4pt}} &
\multicolumn{1}{r!{\huxvb{0}}}{\huxtpad{4pt}\raggedleft {\fontsize{9.5pt}{11.4pt}\selectfont 18.604 ***}\huxbpad{4pt}} &
\multicolumn{1}{r!{\huxvb{0}}}{\huxtpad{4pt}\raggedleft {\fontsize{9.5pt}{11.4pt}\selectfont 18.231 ***}\huxbpad{4pt}} &
\multicolumn{1}{r!{\huxvb{0}}}{\huxtpad{4pt}\raggedleft {\fontsize{9.5pt}{11.4pt}\selectfont 19.784 ***}\huxbpad{4pt}} \tabularnewline[-0.5pt]


\hhline{}
\arrayrulecolor{black}

\multicolumn{1}{!{\huxvb{0}}l!{\huxvb{0}}}{\huxtpad{4pt}\raggedright {\fontsize{9.5pt}{11.4pt}\selectfont }\huxbpad{4pt}} &
\multicolumn{1}{r!{\huxvb{0}}}{\huxtpad{4pt}\raggedleft {\fontsize{9.5pt}{11.4pt}\selectfont (1.113)~~~}\huxbpad{4pt}} &
\multicolumn{1}{r!{\huxvb{0}}}{\huxtpad{4pt}\raggedleft {\fontsize{9.5pt}{11.4pt}\selectfont (1.265)~~~}\huxbpad{4pt}} &
\multicolumn{1}{r!{\huxvb{0}}}{\huxtpad{4pt}\raggedleft {\fontsize{9.5pt}{11.4pt}\selectfont (1.252)~~~}\huxbpad{4pt}} \tabularnewline[-0.5pt]


\hhline{}
\arrayrulecolor{black}

\multicolumn{1}{!{\huxvb{0}}l!{\huxvb{0}}}{\huxtpad{4pt}\raggedright {\fontsize{9.5pt}{11.4pt}\selectfont sample\_tract24031703215}\huxbpad{4pt}} &
\multicolumn{1}{r!{\huxvb{0}}}{\huxtpad{4pt}\raggedleft {\fontsize{9.5pt}{11.4pt}\selectfont 17.952 ***}\huxbpad{4pt}} &
\multicolumn{1}{r!{\huxvb{0}}}{\huxtpad{4pt}\raggedleft {\fontsize{9.5pt}{11.4pt}\selectfont 17.786 ***}\huxbpad{4pt}} &
\multicolumn{1}{r!{\huxvb{0}}}{\huxtpad{4pt}\raggedleft {\fontsize{9.5pt}{11.4pt}\selectfont 18.609 ***}\huxbpad{4pt}} \tabularnewline[-0.5pt]


\hhline{}
\arrayrulecolor{black}

\multicolumn{1}{!{\huxvb{0}}l!{\huxvb{0}}}{\huxtpad{4pt}\raggedright {\fontsize{9.5pt}{11.4pt}\selectfont }\huxbpad{4pt}} &
\multicolumn{1}{r!{\huxvb{0}}}{\huxtpad{4pt}\raggedleft {\fontsize{9.5pt}{11.4pt}\selectfont (1.726)~~~}\huxbpad{4pt}} &
\multicolumn{1}{r!{\huxvb{0}}}{\huxtpad{4pt}\raggedleft {\fontsize{9.5pt}{11.4pt}\selectfont (1.667)~~~}\huxbpad{4pt}} &
\multicolumn{1}{r!{\huxvb{0}}}{\huxtpad{4pt}\raggedleft {\fontsize{9.5pt}{11.4pt}\selectfont (1.505)~~~}\huxbpad{4pt}} \tabularnewline[-0.5pt]


\hhline{}
\arrayrulecolor{black}

\multicolumn{1}{!{\huxvb{0}}l!{\huxvb{0}}}{\huxtpad{4pt}\raggedright {\fontsize{9.5pt}{11.4pt}\selectfont sample\_tract24031703220}\huxbpad{4pt}} &
\multicolumn{1}{r!{\huxvb{0}}}{\huxtpad{4pt}\raggedleft {\fontsize{9.5pt}{11.4pt}\selectfont 19.442 ***}\huxbpad{4pt}} &
\multicolumn{1}{r!{\huxvb{0}}}{\huxtpad{4pt}\raggedleft {\fontsize{9.5pt}{11.4pt}\selectfont 19.063 ***}\huxbpad{4pt}} &
\multicolumn{1}{r!{\huxvb{0}}}{\huxtpad{4pt}\raggedleft {\fontsize{9.5pt}{11.4pt}\selectfont 20.886 ***}\huxbpad{4pt}} \tabularnewline[-0.5pt]


\hhline{}
\arrayrulecolor{black}

\multicolumn{1}{!{\huxvb{0}}l!{\huxvb{0}}}{\huxtpad{4pt}\raggedright {\fontsize{9.5pt}{11.4pt}\selectfont }\huxbpad{4pt}} &
\multicolumn{1}{r!{\huxvb{0}}}{\huxtpad{4pt}\raggedleft {\fontsize{9.5pt}{11.4pt}\selectfont (1.475)~~~}\huxbpad{4pt}} &
\multicolumn{1}{r!{\huxvb{0}}}{\huxtpad{4pt}\raggedleft {\fontsize{9.5pt}{11.4pt}\selectfont (1.837)~~~}\huxbpad{4pt}} &
\multicolumn{1}{r!{\huxvb{0}}}{\huxtpad{4pt}\raggedleft {\fontsize{9.5pt}{11.4pt}\selectfont (1.902)~~~}\huxbpad{4pt}} \tabularnewline[-0.5pt]


\hhline{}
\arrayrulecolor{black}

\multicolumn{1}{!{\huxvb{0}}l!{\huxvb{0}}}{\huxtpad{4pt}\raggedright {\fontsize{9.5pt}{11.4pt}\selectfont sample\_tract24031703221}\huxbpad{4pt}} &
\multicolumn{1}{r!{\huxvb{0}}}{\huxtpad{4pt}\raggedleft {\fontsize{9.5pt}{11.4pt}\selectfont 17.586 ***}\huxbpad{4pt}} &
\multicolumn{1}{r!{\huxvb{0}}}{\huxtpad{4pt}\raggedleft {\fontsize{9.5pt}{11.4pt}\selectfont 17.238 ***}\huxbpad{4pt}} &
\multicolumn{1}{r!{\huxvb{0}}}{\huxtpad{4pt}\raggedleft {\fontsize{9.5pt}{11.4pt}\selectfont 18.474 ***}\huxbpad{4pt}} \tabularnewline[-0.5pt]


\hhline{}
\arrayrulecolor{black}

\multicolumn{1}{!{\huxvb{0}}l!{\huxvb{0}}}{\huxtpad{4pt}\raggedright {\fontsize{9.5pt}{11.4pt}\selectfont }\huxbpad{4pt}} &
\multicolumn{1}{r!{\huxvb{0}}}{\huxtpad{4pt}\raggedleft {\fontsize{9.5pt}{11.4pt}\selectfont (1.132)~~~}\huxbpad{4pt}} &
\multicolumn{1}{r!{\huxvb{0}}}{\huxtpad{4pt}\raggedleft {\fontsize{9.5pt}{11.4pt}\selectfont (1.258)~~~}\huxbpad{4pt}} &
\multicolumn{1}{r!{\huxvb{0}}}{\huxtpad{4pt}\raggedleft {\fontsize{9.5pt}{11.4pt}\selectfont (1.263)~~~}\huxbpad{4pt}} \tabularnewline[-0.5pt]


\hhline{}
\arrayrulecolor{black}

\multicolumn{1}{!{\huxvb{0}}l!{\huxvb{0}}}{\huxtpad{4pt}\raggedright {\fontsize{9.5pt}{11.4pt}\selectfont sample\_tract24031703501}\huxbpad{4pt}} &
\multicolumn{1}{r!{\huxvb{0}}}{\huxtpad{4pt}\raggedleft {\fontsize{9.5pt}{11.4pt}\selectfont 19.771 ***}\huxbpad{4pt}} &
\multicolumn{1}{r!{\huxvb{0}}}{\huxtpad{4pt}\raggedleft {\fontsize{9.5pt}{11.4pt}\selectfont 18.870 ***}\huxbpad{4pt}} &
\multicolumn{1}{r!{\huxvb{0}}}{\huxtpad{4pt}\raggedleft {\fontsize{9.5pt}{11.4pt}\selectfont 20.491 ***}\huxbpad{4pt}} \tabularnewline[-0.5pt]


\hhline{}
\arrayrulecolor{black}

\multicolumn{1}{!{\huxvb{0}}l!{\huxvb{0}}}{\huxtpad{4pt}\raggedright {\fontsize{9.5pt}{11.4pt}\selectfont }\huxbpad{4pt}} &
\multicolumn{1}{r!{\huxvb{0}}}{\huxtpad{4pt}\raggedleft {\fontsize{9.5pt}{11.4pt}\selectfont (1.097)~~~}\huxbpad{4pt}} &
\multicolumn{1}{r!{\huxvb{0}}}{\huxtpad{4pt}\raggedleft {\fontsize{9.5pt}{11.4pt}\selectfont (1.233)~~~}\huxbpad{4pt}} &
\multicolumn{1}{r!{\huxvb{0}}}{\huxtpad{4pt}\raggedleft {\fontsize{9.5pt}{11.4pt}\selectfont (1.303)~~~}\huxbpad{4pt}} \tabularnewline[-0.5pt]


\hhline{}
\arrayrulecolor{black}

\multicolumn{1}{!{\huxvb{0}}l!{\huxvb{0}}}{\huxtpad{4pt}\raggedright {\fontsize{9.5pt}{11.4pt}\selectfont sample\_tract24031703601}\huxbpad{4pt}} &
\multicolumn{1}{r!{\huxvb{0}}}{\huxtpad{4pt}\raggedleft {\fontsize{9.5pt}{11.4pt}\selectfont 19.814 ***}\huxbpad{4pt}} &
\multicolumn{1}{r!{\huxvb{0}}}{\huxtpad{4pt}\raggedleft {\fontsize{9.5pt}{11.4pt}\selectfont 19.290 ***}\huxbpad{4pt}} &
\multicolumn{1}{r!{\huxvb{0}}}{\huxtpad{4pt}\raggedleft {\fontsize{9.5pt}{11.4pt}\selectfont 21.120 ***}\huxbpad{4pt}} \tabularnewline[-0.5pt]


\hhline{}
\arrayrulecolor{black}

\multicolumn{1}{!{\huxvb{0}}l!{\huxvb{0}}}{\huxtpad{4pt}\raggedright {\fontsize{9.5pt}{11.4pt}\selectfont }\huxbpad{4pt}} &
\multicolumn{1}{r!{\huxvb{0}}}{\huxtpad{4pt}\raggedleft {\fontsize{9.5pt}{11.4pt}\selectfont (1.251)~~~}\huxbpad{4pt}} &
\multicolumn{1}{r!{\huxvb{0}}}{\huxtpad{4pt}\raggedleft {\fontsize{9.5pt}{11.4pt}\selectfont (1.313)~~~}\huxbpad{4pt}} &
\multicolumn{1}{r!{\huxvb{0}}}{\huxtpad{4pt}\raggedleft {\fontsize{9.5pt}{11.4pt}\selectfont (1.247)~~~}\huxbpad{4pt}} \tabularnewline[-0.5pt]


\hhline{}
\arrayrulecolor{black}

\multicolumn{1}{!{\huxvb{0}}l!{\huxvb{0}}}{\huxtpad{4pt}\raggedright {\fontsize{9.5pt}{11.4pt}\selectfont sample\_tract24031703902}\huxbpad{4pt}} &
\multicolumn{1}{r!{\huxvb{0}}}{\huxtpad{4pt}\raggedleft {\fontsize{9.5pt}{11.4pt}\selectfont 20.143 ***}\huxbpad{4pt}} &
\multicolumn{1}{r!{\huxvb{0}}}{\huxtpad{4pt}\raggedleft {\fontsize{9.5pt}{11.4pt}\selectfont 19.784 ***}\huxbpad{4pt}} &
\multicolumn{1}{r!{\huxvb{0}}}{\huxtpad{4pt}\raggedleft {\fontsize{9.5pt}{11.4pt}\selectfont 21.314 ***}\huxbpad{4pt}} \tabularnewline[-0.5pt]


\hhline{}
\arrayrulecolor{black}

\multicolumn{1}{!{\huxvb{0}}l!{\huxvb{0}}}{\huxtpad{4pt}\raggedright {\fontsize{9.5pt}{11.4pt}\selectfont }\huxbpad{4pt}} &
\multicolumn{1}{r!{\huxvb{0}}}{\huxtpad{4pt}\raggedleft {\fontsize{9.5pt}{11.4pt}\selectfont (2.261)~~~}\huxbpad{4pt}} &
\multicolumn{1}{r!{\huxvb{0}}}{\huxtpad{4pt}\raggedleft {\fontsize{9.5pt}{11.4pt}\selectfont (2.488)~~~}\huxbpad{4pt}} &
\multicolumn{1}{r!{\huxvb{0}}}{\huxtpad{4pt}\raggedleft {\fontsize{9.5pt}{11.4pt}\selectfont (1.951)~~~}\huxbpad{4pt}} \tabularnewline[-0.5pt]


\hhline{}
\arrayrulecolor{black}

\multicolumn{1}{!{\huxvb{0}}l!{\huxvb{0}}}{\huxtpad{4pt}\raggedright {\fontsize{9.5pt}{11.4pt}\selectfont sample\_tract24031704000}\huxbpad{4pt}} &
\multicolumn{1}{r!{\huxvb{0}}}{\huxtpad{4pt}\raggedleft {\fontsize{9.5pt}{11.4pt}\selectfont 18.395 ***}\huxbpad{4pt}} &
\multicolumn{1}{r!{\huxvb{0}}}{\huxtpad{4pt}\raggedleft {\fontsize{9.5pt}{11.4pt}\selectfont 18.197 ***}\huxbpad{4pt}} &
\multicolumn{1}{r!{\huxvb{0}}}{\huxtpad{4pt}\raggedleft {\fontsize{9.5pt}{11.4pt}\selectfont 19.159 ***}\huxbpad{4pt}} \tabularnewline[-0.5pt]


\hhline{}
\arrayrulecolor{black}

\multicolumn{1}{!{\huxvb{0}}l!{\huxvb{0}}}{\huxtpad{4pt}\raggedright {\fontsize{9.5pt}{11.4pt}\selectfont }\huxbpad{4pt}} &
\multicolumn{1}{r!{\huxvb{0}}}{\huxtpad{4pt}\raggedleft {\fontsize{9.5pt}{11.4pt}\selectfont (1.511)~~~}\huxbpad{4pt}} &
\multicolumn{1}{r!{\huxvb{0}}}{\huxtpad{4pt}\raggedleft {\fontsize{9.5pt}{11.4pt}\selectfont (1.632)~~~}\huxbpad{4pt}} &
\multicolumn{1}{r!{\huxvb{0}}}{\huxtpad{4pt}\raggedleft {\fontsize{9.5pt}{11.4pt}\selectfont (1.554)~~~}\huxbpad{4pt}} \tabularnewline[-0.5pt]


\hhline{}
\arrayrulecolor{black}

\multicolumn{1}{!{\huxvb{0}}l!{\huxvb{0}}}{\huxtpad{4pt}\raggedright {\fontsize{9.5pt}{11.4pt}\selectfont sample\_tract24031706012}\huxbpad{4pt}} &
\multicolumn{1}{r!{\huxvb{0}}}{\huxtpad{4pt}\raggedleft {\fontsize{9.5pt}{11.4pt}\selectfont 18.856 ***}\huxbpad{4pt}} &
\multicolumn{1}{r!{\huxvb{0}}}{\huxtpad{4pt}\raggedleft {\fontsize{9.5pt}{11.4pt}\selectfont 18.498 ***}\huxbpad{4pt}} &
\multicolumn{1}{r!{\huxvb{0}}}{\huxtpad{4pt}\raggedleft {\fontsize{9.5pt}{11.4pt}\selectfont 19.582 ***}\huxbpad{4pt}} \tabularnewline[-0.5pt]


\hhline{}
\arrayrulecolor{black}

\multicolumn{1}{!{\huxvb{0}}l!{\huxvb{0}}}{\huxtpad{4pt}\raggedright {\fontsize{9.5pt}{11.4pt}\selectfont }\huxbpad{4pt}} &
\multicolumn{1}{r!{\huxvb{0}}}{\huxtpad{4pt}\raggedleft {\fontsize{9.5pt}{11.4pt}\selectfont (1.364)~~~}\huxbpad{4pt}} &
\multicolumn{1}{r!{\huxvb{0}}}{\huxtpad{4pt}\raggedleft {\fontsize{9.5pt}{11.4pt}\selectfont (1.793)~~~}\huxbpad{4pt}} &
\multicolumn{1}{r!{\huxvb{0}}}{\huxtpad{4pt}\raggedleft {\fontsize{9.5pt}{11.4pt}\selectfont (1.339)~~~}\huxbpad{4pt}} \tabularnewline[-0.5pt]


\hhline{}
\arrayrulecolor{black}

\multicolumn{1}{!{\huxvb{0}}l!{\huxvb{0}}}{\huxtpad{4pt}\raggedright {\fontsize{9.5pt}{11.4pt}\selectfont sample\_tract24033800103}\huxbpad{4pt}} &
\multicolumn{1}{r!{\huxvb{0}}}{\huxtpad{4pt}\raggedleft {\fontsize{9.5pt}{11.4pt}\selectfont 20.047 ***}\huxbpad{4pt}} &
\multicolumn{1}{r!{\huxvb{0}}}{\huxtpad{4pt}\raggedleft {\fontsize{9.5pt}{11.4pt}\selectfont 19.815 ***}\huxbpad{4pt}} &
\multicolumn{1}{r!{\huxvb{0}}}{\huxtpad{4pt}\raggedleft {\fontsize{9.5pt}{11.4pt}\selectfont 21.379 ***}\huxbpad{4pt}} \tabularnewline[-0.5pt]


\hhline{}
\arrayrulecolor{black}

\multicolumn{1}{!{\huxvb{0}}l!{\huxvb{0}}}{\huxtpad{4pt}\raggedright {\fontsize{9.5pt}{11.4pt}\selectfont }\huxbpad{4pt}} &
\multicolumn{1}{r!{\huxvb{0}}}{\huxtpad{4pt}\raggedleft {\fontsize{9.5pt}{11.4pt}\selectfont (1.277)~~~}\huxbpad{4pt}} &
\multicolumn{1}{r!{\huxvb{0}}}{\huxtpad{4pt}\raggedleft {\fontsize{9.5pt}{11.4pt}\selectfont (1.423)~~~}\huxbpad{4pt}} &
\multicolumn{1}{r!{\huxvb{0}}}{\huxtpad{4pt}\raggedleft {\fontsize{9.5pt}{11.4pt}\selectfont (1.245)~~~}\huxbpad{4pt}} \tabularnewline[-0.5pt]


\hhline{}
\arrayrulecolor{black}

\multicolumn{1}{!{\huxvb{0}}l!{\huxvb{0}}}{\huxtpad{4pt}\raggedright {\fontsize{9.5pt}{11.4pt}\selectfont sample\_tract24033805909}\huxbpad{4pt}} &
\multicolumn{1}{r!{\huxvb{0}}}{\huxtpad{4pt}\raggedleft {\fontsize{9.5pt}{11.4pt}\selectfont 17.692 ***}\huxbpad{4pt}} &
\multicolumn{1}{r!{\huxvb{0}}}{\huxtpad{4pt}\raggedleft {\fontsize{9.5pt}{11.4pt}\selectfont 17.648 ***}\huxbpad{4pt}} &
\multicolumn{1}{r!{\huxvb{0}}}{\huxtpad{4pt}\raggedleft {\fontsize{9.5pt}{11.4pt}\selectfont 19.292 ***}\huxbpad{4pt}} \tabularnewline[-0.5pt]


\hhline{}
\arrayrulecolor{black}

\multicolumn{1}{!{\huxvb{0}}l!{\huxvb{0}}}{\huxtpad{4pt}\raggedright {\fontsize{9.5pt}{11.4pt}\selectfont }\huxbpad{4pt}} &
\multicolumn{1}{r!{\huxvb{0}}}{\huxtpad{4pt}\raggedleft {\fontsize{9.5pt}{11.4pt}\selectfont (1.617)~~~}\huxbpad{4pt}} &
\multicolumn{1}{r!{\huxvb{0}}}{\huxtpad{4pt}\raggedleft {\fontsize{9.5pt}{11.4pt}\selectfont (1.608)~~~}\huxbpad{4pt}} &
\multicolumn{1}{r!{\huxvb{0}}}{\huxtpad{4pt}\raggedleft {\fontsize{9.5pt}{11.4pt}\selectfont (1.332)~~~}\huxbpad{4pt}} \tabularnewline[-0.5pt]


\hhline{}
\arrayrulecolor{black}

\multicolumn{1}{!{\huxvb{0}}l!{\huxvb{0}}}{\huxtpad{4pt}\raggedright {\fontsize{9.5pt}{11.4pt}\selectfont sample\_tract24033806706}\huxbpad{4pt}} &
\multicolumn{1}{r!{\huxvb{0}}}{\huxtpad{4pt}\raggedleft {\fontsize{9.5pt}{11.4pt}\selectfont 17.782 ***}\huxbpad{4pt}} &
\multicolumn{1}{r!{\huxvb{0}}}{\huxtpad{4pt}\raggedleft {\fontsize{9.5pt}{11.4pt}\selectfont 17.238 ***}\huxbpad{4pt}} &
\multicolumn{1}{r!{\huxvb{0}}}{\huxtpad{4pt}\raggedleft {\fontsize{9.5pt}{11.4pt}\selectfont 18.929 ***}\huxbpad{4pt}} \tabularnewline[-0.5pt]


\hhline{}
\arrayrulecolor{black}

\multicolumn{1}{!{\huxvb{0}}l!{\huxvb{0}}}{\huxtpad{4pt}\raggedright {\fontsize{9.5pt}{11.4pt}\selectfont }\huxbpad{4pt}} &
\multicolumn{1}{r!{\huxvb{0}}}{\huxtpad{4pt}\raggedleft {\fontsize{9.5pt}{11.4pt}\selectfont (1.272)~~~}\huxbpad{4pt}} &
\multicolumn{1}{r!{\huxvb{0}}}{\huxtpad{4pt}\raggedleft {\fontsize{9.5pt}{11.4pt}\selectfont (1.278)~~~}\huxbpad{4pt}} &
\multicolumn{1}{r!{\huxvb{0}}}{\huxtpad{4pt}\raggedleft {\fontsize{9.5pt}{11.4pt}\selectfont (1.266)~~~}\huxbpad{4pt}} \tabularnewline[-0.5pt]


\hhline{}
\arrayrulecolor{black}

\multicolumn{1}{!{\huxvb{0}}l!{\huxvb{0}}}{\huxtpad{4pt}\raggedright {\fontsize{9.5pt}{11.4pt}\selectfont sample\_tract24033806900}\huxbpad{4pt}} &
\multicolumn{1}{r!{\huxvb{0}}}{\huxtpad{4pt}\raggedleft {\fontsize{9.5pt}{11.4pt}\selectfont 19.346 ***}\huxbpad{4pt}} &
\multicolumn{1}{r!{\huxvb{0}}}{\huxtpad{4pt}\raggedleft {\fontsize{9.5pt}{11.4pt}\selectfont 19.344 ***}\huxbpad{4pt}} &
\multicolumn{1}{r!{\huxvb{0}}}{\huxtpad{4pt}\raggedleft {\fontsize{9.5pt}{11.4pt}\selectfont 21.753 ***}\huxbpad{4pt}} \tabularnewline[-0.5pt]


\hhline{}
\arrayrulecolor{black}

\multicolumn{1}{!{\huxvb{0}}l!{\huxvb{0}}}{\huxtpad{4pt}\raggedright {\fontsize{9.5pt}{11.4pt}\selectfont }\huxbpad{4pt}} &
\multicolumn{1}{r!{\huxvb{0}}}{\huxtpad{4pt}\raggedleft {\fontsize{9.5pt}{11.4pt}\selectfont (1.400)~~~}\huxbpad{4pt}} &
\multicolumn{1}{r!{\huxvb{0}}}{\huxtpad{4pt}\raggedleft {\fontsize{9.5pt}{11.4pt}\selectfont (1.561)~~~}\huxbpad{4pt}} &
\multicolumn{1}{r!{\huxvb{0}}}{\huxtpad{4pt}\raggedleft {\fontsize{9.5pt}{11.4pt}\selectfont (1.417)~~~}\huxbpad{4pt}} \tabularnewline[-0.5pt]


\hhline{}
\arrayrulecolor{black}

\multicolumn{1}{!{\huxvb{0}}l!{\huxvb{0}}}{\huxtpad{4pt}\raggedright {\fontsize{9.5pt}{11.4pt}\selectfont sample\_tract24033807301}\huxbpad{4pt}} &
\multicolumn{1}{r!{\huxvb{0}}}{\huxtpad{4pt}\raggedleft {\fontsize{9.5pt}{11.4pt}\selectfont 17.983 ***}\huxbpad{4pt}} &
\multicolumn{1}{r!{\huxvb{0}}}{\huxtpad{4pt}\raggedleft {\fontsize{9.5pt}{11.4pt}\selectfont 17.960 ***}\huxbpad{4pt}} &
\multicolumn{1}{r!{\huxvb{0}}}{\huxtpad{4pt}\raggedleft {\fontsize{9.5pt}{11.4pt}\selectfont 19.460 ***}\huxbpad{4pt}} \tabularnewline[-0.5pt]


\hhline{}
\arrayrulecolor{black}

\multicolumn{1}{!{\huxvb{0}}l!{\huxvb{0}}}{\huxtpad{4pt}\raggedright {\fontsize{9.5pt}{11.4pt}\selectfont }\huxbpad{4pt}} &
\multicolumn{1}{r!{\huxvb{0}}}{\huxtpad{4pt}\raggedleft {\fontsize{9.5pt}{11.4pt}\selectfont (1.482)~~~}\huxbpad{4pt}} &
\multicolumn{1}{r!{\huxvb{0}}}{\huxtpad{4pt}\raggedleft {\fontsize{9.5pt}{11.4pt}\selectfont (1.596)~~~}\huxbpad{4pt}} &
\multicolumn{1}{r!{\huxvb{0}}}{\huxtpad{4pt}\raggedleft {\fontsize{9.5pt}{11.4pt}\selectfont (1.287)~~~}\huxbpad{4pt}} \tabularnewline[-0.5pt]


\hhline{}
\arrayrulecolor{black}

\multicolumn{1}{!{\huxvb{0}}l!{\huxvb{0}}}{\huxtpad{4pt}\raggedright {\fontsize{9.5pt}{11.4pt}\selectfont sample\_tract24033807305}\huxbpad{4pt}} &
\multicolumn{1}{r!{\huxvb{0}}}{\huxtpad{4pt}\raggedleft {\fontsize{9.5pt}{11.4pt}\selectfont 19.140 ***}\huxbpad{4pt}} &
\multicolumn{1}{r!{\huxvb{0}}}{\huxtpad{4pt}\raggedleft {\fontsize{9.5pt}{11.4pt}\selectfont 18.696 ***}\huxbpad{4pt}} &
\multicolumn{1}{r!{\huxvb{0}}}{\huxtpad{4pt}\raggedleft {\fontsize{9.5pt}{11.4pt}\selectfont 20.658 ***}\huxbpad{4pt}} \tabularnewline[-0.5pt]


\hhline{}
\arrayrulecolor{black}

\multicolumn{1}{!{\huxvb{0}}l!{\huxvb{0}}}{\huxtpad{4pt}\raggedright {\fontsize{9.5pt}{11.4pt}\selectfont }\huxbpad{4pt}} &
\multicolumn{1}{r!{\huxvb{0}}}{\huxtpad{4pt}\raggedleft {\fontsize{9.5pt}{11.4pt}\selectfont (1.454)~~~}\huxbpad{4pt}} &
\multicolumn{1}{r!{\huxvb{0}}}{\huxtpad{4pt}\raggedleft {\fontsize{9.5pt}{11.4pt}\selectfont (1.393)~~~}\huxbpad{4pt}} &
\multicolumn{1}{r!{\huxvb{0}}}{\huxtpad{4pt}\raggedleft {\fontsize{9.5pt}{11.4pt}\selectfont (1.283)~~~}\huxbpad{4pt}} \tabularnewline[-0.5pt]


\hhline{}
\arrayrulecolor{black}

\multicolumn{1}{!{\huxvb{0}}l!{\huxvb{0}}}{\huxtpad{4pt}\raggedright {\fontsize{9.5pt}{11.4pt}\selectfont sample\_tract24033807404}\huxbpad{4pt}} &
\multicolumn{1}{r!{\huxvb{0}}}{\huxtpad{4pt}\raggedleft {\fontsize{9.5pt}{11.4pt}\selectfont 21.266 ***}\huxbpad{4pt}} &
\multicolumn{1}{r!{\huxvb{0}}}{\huxtpad{4pt}\raggedleft {\fontsize{9.5pt}{11.4pt}\selectfont 20.872 ***}\huxbpad{4pt}} &
\multicolumn{1}{r!{\huxvb{0}}}{\huxtpad{4pt}\raggedleft {\fontsize{9.5pt}{11.4pt}\selectfont 23.149 ***}\huxbpad{4pt}} \tabularnewline[-0.5pt]


\hhline{}
\arrayrulecolor{black}

\multicolumn{1}{!{\huxvb{0}}l!{\huxvb{0}}}{\huxtpad{4pt}\raggedright {\fontsize{9.5pt}{11.4pt}\selectfont }\huxbpad{4pt}} &
\multicolumn{1}{r!{\huxvb{0}}}{\huxtpad{4pt}\raggedleft {\fontsize{9.5pt}{11.4pt}\selectfont (1.644)~~~}\huxbpad{4pt}} &
\multicolumn{1}{r!{\huxvb{0}}}{\huxtpad{4pt}\raggedleft {\fontsize{9.5pt}{11.4pt}\selectfont (1.739)~~~}\huxbpad{4pt}} &
\multicolumn{1}{r!{\huxvb{0}}}{\huxtpad{4pt}\raggedleft {\fontsize{9.5pt}{11.4pt}\selectfont (1.776)~~~}\huxbpad{4pt}} \tabularnewline[-0.5pt]


\hhline{}
\arrayrulecolor{black}

\multicolumn{1}{!{\huxvb{0}}l!{\huxvb{0}}}{\huxtpad{4pt}\raggedright {\fontsize{9.5pt}{11.4pt}\selectfont sample\_tract24033807405}\huxbpad{4pt}} &
\multicolumn{1}{r!{\huxvb{0}}}{\huxtpad{4pt}\raggedleft {\fontsize{9.5pt}{11.4pt}\selectfont 20.781 ***}\huxbpad{4pt}} &
\multicolumn{1}{r!{\huxvb{0}}}{\huxtpad{4pt}\raggedleft {\fontsize{9.5pt}{11.4pt}\selectfont 20.632 ***}\huxbpad{4pt}} &
\multicolumn{1}{r!{\huxvb{0}}}{\huxtpad{4pt}\raggedleft {\fontsize{9.5pt}{11.4pt}\selectfont 22.259 ***}\huxbpad{4pt}} \tabularnewline[-0.5pt]


\hhline{}
\arrayrulecolor{black}

\multicolumn{1}{!{\huxvb{0}}l!{\huxvb{0}}}{\huxtpad{4pt}\raggedright {\fontsize{9.5pt}{11.4pt}\selectfont }\huxbpad{4pt}} &
\multicolumn{1}{r!{\huxvb{0}}}{\huxtpad{4pt}\raggedleft {\fontsize{9.5pt}{11.4pt}\selectfont (1.516)~~~}\huxbpad{4pt}} &
\multicolumn{1}{r!{\huxvb{0}}}{\huxtpad{4pt}\raggedleft {\fontsize{9.5pt}{11.4pt}\selectfont (1.598)~~~}\huxbpad{4pt}} &
\multicolumn{1}{r!{\huxvb{0}}}{\huxtpad{4pt}\raggedleft {\fontsize{9.5pt}{11.4pt}\selectfont (1.224)~~~}\huxbpad{4pt}} \tabularnewline[-0.5pt]


\hhline{}
\arrayrulecolor{black}

\multicolumn{1}{!{\huxvb{0}}l!{\huxvb{0}}}{\huxtpad{4pt}\raggedright {\fontsize{9.5pt}{11.4pt}\selectfont sample\_tract24033807407}\huxbpad{4pt}} &
\multicolumn{1}{r!{\huxvb{0}}}{\huxtpad{4pt}\raggedleft {\fontsize{9.5pt}{11.4pt}\selectfont 19.665 ***}\huxbpad{4pt}} &
\multicolumn{1}{r!{\huxvb{0}}}{\huxtpad{4pt}\raggedleft {\fontsize{9.5pt}{11.4pt}\selectfont 19.176 ***}\huxbpad{4pt}} &
\multicolumn{1}{r!{\huxvb{0}}}{\huxtpad{4pt}\raggedleft {\fontsize{9.5pt}{11.4pt}\selectfont 20.701 ***}\huxbpad{4pt}} \tabularnewline[-0.5pt]


\hhline{}
\arrayrulecolor{black}

\multicolumn{1}{!{\huxvb{0}}l!{\huxvb{0}}}{\huxtpad{4pt}\raggedright {\fontsize{9.5pt}{11.4pt}\selectfont }\huxbpad{4pt}} &
\multicolumn{1}{r!{\huxvb{0}}}{\huxtpad{4pt}\raggedleft {\fontsize{9.5pt}{11.4pt}\selectfont (1.311)~~~}\huxbpad{4pt}} &
\multicolumn{1}{r!{\huxvb{0}}}{\huxtpad{4pt}\raggedleft {\fontsize{9.5pt}{11.4pt}\selectfont (1.497)~~~}\huxbpad{4pt}} &
\multicolumn{1}{r!{\huxvb{0}}}{\huxtpad{4pt}\raggedleft {\fontsize{9.5pt}{11.4pt}\selectfont (1.809)~~~}\huxbpad{4pt}} \tabularnewline[-0.5pt]


\hhline{}
\arrayrulecolor{black}

\multicolumn{1}{!{\huxvb{0}}l!{\huxvb{0}}}{\huxtpad{4pt}\raggedright {\fontsize{9.5pt}{11.4pt}\selectfont sample\_tract51013102500}\huxbpad{4pt}} &
\multicolumn{1}{r!{\huxvb{0}}}{\huxtpad{4pt}\raggedleft {\fontsize{9.5pt}{11.4pt}\selectfont 39.336 ***}\huxbpad{4pt}} &
\multicolumn{1}{r!{\huxvb{0}}}{\huxtpad{4pt}\raggedleft {\fontsize{9.5pt}{11.4pt}\selectfont 39.152 ***}\huxbpad{4pt}} &
\multicolumn{1}{r!{\huxvb{0}}}{\huxtpad{4pt}\raggedleft {\fontsize{9.5pt}{11.4pt}\selectfont 41.308 ***}\huxbpad{4pt}} \tabularnewline[-0.5pt]


\hhline{}
\arrayrulecolor{black}

\multicolumn{1}{!{\huxvb{0}}l!{\huxvb{0}}}{\huxtpad{4pt}\raggedright {\fontsize{9.5pt}{11.4pt}\selectfont }\huxbpad{4pt}} &
\multicolumn{1}{r!{\huxvb{0}}}{\huxtpad{4pt}\raggedleft {\fontsize{9.5pt}{11.4pt}\selectfont (1.052)~~~}\huxbpad{4pt}} &
\multicolumn{1}{r!{\huxvb{0}}}{\huxtpad{4pt}\raggedleft {\fontsize{9.5pt}{11.4pt}\selectfont (1.151)~~~}\huxbpad{4pt}} &
\multicolumn{1}{r!{\huxvb{0}}}{\huxtpad{4pt}\raggedleft {\fontsize{9.5pt}{11.4pt}\selectfont (1.094)~~~}\huxbpad{4pt}} \tabularnewline[-0.5pt]


\hhline{}
\arrayrulecolor{black}

\multicolumn{1}{!{\huxvb{0}}l!{\huxvb{0}}}{\huxtpad{4pt}\raggedright {\fontsize{9.5pt}{11.4pt}\selectfont sample\_tract51013102701}\huxbpad{4pt}} &
\multicolumn{1}{r!{\huxvb{0}}}{\huxtpad{4pt}\raggedleft {\fontsize{9.5pt}{11.4pt}\selectfont 18.695 ***}\huxbpad{4pt}} &
\multicolumn{1}{r!{\huxvb{0}}}{\huxtpad{4pt}\raggedleft {\fontsize{9.5pt}{11.4pt}\selectfont 18.145 ***}\huxbpad{4pt}} &
\multicolumn{1}{r!{\huxvb{0}}}{\huxtpad{4pt}\raggedleft {\fontsize{9.5pt}{11.4pt}\selectfont 19.419 ***}\huxbpad{4pt}} \tabularnewline[-0.5pt]


\hhline{}
\arrayrulecolor{black}

\multicolumn{1}{!{\huxvb{0}}l!{\huxvb{0}}}{\huxtpad{4pt}\raggedright {\fontsize{9.5pt}{11.4pt}\selectfont }\huxbpad{4pt}} &
\multicolumn{1}{r!{\huxvb{0}}}{\huxtpad{4pt}\raggedleft {\fontsize{9.5pt}{11.4pt}\selectfont (1.644)~~~}\huxbpad{4pt}} &
\multicolumn{1}{r!{\huxvb{0}}}{\huxtpad{4pt}\raggedleft {\fontsize{9.5pt}{11.4pt}\selectfont (1.767)~~~}\huxbpad{4pt}} &
\multicolumn{1}{r!{\huxvb{0}}}{\huxtpad{4pt}\raggedleft {\fontsize{9.5pt}{11.4pt}\selectfont (1.465)~~~}\huxbpad{4pt}} \tabularnewline[-0.5pt]


\hhline{}
\arrayrulecolor{black}

\multicolumn{1}{!{\huxvb{0}}l!{\huxvb{0}}}{\huxtpad{4pt}\raggedright {\fontsize{9.5pt}{11.4pt}\selectfont sample\_tract51013102801}\huxbpad{4pt}} &
\multicolumn{1}{r!{\huxvb{0}}}{\huxtpad{4pt}\raggedleft {\fontsize{9.5pt}{11.4pt}\selectfont 19.094 ***}\huxbpad{4pt}} &
\multicolumn{1}{r!{\huxvb{0}}}{\huxtpad{4pt}\raggedleft {\fontsize{9.5pt}{11.4pt}\selectfont 18.891 ***}\huxbpad{4pt}} &
\multicolumn{1}{r!{\huxvb{0}}}{\huxtpad{4pt}\raggedleft {\fontsize{9.5pt}{11.4pt}\selectfont 20.285 ***}\huxbpad{4pt}} \tabularnewline[-0.5pt]


\hhline{}
\arrayrulecolor{black}

\multicolumn{1}{!{\huxvb{0}}l!{\huxvb{0}}}{\huxtpad{4pt}\raggedright {\fontsize{9.5pt}{11.4pt}\selectfont }\huxbpad{4pt}} &
\multicolumn{1}{r!{\huxvb{0}}}{\huxtpad{4pt}\raggedleft {\fontsize{9.5pt}{11.4pt}\selectfont (1.312)~~~}\huxbpad{4pt}} &
\multicolumn{1}{r!{\huxvb{0}}}{\huxtpad{4pt}\raggedleft {\fontsize{9.5pt}{11.4pt}\selectfont (1.556)~~~}\huxbpad{4pt}} &
\multicolumn{1}{r!{\huxvb{0}}}{\huxtpad{4pt}\raggedleft {\fontsize{9.5pt}{11.4pt}\selectfont (1.273)~~~}\huxbpad{4pt}} \tabularnewline[-0.5pt]


\hhline{}
\arrayrulecolor{black}

\multicolumn{1}{!{\huxvb{0}}l!{\huxvb{0}}}{\huxtpad{4pt}\raggedright {\fontsize{9.5pt}{11.4pt}\selectfont sample\_tract51013103200}\huxbpad{4pt}} &
\multicolumn{1}{r!{\huxvb{0}}}{\huxtpad{4pt}\raggedleft {\fontsize{9.5pt}{11.4pt}\selectfont 20.909 ***}\huxbpad{4pt}} &
\multicolumn{1}{r!{\huxvb{0}}}{\huxtpad{4pt}\raggedleft {\fontsize{9.5pt}{11.4pt}\selectfont 20.907 ***}\huxbpad{4pt}} &
\multicolumn{1}{r!{\huxvb{0}}}{\huxtpad{4pt}\raggedleft {\fontsize{9.5pt}{11.4pt}\selectfont 22.241 ***}\huxbpad{4pt}} \tabularnewline[-0.5pt]


\hhline{}
\arrayrulecolor{black}

\multicolumn{1}{!{\huxvb{0}}l!{\huxvb{0}}}{\huxtpad{4pt}\raggedright {\fontsize{9.5pt}{11.4pt}\selectfont }\huxbpad{4pt}} &
\multicolumn{1}{r!{\huxvb{0}}}{\huxtpad{4pt}\raggedleft {\fontsize{9.5pt}{11.4pt}\selectfont (1.417)~~~}\huxbpad{4pt}} &
\multicolumn{1}{r!{\huxvb{0}}}{\huxtpad{4pt}\raggedleft {\fontsize{9.5pt}{11.4pt}\selectfont (1.419)~~~}\huxbpad{4pt}} &
\multicolumn{1}{r!{\huxvb{0}}}{\huxtpad{4pt}\raggedleft {\fontsize{9.5pt}{11.4pt}\selectfont (1.404)~~~}\huxbpad{4pt}} \tabularnewline[-0.5pt]


\hhline{}
\arrayrulecolor{black}

\multicolumn{1}{!{\huxvb{0}}l!{\huxvb{0}}}{\huxtpad{4pt}\raggedright {\fontsize{9.5pt}{11.4pt}\selectfont sample\_tract51059420100}\huxbpad{4pt}} &
\multicolumn{1}{r!{\huxvb{0}}}{\huxtpad{4pt}\raggedleft {\fontsize{9.5pt}{11.4pt}\selectfont 38.670 ***}\huxbpad{4pt}} &
\multicolumn{1}{r!{\huxvb{0}}}{\huxtpad{4pt}\raggedleft {\fontsize{9.5pt}{11.4pt}\selectfont 38.519 ***}\huxbpad{4pt}} &
\multicolumn{1}{r!{\huxvb{0}}}{\huxtpad{4pt}\raggedleft {\fontsize{9.5pt}{11.4pt}\selectfont 41.000 ***}\huxbpad{4pt}} \tabularnewline[-0.5pt]


\hhline{}
\arrayrulecolor{black}

\multicolumn{1}{!{\huxvb{0}}l!{\huxvb{0}}}{\huxtpad{4pt}\raggedright {\fontsize{9.5pt}{11.4pt}\selectfont }\huxbpad{4pt}} &
\multicolumn{1}{r!{\huxvb{0}}}{\huxtpad{4pt}\raggedleft {\fontsize{9.5pt}{11.4pt}\selectfont (1.289)~~~}\huxbpad{4pt}} &
\multicolumn{1}{r!{\huxvb{0}}}{\huxtpad{4pt}\raggedleft {\fontsize{9.5pt}{11.4pt}\selectfont (1.386)~~~}\huxbpad{4pt}} &
\multicolumn{1}{r!{\huxvb{0}}}{\huxtpad{4pt}\raggedleft {\fontsize{9.5pt}{11.4pt}\selectfont (1.319)~~~}\huxbpad{4pt}} \tabularnewline[-0.5pt]


\hhline{}
\arrayrulecolor{black}

\multicolumn{1}{!{\huxvb{0}}l!{\huxvb{0}}}{\huxtpad{4pt}\raggedright {\fontsize{9.5pt}{11.4pt}\selectfont sample\_tract51059420201}\huxbpad{4pt}} &
\multicolumn{1}{r!{\huxvb{0}}}{\huxtpad{4pt}\raggedleft {\fontsize{9.5pt}{11.4pt}\selectfont 20.332 ***}\huxbpad{4pt}} &
\multicolumn{1}{r!{\huxvb{0}}}{\huxtpad{4pt}\raggedleft {\fontsize{9.5pt}{11.4pt}\selectfont 20.087 ***}\huxbpad{4pt}} &
\multicolumn{1}{r!{\huxvb{0}}}{\huxtpad{4pt}\raggedleft {\fontsize{9.5pt}{11.4pt}\selectfont 21.898 ***}\huxbpad{4pt}} \tabularnewline[-0.5pt]


\hhline{}
\arrayrulecolor{black}

\multicolumn{1}{!{\huxvb{0}}l!{\huxvb{0}}}{\huxtpad{4pt}\raggedright {\fontsize{9.5pt}{11.4pt}\selectfont }\huxbpad{4pt}} &
\multicolumn{1}{r!{\huxvb{0}}}{\huxtpad{4pt}\raggedleft {\fontsize{9.5pt}{11.4pt}\selectfont (1.626)~~~}\huxbpad{4pt}} &
\multicolumn{1}{r!{\huxvb{0}}}{\huxtpad{4pt}\raggedleft {\fontsize{9.5pt}{11.4pt}\selectfont (1.778)~~~}\huxbpad{4pt}} &
\multicolumn{1}{r!{\huxvb{0}}}{\huxtpad{4pt}\raggedleft {\fontsize{9.5pt}{11.4pt}\selectfont (1.699)~~~}\huxbpad{4pt}} \tabularnewline[-0.5pt]


\hhline{}
\arrayrulecolor{black}

\multicolumn{1}{!{\huxvb{0}}l!{\huxvb{0}}}{\huxtpad{4pt}\raggedright {\fontsize{9.5pt}{11.4pt}\selectfont sample\_tract51059420400}\huxbpad{4pt}} &
\multicolumn{1}{r!{\huxvb{0}}}{\huxtpad{4pt}\raggedleft {\fontsize{9.5pt}{11.4pt}\selectfont 20.185 ***}\huxbpad{4pt}} &
\multicolumn{1}{r!{\huxvb{0}}}{\huxtpad{4pt}\raggedleft {\fontsize{9.5pt}{11.4pt}\selectfont 19.486 ***}\huxbpad{4pt}} &
\multicolumn{1}{r!{\huxvb{0}}}{\huxtpad{4pt}\raggedleft {\fontsize{9.5pt}{11.4pt}\selectfont 20.967 ***}\huxbpad{4pt}} \tabularnewline[-0.5pt]


\hhline{}
\arrayrulecolor{black}

\multicolumn{1}{!{\huxvb{0}}l!{\huxvb{0}}}{\huxtpad{4pt}\raggedright {\fontsize{9.5pt}{11.4pt}\selectfont }\huxbpad{4pt}} &
\multicolumn{1}{r!{\huxvb{0}}}{\huxtpad{4pt}\raggedleft {\fontsize{9.5pt}{11.4pt}\selectfont (1.432)~~~}\huxbpad{4pt}} &
\multicolumn{1}{r!{\huxvb{0}}}{\huxtpad{4pt}\raggedleft {\fontsize{9.5pt}{11.4pt}\selectfont (1.449)~~~}\huxbpad{4pt}} &
\multicolumn{1}{r!{\huxvb{0}}}{\huxtpad{4pt}\raggedleft {\fontsize{9.5pt}{11.4pt}\selectfont (1.371)~~~}\huxbpad{4pt}} \tabularnewline[-0.5pt]


\hhline{}
\arrayrulecolor{black}

\multicolumn{1}{!{\huxvb{0}}l!{\huxvb{0}}}{\huxtpad{4pt}\raggedright {\fontsize{9.5pt}{11.4pt}\selectfont sample\_tract51059420503}\huxbpad{4pt}} &
\multicolumn{1}{r!{\huxvb{0}}}{\huxtpad{4pt}\raggedleft {\fontsize{9.5pt}{11.4pt}\selectfont -0.213~~~~}\huxbpad{4pt}} &
\multicolumn{1}{r!{\huxvb{0}}}{\huxtpad{4pt}\raggedleft {\fontsize{9.5pt}{11.4pt}\selectfont -0.029~~~~}\huxbpad{4pt}} &
\multicolumn{1}{r!{\huxvb{0}}}{\huxtpad{4pt}\raggedleft {\fontsize{9.5pt}{11.4pt}\selectfont 0.751~~~~}\huxbpad{4pt}} \tabularnewline[-0.5pt]


\hhline{}
\arrayrulecolor{black}

\multicolumn{1}{!{\huxvb{0}}l!{\huxvb{0}}}{\huxtpad{4pt}\raggedright {\fontsize{9.5pt}{11.4pt}\selectfont }\huxbpad{4pt}} &
\multicolumn{1}{r!{\huxvb{0}}}{\huxtpad{4pt}\raggedleft {\fontsize{9.5pt}{11.4pt}\selectfont (1.322)~~~}\huxbpad{4pt}} &
\multicolumn{1}{r!{\huxvb{0}}}{\huxtpad{4pt}\raggedleft {\fontsize{9.5pt}{11.4pt}\selectfont (1.403)~~~}\huxbpad{4pt}} &
\multicolumn{1}{r!{\huxvb{0}}}{\huxtpad{4pt}\raggedleft {\fontsize{9.5pt}{11.4pt}\selectfont (1.364)~~~}\huxbpad{4pt}} \tabularnewline[-0.5pt]


\hhline{}
\arrayrulecolor{black}

\multicolumn{1}{!{\huxvb{0}}l!{\huxvb{0}}}{\huxtpad{4pt}\raggedright {\fontsize{9.5pt}{11.4pt}\selectfont sample\_tract51059421001}\huxbpad{4pt}} &
\multicolumn{1}{r!{\huxvb{0}}}{\huxtpad{4pt}\raggedleft {\fontsize{9.5pt}{11.4pt}\selectfont 20.004 ***}\huxbpad{4pt}} &
\multicolumn{1}{r!{\huxvb{0}}}{\huxtpad{4pt}\raggedleft {\fontsize{9.5pt}{11.4pt}\selectfont 19.817 ***}\huxbpad{4pt}} &
\multicolumn{1}{r!{\huxvb{0}}}{\huxtpad{4pt}\raggedleft {\fontsize{9.5pt}{11.4pt}\selectfont 20.897 ***}\huxbpad{4pt}} \tabularnewline[-0.5pt]


\hhline{}
\arrayrulecolor{black}

\multicolumn{1}{!{\huxvb{0}}l!{\huxvb{0}}}{\huxtpad{4pt}\raggedright {\fontsize{9.5pt}{11.4pt}\selectfont }\huxbpad{4pt}} &
\multicolumn{1}{r!{\huxvb{0}}}{\huxtpad{4pt}\raggedleft {\fontsize{9.5pt}{11.4pt}\selectfont (1.383)~~~}\huxbpad{4pt}} &
\multicolumn{1}{r!{\huxvb{0}}}{\huxtpad{4pt}\raggedleft {\fontsize{9.5pt}{11.4pt}\selectfont (1.512)~~~}\huxbpad{4pt}} &
\multicolumn{1}{r!{\huxvb{0}}}{\huxtpad{4pt}\raggedleft {\fontsize{9.5pt}{11.4pt}\selectfont (1.330)~~~}\huxbpad{4pt}} \tabularnewline[-0.5pt]


\hhline{}
\arrayrulecolor{black}

\multicolumn{1}{!{\huxvb{0}}l!{\huxvb{0}}}{\huxtpad{4pt}\raggedright {\fontsize{9.5pt}{11.4pt}\selectfont sample\_tract51059421002}\huxbpad{4pt}} &
\multicolumn{1}{r!{\huxvb{0}}}{\huxtpad{4pt}\raggedleft {\fontsize{9.5pt}{11.4pt}\selectfont 21.603 ***}\huxbpad{4pt}} &
\multicolumn{1}{r!{\huxvb{0}}}{\huxtpad{4pt}\raggedleft {\fontsize{9.5pt}{11.4pt}\selectfont 21.299 ***}\huxbpad{4pt}} &
\multicolumn{1}{r!{\huxvb{0}}}{\huxtpad{4pt}\raggedleft {\fontsize{9.5pt}{11.4pt}\selectfont 22.416 ***}\huxbpad{4pt}} \tabularnewline[-0.5pt]


\hhline{}
\arrayrulecolor{black}

\multicolumn{1}{!{\huxvb{0}}l!{\huxvb{0}}}{\huxtpad{4pt}\raggedright {\fontsize{9.5pt}{11.4pt}\selectfont }\huxbpad{4pt}} &
\multicolumn{1}{r!{\huxvb{0}}}{\huxtpad{4pt}\raggedleft {\fontsize{9.5pt}{11.4pt}\selectfont (1.512)~~~}\huxbpad{4pt}} &
\multicolumn{1}{r!{\huxvb{0}}}{\huxtpad{4pt}\raggedleft {\fontsize{9.5pt}{11.4pt}\selectfont (1.621)~~~}\huxbpad{4pt}} &
\multicolumn{1}{r!{\huxvb{0}}}{\huxtpad{4pt}\raggedleft {\fontsize{9.5pt}{11.4pt}\selectfont (1.627)~~~}\huxbpad{4pt}} \tabularnewline[-0.5pt]


\hhline{}
\arrayrulecolor{black}

\multicolumn{1}{!{\huxvb{0}}l!{\huxvb{0}}}{\huxtpad{4pt}\raggedright {\fontsize{9.5pt}{11.4pt}\selectfont sample\_tract51059421101}\huxbpad{4pt}} &
\multicolumn{1}{r!{\huxvb{0}}}{\huxtpad{4pt}\raggedleft {\fontsize{9.5pt}{11.4pt}\selectfont 19.237 ***}\huxbpad{4pt}} &
\multicolumn{1}{r!{\huxvb{0}}}{\huxtpad{4pt}\raggedleft {\fontsize{9.5pt}{11.4pt}\selectfont 18.875 ***}\huxbpad{4pt}} &
\multicolumn{1}{r!{\huxvb{0}}}{\huxtpad{4pt}\raggedleft {\fontsize{9.5pt}{11.4pt}\selectfont 19.914 ***}\huxbpad{4pt}} \tabularnewline[-0.5pt]


\hhline{}
\arrayrulecolor{black}

\multicolumn{1}{!{\huxvb{0}}l!{\huxvb{0}}}{\huxtpad{4pt}\raggedright {\fontsize{9.5pt}{11.4pt}\selectfont }\huxbpad{4pt}} &
\multicolumn{1}{r!{\huxvb{0}}}{\huxtpad{4pt}\raggedleft {\fontsize{9.5pt}{11.4pt}\selectfont (1.345)~~~}\huxbpad{4pt}} &
\multicolumn{1}{r!{\huxvb{0}}}{\huxtpad{4pt}\raggedleft {\fontsize{9.5pt}{11.4pt}\selectfont (1.731)~~~}\huxbpad{4pt}} &
\multicolumn{1}{r!{\huxvb{0}}}{\huxtpad{4pt}\raggedleft {\fontsize{9.5pt}{11.4pt}\selectfont (1.371)~~~}\huxbpad{4pt}} \tabularnewline[-0.5pt]


\hhline{}
\arrayrulecolor{black}

\multicolumn{1}{!{\huxvb{0}}l!{\huxvb{0}}}{\huxtpad{4pt}\raggedright {\fontsize{9.5pt}{11.4pt}\selectfont sample\_tract51059421102}\huxbpad{4pt}} &
\multicolumn{1}{r!{\huxvb{0}}}{\huxtpad{4pt}\raggedleft {\fontsize{9.5pt}{11.4pt}\selectfont 20.161 ***}\huxbpad{4pt}} &
\multicolumn{1}{r!{\huxvb{0}}}{\huxtpad{4pt}\raggedleft {\fontsize{9.5pt}{11.4pt}\selectfont 19.873 ***}\huxbpad{4pt}} &
\multicolumn{1}{r!{\huxvb{0}}}{\huxtpad{4pt}\raggedleft {\fontsize{9.5pt}{11.4pt}\selectfont 20.565 ***}\huxbpad{4pt}} \tabularnewline[-0.5pt]


\hhline{}
\arrayrulecolor{black}

\multicolumn{1}{!{\huxvb{0}}l!{\huxvb{0}}}{\huxtpad{4pt}\raggedright {\fontsize{9.5pt}{11.4pt}\selectfont }\huxbpad{4pt}} &
\multicolumn{1}{r!{\huxvb{0}}}{\huxtpad{4pt}\raggedleft {\fontsize{9.5pt}{11.4pt}\selectfont (1.350)~~~}\huxbpad{4pt}} &
\multicolumn{1}{r!{\huxvb{0}}}{\huxtpad{4pt}\raggedleft {\fontsize{9.5pt}{11.4pt}\selectfont (1.448)~~~}\huxbpad{4pt}} &
\multicolumn{1}{r!{\huxvb{0}}}{\huxtpad{4pt}\raggedleft {\fontsize{9.5pt}{11.4pt}\selectfont (1.718)~~~}\huxbpad{4pt}} \tabularnewline[-0.5pt]


\hhline{}
\arrayrulecolor{black}

\multicolumn{1}{!{\huxvb{0}}l!{\huxvb{0}}}{\huxtpad{4pt}\raggedright {\fontsize{9.5pt}{11.4pt}\selectfont sample\_tract51059421702}\huxbpad{4pt}} &
\multicolumn{1}{r!{\huxvb{0}}}{\huxtpad{4pt}\raggedleft {\fontsize{9.5pt}{11.4pt}\selectfont 2.033~~~~}\huxbpad{4pt}} &
\multicolumn{1}{r!{\huxvb{0}}}{\huxtpad{4pt}\raggedleft {\fontsize{9.5pt}{11.4pt}\selectfont 1.522~~~~}\huxbpad{4pt}} &
\multicolumn{1}{r!{\huxvb{0}}}{\huxtpad{4pt}\raggedleft {\fontsize{9.5pt}{11.4pt}\selectfont 1.854~~~~}\huxbpad{4pt}} \tabularnewline[-0.5pt]


\hhline{}
\arrayrulecolor{black}

\multicolumn{1}{!{\huxvb{0}}l!{\huxvb{0}}}{\huxtpad{4pt}\raggedright {\fontsize{9.5pt}{11.4pt}\selectfont }\huxbpad{4pt}} &
\multicolumn{1}{r!{\huxvb{0}}}{\huxtpad{4pt}\raggedleft {\fontsize{9.5pt}{11.4pt}\selectfont (1.232)~~~}\huxbpad{4pt}} &
\multicolumn{1}{r!{\huxvb{0}}}{\huxtpad{4pt}\raggedleft {\fontsize{9.5pt}{11.4pt}\selectfont (1.457)~~~}\huxbpad{4pt}} &
\multicolumn{1}{r!{\huxvb{0}}}{\huxtpad{4pt}\raggedleft {\fontsize{9.5pt}{11.4pt}\selectfont (1.442)~~~}\huxbpad{4pt}} \tabularnewline[-0.5pt]


\hhline{}
\arrayrulecolor{black}

\multicolumn{1}{!{\huxvb{0}}l!{\huxvb{0}}}{\huxtpad{4pt}\raggedright {\fontsize{9.5pt}{11.4pt}\selectfont sample\_tract51059421800}\huxbpad{4pt}} &
\multicolumn{1}{r!{\huxvb{0}}}{\huxtpad{4pt}\raggedleft {\fontsize{9.5pt}{11.4pt}\selectfont 0.873~~~~}\huxbpad{4pt}} &
\multicolumn{1}{r!{\huxvb{0}}}{\huxtpad{4pt}\raggedleft {\fontsize{9.5pt}{11.4pt}\selectfont -0.192~~~~}\huxbpad{4pt}} &
\multicolumn{1}{r!{\huxvb{0}}}{\huxtpad{4pt}\raggedleft {\fontsize{9.5pt}{11.4pt}\selectfont 0.339~~~~}\huxbpad{4pt}} \tabularnewline[-0.5pt]


\hhline{}
\arrayrulecolor{black}

\multicolumn{1}{!{\huxvb{0}}l!{\huxvb{0}}}{\huxtpad{4pt}\raggedright {\fontsize{9.5pt}{11.4pt}\selectfont }\huxbpad{4pt}} &
\multicolumn{1}{r!{\huxvb{0}}}{\huxtpad{4pt}\raggedleft {\fontsize{9.5pt}{11.4pt}\selectfont (0.998)~~~}\huxbpad{4pt}} &
\multicolumn{1}{r!{\huxvb{0}}}{\huxtpad{4pt}\raggedleft {\fontsize{9.5pt}{11.4pt}\selectfont (1.300)~~~}\huxbpad{4pt}} &
\multicolumn{1}{r!{\huxvb{0}}}{\huxtpad{4pt}\raggedleft {\fontsize{9.5pt}{11.4pt}\selectfont (1.478)~~~}\huxbpad{4pt}} \tabularnewline[-0.5pt]


\hhline{}
\arrayrulecolor{black}

\multicolumn{1}{!{\huxvb{0}}l!{\huxvb{0}}}{\huxtpad{4pt}\raggedright {\fontsize{9.5pt}{11.4pt}\selectfont sample\_tract51059422000}\huxbpad{4pt}} &
\multicolumn{1}{r!{\huxvb{0}}}{\huxtpad{4pt}\raggedleft {\fontsize{9.5pt}{11.4pt}\selectfont 0.612~~~~}\huxbpad{4pt}} &
\multicolumn{1}{r!{\huxvb{0}}}{\huxtpad{4pt}\raggedleft {\fontsize{9.5pt}{11.4pt}\selectfont -0.602~~~~}\huxbpad{4pt}} &
\multicolumn{1}{r!{\huxvb{0}}}{\huxtpad{4pt}\raggedleft {\fontsize{9.5pt}{11.4pt}\selectfont 0.539~~~~}\huxbpad{4pt}} \tabularnewline[-0.5pt]


\hhline{}
\arrayrulecolor{black}

\multicolumn{1}{!{\huxvb{0}}l!{\huxvb{0}}}{\huxtpad{4pt}\raggedright {\fontsize{9.5pt}{11.4pt}\selectfont }\huxbpad{4pt}} &
\multicolumn{1}{r!{\huxvb{0}}}{\huxtpad{4pt}\raggedleft {\fontsize{9.5pt}{11.4pt}\selectfont (1.217)~~~}\huxbpad{4pt}} &
\multicolumn{1}{r!{\huxvb{0}}}{\huxtpad{4pt}\raggedleft {\fontsize{9.5pt}{11.4pt}\selectfont (1.487)~~~}\huxbpad{4pt}} &
\multicolumn{1}{r!{\huxvb{0}}}{\huxtpad{4pt}\raggedleft {\fontsize{9.5pt}{11.4pt}\selectfont (1.217)~~~}\huxbpad{4pt}} \tabularnewline[-0.5pt]


\hhline{}
\arrayrulecolor{black}

\multicolumn{1}{!{\huxvb{0}}l!{\huxvb{0}}}{\huxtpad{4pt}\raggedright {\fontsize{9.5pt}{11.4pt}\selectfont sample\_tract51059422101}\huxbpad{4pt}} &
\multicolumn{1}{r!{\huxvb{0}}}{\huxtpad{4pt}\raggedleft {\fontsize{9.5pt}{11.4pt}\selectfont 19.381 ***}\huxbpad{4pt}} &
\multicolumn{1}{r!{\huxvb{0}}}{\huxtpad{4pt}\raggedleft {\fontsize{9.5pt}{11.4pt}\selectfont 18.767 ***}\huxbpad{4pt}} &
\multicolumn{1}{r!{\huxvb{0}}}{\huxtpad{4pt}\raggedleft {\fontsize{9.5pt}{11.4pt}\selectfont 20.453 ***}\huxbpad{4pt}} \tabularnewline[-0.5pt]


\hhline{}
\arrayrulecolor{black}

\multicolumn{1}{!{\huxvb{0}}l!{\huxvb{0}}}{\huxtpad{4pt}\raggedright {\fontsize{9.5pt}{11.4pt}\selectfont }\huxbpad{4pt}} &
\multicolumn{1}{r!{\huxvb{0}}}{\huxtpad{4pt}\raggedleft {\fontsize{9.5pt}{11.4pt}\selectfont (1.211)~~~}\huxbpad{4pt}} &
\multicolumn{1}{r!{\huxvb{0}}}{\huxtpad{4pt}\raggedleft {\fontsize{9.5pt}{11.4pt}\selectfont (1.301)~~~}\huxbpad{4pt}} &
\multicolumn{1}{r!{\huxvb{0}}}{\huxtpad{4pt}\raggedleft {\fontsize{9.5pt}{11.4pt}\selectfont (1.154)~~~}\huxbpad{4pt}} \tabularnewline[-0.5pt]


\hhline{}
\arrayrulecolor{black}

\multicolumn{1}{!{\huxvb{0}}l!{\huxvb{0}}}{\huxtpad{4pt}\raggedright {\fontsize{9.5pt}{11.4pt}\selectfont sample\_tract51059422102}\huxbpad{4pt}} &
\multicolumn{1}{r!{\huxvb{0}}}{\huxtpad{4pt}\raggedleft {\fontsize{9.5pt}{11.4pt}\selectfont 17.912 ***}\huxbpad{4pt}} &
\multicolumn{1}{r!{\huxvb{0}}}{\huxtpad{4pt}\raggedleft {\fontsize{9.5pt}{11.4pt}\selectfont 17.595 ***}\huxbpad{4pt}} &
\multicolumn{1}{r!{\huxvb{0}}}{\huxtpad{4pt}\raggedleft {\fontsize{9.5pt}{11.4pt}\selectfont 19.650 ***}\huxbpad{4pt}} \tabularnewline[-0.5pt]


\hhline{}
\arrayrulecolor{black}

\multicolumn{1}{!{\huxvb{0}}l!{\huxvb{0}}}{\huxtpad{4pt}\raggedright {\fontsize{9.5pt}{11.4pt}\selectfont }\huxbpad{4pt}} &
\multicolumn{1}{r!{\huxvb{0}}}{\huxtpad{4pt}\raggedleft {\fontsize{9.5pt}{11.4pt}\selectfont (1.538)~~~}\huxbpad{4pt}} &
\multicolumn{1}{r!{\huxvb{0}}}{\huxtpad{4pt}\raggedleft {\fontsize{9.5pt}{11.4pt}\selectfont (1.563)~~~}\huxbpad{4pt}} &
\multicolumn{1}{r!{\huxvb{0}}}{\huxtpad{4pt}\raggedleft {\fontsize{9.5pt}{11.4pt}\selectfont (1.187)~~~}\huxbpad{4pt}} \tabularnewline[-0.5pt]


\hhline{}
\arrayrulecolor{black}

\multicolumn{1}{!{\huxvb{0}}l!{\huxvb{0}}}{\huxtpad{4pt}\raggedright {\fontsize{9.5pt}{11.4pt}\selectfont sample\_tract51059422401}\huxbpad{4pt}} &
\multicolumn{1}{r!{\huxvb{0}}}{\huxtpad{4pt}\raggedleft {\fontsize{9.5pt}{11.4pt}\selectfont 38.728 ***}\huxbpad{4pt}} &
\multicolumn{1}{r!{\huxvb{0}}}{\huxtpad{4pt}\raggedleft {\fontsize{9.5pt}{11.4pt}\selectfont 38.159 ***}\huxbpad{4pt}} &
\multicolumn{1}{r!{\huxvb{0}}}{\huxtpad{4pt}\raggedleft {\fontsize{9.5pt}{11.4pt}\selectfont 40.086 ***}\huxbpad{4pt}} \tabularnewline[-0.5pt]


\hhline{}
\arrayrulecolor{black}

\multicolumn{1}{!{\huxvb{0}}l!{\huxvb{0}}}{\huxtpad{4pt}\raggedright {\fontsize{9.5pt}{11.4pt}\selectfont }\huxbpad{4pt}} &
\multicolumn{1}{r!{\huxvb{0}}}{\huxtpad{4pt}\raggedleft {\fontsize{9.5pt}{11.4pt}\selectfont (1.248)~~~}\huxbpad{4pt}} &
\multicolumn{1}{r!{\huxvb{0}}}{\huxtpad{4pt}\raggedleft {\fontsize{9.5pt}{11.4pt}\selectfont (1.339)~~~}\huxbpad{4pt}} &
\multicolumn{1}{r!{\huxvb{0}}}{\huxtpad{4pt}\raggedleft {\fontsize{9.5pt}{11.4pt}\selectfont (1.402)~~~}\huxbpad{4pt}} \tabularnewline[-0.5pt]


\hhline{}
\arrayrulecolor{black}

\multicolumn{1}{!{\huxvb{0}}l!{\huxvb{0}}}{\huxtpad{4pt}\raggedright {\fontsize{9.5pt}{11.4pt}\selectfont sample\_tract51059430901}\huxbpad{4pt}} &
\multicolumn{1}{r!{\huxvb{0}}}{\huxtpad{4pt}\raggedleft {\fontsize{9.5pt}{11.4pt}\selectfont 19.925 ***}\huxbpad{4pt}} &
\multicolumn{1}{r!{\huxvb{0}}}{\huxtpad{4pt}\raggedleft {\fontsize{9.5pt}{11.4pt}\selectfont 19.422 ***}\huxbpad{4pt}} &
\multicolumn{1}{r!{\huxvb{0}}}{\huxtpad{4pt}\raggedleft {\fontsize{9.5pt}{11.4pt}\selectfont 20.940 ***}\huxbpad{4pt}} \tabularnewline[-0.5pt]


\hhline{}
\arrayrulecolor{black}

\multicolumn{1}{!{\huxvb{0}}l!{\huxvb{0}}}{\huxtpad{4pt}\raggedright {\fontsize{9.5pt}{11.4pt}\selectfont }\huxbpad{4pt}} &
\multicolumn{1}{r!{\huxvb{0}}}{\huxtpad{4pt}\raggedleft {\fontsize{9.5pt}{11.4pt}\selectfont (1.216)~~~}\huxbpad{4pt}} &
\multicolumn{1}{r!{\huxvb{0}}}{\huxtpad{4pt}\raggedleft {\fontsize{9.5pt}{11.4pt}\selectfont (1.365)~~~}\huxbpad{4pt}} &
\multicolumn{1}{r!{\huxvb{0}}}{\huxtpad{4pt}\raggedleft {\fontsize{9.5pt}{11.4pt}\selectfont (1.222)~~~}\huxbpad{4pt}} \tabularnewline[-0.5pt]


\hhline{}
\arrayrulecolor{black}

\multicolumn{1}{!{\huxvb{0}}l!{\huxvb{0}}}{\huxtpad{4pt}\raggedright {\fontsize{9.5pt}{11.4pt}\selectfont sample\_tract51059431001}\huxbpad{4pt}} &
\multicolumn{1}{r!{\huxvb{0}}}{\huxtpad{4pt}\raggedleft {\fontsize{9.5pt}{11.4pt}\selectfont 18.209 ***}\huxbpad{4pt}} &
\multicolumn{1}{r!{\huxvb{0}}}{\huxtpad{4pt}\raggedleft {\fontsize{9.5pt}{11.4pt}\selectfont 18.027 ***}\huxbpad{4pt}} &
\multicolumn{1}{r!{\huxvb{0}}}{\huxtpad{4pt}\raggedleft {\fontsize{9.5pt}{11.4pt}\selectfont 19.140 ***}\huxbpad{4pt}} \tabularnewline[-0.5pt]


\hhline{}
\arrayrulecolor{black}

\multicolumn{1}{!{\huxvb{0}}l!{\huxvb{0}}}{\huxtpad{4pt}\raggedright {\fontsize{9.5pt}{11.4pt}\selectfont }\huxbpad{4pt}} &
\multicolumn{1}{r!{\huxvb{0}}}{\huxtpad{4pt}\raggedleft {\fontsize{9.5pt}{11.4pt}\selectfont (1.518)~~~}\huxbpad{4pt}} &
\multicolumn{1}{r!{\huxvb{0}}}{\huxtpad{4pt}\raggedleft {\fontsize{9.5pt}{11.4pt}\selectfont (1.505)~~~}\huxbpad{4pt}} &
\multicolumn{1}{r!{\huxvb{0}}}{\huxtpad{4pt}\raggedleft {\fontsize{9.5pt}{11.4pt}\selectfont (1.179)~~~}\huxbpad{4pt}} \tabularnewline[-0.5pt]


\hhline{}
\arrayrulecolor{black}

\multicolumn{1}{!{\huxvb{0}}l!{\huxvb{0}}}{\huxtpad{4pt}\raggedright {\fontsize{9.5pt}{11.4pt}\selectfont sample\_tract51059431002}\huxbpad{4pt}} &
\multicolumn{1}{r!{\huxvb{0}}}{\huxtpad{4pt}\raggedleft {\fontsize{9.5pt}{11.4pt}\selectfont 37.825 ***}\huxbpad{4pt}} &
\multicolumn{1}{r!{\huxvb{0}}}{\huxtpad{4pt}\raggedleft {\fontsize{9.5pt}{11.4pt}\selectfont 37.191 ***}\huxbpad{4pt}} &
\multicolumn{1}{r!{\huxvb{0}}}{\huxtpad{4pt}\raggedleft {\fontsize{9.5pt}{11.4pt}\selectfont 39.218 ***}\huxbpad{4pt}} \tabularnewline[-0.5pt]


\hhline{}
\arrayrulecolor{black}

\multicolumn{1}{!{\huxvb{0}}l!{\huxvb{0}}}{\huxtpad{4pt}\raggedright {\fontsize{9.5pt}{11.4pt}\selectfont }\huxbpad{4pt}} &
\multicolumn{1}{r!{\huxvb{0}}}{\huxtpad{4pt}\raggedleft {\fontsize{9.5pt}{11.4pt}\selectfont (1.385)~~~}\huxbpad{4pt}} &
\multicolumn{1}{r!{\huxvb{0}}}{\huxtpad{4pt}\raggedleft {\fontsize{9.5pt}{11.4pt}\selectfont (1.513)~~~}\huxbpad{4pt}} &
\multicolumn{1}{r!{\huxvb{0}}}{\huxtpad{4pt}\raggedleft {\fontsize{9.5pt}{11.4pt}\selectfont (1.464)~~~}\huxbpad{4pt}} \tabularnewline[-0.5pt]


\hhline{}
\arrayrulecolor{black}

\multicolumn{1}{!{\huxvb{0}}l!{\huxvb{0}}}{\huxtpad{4pt}\raggedright {\fontsize{9.5pt}{11.4pt}\selectfont sample\_tract51059432702}\huxbpad{4pt}} &
\multicolumn{1}{r!{\huxvb{0}}}{\huxtpad{4pt}\raggedleft {\fontsize{9.5pt}{11.4pt}\selectfont 21.654 ***}\huxbpad{4pt}} &
\multicolumn{1}{r!{\huxvb{0}}}{\huxtpad{4pt}\raggedleft {\fontsize{9.5pt}{11.4pt}\selectfont 21.307 ***}\huxbpad{4pt}} &
\multicolumn{1}{r!{\huxvb{0}}}{\huxtpad{4pt}\raggedleft {\fontsize{9.5pt}{11.4pt}\selectfont 21.821 ***}\huxbpad{4pt}} \tabularnewline[-0.5pt]


\hhline{}
\arrayrulecolor{black}

\multicolumn{1}{!{\huxvb{0}}l!{\huxvb{0}}}{\huxtpad{4pt}\raggedright {\fontsize{9.5pt}{11.4pt}\selectfont }\huxbpad{4pt}} &
\multicolumn{1}{r!{\huxvb{0}}}{\huxtpad{4pt}\raggedleft {\fontsize{9.5pt}{11.4pt}\selectfont (1.577)~~~}\huxbpad{4pt}} &
\multicolumn{1}{r!{\huxvb{0}}}{\huxtpad{4pt}\raggedleft {\fontsize{9.5pt}{11.4pt}\selectfont (1.676)~~~}\huxbpad{4pt}} &
\multicolumn{1}{r!{\huxvb{0}}}{\huxtpad{4pt}\raggedleft {\fontsize{9.5pt}{11.4pt}\selectfont (1.594)~~~}\huxbpad{4pt}} \tabularnewline[-0.5pt]


\hhline{}
\arrayrulecolor{black}

\multicolumn{1}{!{\huxvb{0}}l!{\huxvb{0}}}{\huxtpad{4pt}\raggedright {\fontsize{9.5pt}{11.4pt}\selectfont sample\_tract51059432800}\huxbpad{4pt}} &
\multicolumn{1}{r!{\huxvb{0}}}{\huxtpad{4pt}\raggedleft {\fontsize{9.5pt}{11.4pt}\selectfont 1.605~~~~}\huxbpad{4pt}} &
\multicolumn{1}{r!{\huxvb{0}}}{\huxtpad{4pt}\raggedleft {\fontsize{9.5pt}{11.4pt}\selectfont 0.543~~~~}\huxbpad{4pt}} &
\multicolumn{1}{r!{\huxvb{0}}}{\huxtpad{4pt}\raggedleft {\fontsize{9.5pt}{11.4pt}\selectfont 1.682~~~~}\huxbpad{4pt}} \tabularnewline[-0.5pt]


\hhline{}
\arrayrulecolor{black}

\multicolumn{1}{!{\huxvb{0}}l!{\huxvb{0}}}{\huxtpad{4pt}\raggedright {\fontsize{9.5pt}{11.4pt}\selectfont }\huxbpad{4pt}} &
\multicolumn{1}{r!{\huxvb{0}}}{\huxtpad{4pt}\raggedleft {\fontsize{9.5pt}{11.4pt}\selectfont (1.383)~~~}\huxbpad{4pt}} &
\multicolumn{1}{r!{\huxvb{0}}}{\huxtpad{4pt}\raggedleft {\fontsize{9.5pt}{11.4pt}\selectfont (1.584)~~~}\huxbpad{4pt}} &
\multicolumn{1}{r!{\huxvb{0}}}{\huxtpad{4pt}\raggedleft {\fontsize{9.5pt}{11.4pt}\selectfont (1.640)~~~}\huxbpad{4pt}} \tabularnewline[-0.5pt]


\hhline{}
\arrayrulecolor{black}

\multicolumn{1}{!{\huxvb{0}}l!{\huxvb{0}}}{\huxtpad{4pt}\raggedright {\fontsize{9.5pt}{11.4pt}\selectfont sample\_tract51059451900}\huxbpad{4pt}} &
\multicolumn{1}{r!{\huxvb{0}}}{\huxtpad{4pt}\raggedleft {\fontsize{9.5pt}{11.4pt}\selectfont 1.320~~~~}\huxbpad{4pt}} &
\multicolumn{1}{r!{\huxvb{0}}}{\huxtpad{4pt}\raggedleft {\fontsize{9.5pt}{11.4pt}\selectfont 0.968~~~~}\huxbpad{4pt}} &
\multicolumn{1}{r!{\huxvb{0}}}{\huxtpad{4pt}\raggedleft {\fontsize{9.5pt}{11.4pt}\selectfont 1.013~~~~}\huxbpad{4pt}} \tabularnewline[-0.5pt]


\hhline{}
\arrayrulecolor{black}

\multicolumn{1}{!{\huxvb{0}}l!{\huxvb{0}}}{\huxtpad{4pt}\raggedright {\fontsize{9.5pt}{11.4pt}\selectfont }\huxbpad{4pt}} &
\multicolumn{1}{r!{\huxvb{0}}}{\huxtpad{4pt}\raggedleft {\fontsize{9.5pt}{11.4pt}\selectfont (1.231)~~~}\huxbpad{4pt}} &
\multicolumn{1}{r!{\huxvb{0}}}{\huxtpad{4pt}\raggedleft {\fontsize{9.5pt}{11.4pt}\selectfont (1.378)~~~}\huxbpad{4pt}} &
\multicolumn{1}{r!{\huxvb{0}}}{\huxtpad{4pt}\raggedleft {\fontsize{9.5pt}{11.4pt}\selectfont (1.341)~~~}\huxbpad{4pt}} \tabularnewline[-0.5pt]


\hhline{}
\arrayrulecolor{black}

\multicolumn{1}{!{\huxvb{0}}l!{\huxvb{0}}}{\huxtpad{4pt}\raggedright {\fontsize{9.5pt}{11.4pt}\selectfont sample\_tract51059452101}\huxbpad{4pt}} &
\multicolumn{1}{r!{\huxvb{0}}}{\huxtpad{4pt}\raggedleft {\fontsize{9.5pt}{11.4pt}\selectfont 16.150 ***}\huxbpad{4pt}} &
\multicolumn{1}{r!{\huxvb{0}}}{\huxtpad{4pt}\raggedleft {\fontsize{9.5pt}{11.4pt}\selectfont 15.854 ***}\huxbpad{4pt}} &
\multicolumn{1}{r!{\huxvb{0}}}{\huxtpad{4pt}\raggedleft {\fontsize{9.5pt}{11.4pt}\selectfont 18.529 ***}\huxbpad{4pt}} \tabularnewline[-0.5pt]


\hhline{}
\arrayrulecolor{black}

\multicolumn{1}{!{\huxvb{0}}l!{\huxvb{0}}}{\huxtpad{4pt}\raggedright {\fontsize{9.5pt}{11.4pt}\selectfont }\huxbpad{4pt}} &
\multicolumn{1}{r!{\huxvb{0}}}{\huxtpad{4pt}\raggedleft {\fontsize{9.5pt}{11.4pt}\selectfont (1.698)~~~}\huxbpad{4pt}} &
\multicolumn{1}{r!{\huxvb{0}}}{\huxtpad{4pt}\raggedleft {\fontsize{9.5pt}{11.4pt}\selectfont (1.780)~~~}\huxbpad{4pt}} &
\multicolumn{1}{r!{\huxvb{0}}}{\huxtpad{4pt}\raggedleft {\fontsize{9.5pt}{11.4pt}\selectfont (1.793)~~~}\huxbpad{4pt}} \tabularnewline[-0.5pt]


\hhline{}
\arrayrulecolor{black}

\multicolumn{1}{!{\huxvb{0}}l!{\huxvb{0}}}{\huxtpad{4pt}\raggedright {\fontsize{9.5pt}{11.4pt}\selectfont sample\_tract51059452200}\huxbpad{4pt}} &
\multicolumn{1}{r!{\huxvb{0}}}{\huxtpad{4pt}\raggedleft {\fontsize{9.5pt}{11.4pt}\selectfont 18.920 ***}\huxbpad{4pt}} &
\multicolumn{1}{r!{\huxvb{0}}}{\huxtpad{4pt}\raggedleft {\fontsize{9.5pt}{11.4pt}\selectfont 18.570 ***}\huxbpad{4pt}} &
\multicolumn{1}{r!{\huxvb{0}}}{\huxtpad{4pt}\raggedleft {\fontsize{9.5pt}{11.4pt}\selectfont 20.156 ***}\huxbpad{4pt}} \tabularnewline[-0.5pt]


\hhline{}
\arrayrulecolor{black}

\multicolumn{1}{!{\huxvb{0}}l!{\huxvb{0}}}{\huxtpad{4pt}\raggedright {\fontsize{9.5pt}{11.4pt}\selectfont }\huxbpad{4pt}} &
\multicolumn{1}{r!{\huxvb{0}}}{\huxtpad{4pt}\raggedleft {\fontsize{9.5pt}{11.4pt}\selectfont (1.238)~~~}\huxbpad{4pt}} &
\multicolumn{1}{r!{\huxvb{0}}}{\huxtpad{4pt}\raggedleft {\fontsize{9.5pt}{11.4pt}\selectfont (1.288)~~~}\huxbpad{4pt}} &
\multicolumn{1}{r!{\huxvb{0}}}{\huxtpad{4pt}\raggedleft {\fontsize{9.5pt}{11.4pt}\selectfont (1.288)~~~}\huxbpad{4pt}} \tabularnewline[-0.5pt]


\hhline{}
\arrayrulecolor{black}

\multicolumn{1}{!{\huxvb{0}}l!{\huxvb{0}}}{\huxtpad{4pt}\raggedright {\fontsize{9.5pt}{11.4pt}\selectfont sample\_tract51059452301}\huxbpad{4pt}} &
\multicolumn{1}{r!{\huxvb{0}}}{\huxtpad{4pt}\raggedleft {\fontsize{9.5pt}{11.4pt}\selectfont -0.203~~~~}\huxbpad{4pt}} &
\multicolumn{1}{r!{\huxvb{0}}}{\huxtpad{4pt}\raggedleft {\fontsize{9.5pt}{11.4pt}\selectfont -0.558~~~~}\huxbpad{4pt}} &
\multicolumn{1}{r!{\huxvb{0}}}{\huxtpad{4pt}\raggedleft {\fontsize{9.5pt}{11.4pt}\selectfont -0.482~~~~}\huxbpad{4pt}} \tabularnewline[-0.5pt]


\hhline{}
\arrayrulecolor{black}

\multicolumn{1}{!{\huxvb{0}}l!{\huxvb{0}}}{\huxtpad{4pt}\raggedright {\fontsize{9.5pt}{11.4pt}\selectfont }\huxbpad{4pt}} &
\multicolumn{1}{r!{\huxvb{0}}}{\huxtpad{4pt}\raggedleft {\fontsize{9.5pt}{11.4pt}\selectfont (1.390)~~~}\huxbpad{4pt}} &
\multicolumn{1}{r!{\huxvb{0}}}{\huxtpad{4pt}\raggedleft {\fontsize{9.5pt}{11.4pt}\selectfont (1.589)~~~}\huxbpad{4pt}} &
\multicolumn{1}{r!{\huxvb{0}}}{\huxtpad{4pt}\raggedleft {\fontsize{9.5pt}{11.4pt}\selectfont (1.487)~~~}\huxbpad{4pt}} \tabularnewline[-0.5pt]


\hhline{}
\arrayrulecolor{black}

\multicolumn{1}{!{\huxvb{0}}l!{\huxvb{0}}}{\huxtpad{4pt}\raggedright {\fontsize{9.5pt}{11.4pt}\selectfont sample\_tract51059452502}\huxbpad{4pt}} &
\multicolumn{1}{r!{\huxvb{0}}}{\huxtpad{4pt}\raggedleft {\fontsize{9.5pt}{11.4pt}\selectfont 17.468 ***}\huxbpad{4pt}} &
\multicolumn{1}{r!{\huxvb{0}}}{\huxtpad{4pt}\raggedleft {\fontsize{9.5pt}{11.4pt}\selectfont 16.663 ***}\huxbpad{4pt}} &
\multicolumn{1}{r!{\huxvb{0}}}{\huxtpad{4pt}\raggedleft {\fontsize{9.5pt}{11.4pt}\selectfont 18.338 ***}\huxbpad{4pt}} \tabularnewline[-0.5pt]


\hhline{}
\arrayrulecolor{black}

\multicolumn{1}{!{\huxvb{0}}l!{\huxvb{0}}}{\huxtpad{4pt}\raggedright {\fontsize{9.5pt}{11.4pt}\selectfont }\huxbpad{4pt}} &
\multicolumn{1}{r!{\huxvb{0}}}{\huxtpad{4pt}\raggedleft {\fontsize{9.5pt}{11.4pt}\selectfont (1.316)~~~}\huxbpad{4pt}} &
\multicolumn{1}{r!{\huxvb{0}}}{\huxtpad{4pt}\raggedleft {\fontsize{9.5pt}{11.4pt}\selectfont (1.414)~~~}\huxbpad{4pt}} &
\multicolumn{1}{r!{\huxvb{0}}}{\huxtpad{4pt}\raggedleft {\fontsize{9.5pt}{11.4pt}\selectfont (1.409)~~~}\huxbpad{4pt}} \tabularnewline[-0.5pt]


\hhline{}
\arrayrulecolor{black}

\multicolumn{1}{!{\huxvb{0}}l!{\huxvb{0}}}{\huxtpad{4pt}\raggedright {\fontsize{9.5pt}{11.4pt}\selectfont sample\_tract51059452600}\huxbpad{4pt}} &
\multicolumn{1}{r!{\huxvb{0}}}{\huxtpad{4pt}\raggedleft {\fontsize{9.5pt}{11.4pt}\selectfont 19.221 ***}\huxbpad{4pt}} &
\multicolumn{1}{r!{\huxvb{0}}}{\huxtpad{4pt}\raggedleft {\fontsize{9.5pt}{11.4pt}\selectfont 18.976 ***}\huxbpad{4pt}} &
\multicolumn{1}{r!{\huxvb{0}}}{\huxtpad{4pt}\raggedleft {\fontsize{9.5pt}{11.4pt}\selectfont 20.510 ***}\huxbpad{4pt}} \tabularnewline[-0.5pt]


\hhline{}
\arrayrulecolor{black}

\multicolumn{1}{!{\huxvb{0}}l!{\huxvb{0}}}{\huxtpad{4pt}\raggedright {\fontsize{9.5pt}{11.4pt}\selectfont }\huxbpad{4pt}} &
\multicolumn{1}{r!{\huxvb{0}}}{\huxtpad{4pt}\raggedleft {\fontsize{9.5pt}{11.4pt}\selectfont (1.117)~~~}\huxbpad{4pt}} &
\multicolumn{1}{r!{\huxvb{0}}}{\huxtpad{4pt}\raggedleft {\fontsize{9.5pt}{11.4pt}\selectfont (1.252)~~~}\huxbpad{4pt}} &
\multicolumn{1}{r!{\huxvb{0}}}{\huxtpad{4pt}\raggedleft {\fontsize{9.5pt}{11.4pt}\selectfont (1.210)~~~}\huxbpad{4pt}} \tabularnewline[-0.5pt]


\hhline{}
\arrayrulecolor{black}

\multicolumn{1}{!{\huxvb{0}}l!{\huxvb{0}}}{\huxtpad{4pt}\raggedright {\fontsize{9.5pt}{11.4pt}\selectfont sample\_tract51059452700}\huxbpad{4pt}} &
\multicolumn{1}{r!{\huxvb{0}}}{\huxtpad{4pt}\raggedleft {\fontsize{9.5pt}{11.4pt}\selectfont 1.084~~~~}\huxbpad{4pt}} &
\multicolumn{1}{r!{\huxvb{0}}}{\huxtpad{4pt}\raggedleft {\fontsize{9.5pt}{11.4pt}\selectfont 0.538~~~~}\huxbpad{4pt}} &
\multicolumn{1}{r!{\huxvb{0}}}{\huxtpad{4pt}\raggedleft {\fontsize{9.5pt}{11.4pt}\selectfont 1.565~~~~}\huxbpad{4pt}} \tabularnewline[-0.5pt]


\hhline{}
\arrayrulecolor{black}

\multicolumn{1}{!{\huxvb{0}}l!{\huxvb{0}}}{\huxtpad{4pt}\raggedright {\fontsize{9.5pt}{11.4pt}\selectfont }\huxbpad{4pt}} &
\multicolumn{1}{r!{\huxvb{0}}}{\huxtpad{4pt}\raggedleft {\fontsize{9.5pt}{11.4pt}\selectfont (1.263)~~~}\huxbpad{4pt}} &
\multicolumn{1}{r!{\huxvb{0}}}{\huxtpad{4pt}\raggedleft {\fontsize{9.5pt}{11.4pt}\selectfont (1.459)~~~}\huxbpad{4pt}} &
\multicolumn{1}{r!{\huxvb{0}}}{\huxtpad{4pt}\raggedleft {\fontsize{9.5pt}{11.4pt}\selectfont (1.442)~~~}\huxbpad{4pt}} \tabularnewline[-0.5pt]


\hhline{}
\arrayrulecolor{black}

\multicolumn{1}{!{\huxvb{0}}l!{\huxvb{0}}}{\huxtpad{4pt}\raggedright {\fontsize{9.5pt}{11.4pt}\selectfont sample\_tract51059452801}\huxbpad{4pt}} &
\multicolumn{1}{r!{\huxvb{0}}}{\huxtpad{4pt}\raggedleft {\fontsize{9.5pt}{11.4pt}\selectfont 0.925~~~~}\huxbpad{4pt}} &
\multicolumn{1}{r!{\huxvb{0}}}{\huxtpad{4pt}\raggedleft {\fontsize{9.5pt}{11.4pt}\selectfont 0.355~~~~}\huxbpad{4pt}} &
\multicolumn{1}{r!{\huxvb{0}}}{\huxtpad{4pt}\raggedleft {\fontsize{9.5pt}{11.4pt}\selectfont 0.331~~~~}\huxbpad{4pt}} \tabularnewline[-0.5pt]


\hhline{}
\arrayrulecolor{black}

\multicolumn{1}{!{\huxvb{0}}l!{\huxvb{0}}}{\huxtpad{4pt}\raggedright {\fontsize{9.5pt}{11.4pt}\selectfont }\huxbpad{4pt}} &
\multicolumn{1}{r!{\huxvb{0}}}{\huxtpad{4pt}\raggedleft {\fontsize{9.5pt}{11.4pt}\selectfont (1.138)~~~}\huxbpad{4pt}} &
\multicolumn{1}{r!{\huxvb{0}}}{\huxtpad{4pt}\raggedleft {\fontsize{9.5pt}{11.4pt}\selectfont (1.337)~~~}\huxbpad{4pt}} &
\multicolumn{1}{r!{\huxvb{0}}}{\huxtpad{4pt}\raggedleft {\fontsize{9.5pt}{11.4pt}\selectfont (1.446)~~~}\huxbpad{4pt}} \tabularnewline[-0.5pt]


\hhline{}
\arrayrulecolor{black}

\multicolumn{1}{!{\huxvb{0}}l!{\huxvb{0}}}{\huxtpad{4pt}\raggedright {\fontsize{9.5pt}{11.4pt}\selectfont sample\_tract51059461901}\huxbpad{4pt}} &
\multicolumn{1}{r!{\huxvb{0}}}{\huxtpad{4pt}\raggedleft {\fontsize{9.5pt}{11.4pt}\selectfont 37.659 ***}\huxbpad{4pt}} &
\multicolumn{1}{r!{\huxvb{0}}}{\huxtpad{4pt}\raggedleft {\fontsize{9.5pt}{11.4pt}\selectfont 37.242 ***}\huxbpad{4pt}} &
\multicolumn{1}{r!{\huxvb{0}}}{\huxtpad{4pt}\raggedleft {\fontsize{9.5pt}{11.4pt}\selectfont 40.713 ***}\huxbpad{4pt}} \tabularnewline[-0.5pt]


\hhline{}
\arrayrulecolor{black}

\multicolumn{1}{!{\huxvb{0}}l!{\huxvb{0}}}{\huxtpad{4pt}\raggedright {\fontsize{9.5pt}{11.4pt}\selectfont }\huxbpad{4pt}} &
\multicolumn{1}{r!{\huxvb{0}}}{\huxtpad{4pt}\raggedleft {\fontsize{9.5pt}{11.4pt}\selectfont (1.051)~~~}\huxbpad{4pt}} &
\multicolumn{1}{r!{\huxvb{0}}}{\huxtpad{4pt}\raggedleft {\fontsize{9.5pt}{11.4pt}\selectfont (1.212)~~~}\huxbpad{4pt}} &
\multicolumn{1}{r!{\huxvb{0}}}{\huxtpad{4pt}\raggedleft {\fontsize{9.5pt}{11.4pt}\selectfont (1.285)~~~}\huxbpad{4pt}} \tabularnewline[-0.5pt]


\hhline{}
\arrayrulecolor{black}

\multicolumn{1}{!{\huxvb{0}}l!{\huxvb{0}}}{\huxtpad{4pt}\raggedright {\fontsize{9.5pt}{11.4pt}\selectfont sample\_tract51059471402}\huxbpad{4pt}} &
\multicolumn{1}{r!{\huxvb{0}}}{\huxtpad{4pt}\raggedleft {\fontsize{9.5pt}{11.4pt}\selectfont 18.270 ***}\huxbpad{4pt}} &
\multicolumn{1}{r!{\huxvb{0}}}{\huxtpad{4pt}\raggedleft {\fontsize{9.5pt}{11.4pt}\selectfont 18.382 ***}\huxbpad{4pt}} &
\multicolumn{1}{r!{\huxvb{0}}}{\huxtpad{4pt}\raggedleft {\fontsize{9.5pt}{11.4pt}\selectfont 20.066 ***}\huxbpad{4pt}} \tabularnewline[-0.5pt]


\hhline{}
\arrayrulecolor{black}

\multicolumn{1}{!{\huxvb{0}}l!{\huxvb{0}}}{\huxtpad{4pt}\raggedright {\fontsize{9.5pt}{11.4pt}\selectfont }\huxbpad{4pt}} &
\multicolumn{1}{r!{\huxvb{0}}}{\huxtpad{4pt}\raggedleft {\fontsize{9.5pt}{11.4pt}\selectfont (1.700)~~~}\huxbpad{4pt}} &
\multicolumn{1}{r!{\huxvb{0}}}{\huxtpad{4pt}\raggedleft {\fontsize{9.5pt}{11.4pt}\selectfont (1.834)~~~}\huxbpad{4pt}} &
\multicolumn{1}{r!{\huxvb{0}}}{\huxtpad{4pt}\raggedleft {\fontsize{9.5pt}{11.4pt}\selectfont (1.792)~~~}\huxbpad{4pt}} \tabularnewline[-0.5pt]


\hhline{}
\arrayrulecolor{black}

\multicolumn{1}{!{\huxvb{0}}l!{\huxvb{0}}}{\huxtpad{4pt}\raggedright {\fontsize{9.5pt}{11.4pt}\selectfont sample\_tract51059480901}\huxbpad{4pt}} &
\multicolumn{1}{r!{\huxvb{0}}}{\huxtpad{4pt}\raggedleft {\fontsize{9.5pt}{11.4pt}\selectfont 19.617 ***}\huxbpad{4pt}} &
\multicolumn{1}{r!{\huxvb{0}}}{\huxtpad{4pt}\raggedleft {\fontsize{9.5pt}{11.4pt}\selectfont 18.717 ***}\huxbpad{4pt}} &
\multicolumn{1}{r!{\huxvb{0}}}{\huxtpad{4pt}\raggedleft {\fontsize{9.5pt}{11.4pt}\selectfont 19.108 ***}\huxbpad{4pt}} \tabularnewline[-0.5pt]


\hhline{}
\arrayrulecolor{black}

\multicolumn{1}{!{\huxvb{0}}l!{\huxvb{0}}}{\huxtpad{4pt}\raggedright {\fontsize{9.5pt}{11.4pt}\selectfont }\huxbpad{4pt}} &
\multicolumn{1}{r!{\huxvb{0}}}{\huxtpad{4pt}\raggedleft {\fontsize{9.5pt}{11.4pt}\selectfont (1.359)~~~}\huxbpad{4pt}} &
\multicolumn{1}{r!{\huxvb{0}}}{\huxtpad{4pt}\raggedleft {\fontsize{9.5pt}{11.4pt}\selectfont (1.383)~~~}\huxbpad{4pt}} &
\multicolumn{1}{r!{\huxvb{0}}}{\huxtpad{4pt}\raggedleft {\fontsize{9.5pt}{11.4pt}\selectfont (1.198)~~~}\huxbpad{4pt}} \tabularnewline[-0.5pt]


\hhline{}
\arrayrulecolor{black}

\multicolumn{1}{!{\huxvb{0}}l!{\huxvb{0}}}{\huxtpad{4pt}\raggedright {\fontsize{9.5pt}{11.4pt}\selectfont sample\_tract51059480902}\huxbpad{4pt}} &
\multicolumn{1}{r!{\huxvb{0}}}{\huxtpad{4pt}\raggedleft {\fontsize{9.5pt}{11.4pt}\selectfont 20.155 ***}\huxbpad{4pt}} &
\multicolumn{1}{r!{\huxvb{0}}}{\huxtpad{4pt}\raggedleft {\fontsize{9.5pt}{11.4pt}\selectfont 19.766 ***}\huxbpad{4pt}} &
\multicolumn{1}{r!{\huxvb{0}}}{\huxtpad{4pt}\raggedleft {\fontsize{9.5pt}{11.4pt}\selectfont 19.720 ***}\huxbpad{4pt}} \tabularnewline[-0.5pt]


\hhline{}
\arrayrulecolor{black}

\multicolumn{1}{!{\huxvb{0}}l!{\huxvb{0}}}{\huxtpad{4pt}\raggedright {\fontsize{9.5pt}{11.4pt}\selectfont }\huxbpad{4pt}} &
\multicolumn{1}{r!{\huxvb{0}}}{\huxtpad{4pt}\raggedleft {\fontsize{9.5pt}{11.4pt}\selectfont (1.385)~~~}\huxbpad{4pt}} &
\multicolumn{1}{r!{\huxvb{0}}}{\huxtpad{4pt}\raggedleft {\fontsize{9.5pt}{11.4pt}\selectfont (1.677)~~~}\huxbpad{4pt}} &
\multicolumn{1}{r!{\huxvb{0}}}{\huxtpad{4pt}\raggedleft {\fontsize{9.5pt}{11.4pt}\selectfont (1.829)~~~}\huxbpad{4pt}} \tabularnewline[-0.5pt]


\hhline{}
\arrayrulecolor{black}

\multicolumn{1}{!{\huxvb{0}}l!{\huxvb{0}}}{\huxtpad{4pt}\raggedright {\fontsize{9.5pt}{11.4pt}\selectfont sample\_tract51059481103}\huxbpad{4pt}} &
\multicolumn{1}{r!{\huxvb{0}}}{\huxtpad{4pt}\raggedleft {\fontsize{9.5pt}{11.4pt}\selectfont 1.025~~~~}\huxbpad{4pt}} &
\multicolumn{1}{r!{\huxvb{0}}}{\huxtpad{4pt}\raggedleft {\fontsize{9.5pt}{11.4pt}\selectfont 0.795~~~~}\huxbpad{4pt}} &
\multicolumn{1}{r!{\huxvb{0}}}{\huxtpad{4pt}\raggedleft {\fontsize{9.5pt}{11.4pt}\selectfont 2.093~~~~}\huxbpad{4pt}} \tabularnewline[-0.5pt]


\hhline{}
\arrayrulecolor{black}

\multicolumn{1}{!{\huxvb{0}}l!{\huxvb{0}}}{\huxtpad{4pt}\raggedright {\fontsize{9.5pt}{11.4pt}\selectfont }\huxbpad{4pt}} &
\multicolumn{1}{r!{\huxvb{0}}}{\huxtpad{4pt}\raggedleft {\fontsize{9.5pt}{11.4pt}\selectfont (1.371)~~~}\huxbpad{4pt}} &
\multicolumn{1}{r!{\huxvb{0}}}{\huxtpad{4pt}\raggedleft {\fontsize{9.5pt}{11.4pt}\selectfont (1.532)~~~}\huxbpad{4pt}} &
\multicolumn{1}{r!{\huxvb{0}}}{\huxtpad{4pt}\raggedleft {\fontsize{9.5pt}{11.4pt}\selectfont (1.502)~~~}\huxbpad{4pt}} \tabularnewline[-0.5pt]


\hhline{}
\arrayrulecolor{black}

\multicolumn{1}{!{\huxvb{0}}l!{\huxvb{0}}}{\huxtpad{4pt}\raggedright {\fontsize{9.5pt}{11.4pt}\selectfont sample\_tract51059481202}\huxbpad{4pt}} &
\multicolumn{1}{r!{\huxvb{0}}}{\huxtpad{4pt}\raggedleft {\fontsize{9.5pt}{11.4pt}\selectfont 38.297 ***}\huxbpad{4pt}} &
\multicolumn{1}{r!{\huxvb{0}}}{\huxtpad{4pt}\raggedleft {\fontsize{9.5pt}{11.4pt}\selectfont 37.385 ***}\huxbpad{4pt}} &
\multicolumn{1}{r!{\huxvb{0}}}{\huxtpad{4pt}\raggedleft {\fontsize{9.5pt}{11.4pt}\selectfont 39.066 ***}\huxbpad{4pt}} \tabularnewline[-0.5pt]


\hhline{}
\arrayrulecolor{black}

\multicolumn{1}{!{\huxvb{0}}l!{\huxvb{0}}}{\huxtpad{4pt}\raggedright {\fontsize{9.5pt}{11.4pt}\selectfont }\huxbpad{4pt}} &
\multicolumn{1}{r!{\huxvb{0}}}{\huxtpad{4pt}\raggedleft {\fontsize{9.5pt}{11.4pt}\selectfont (1.268)~~~}\huxbpad{4pt}} &
\multicolumn{1}{r!{\huxvb{0}}}{\huxtpad{4pt}\raggedleft {\fontsize{9.5pt}{11.4pt}\selectfont (1.375)~~~}\huxbpad{4pt}} &
\multicolumn{1}{r!{\huxvb{0}}}{\huxtpad{4pt}\raggedleft {\fontsize{9.5pt}{11.4pt}\selectfont (1.456)~~~}\huxbpad{4pt}} \tabularnewline[-0.5pt]


\hhline{}
\arrayrulecolor{black}

\multicolumn{1}{!{\huxvb{0}}l!{\huxvb{0}}}{\huxtpad{4pt}\raggedright {\fontsize{9.5pt}{11.4pt}\selectfont sample\_tract51059482302}\huxbpad{4pt}} &
\multicolumn{1}{r!{\huxvb{0}}}{\huxtpad{4pt}\raggedleft {\fontsize{9.5pt}{11.4pt}\selectfont 39.005 ***}\huxbpad{4pt}} &
\multicolumn{1}{r!{\huxvb{0}}}{\huxtpad{4pt}\raggedleft {\fontsize{9.5pt}{11.4pt}\selectfont 38.691 ***}\huxbpad{4pt}} &
\multicolumn{1}{r!{\huxvb{0}}}{\huxtpad{4pt}\raggedleft {\fontsize{9.5pt}{11.4pt}\selectfont 42.789 ***}\huxbpad{4pt}} \tabularnewline[-0.5pt]


\hhline{}
\arrayrulecolor{black}

\multicolumn{1}{!{\huxvb{0}}l!{\huxvb{0}}}{\huxtpad{4pt}\raggedright {\fontsize{9.5pt}{11.4pt}\selectfont }\huxbpad{4pt}} &
\multicolumn{1}{r!{\huxvb{0}}}{\huxtpad{4pt}\raggedleft {\fontsize{9.5pt}{11.4pt}\selectfont (1.372)~~~}\huxbpad{4pt}} &
\multicolumn{1}{r!{\huxvb{0}}}{\huxtpad{4pt}\raggedleft {\fontsize{9.5pt}{11.4pt}\selectfont (1.493)~~~}\huxbpad{4pt}} &
\multicolumn{1}{r!{\huxvb{0}}}{\huxtpad{4pt}\raggedleft {\fontsize{9.5pt}{11.4pt}\selectfont (1.498)~~~}\huxbpad{4pt}} \tabularnewline[-0.5pt]


\hhline{}
\arrayrulecolor{black}

\multicolumn{1}{!{\huxvb{0}}l!{\huxvb{0}}}{\huxtpad{4pt}\raggedright {\fontsize{9.5pt}{11.4pt}\selectfont sample\_tract51059491301}\huxbpad{4pt}} &
\multicolumn{1}{r!{\huxvb{0}}}{\huxtpad{4pt}\raggedleft {\fontsize{9.5pt}{11.4pt}\selectfont 19.804 ***}\huxbpad{4pt}} &
\multicolumn{1}{r!{\huxvb{0}}}{\huxtpad{4pt}\raggedleft {\fontsize{9.5pt}{11.4pt}\selectfont 19.236 ***}\huxbpad{4pt}} &
\multicolumn{1}{r!{\huxvb{0}}}{\huxtpad{4pt}\raggedleft {\fontsize{9.5pt}{11.4pt}\selectfont 20.874 ***}\huxbpad{4pt}} \tabularnewline[-0.5pt]


\hhline{}
\arrayrulecolor{black}

\multicolumn{1}{!{\huxvb{0}}l!{\huxvb{0}}}{\huxtpad{4pt}\raggedright {\fontsize{9.5pt}{11.4pt}\selectfont }\huxbpad{4pt}} &
\multicolumn{1}{r!{\huxvb{0}}}{\huxtpad{4pt}\raggedleft {\fontsize{9.5pt}{11.4pt}\selectfont (1.279)~~~}\huxbpad{4pt}} &
\multicolumn{1}{r!{\huxvb{0}}}{\huxtpad{4pt}\raggedleft {\fontsize{9.5pt}{11.4pt}\selectfont (1.361)~~~}\huxbpad{4pt}} &
\multicolumn{1}{r!{\huxvb{0}}}{\huxtpad{4pt}\raggedleft {\fontsize{9.5pt}{11.4pt}\selectfont (1.324)~~~}\huxbpad{4pt}} \tabularnewline[-0.5pt]


\hhline{}
\arrayrulecolor{black}

\multicolumn{1}{!{\huxvb{0}}l!{\huxvb{0}}}{\huxtpad{4pt}\raggedright {\fontsize{9.5pt}{11.4pt}\selectfont sample\_tract51059491601}\huxbpad{4pt}} &
\multicolumn{1}{r!{\huxvb{0}}}{\huxtpad{4pt}\raggedleft {\fontsize{9.5pt}{11.4pt}\selectfont 21.609 ***}\huxbpad{4pt}} &
\multicolumn{1}{r!{\huxvb{0}}}{\huxtpad{4pt}\raggedleft {\fontsize{9.5pt}{11.4pt}\selectfont 21.219 ***}\huxbpad{4pt}} &
\multicolumn{1}{r!{\huxvb{0}}}{\huxtpad{4pt}\raggedleft {\fontsize{9.5pt}{11.4pt}\selectfont 22.479 ***}\huxbpad{4pt}} \tabularnewline[-0.5pt]


\hhline{}
\arrayrulecolor{black}

\multicolumn{1}{!{\huxvb{0}}l!{\huxvb{0}}}{\huxtpad{4pt}\raggedright {\fontsize{9.5pt}{11.4pt}\selectfont }\huxbpad{4pt}} &
\multicolumn{1}{r!{\huxvb{0}}}{\huxtpad{4pt}\raggedleft {\fontsize{9.5pt}{11.4pt}\selectfont (1.233)~~~}\huxbpad{4pt}} &
\multicolumn{1}{r!{\huxvb{0}}}{\huxtpad{4pt}\raggedleft {\fontsize{9.5pt}{11.4pt}\selectfont (1.363)~~~}\huxbpad{4pt}} &
\multicolumn{1}{r!{\huxvb{0}}}{\huxtpad{4pt}\raggedleft {\fontsize{9.5pt}{11.4pt}\selectfont (1.209)~~~}\huxbpad{4pt}} \tabularnewline[-0.5pt]


\hhline{}
\arrayrulecolor{black}

\multicolumn{1}{!{\huxvb{0}}l!{\huxvb{0}}}{\huxtpad{4pt}\raggedright {\fontsize{9.5pt}{11.4pt}\selectfont sample\_tract51059491702}\huxbpad{4pt}} &
\multicolumn{1}{r!{\huxvb{0}}}{\huxtpad{4pt}\raggedleft {\fontsize{9.5pt}{11.4pt}\selectfont 23.199 ***}\huxbpad{4pt}} &
\multicolumn{1}{r!{\huxvb{0}}}{\huxtpad{4pt}\raggedleft {\fontsize{9.5pt}{11.4pt}\selectfont 23.046 ***}\huxbpad{4pt}} &
\multicolumn{1}{r!{\huxvb{0}}}{\huxtpad{4pt}\raggedleft {\fontsize{9.5pt}{11.4pt}\selectfont 23.529 ***}\huxbpad{4pt}} \tabularnewline[-0.5pt]


\hhline{}
\arrayrulecolor{black}

\multicolumn{1}{!{\huxvb{0}}l!{\huxvb{0}}}{\huxtpad{4pt}\raggedright {\fontsize{9.5pt}{11.4pt}\selectfont }\huxbpad{4pt}} &
\multicolumn{1}{r!{\huxvb{0}}}{\huxtpad{4pt}\raggedleft {\fontsize{9.5pt}{11.4pt}\selectfont (1.712)~~~}\huxbpad{4pt}} &
\multicolumn{1}{r!{\huxvb{0}}}{\huxtpad{4pt}\raggedleft {\fontsize{9.5pt}{11.4pt}\selectfont (1.842)~~~}\huxbpad{4pt}} &
\multicolumn{1}{r!{\huxvb{0}}}{\huxtpad{4pt}\raggedleft {\fontsize{9.5pt}{11.4pt}\selectfont (1.822)~~~}\huxbpad{4pt}} \tabularnewline[-0.5pt]


\hhline{}
\arrayrulecolor{black}

\multicolumn{1}{!{\huxvb{0}}l!{\huxvb{0}}}{\huxtpad{4pt}\raggedright {\fontsize{9.5pt}{11.4pt}\selectfont sample\_tract51059491801}\huxbpad{4pt}} &
\multicolumn{1}{r!{\huxvb{0}}}{\huxtpad{4pt}\raggedleft {\fontsize{9.5pt}{11.4pt}\selectfont -0.172~~~~}\huxbpad{4pt}} &
\multicolumn{1}{r!{\huxvb{0}}}{\huxtpad{4pt}\raggedleft {\fontsize{9.5pt}{11.4pt}\selectfont -0.309~~~~}\huxbpad{4pt}} &
\multicolumn{1}{r!{\huxvb{0}}}{\huxtpad{4pt}\raggedleft {\fontsize{9.5pt}{11.4pt}\selectfont 0.626~~~~}\huxbpad{4pt}} \tabularnewline[-0.5pt]


\hhline{}
\arrayrulecolor{black}

\multicolumn{1}{!{\huxvb{0}}l!{\huxvb{0}}}{\huxtpad{4pt}\raggedright {\fontsize{9.5pt}{11.4pt}\selectfont }\huxbpad{4pt}} &
\multicolumn{1}{r!{\huxvb{0}}}{\huxtpad{4pt}\raggedleft {\fontsize{9.5pt}{11.4pt}\selectfont (1.189)~~~}\huxbpad{4pt}} &
\multicolumn{1}{r!{\huxvb{0}}}{\huxtpad{4pt}\raggedleft {\fontsize{9.5pt}{11.4pt}\selectfont (1.320)~~~}\huxbpad{4pt}} &
\multicolumn{1}{r!{\huxvb{0}}}{\huxtpad{4pt}\raggedleft {\fontsize{9.5pt}{11.4pt}\selectfont (1.249)~~~}\huxbpad{4pt}} \tabularnewline[-0.5pt]


\hhline{}
\arrayrulecolor{black}

\multicolumn{1}{!{\huxvb{0}}l!{\huxvb{0}}}{\huxtpad{4pt}\raggedright {\fontsize{9.5pt}{11.4pt}\selectfont sample\_tract51059492400}\huxbpad{4pt}} &
\multicolumn{1}{r!{\huxvb{0}}}{\huxtpad{4pt}\raggedleft {\fontsize{9.5pt}{11.4pt}\selectfont 0.725~~~~}\huxbpad{4pt}} &
\multicolumn{1}{r!{\huxvb{0}}}{\huxtpad{4pt}\raggedleft {\fontsize{9.5pt}{11.4pt}\selectfont -0.422~~~~}\huxbpad{4pt}} &
\multicolumn{1}{r!{\huxvb{0}}}{\huxtpad{4pt}\raggedleft {\fontsize{9.5pt}{11.4pt}\selectfont 0.176~~~~}\huxbpad{4pt}} \tabularnewline[-0.5pt]


\hhline{}
\arrayrulecolor{black}

\multicolumn{1}{!{\huxvb{0}}l!{\huxvb{0}}}{\huxtpad{4pt}\raggedright {\fontsize{9.5pt}{11.4pt}\selectfont }\huxbpad{4pt}} &
\multicolumn{1}{r!{\huxvb{0}}}{\huxtpad{4pt}\raggedleft {\fontsize{9.5pt}{11.4pt}\selectfont (1.100)~~~}\huxbpad{4pt}} &
\multicolumn{1}{r!{\huxvb{0}}}{\huxtpad{4pt}\raggedleft {\fontsize{9.5pt}{11.4pt}\selectfont (1.528)~~~}\huxbpad{4pt}} &
\multicolumn{1}{r!{\huxvb{0}}}{\huxtpad{4pt}\raggedleft {\fontsize{9.5pt}{11.4pt}\selectfont (1.357)~~~}\huxbpad{4pt}} \tabularnewline[-0.5pt]


\hhline{}
\arrayrulecolor{black}

\multicolumn{1}{!{\huxvb{0}}l!{\huxvb{0}}}{\huxtpad{4pt}\raggedright {\fontsize{9.5pt}{11.4pt}\selectfont sample\_tract51510200102}\huxbpad{4pt}} &
\multicolumn{1}{r!{\huxvb{0}}}{\huxtpad{4pt}\raggedleft {\fontsize{9.5pt}{11.4pt}\selectfont 17.312 ***}\huxbpad{4pt}} &
\multicolumn{1}{r!{\huxvb{0}}}{\huxtpad{4pt}\raggedleft {\fontsize{9.5pt}{11.4pt}\selectfont 16.794 ***}\huxbpad{4pt}} &
\multicolumn{1}{r!{\huxvb{0}}}{\huxtpad{4pt}\raggedleft {\fontsize{9.5pt}{11.4pt}\selectfont 17.436 ***}\huxbpad{4pt}} \tabularnewline[-0.5pt]


\hhline{}
\arrayrulecolor{black}

\multicolumn{1}{!{\huxvb{0}}l!{\huxvb{0}}}{\huxtpad{4pt}\raggedright {\fontsize{9.5pt}{11.4pt}\selectfont }\huxbpad{4pt}} &
\multicolumn{1}{r!{\huxvb{0}}}{\huxtpad{4pt}\raggedleft {\fontsize{9.5pt}{11.4pt}\selectfont (1.895)~~~}\huxbpad{4pt}} &
\multicolumn{1}{r!{\huxvb{0}}}{\huxtpad{4pt}\raggedleft {\fontsize{9.5pt}{11.4pt}\selectfont (2.125)~~~}\huxbpad{4pt}} &
\multicolumn{1}{r!{\huxvb{0}}}{\huxtpad{4pt}\raggedleft {\fontsize{9.5pt}{11.4pt}\selectfont (1.732)~~~}\huxbpad{4pt}} \tabularnewline[-0.5pt]


\hhline{}
\arrayrulecolor{black}

\multicolumn{1}{!{\huxvb{0}}l!{\huxvb{0}}}{\huxtpad{4pt}\raggedright {\fontsize{9.5pt}{11.4pt}\selectfont sample\_tract51510200103}\huxbpad{4pt}} &
\multicolumn{1}{r!{\huxvb{0}}}{\huxtpad{4pt}\raggedleft {\fontsize{9.5pt}{11.4pt}\selectfont 1.223~~~~}\huxbpad{4pt}} &
\multicolumn{1}{r!{\huxvb{0}}}{\huxtpad{4pt}\raggedleft {\fontsize{9.5pt}{11.4pt}\selectfont 0.569~~~~}\huxbpad{4pt}} &
\multicolumn{1}{r!{\huxvb{0}}}{\huxtpad{4pt}\raggedleft {\fontsize{9.5pt}{11.4pt}\selectfont 0.407~~~~}\huxbpad{4pt}} \tabularnewline[-0.5pt]


\hhline{}
\arrayrulecolor{black}

\multicolumn{1}{!{\huxvb{0}}l!{\huxvb{0}}}{\huxtpad{4pt}\raggedright {\fontsize{9.5pt}{11.4pt}\selectfont }\huxbpad{4pt}} &
\multicolumn{1}{r!{\huxvb{0}}}{\huxtpad{4pt}\raggedleft {\fontsize{9.5pt}{11.4pt}\selectfont (0.990)~~~}\huxbpad{4pt}} &
\multicolumn{1}{r!{\huxvb{0}}}{\huxtpad{4pt}\raggedleft {\fontsize{9.5pt}{11.4pt}\selectfont (1.141)~~~}\huxbpad{4pt}} &
\multicolumn{1}{r!{\huxvb{0}}}{\huxtpad{4pt}\raggedleft {\fontsize{9.5pt}{11.4pt}\selectfont (1.239)~~~}\huxbpad{4pt}} \tabularnewline[-0.5pt]


\hhline{}
\arrayrulecolor{black}

\multicolumn{1}{!{\huxvb{0}}l!{\huxvb{0}}}{\huxtpad{4pt}\raggedright {\fontsize{9.5pt}{11.4pt}\selectfont sample\_tract51510200107}\huxbpad{4pt}} &
\multicolumn{1}{r!{\huxvb{0}}}{\huxtpad{4pt}\raggedleft {\fontsize{9.5pt}{11.4pt}\selectfont 1.169~~~~}\huxbpad{4pt}} &
\multicolumn{1}{r!{\huxvb{0}}}{\huxtpad{4pt}\raggedleft {\fontsize{9.5pt}{11.4pt}\selectfont 0.986~~~~}\huxbpad{4pt}} &
\multicolumn{1}{r!{\huxvb{0}}}{\huxtpad{4pt}\raggedleft {\fontsize{9.5pt}{11.4pt}\selectfont 1.617~~~~}\huxbpad{4pt}} \tabularnewline[-0.5pt]


\hhline{}
\arrayrulecolor{black}

\multicolumn{1}{!{\huxvb{0}}l!{\huxvb{0}}}{\huxtpad{4pt}\raggedright {\fontsize{9.5pt}{11.4pt}\selectfont }\huxbpad{4pt}} &
\multicolumn{1}{r!{\huxvb{0}}}{\huxtpad{4pt}\raggedleft {\fontsize{9.5pt}{11.4pt}\selectfont (1.123)~~~}\huxbpad{4pt}} &
\multicolumn{1}{r!{\huxvb{0}}}{\huxtpad{4pt}\raggedleft {\fontsize{9.5pt}{11.4pt}\selectfont (1.224)~~~}\huxbpad{4pt}} &
\multicolumn{1}{r!{\huxvb{0}}}{\huxtpad{4pt}\raggedleft {\fontsize{9.5pt}{11.4pt}\selectfont (1.165)~~~}\huxbpad{4pt}} \tabularnewline[-0.5pt]


\hhline{}
\arrayrulecolor{black}

\multicolumn{1}{!{\huxvb{0}}l!{\huxvb{0}}}{\huxtpad{4pt}\raggedright {\fontsize{9.5pt}{11.4pt}\selectfont sample\_tract51510200301}\huxbpad{4pt}} &
\multicolumn{1}{r!{\huxvb{0}}}{\huxtpad{4pt}\raggedleft {\fontsize{9.5pt}{11.4pt}\selectfont 20.240 ***}\huxbpad{4pt}} &
\multicolumn{1}{r!{\huxvb{0}}}{\huxtpad{4pt}\raggedleft {\fontsize{9.5pt}{11.4pt}\selectfont 20.075 ***}\huxbpad{4pt}} &
\multicolumn{1}{r!{\huxvb{0}}}{\huxtpad{4pt}\raggedleft {\fontsize{9.5pt}{11.4pt}\selectfont 20.817 ***}\huxbpad{4pt}} \tabularnewline[-0.5pt]


\hhline{}
\arrayrulecolor{black}

\multicolumn{1}{!{\huxvb{0}}l!{\huxvb{0}}}{\huxtpad{4pt}\raggedright {\fontsize{9.5pt}{11.4pt}\selectfont }\huxbpad{4pt}} &
\multicolumn{1}{r!{\huxvb{0}}}{\huxtpad{4pt}\raggedleft {\fontsize{9.5pt}{11.4pt}\selectfont (1.637)~~~}\huxbpad{4pt}} &
\multicolumn{1}{r!{\huxvb{0}}}{\huxtpad{4pt}\raggedleft {\fontsize{9.5pt}{11.4pt}\selectfont (1.653)~~~}\huxbpad{4pt}} &
\multicolumn{1}{r!{\huxvb{0}}}{\huxtpad{4pt}\raggedleft {\fontsize{9.5pt}{11.4pt}\selectfont (1.495)~~~}\huxbpad{4pt}} \tabularnewline[-0.5pt]


\hhline{}
\arrayrulecolor{black}

\multicolumn{1}{!{\huxvb{0}}l!{\huxvb{0}}}{\huxtpad{4pt}\raggedright {\fontsize{9.5pt}{11.4pt}\selectfont sample\_tract51510200303}\huxbpad{4pt}} &
\multicolumn{1}{r!{\huxvb{0}}}{\huxtpad{4pt}\raggedleft {\fontsize{9.5pt}{11.4pt}\selectfont 20.531 ***}\huxbpad{4pt}} &
\multicolumn{1}{r!{\huxvb{0}}}{\huxtpad{4pt}\raggedleft {\fontsize{9.5pt}{11.4pt}\selectfont 20.395 ***}\huxbpad{4pt}} &
\multicolumn{1}{r!{\huxvb{0}}}{\huxtpad{4pt}\raggedleft {\fontsize{9.5pt}{11.4pt}\selectfont 21.085 ***}\huxbpad{4pt}} \tabularnewline[-0.5pt]


\hhline{}
\arrayrulecolor{black}

\multicolumn{1}{!{\huxvb{0}}l!{\huxvb{0}}}{\huxtpad{4pt}\raggedright {\fontsize{9.5pt}{11.4pt}\selectfont }\huxbpad{4pt}} &
\multicolumn{1}{r!{\huxvb{0}}}{\huxtpad{4pt}\raggedleft {\fontsize{9.5pt}{11.4pt}\selectfont (1.208)~~~}\huxbpad{4pt}} &
\multicolumn{1}{r!{\huxvb{0}}}{\huxtpad{4pt}\raggedleft {\fontsize{9.5pt}{11.4pt}\selectfont (1.340)~~~}\huxbpad{4pt}} &
\multicolumn{1}{r!{\huxvb{0}}}{\huxtpad{4pt}\raggedleft {\fontsize{9.5pt}{11.4pt}\selectfont (1.190)~~~}\huxbpad{4pt}} \tabularnewline[-0.5pt]


\hhline{}
\arrayrulecolor{black}

\multicolumn{1}{!{\huxvb{0}}l!{\huxvb{0}}}{\huxtpad{4pt}\raggedright {\fontsize{9.5pt}{11.4pt}\selectfont sample\_tract51510200600}\huxbpad{4pt}} &
\multicolumn{1}{r!{\huxvb{0}}}{\huxtpad{4pt}\raggedleft {\fontsize{9.5pt}{11.4pt}\selectfont 19.196 ***}\huxbpad{4pt}} &
\multicolumn{1}{r!{\huxvb{0}}}{\huxtpad{4pt}\raggedleft {\fontsize{9.5pt}{11.4pt}\selectfont 18.695 ***}\huxbpad{4pt}} &
\multicolumn{1}{r!{\huxvb{0}}}{\huxtpad{4pt}\raggedleft {\fontsize{9.5pt}{11.4pt}\selectfont 20.469 ***}\huxbpad{4pt}} \tabularnewline[-0.5pt]


\hhline{}
\arrayrulecolor{black}

\multicolumn{1}{!{\huxvb{0}}l!{\huxvb{0}}}{\huxtpad{4pt}\raggedright {\fontsize{9.5pt}{11.4pt}\selectfont }\huxbpad{4pt}} &
\multicolumn{1}{r!{\huxvb{0}}}{\huxtpad{4pt}\raggedleft {\fontsize{9.5pt}{11.4pt}\selectfont (1.366)~~~}\huxbpad{4pt}} &
\multicolumn{1}{r!{\huxvb{0}}}{\huxtpad{4pt}\raggedleft {\fontsize{9.5pt}{11.4pt}\selectfont (1.473)~~~}\huxbpad{4pt}} &
\multicolumn{1}{r!{\huxvb{0}}}{\huxtpad{4pt}\raggedleft {\fontsize{9.5pt}{11.4pt}\selectfont (1.429)~~~}\huxbpad{4pt}} \tabularnewline[-0.5pt]


\hhline{}
\arrayrulecolor{black}

\multicolumn{1}{!{\huxvb{0}}l!{\huxvb{0}}}{\huxtpad{4pt}\raggedright {\fontsize{9.5pt}{11.4pt}\selectfont age}\huxbpad{4pt}} &
\multicolumn{1}{r!{\huxvb{0}}}{\huxtpad{4pt}\raggedleft {\fontsize{9.5pt}{11.4pt}\selectfont ~~~~~~~~}\huxbpad{4pt}} &
\multicolumn{1}{r!{\huxvb{0}}}{\huxtpad{4pt}\raggedleft {\fontsize{9.5pt}{11.4pt}\selectfont -0.002~~~~}\huxbpad{4pt}} &
\multicolumn{1}{r!{\huxvb{0}}}{\huxtpad{4pt}\raggedleft {\fontsize{9.5pt}{11.4pt}\selectfont 0.030 *~~}\huxbpad{4pt}} \tabularnewline[-0.5pt]


\hhline{}
\arrayrulecolor{black}

\multicolumn{1}{!{\huxvb{0}}l!{\huxvb{0}}}{\huxtpad{4pt}\raggedright {\fontsize{9.5pt}{11.4pt}\selectfont }\huxbpad{4pt}} &
\multicolumn{1}{r!{\huxvb{0}}}{\huxtpad{4pt}\raggedleft {\fontsize{9.5pt}{11.4pt}\selectfont ~~~~~~~~}\huxbpad{4pt}} &
\multicolumn{1}{r!{\huxvb{0}}}{\huxtpad{4pt}\raggedleft {\fontsize{9.5pt}{11.4pt}\selectfont (0.010)~~~}\huxbpad{4pt}} &
\multicolumn{1}{r!{\huxvb{0}}}{\huxtpad{4pt}\raggedleft {\fontsize{9.5pt}{11.4pt}\selectfont (0.013)~~~}\huxbpad{4pt}} \tabularnewline[-0.5pt]


\hhline{}
\arrayrulecolor{black}

\multicolumn{1}{!{\huxvb{0}}l!{\huxvb{0}}}{\huxtpad{4pt}\raggedright {\fontsize{9.5pt}{11.4pt}\selectfont forbornTRUE}\huxbpad{4pt}} &
\multicolumn{1}{r!{\huxvb{0}}}{\huxtpad{4pt}\raggedleft {\fontsize{9.5pt}{11.4pt}\selectfont ~~~~~~~~}\huxbpad{4pt}} &
\multicolumn{1}{r!{\huxvb{0}}}{\huxtpad{4pt}\raggedleft {\fontsize{9.5pt}{11.4pt}\selectfont 0.586~~~~}\huxbpad{4pt}} &
\multicolumn{1}{r!{\huxvb{0}}}{\huxtpad{4pt}\raggedleft {\fontsize{9.5pt}{11.4pt}\selectfont 0.618~~~~}\huxbpad{4pt}} \tabularnewline[-0.5pt]


\hhline{}
\arrayrulecolor{black}

\multicolumn{1}{!{\huxvb{0}}l!{\huxvb{0}}}{\huxtpad{4pt}\raggedright {\fontsize{9.5pt}{11.4pt}\selectfont }\huxbpad{4pt}} &
\multicolumn{1}{r!{\huxvb{0}}}{\huxtpad{4pt}\raggedleft {\fontsize{9.5pt}{11.4pt}\selectfont ~~~~~~~~}\huxbpad{4pt}} &
\multicolumn{1}{r!{\huxvb{0}}}{\huxtpad{4pt}\raggedleft {\fontsize{9.5pt}{11.4pt}\selectfont (0.414)~~~}\huxbpad{4pt}} &
\multicolumn{1}{r!{\huxvb{0}}}{\huxtpad{4pt}\raggedleft {\fontsize{9.5pt}{11.4pt}\selectfont (0.409)~~~}\huxbpad{4pt}} \tabularnewline[-0.5pt]


\hhline{}
\arrayrulecolor{black}

\multicolumn{1}{!{\huxvb{0}}l!{\huxvb{0}}}{\huxtpad{4pt}\raggedright {\fontsize{9.5pt}{11.4pt}\selectfont manTRUE}\huxbpad{4pt}} &
\multicolumn{1}{r!{\huxvb{0}}}{\huxtpad{4pt}\raggedleft {\fontsize{9.5pt}{11.4pt}\selectfont ~~~~~~~~}\huxbpad{4pt}} &
\multicolumn{1}{r!{\huxvb{0}}}{\huxtpad{4pt}\raggedleft {\fontsize{9.5pt}{11.4pt}\selectfont 0.195~~~~}\huxbpad{4pt}} &
\multicolumn{1}{r!{\huxvb{0}}}{\huxtpad{4pt}\raggedleft {\fontsize{9.5pt}{11.4pt}\selectfont 0.223~~~~}\huxbpad{4pt}} \tabularnewline[-0.5pt]


\hhline{}
\arrayrulecolor{black}

\multicolumn{1}{!{\huxvb{0}}l!{\huxvb{0}}}{\huxtpad{4pt}\raggedright {\fontsize{9.5pt}{11.4pt}\selectfont }\huxbpad{4pt}} &
\multicolumn{1}{r!{\huxvb{0}}}{\huxtpad{4pt}\raggedleft {\fontsize{9.5pt}{11.4pt}\selectfont ~~~~~~~~}\huxbpad{4pt}} &
\multicolumn{1}{r!{\huxvb{0}}}{\huxtpad{4pt}\raggedleft {\fontsize{9.5pt}{11.4pt}\selectfont (0.318)~~~}\huxbpad{4pt}} &
\multicolumn{1}{r!{\huxvb{0}}}{\huxtpad{4pt}\raggedleft {\fontsize{9.5pt}{11.4pt}\selectfont (0.316)~~~}\huxbpad{4pt}} \tabularnewline[-0.5pt]


\hhline{}
\arrayrulecolor{black}

\multicolumn{1}{!{\huxvb{0}}l!{\huxvb{0}}}{\huxtpad{4pt}\raggedright {\fontsize{9.5pt}{11.4pt}\selectfont kidsTRUE}\huxbpad{4pt}} &
\multicolumn{1}{r!{\huxvb{0}}}{\huxtpad{4pt}\raggedleft {\fontsize{9.5pt}{11.4pt}\selectfont ~~~~~~~~}\huxbpad{4pt}} &
\multicolumn{1}{r!{\huxvb{0}}}{\huxtpad{4pt}\raggedleft {\fontsize{9.5pt}{11.4pt}\selectfont 0.067~~~~}\huxbpad{4pt}} &
\multicolumn{1}{r!{\huxvb{0}}}{\huxtpad{4pt}\raggedleft {\fontsize{9.5pt}{11.4pt}\selectfont -0.088~~~~}\huxbpad{4pt}} \tabularnewline[-0.5pt]


\hhline{}
\arrayrulecolor{black}

\multicolumn{1}{!{\huxvb{0}}l!{\huxvb{0}}}{\huxtpad{4pt}\raggedright {\fontsize{9.5pt}{11.4pt}\selectfont }\huxbpad{4pt}} &
\multicolumn{1}{r!{\huxvb{0}}}{\huxtpad{4pt}\raggedleft {\fontsize{9.5pt}{11.4pt}\selectfont ~~~~~~~~}\huxbpad{4pt}} &
\multicolumn{1}{r!{\huxvb{0}}}{\huxtpad{4pt}\raggedleft {\fontsize{9.5pt}{11.4pt}\selectfont (0.326)~~~}\huxbpad{4pt}} &
\multicolumn{1}{r!{\huxvb{0}}}{\huxtpad{4pt}\raggedleft {\fontsize{9.5pt}{11.4pt}\selectfont (0.346)~~~}\huxbpad{4pt}} \tabularnewline[-0.5pt]


\hhline{}
\arrayrulecolor{black}

\multicolumn{1}{!{\huxvb{0}}l!{\huxvb{0}}}{\huxtpad{4pt}\raggedright {\fontsize{9.5pt}{11.4pt}\selectfont marriedTRUE}\huxbpad{4pt}} &
\multicolumn{1}{r!{\huxvb{0}}}{\huxtpad{4pt}\raggedleft {\fontsize{9.5pt}{11.4pt}\selectfont ~~~~~~~~}\huxbpad{4pt}} &
\multicolumn{1}{r!{\huxvb{0}}}{\huxtpad{4pt}\raggedleft {\fontsize{9.5pt}{11.4pt}\selectfont 0.336~~~~}\huxbpad{4pt}} &
\multicolumn{1}{r!{\huxvb{0}}}{\huxtpad{4pt}\raggedleft {\fontsize{9.5pt}{11.4pt}\selectfont -0.155~~~~}\huxbpad{4pt}} \tabularnewline[-0.5pt]


\hhline{}
\arrayrulecolor{black}

\multicolumn{1}{!{\huxvb{0}}l!{\huxvb{0}}}{\huxtpad{4pt}\raggedright {\fontsize{9.5pt}{11.4pt}\selectfont }\huxbpad{4pt}} &
\multicolumn{1}{r!{\huxvb{0}}}{\huxtpad{4pt}\raggedleft {\fontsize{9.5pt}{11.4pt}\selectfont ~~~~~~~~}\huxbpad{4pt}} &
\multicolumn{1}{r!{\huxvb{0}}}{\huxtpad{4pt}\raggedleft {\fontsize{9.5pt}{11.4pt}\selectfont (0.331)~~~}\huxbpad{4pt}} &
\multicolumn{1}{r!{\huxvb{0}}}{\huxtpad{4pt}\raggedleft {\fontsize{9.5pt}{11.4pt}\selectfont (0.330)~~~}\huxbpad{4pt}} \tabularnewline[-0.5pt]


\hhline{}
\arrayrulecolor{black}

\multicolumn{1}{!{\huxvb{0}}l!{\huxvb{0}}}{\huxtpad{4pt}\raggedright {\fontsize{9.5pt}{11.4pt}\selectfont educ.L}\huxbpad{4pt}} &
\multicolumn{1}{r!{\huxvb{0}}}{\huxtpad{4pt}\raggedleft {\fontsize{9.5pt}{11.4pt}\selectfont ~~~~~~~~}\huxbpad{4pt}} &
\multicolumn{1}{r!{\huxvb{0}}}{\huxtpad{4pt}\raggedleft {\fontsize{9.5pt}{11.4pt}\selectfont -1.188~~~~}\huxbpad{4pt}} &
\multicolumn{1}{r!{\huxvb{0}}}{\huxtpad{4pt}\raggedleft {\fontsize{9.5pt}{11.4pt}\selectfont -1.222 *~~}\huxbpad{4pt}} \tabularnewline[-0.5pt]


\hhline{}
\arrayrulecolor{black}

\multicolumn{1}{!{\huxvb{0}}l!{\huxvb{0}}}{\huxtpad{4pt}\raggedright {\fontsize{9.5pt}{11.4pt}\selectfont }\huxbpad{4pt}} &
\multicolumn{1}{r!{\huxvb{0}}}{\huxtpad{4pt}\raggedleft {\fontsize{9.5pt}{11.4pt}\selectfont ~~~~~~~~}\huxbpad{4pt}} &
\multicolumn{1}{r!{\huxvb{0}}}{\huxtpad{4pt}\raggedleft {\fontsize{9.5pt}{11.4pt}\selectfont (0.607)~~~}\huxbpad{4pt}} &
\multicolumn{1}{r!{\huxvb{0}}}{\huxtpad{4pt}\raggedleft {\fontsize{9.5pt}{11.4pt}\selectfont (0.559)~~~}\huxbpad{4pt}} \tabularnewline[-0.5pt]


\hhline{}
\arrayrulecolor{black}

\multicolumn{1}{!{\huxvb{0}}l!{\huxvb{0}}}{\huxtpad{4pt}\raggedright {\fontsize{9.5pt}{11.4pt}\selectfont educ.Q}\huxbpad{4pt}} &
\multicolumn{1}{r!{\huxvb{0}}}{\huxtpad{4pt}\raggedleft {\fontsize{9.5pt}{11.4pt}\selectfont ~~~~~~~~}\huxbpad{4pt}} &
\multicolumn{1}{r!{\huxvb{0}}}{\huxtpad{4pt}\raggedleft {\fontsize{9.5pt}{11.4pt}\selectfont 0.455~~~~}\huxbpad{4pt}} &
\multicolumn{1}{r!{\huxvb{0}}}{\huxtpad{4pt}\raggedleft {\fontsize{9.5pt}{11.4pt}\selectfont 0.420~~~~}\huxbpad{4pt}} \tabularnewline[-0.5pt]


\hhline{}
\arrayrulecolor{black}

\multicolumn{1}{!{\huxvb{0}}l!{\huxvb{0}}}{\huxtpad{4pt}\raggedright {\fontsize{9.5pt}{11.4pt}\selectfont }\huxbpad{4pt}} &
\multicolumn{1}{r!{\huxvb{0}}}{\huxtpad{4pt}\raggedleft {\fontsize{9.5pt}{11.4pt}\selectfont ~~~~~~~~}\huxbpad{4pt}} &
\multicolumn{1}{r!{\huxvb{0}}}{\huxtpad{4pt}\raggedleft {\fontsize{9.5pt}{11.4pt}\selectfont (0.545)~~~}\huxbpad{4pt}} &
\multicolumn{1}{r!{\huxvb{0}}}{\huxtpad{4pt}\raggedleft {\fontsize{9.5pt}{11.4pt}\selectfont (0.532)~~~}\huxbpad{4pt}} \tabularnewline[-0.5pt]


\hhline{}
\arrayrulecolor{black}

\multicolumn{1}{!{\huxvb{0}}l!{\huxvb{0}}}{\huxtpad{4pt}\raggedright {\fontsize{9.5pt}{11.4pt}\selectfont educ.C}\huxbpad{4pt}} &
\multicolumn{1}{r!{\huxvb{0}}}{\huxtpad{4pt}\raggedleft {\fontsize{9.5pt}{11.4pt}\selectfont ~~~~~~~~}\huxbpad{4pt}} &
\multicolumn{1}{r!{\huxvb{0}}}{\huxtpad{4pt}\raggedleft {\fontsize{9.5pt}{11.4pt}\selectfont 0.083~~~~}\huxbpad{4pt}} &
\multicolumn{1}{r!{\huxvb{0}}}{\huxtpad{4pt}\raggedleft {\fontsize{9.5pt}{11.4pt}\selectfont -0.020~~~~}\huxbpad{4pt}} \tabularnewline[-0.5pt]


\hhline{}
\arrayrulecolor{black}

\multicolumn{1}{!{\huxvb{0}}l!{\huxvb{0}}}{\huxtpad{4pt}\raggedright {\fontsize{9.5pt}{11.4pt}\selectfont }\huxbpad{4pt}} &
\multicolumn{1}{r!{\huxvb{0}}}{\huxtpad{4pt}\raggedleft {\fontsize{9.5pt}{11.4pt}\selectfont ~~~~~~~~}\huxbpad{4pt}} &
\multicolumn{1}{r!{\huxvb{0}}}{\huxtpad{4pt}\raggedleft {\fontsize{9.5pt}{11.4pt}\selectfont (0.460)~~~}\huxbpad{4pt}} &
\multicolumn{1}{r!{\huxvb{0}}}{\huxtpad{4pt}\raggedleft {\fontsize{9.5pt}{11.4pt}\selectfont (0.447)~~~}\huxbpad{4pt}} \tabularnewline[-0.5pt]


\hhline{}
\arrayrulecolor{black}

\multicolumn{1}{!{\huxvb{0}}l!{\huxvb{0}}}{\huxtpad{4pt}\raggedright {\fontsize{9.5pt}{11.4pt}\selectfont educ\verb|^|4}\huxbpad{4pt}} &
\multicolumn{1}{r!{\huxvb{0}}}{\huxtpad{4pt}\raggedleft {\fontsize{9.5pt}{11.4pt}\selectfont ~~~~~~~~}\huxbpad{4pt}} &
\multicolumn{1}{r!{\huxvb{0}}}{\huxtpad{4pt}\raggedleft {\fontsize{9.5pt}{11.4pt}\selectfont 0.025~~~~}\huxbpad{4pt}} &
\multicolumn{1}{r!{\huxvb{0}}}{\huxtpad{4pt}\raggedleft {\fontsize{9.5pt}{11.4pt}\selectfont 0.073~~~~}\huxbpad{4pt}} \tabularnewline[-0.5pt]


\hhline{}
\arrayrulecolor{black}

\multicolumn{1}{!{\huxvb{0}}l!{\huxvb{0}}}{\huxtpad{4pt}\raggedright {\fontsize{9.5pt}{11.4pt}\selectfont }\huxbpad{4pt}} &
\multicolumn{1}{r!{\huxvb{0}}}{\huxtpad{4pt}\raggedleft {\fontsize{9.5pt}{11.4pt}\selectfont ~~~~~~~~}\huxbpad{4pt}} &
\multicolumn{1}{r!{\huxvb{0}}}{\huxtpad{4pt}\raggedleft {\fontsize{9.5pt}{11.4pt}\selectfont (0.333)~~~}\huxbpad{4pt}} &
\multicolumn{1}{r!{\huxvb{0}}}{\huxtpad{4pt}\raggedleft {\fontsize{9.5pt}{11.4pt}\selectfont (0.354)~~~}\huxbpad{4pt}} \tabularnewline[-0.5pt]


\hhline{}
\arrayrulecolor{black}

\multicolumn{1}{!{\huxvb{0}}l!{\huxvb{0}}}{\huxtpad{4pt}\raggedright {\fontsize{9.5pt}{11.4pt}\selectfont nhdyrs}\huxbpad{4pt}} &
\multicolumn{1}{r!{\huxvb{0}}}{\huxtpad{4pt}\raggedleft {\fontsize{9.5pt}{11.4pt}\selectfont ~~~~~~~~}\huxbpad{4pt}} &
\multicolumn{1}{r!{\huxvb{0}}}{\huxtpad{4pt}\raggedleft {\fontsize{9.5pt}{11.4pt}\selectfont ~~~~~~~~}\huxbpad{4pt}} &
\multicolumn{1}{r!{\huxvb{0}}}{\huxtpad{4pt}\raggedleft {\fontsize{9.5pt}{11.4pt}\selectfont -0.090 ***}\huxbpad{4pt}} \tabularnewline[-0.5pt]


\hhline{}
\arrayrulecolor{black}

\multicolumn{1}{!{\huxvb{0}}l!{\huxvb{0}}}{\huxtpad{4pt}\raggedright {\fontsize{9.5pt}{11.4pt}\selectfont }\huxbpad{4pt}} &
\multicolumn{1}{r!{\huxvb{0}}}{\huxtpad{4pt}\raggedleft {\fontsize{9.5pt}{11.4pt}\selectfont ~~~~~~~~}\huxbpad{4pt}} &
\multicolumn{1}{r!{\huxvb{0}}}{\huxtpad{4pt}\raggedleft {\fontsize{9.5pt}{11.4pt}\selectfont ~~~~~~~~}\huxbpad{4pt}} &
\multicolumn{1}{r!{\huxvb{0}}}{\huxtpad{4pt}\raggedleft {\fontsize{9.5pt}{11.4pt}\selectfont (0.021)~~~}\huxbpad{4pt}} \tabularnewline[-0.5pt]


\hhline{}
\arrayrulecolor{black}

\multicolumn{1}{!{\huxvb{0}}l!{\huxvb{0}}}{\huxtpad{4pt}\raggedright {\fontsize{9.5pt}{11.4pt}\selectfont nhdsize.L}\huxbpad{4pt}} &
\multicolumn{1}{r!{\huxvb{0}}}{\huxtpad{4pt}\raggedleft {\fontsize{9.5pt}{11.4pt}\selectfont ~~~~~~~~}\huxbpad{4pt}} &
\multicolumn{1}{r!{\huxvb{0}}}{\huxtpad{4pt}\raggedleft {\fontsize{9.5pt}{11.4pt}\selectfont ~~~~~~~~}\huxbpad{4pt}} &
\multicolumn{1}{r!{\huxvb{0}}}{\huxtpad{4pt}\raggedleft {\fontsize{9.5pt}{11.4pt}\selectfont 0.129~~~~}\huxbpad{4pt}} \tabularnewline[-0.5pt]


\hhline{}
\arrayrulecolor{black}

\multicolumn{1}{!{\huxvb{0}}l!{\huxvb{0}}}{\huxtpad{4pt}\raggedright {\fontsize{9.5pt}{11.4pt}\selectfont }\huxbpad{4pt}} &
\multicolumn{1}{r!{\huxvb{0}}}{\huxtpad{4pt}\raggedleft {\fontsize{9.5pt}{11.4pt}\selectfont ~~~~~~~~}\huxbpad{4pt}} &
\multicolumn{1}{r!{\huxvb{0}}}{\huxtpad{4pt}\raggedleft {\fontsize{9.5pt}{11.4pt}\selectfont ~~~~~~~~}\huxbpad{4pt}} &
\multicolumn{1}{r!{\huxvb{0}}}{\huxtpad{4pt}\raggedleft {\fontsize{9.5pt}{11.4pt}\selectfont (0.493)~~~}\huxbpad{4pt}} \tabularnewline[-0.5pt]


\hhline{}
\arrayrulecolor{black}

\multicolumn{1}{!{\huxvb{0}}l!{\huxvb{0}}}{\huxtpad{4pt}\raggedright {\fontsize{9.5pt}{11.4pt}\selectfont nhdsize.Q}\huxbpad{4pt}} &
\multicolumn{1}{r!{\huxvb{0}}}{\huxtpad{4pt}\raggedleft {\fontsize{9.5pt}{11.4pt}\selectfont ~~~~~~~~}\huxbpad{4pt}} &
\multicolumn{1}{r!{\huxvb{0}}}{\huxtpad{4pt}\raggedleft {\fontsize{9.5pt}{11.4pt}\selectfont ~~~~~~~~}\huxbpad{4pt}} &
\multicolumn{1}{r!{\huxvb{0}}}{\huxtpad{4pt}\raggedleft {\fontsize{9.5pt}{11.4pt}\selectfont -0.411~~~~}\huxbpad{4pt}} \tabularnewline[-0.5pt]


\hhline{}
\arrayrulecolor{black}

\multicolumn{1}{!{\huxvb{0}}l!{\huxvb{0}}}{\huxtpad{4pt}\raggedright {\fontsize{9.5pt}{11.4pt}\selectfont }\huxbpad{4pt}} &
\multicolumn{1}{r!{\huxvb{0}}}{\huxtpad{4pt}\raggedleft {\fontsize{9.5pt}{11.4pt}\selectfont ~~~~~~~~}\huxbpad{4pt}} &
\multicolumn{1}{r!{\huxvb{0}}}{\huxtpad{4pt}\raggedleft {\fontsize{9.5pt}{11.4pt}\selectfont ~~~~~~~~}\huxbpad{4pt}} &
\multicolumn{1}{r!{\huxvb{0}}}{\huxtpad{4pt}\raggedleft {\fontsize{9.5pt}{11.4pt}\selectfont (0.330)~~~}\huxbpad{4pt}} \tabularnewline[-0.5pt]


\hhline{}
\arrayrulecolor{black}

\multicolumn{1}{!{\huxvb{0}}l!{\huxvb{0}}}{\huxtpad{4pt}\raggedright {\fontsize{9.5pt}{11.4pt}\selectfont extremely\_satisfied}\huxbpad{4pt}} &
\multicolumn{1}{r!{\huxvb{0}}}{\huxtpad{4pt}\raggedleft {\fontsize{9.5pt}{11.4pt}\selectfont ~~~~~~~~}\huxbpad{4pt}} &
\multicolumn{1}{r!{\huxvb{0}}}{\huxtpad{4pt}\raggedleft {\fontsize{9.5pt}{11.4pt}\selectfont ~~~~~~~~}\huxbpad{4pt}} &
\multicolumn{1}{r!{\huxvb{0}}}{\huxtpad{4pt}\raggedleft {\fontsize{9.5pt}{11.4pt}\selectfont 1.377 ***}\huxbpad{4pt}} \tabularnewline[-0.5pt]


\hhline{}
\arrayrulecolor{black}

\multicolumn{1}{!{\huxvb{0}}l!{\huxvb{0}}}{\huxtpad{4pt}\raggedright {\fontsize{9.5pt}{11.4pt}\selectfont }\huxbpad{4pt}} &
\multicolumn{1}{r!{\huxvb{0}}}{\huxtpad{4pt}\raggedleft {\fontsize{9.5pt}{11.4pt}\selectfont ~~~~~~~~}\huxbpad{4pt}} &
\multicolumn{1}{r!{\huxvb{0}}}{\huxtpad{4pt}\raggedleft {\fontsize{9.5pt}{11.4pt}\selectfont ~~~~~~~~}\huxbpad{4pt}} &
\multicolumn{1}{r!{\huxvb{0}}}{\huxtpad{4pt}\raggedleft {\fontsize{9.5pt}{11.4pt}\selectfont (0.393)~~~}\huxbpad{4pt}} \tabularnewline[-0.5pt]


\hhline{>{\huxb{1}}->{\huxb{1}}->{\huxb{1}}->{\huxb{1}}-}
\arrayrulecolor{black}

\multicolumn{1}{!{\huxvb{0}}l!{\huxvb{0}}}{\huxtpad{4pt}\raggedright {\fontsize{9.5pt}{11.4pt}\selectfont nobs}\huxbpad{4pt}} &
\multicolumn{1}{r!{\huxvb{0}}}{\huxtpad{4pt}\raggedleft {\fontsize{9.5pt}{11.4pt}\selectfont ~~~~~~~~}\huxbpad{4pt}} &
\multicolumn{1}{r!{\huxvb{0}}}{\huxtpad{4pt}\raggedleft {\fontsize{9.5pt}{11.4pt}\selectfont ~~~~~~~~}\huxbpad{4pt}} &
\multicolumn{1}{r!{\huxvb{0}}}{\huxtpad{4pt}\raggedleft {\fontsize{9.5pt}{11.4pt}\selectfont ~~~~~~~~}\huxbpad{4pt}} \tabularnewline[-0.5pt]


\hhline{>{\huxb{0.8}}->{\huxb{0.8}}->{\huxb{0.8}}->{\huxb{0.8}}-}
\arrayrulecolor{black}

\multicolumn{4}{!{\huxvb{0}}p{0.5\textwidth+6\tabcolsep}!{\huxvb{0}}}{\parbox[b]{0.5\textwidth+6\tabcolsep-4pt-4pt}{\huxtpad{4pt}\raggedright {\fontsize{9.5pt}{11.4pt}\selectfont  *** p $<$ 0.001;  ** p $<$ 0.01;  * p $<$ 0.05.}\huxbpad{4pt}}} \tabularnewline[-0.5pt]


\hhline{}
\arrayrulecolor{black}
\end{tabularx}
\end{table}


% Table created by stargazer v.5.2.2 by Marek Hlavac, Harvard University. E-mail: hlavac at fas.harvard.edu
% Date and time: Fri, Oct 26, 2018 - 13:53:08
% Requires LaTeX packages: dcolumn 
\begin{table}[!htbp] \centering 
  \caption{Estimated coefficients for logistic regression of seriously considering communities} 
  \label{tab:consider} 
\begin{tabular}{@{\extracolsep{5pt}}lD{.}{.}{-3} D{.}{.}{-3} } 
\\[-1.8ex]\hline 
\hline \\[-1.8ex] 
\\[-1.8ex] & \multicolumn{1}{c}{(1)} & \multicolumn{1}{c}{(2)}\\ 
\hline \\[-1.8ex] 
 Entropy & 0.010 & 0.011 \\ 
  & (0.016) & (0.017) \\ 
  \quad $\times$ Asian & -0.021 & -0.023 \\ 
  & (0.024) & (0.025) \\ 
  \quad $\times$ Black & -0.020 & -0.024 \\ 
  & (0.022) & (0.022) \\ 
  \quad $\times$ Latino & -0.023 & -0.027 \\ 
  & (0.023) & (0.023) \\ 
  Median Home Value &  & -0.0003 \\ 
  &  & (0.001) \\ 
  \quad $\times$ High income &  & 0.001 \\ 
  &  & (0.001) \\ 
  Asian & 1.330 & 1.498 \\ 
  & (2.130) & (2.143) \\ 
  Black & 1.891 & 2.089 \\ 
  & (1.885) & (1.930) \\ 
  Latino & 2.083 & 2.301 \\ 
  & (1.973) & (1.972) \\ 
  High income &  & -0.949^{*} \\ 
  &  & (0.518) \\ 
  Lives in the community & 0.290 & 0.291 \\ 
  & (0.246) & (0.245) \\ 
  Constant & -3.573^{***} & -3.412^{**} \\ 
  & (1.377) & (1.544) \\ 
 \hline \\[-1.8ex] 
Observations & \multicolumn{1}{c}{11,620} & \multicolumn{1}{c}{11,620} \\ 
Log Likelihood & \multicolumn{1}{c}{-2,602.205} & \multicolumn{1}{c}{-2,595.259} \\ 
Akaike Inf. Crit. & \multicolumn{1}{c}{5,222.411} & \multicolumn{1}{c}{5,214.519} \\ 
\hline 
\hline \\[-1.8ex] 
\textit{Note:}  & \multicolumn{2}{r}{$^{*}$p$<$0.1; $^{**}$p$<$0.05; $^{***}$p$<$0.01} \\ 
\end{tabular} 
\end{table} 


% Table created by stargazer v.5.2.2 by Marek Hlavac, Harvard University. E-mail: hlavac at fas.harvard.edu
% Date and time: Fri, Oct 26, 2018 - 13:53:08
% Requires LaTeX packages: dcolumn 
\begin{table}[!htbp] \centering 
  \caption{Estimated coefficients for logistic regression of never considering communities} 
  \label{tab:neverconsider} 
\begin{tabular}{@{\extracolsep{5pt}}lD{.}{.}{-3} D{.}{.}{-3} } 
\\[-1.8ex]\hline 
\hline \\[-1.8ex] 
\\[-1.8ex] & \multicolumn{1}{c}{(1)} & \multicolumn{1}{c}{(2)}\\ 
\hline \\[-1.8ex] 
 Entropy & 0.003 & 0.002 \\ 
  & (0.008) & (0.009) \\ 
  \quad $\times$ Asian & 0.002 & 0.004 \\ 
  & (0.016) & (0.016) \\ 
  \quad $\times$ Black & 0.005 & 0.007 \\ 
  & (0.012) & (0.012) \\ 
  \quad $\times$ Latino & -0.006 & -0.004 \\ 
  & (0.012) & (0.012) \\ 
  Median Home Value &  & 0.0001 \\ 
  &  & (0.0003) \\ 
  \quad $\times$ High income &  & -0.001 \\ 
  &  & (0.001) \\ 
  Asian & -1.152 & -1.207 \\ 
  & (1.375) & (1.383) \\ 
  Black & -1.008 & -1.090 \\ 
  & (1.069) & (1.078) \\ 
  Latino & 0.320 & 0.219 \\ 
  & (1.067) & (1.080) \\ 
  High income &  & 0.681^{***} \\ 
  &  & (0.260) \\ 
  Lives in the community & -1.755^{***} & -1.760^{***} \\ 
  & (0.297) & (0.299) \\ 
  Constant & -1.133 & -1.156 \\ 
  & (0.690) & (0.807) \\ 
 \hline \\[-1.8ex] 
Observations & \multicolumn{1}{c}{11,620} & \multicolumn{1}{c}{11,620} \\ 
Log Likelihood & \multicolumn{1}{c}{-5,591.998} & \multicolumn{1}{c}{-5,568.267} \\ 
Akaike Inf. Crit. & \multicolumn{1}{c}{11,202.000} & \multicolumn{1}{c}{11,160.530} \\ 
\hline 
\hline \\[-1.8ex] 
\textit{Note:}  & \multicolumn{2}{r}{$^{*}$p$<$0.1; $^{**}$p$<$0.05; $^{***}$p$<$0.01} \\ 
\end{tabular} 
\end{table} 



\section{Figures}
\begin{figure}
\centering
\includegraphics[scale=.35]{../analysis/images/satisfied_descriptive.png}
\caption{Proportion of respondents reporting that they are extremely or very satisfied with their neighborhood, by race}
\label{fig:satisfied}
\end{figure}

\end{document}